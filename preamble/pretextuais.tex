%Capa
\imprimircapa


%Folha de rosto sem número de página
\setcounter{page}{3}
\imprimirfolhaderosto*


% Ficha Catalográfica
% Não sei o que é ainda, mas parece que a universidade vai me fornecer uma
%após a defesa da minha tese. Quando ela fizer isso, tenho que comentar as
%linhas abaixo e descomentar o \includepdf:
\begin{fichacatalografica}
    \vspace*{\fill}
    \begin{center}
        \textsc{Inclua aqui o pdf com a ficha catalográfica fornecida pela BAE.}
    \end{center}
    \vspace*{\fill}
    %\includepdf{ficha-catalografica.pdf}
\end{fichacatalografica}


% Folha de aprovação
% Na versão final, tenho que excluir essas linhas e incluir o \includepdf
\newpage
\vspace*{\fill}
\begin{center}
    \textsc{Inclua aqui a folha de assinaturas.}
\end{center}
\vspace*{\fill}
\newpage
%\includepdf[pagecommand={\thispagestyle{plain}}]{folha-assinaturas.pdf}
\cleardoublepage


% Dedicatória
\begin{dedicatoria}
    \vspace*{\fill}
    \centering
    \noindent
    \textit{Dedico esta tese à todo mundo.}
    \vspace*{\fill}
\end{dedicatoria}

% Agradecimentos
\begin{agradecimentos}
    \lipsum[1-4]
\end{agradecimentos}

% Epígrafe
\begin{epigrafe}
    \vspace*{\fill}
    \begin{flushright}
        \textit{``Aqui jaz quatro anos da minha vida.''\\
        (Fernando Henrique de Sá)}
    \end{flushright}
\end{epigrafe}

% Resumos em português
\begin{resumo}
    \lipsum[1-2]
    \vspace{\onelineskip}
    \noindent\textbf{Palavras-chaves}: palavra-chave 1; palavra-chave 2; palavra-chave 3.
    \vspace{\fill}
\end{resumo}

% Resumo em Inglês
\begin{otherlanguage*}{english}
    \begin{center}{\ABNTEXchapterfont\huge Abstract}\end{center}
    \lipsum[1-2]
    \vspace{\onelineskip}
    \noindent\textbf{Keywords}: keyword 1; keyword 2; keyword 3.
    \vspace{\fill}
\end{otherlanguage*}
\cleardoublepage


% Lista de ilustrações
\pdfbookmark[0]{\listfigurename}{lof}
\listoffigures*
\cleardoublepage


% Lista de tabelas
\pdfbookmark[0]{\listtablename}{lot}
\listoftables*
\cleardoublepage


% Lista de Acronimos e Abreviações
\renewcommand{\nomname}{Lista de Acrônimos e Abreviações}
\pdfbookmark[0]{\nomname}{las}
\printnomenclature
\cleardoublepage


% Sumário
\pdfbookmark[0]{\contentsname}{toc}
\tableofcontents*
\cleardoublepage
