% --------------- PACOTES -----------------%

% Pacotes fundamentais
\usepackage{cmap}				% Mapear caracteres especiais no PDF
\usepackage{lmodern}			% Usa a fonte Latin Modern
\usepackage[T1]{fontenc}		% Selecão de códigos de fonte. hifenação correta.
\usepackage[utf8]{inputenc}		% Codificacao do documento (conversão automática dos acentos)
\usepackage{lastpage}			% Usado pela Ficha catalográfica
\usepackage{indentfirst}		% Indenta o primeiro parágrafo de cada seção.
\usepackage{xcolor}				% Controle das cores
\usepackage[pdftex]{graphicx}	% Inclusão de gráficos
\usepackage{epstopdf}           % Pacote que converte as figuras em eps para pdf


% Pacotes adicionais, usados apenas no âmbito do Modelo Canônico do abnteX2
\usepackage{nomencl}
\usepackage{amsmath}
\usepackage{bbm}
\usepackage[chapter]{algorithm}
\usepackage{algorithmic}
\usepackage{multirow}
\usepackage{rotating}
\usepackage{pdfpages}       % insere páginas pdf no arquivo
\usepackage{siunitx}        % pacote para padronizar display de números e unidades

\usepackage[brazilian,hyperpageref]{backref}	 % Paginas com as citacões na bibl

% Pacote que faz Citações padrão ABNT
\usepackage[alf,abnt-etal-cite=2,abnt-etal-list=0,abnt-etal-text=emph]{abntex2cite}

% Pacote de customização - Unicamp
\usepackage{unicamp}


% Pacotes de Debbuging:
\usepackage{lipsum}     % Pacote que gera texto dummy
%\usepackage{blindtext}  % Pacote que gera texto dummy
\usepackage{showlabels} % Pacote que mostra os labels das equações no pdf
\usepackage{todonotes}  % Pacote que insere notas no pdf 

% Posso criar definições de acrônimos e ele automaticamente coloca o nome completo
% no texto na primeira ocorrência ou o acrônimo nas referências subsequentes
%\usepackage[toc,acronym,nomain,nogroupskip,nonumberlist]{glossaries}%nonumberlist
%\makenoidxglossaries
%\frenchspacing
%\renewcommand{\finalnamedelim}{\ \&\ }
%\newacronym{linac}{LINAC}{Linear Accelerator}
\newacronym{cnpem}{CNPEM}{National Center for Research in Energy and Materials}
\newacronym{lnls}{LNLS}{Brazilian Synchrotron Light Laboratory}
\newacronym{3gls}{3th GLS}{Third Generation Light Sources}
\newacronym{4gls}{4th GLS}{Fourth Generation Light Sources}
\newacronym{mac}{MAC}{Machine Advisory Committee}
\newacronym{csr}{CSR}{coherent synchrotron radiation}
\newacronym{dsp}{DSP}{direct space charge}
\newacronym{isp}{ISP}{indirect space charge}
\newacronym{bcon}{BC}{boundary conditions}
\newacronym{maxeq}{ME}{Maxwell Equations}
\newacronym{bpms}{BPMs}{Beam Position Monitors}
\newacronym{lhs}{l.h.s.}{left hand side}
\newacronym{rhs}{r.h.s.}{right hand side}
\newacronym{sls}{SLS}{Synchrotron Light Source}
\newacronym{fel}{FEL}{Free Electron Lasers}
\newacronym{id}{ID}{Insertion Device}
\newacronym{ids}{IDs}{Insertion Devices}
\newacronym{als}{ALS}{Advanced Light Source}
\newacronym{cern}{CERN}{Centre Elkjoe Rlekoe Nleoik}
\newacronym{neg}{NEG}{NEG sigla}
\newacronym{pec}{PEC}{Perfect Electric Conductor}
\newacronym{esrf}{ESRF}{European Synchrotron Radiation Facility}
\newacronym{mba}{MBA}{Multi-Bend-Achromat}
\newacronym{tmci}{TMCI}{Transverse Mode-Coupling Instability}
\newacronym{lmci}{LMCI}{Longitudinal Mode-Coupling Instability}

% comando \gls


%---------------CONFIGURAÇÕES------------%

% informações do PDF. O pacote hyperref já foi incluido em abntex2, eu acho
\makeatletter
\hypersetup{
     	%pagebackref=true,
		pdftitle={\@title},
		pdfauthor={\@author},
    	pdfsubject={\imprimirpreambulo},
	    pdfcreator={LaTeX with abnTeX2},
		pdfkeywords={abnt}{latex}{abntex}{abntex2}{trabalho acadêmico},
		hidelinks,					% desabilita as bordas dos links
		colorlinks=false,       	% false: boxed links; true: colored links
    	linkcolor=blue,          	% color of internal links
    	citecolor=blue,        		% color of links to bibliography
    	filecolor=magenta,      	% color of file links
		urlcolor=blue,
%		linkbordercolor={1 1 1},	% set to white
		bookmarksdepth=4
}
\makeatother


% Espaçamentos entre linhas e parágrafos
\setlength{\parindent}{1.3cm} % Tamanho da identação do parágrafo
% Controle do espaçamento entre um parágrafo e outro:
\setlength{\parskip}{0.2cm}  % tente também \onelineskip


% Compila o índice
\makeindex
\makenomenclature
