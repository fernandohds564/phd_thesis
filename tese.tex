\documentclass[
	% -- opções da classe memoir --
	12pt,				% tamanho da fonte
	openright,			% capítulos começam em páginas ímpar (insere página vazia caso preciso)
	oneside,			% para impressão em verso e anverso. Oposto a oneside
	a4paper,		% tamanho do papel.
	% -- opções da classe abntex2 --
	chapter=TITLE,		% títulos de capítulos convertidos em letras maiúsculas
	section=TITLE,		% títulos de seções convertidos em letras maiúsculas
	%subsection=TITLE,	% títulos de subseções convertidos em letras maiúsculas
	%subsubsection=TITLE,% títulos de subsubseções convertidos em letras maiúsculas
	% -- opções do pacote babel --
    brazil,				% o último idioma é o principal do documento
	english,			% idioma adicional para hifenização
	%french,			% idioma adicional para hifenização
	%spanish,			% idioma adicional para hifenização
	sumario=tradicional,
	]{abntex2}

%%%%%% IMPORTAÇÃO DOS PACOTES %%%%%%
\documentclass[
	% -- opções da classe memoir --
	12pt,				% tamanho da fonte
	openright,			% capítulos começam em páginas ímpar (insere página vazia caso preciso)
	oneside,			% para impressão em verso e anverso. Oposto a oneside
	a4paper,		% tamanho do papel.
	% -- opções da classe abntex2 --
	chapter=TITLE,		% títulos de capítulos convertidos em letras maiúsculas
	section=TITLE,		% títulos de seções convertidos em letras maiúsculas
	%subsection=TITLE,	% títulos de subseções convertidos em letras maiúsculas
	%subsubsection=TITLE,% títulos de subsubseções convertidos em letras maiúsculas
	% -- opções do pacote babel --
    brazil,				% o último idioma é o principal do documento
	english,			% idioma adicional para hifenização
	%french,			% idioma adicional para hifenização
	%spanish,			% idioma adicional para hifenização
	sumario=tradicional,
	]{abntex2}

% --------------- PACOTES -----------------%
%%%%% conflict with float package, loaded by one of the packages bellow.
% \usepackage{floatrow} % pacote para criar ambientes float genéricos
% \floatsetup[table]{style=plaintop} % poe legendas de tabelas em cima.

% Pacotes fundamentais
\usepackage{cmap}				% Mapear caracteres especiais no PDF
\usepackage{lmodern}			% Usa a fonte Latin Modern
\usepackage[T1]{fontenc}		% Selecão de códigos de fonte. hifenação correta.
\usepackage[utf8]{inputenc}		% Codificacao do documento (conversão automática dos acentos)
\usepackage{lastpage}			% Usado pela Ficha catalográfica
\usepackage{indentfirst}		% Indenta o primeiro parágrafo de cada seção.
\usepackage{xcolor}				% Controle das cores
\usepackage[pdftex]{graphicx}	% Inclusão de gráficos
\graphicspath{{figures/}}
\usepackage{subcaption}
\usepackage{epstopdf}           % Pacote que converte as figuras em eps para pdf


% Pacotes adicionais, usados apenas no âmbito do Modelo Canônico do abnteX2
\usepackage{nomencl}
\usepackage{amsmath}   % usado para ter o environment pmatrix, align
\usepackage{mathtools} % usado para ter o environment pmatrix*
\usepackage{mathrsfs}  % usado para ter o curly H
\usepackage{xfrac}     % usado para ter o comando \sfrac
\usepackage{bbm}
\usepackage[chapter]{algorithm}
\usepackage{algorithmic}
\usepackage{multirow}
\usepackage{rotating}
\usepackage{pdfpages}       % insere páginas pdf no arquivo
\usepackage{siunitx}        % pacote para padronizar display de números e unidades

\usepackage[brazilian,hyperpageref]{backref}	 % Paginas com as citacões na bibl

\usepackage[toc,acronym,nomain,nogroupskip,nonumberlist]{glossaries}%nonumberlist
\newacronym{linac}{LINAC}{Linear Accelerator}
\newacronym{cnpem}{CNPEM}{National Center for Research in Energy and Materials}
\newacronym{lnls}{LNLS}{Brazilian Synchrotron Light Laboratory}
\newacronym{3gls}{3$^\text{rd}$ GLS}{Third Generation Light Sources}
\newacronym{4gls}{4$^\text{th}$ GLS}{Fourth Generation Light Sources}
\newacronym{mac}{MAC}{Machine Advisory Committee}
\newacronym{bsc}{BSC}{Beam Stay Clear}
\newacronym{supcond}{SC-RF}{Superconducting RF Cavity}
\newacronym{csr}{CSR}{coherent synchrotron radiation}
\newacronym{dsp}{DSP}{direct space charge}
\newacronym{isp}{ISP}{indirect space charge}
\newacronym{maxeq}{ME}{Maxwell Equations}
\newacronym{lhs}{l.h.s.}{left hand side}
\newacronym{rhs}{r.h.s.}{right hand side}
\newacronym{xfel}{FEL}{X-Ray Free Electron Lasers}
\newacronym{als}{ALS}{Advanced Light Source}
\newacronym{cern}{CERN}{European Organization for Nuclear Research}
\newacronym{neg}{NEG}{Non--Evaporable Getter}
\newacronym{pec}{PEC}{Perfect Electric Conductor}
\newacronym{esrf}{ESRF}{European Synchrotron Radiation Facility}
\newacronym{mba}{MBA}{Multi-Bend-Achromat}
\newacronym{tmci}{TMCI}{Transverse Mode-Coupling Instability}
\newacronym{lmci}{LMCI}{Longitudinal Mode-Coupling Instability}
\newacronym{fofb}{FOFB}{Fast Orbit Feedback System}
\newacronym{apu}{APU}{Adjustable Phase Undulator}
\newacronym{rms}{rms}{root-mean-square}
\newacronym{vfp}{VFP}{Vlasov Fokker Planck}
\newacronym{si}{SI}{International System of Units}
\newacronym{ibs}{IBS}{intrabeam scattering}
\newacronym{pic}{PIC}{particle in cell}
\newacronym{dc}{DC}{direct current}

\newglossaryentry{bpm}
{
  name={BPM},
  description={Beam Position Monitor},
  first={\glsentrydesc{bpm} (\glsentrytext{bpm})},
  plural={BPMs},
  descriptionplural={Beam Position Monitors},
  firstplural={\glsentrydescplural{bpm} (\glsentryplural{bpm})}
}
\newglossaryentry{sls}
{
  name={SLS},
  description={Synchrotron Light Source},
  first={\glsentrydesc{sls} (\glsentrytext{sls})},
  plural={SLSs},
  descriptionplural={Synchrotron Light Sources},
  firstplural={\glsentrydescplural{sls} (\glsentryplural{sls})}
}
\newglossaryentry{id}
{
  name={ID},
  description={Insertion Device},
  first={\glsentrydesc{id} (\glsentrytext{id})},
  plural={IDs},
  descriptionplural={Insertion Devices},
  firstplural={\glsentrydescplural{id} (\glsentryplural{id})}
}
\newglossaryentry{bbr}
{
  name={BBR},
  description={broad band resonator},
  first={\glsentrydesc{bbr} (\glsentrytext{bbr})},
  plural={BBRs},
  descriptionplural={broad band resonators},
  firstplural={\glsentrydescplural{bbr} (\glsentryplural{bbr})}
}
\newglossaryentry{hom}
{
  name={HOM},
  description={higher order mode},
  first={\glsentrydesc{hom} (\glsentrytext{hom})},
  plural={HOMs},
  descriptionplural={higher order modes},
  firstplural={\glsentrydescplural{hom} (\glsentryplural{hom})}
}


% Pacote que faz Citações padrão ABNT
\usepackage[
alf,
versalete,
abnt-and-type=&,
abnt-etal-cite=2,
abnt-etal-list=3,
abnt-doi=link,
% abnt-repeated-author-omit=yes,
abnt-etal-text=emph]{abntex2cite}
% \citebrackets[]
% \usepackage[style=numeric-comp]{biblatex}
\renewcommand{\authorcapstyle}{\small}
\renewcommand{\authorstyle}{\relax}

% Pacote de customização - Unicamp
\usepackage{unicamp}

% Pacote para fazer desenhos
\usepackage{tikz}
%\usetikzlibrary{⟨list of libraries separated by commas⟩} % carrega bibliotecas adicionais
\usetikzlibrary{calc,arrows.meta}


% Pacotes de Debbuging:
\usepackage{lipsum}     % Pacote que gera texto dummy
%\usepackage{blindtext}  % Pacote que gera texto dummy
% \usepackage{showlabels} % Pacote que mostra os labels das equações no pdf
\usepackage{todonotes}  % Pacote que insere notas no pdf

% Posso criar definições de acrônimos e ele automaticamente coloca o nome completo
% no texto na primeira ocorrência ou o acrônimo nas referências subsequentes
% \frenchspacing
% \renewcommand{\finalnamedelim}{\ \&\ }
% \makenoidxglossaries
% \newacronym{linac}{LINAC}{Linear Accelerator}
\newacronym{cnpem}{CNPEM}{National Center for Research in Energy and Materials}
\newacronym{lnls}{LNLS}{Brazilian Synchrotron Light Laboratory}
\newacronym{3gls}{3$^\text{rd}$ GLS}{Third Generation Light Sources}
\newacronym{4gls}{4$^\text{th}$ GLS}{Fourth Generation Light Sources}
\newacronym{mac}{MAC}{Machine Advisory Committee}
\newacronym{bsc}{BSC}{Beam Stay Clear}
\newacronym{supcond}{SC-RF}{Superconducting RF Cavity}
\newacronym{csr}{CSR}{coherent synchrotron radiation}
\newacronym{dsp}{DSP}{direct space charge}
\newacronym{isp}{ISP}{indirect space charge}
\newacronym{maxeq}{ME}{Maxwell Equations}
\newacronym{lhs}{l.h.s.}{left hand side}
\newacronym{rhs}{r.h.s.}{right hand side}
\newacronym{xfel}{FEL}{X-Ray Free Electron Lasers}
\newacronym{als}{ALS}{Advanced Light Source}
\newacronym{cern}{CERN}{European Organization for Nuclear Research}
\newacronym{neg}{NEG}{Non--Evaporable Getter}
\newacronym{pec}{PEC}{Perfect Electric Conductor}
\newacronym{esrf}{ESRF}{European Synchrotron Radiation Facility}
\newacronym{mba}{MBA}{Multi-Bend-Achromat}
\newacronym{tmci}{TMCI}{Transverse Mode-Coupling Instability}
\newacronym{lmci}{LMCI}{Longitudinal Mode-Coupling Instability}
\newacronym{fofb}{FOFB}{Fast Orbit Feedback System}
\newacronym{apu}{APU}{Adjustable Phase Undulator}
\newacronym{rms}{rms}{root-mean-square}
\newacronym{vfp}{VFP}{Vlasov Fokker Planck}
\newacronym{si}{SI}{International System of Units}
\newacronym{ibs}{IBS}{intrabeam scattering}
\newacronym{pic}{PIC}{particle in cell}
\newacronym{dc}{DC}{direct current}

\newglossaryentry{bpm}
{
  name={BPM},
  description={Beam Position Monitor},
  first={\glsentrydesc{bpm} (\glsentrytext{bpm})},
  plural={BPMs},
  descriptionplural={Beam Position Monitors},
  firstplural={\glsentrydescplural{bpm} (\glsentryplural{bpm})}
}
\newglossaryentry{sls}
{
  name={SLS},
  description={Synchrotron Light Source},
  first={\glsentrydesc{sls} (\glsentrytext{sls})},
  plural={SLSs},
  descriptionplural={Synchrotron Light Sources},
  firstplural={\glsentrydescplural{sls} (\glsentryplural{sls})}
}
\newglossaryentry{id}
{
  name={ID},
  description={Insertion Device},
  first={\glsentrydesc{id} (\glsentrytext{id})},
  plural={IDs},
  descriptionplural={Insertion Devices},
  firstplural={\glsentrydescplural{id} (\glsentryplural{id})}
}
\newglossaryentry{bbr}
{
  name={BBR},
  description={broad band resonator},
  first={\glsentrydesc{bbr} (\glsentrytext{bbr})},
  plural={BBRs},
  descriptionplural={broad band resonators},
  firstplural={\glsentrydescplural{bbr} (\glsentryplural{bbr})}
}
\newglossaryentry{hom}
{
  name={HOM},
  description={higher order mode},
  first={\glsentrydesc{hom} (\glsentrytext{hom})},
  plural={HOMs},
  descriptionplural={higher order modes},
  firstplural={\glsentrydescplural{hom} (\glsentryplural{hom})}
}

% comando \gls


%---------------CONFIGURAÇÕES------------%

% informações do PDF. O pacote hyperref já foi incluido em abntex2, eu acho
\makeatletter
\hypersetup{
     	%pagebackref=true,
		pdftitle={\@title},
		pdfauthor={\@author},
    	pdfsubject={\imprimirpreambulo},
	    pdfcreator={LaTeX with abnTeX2},
		pdfkeywords={abnt}{latex}{abntex}{abntex2}{trabalho acadêmico},
		hidelinks,					% desabilita as bordas dos links
		colorlinks=false,       	% false: boxed links; true: colored links
    	linkcolor=blue,          	% color of internal links
    	citecolor=blue,        		% color of links to bibliography
    	filecolor=magenta,      	% color of file links
		urlcolor=blue,
%		linkbordercolor={1 1 1},	% set to white
		bookmarksdepth=4
}
\makeatother

% Espaçamentos entre linhas e parágrafos
\setlength{\parindent}{1.3cm} % Tamanho da identação do parágrafo
% Controle do espaçamento entre um parágrafo e outro:
\setlength{\parskip}{0.2cm}  % tente também \onelineskip

%\setlength{\mathindent}{0cm}

% Compila o índice
\makeindex
\makenomenclature


% Informacoes de dados para CAPA e FOLHA DE ROSTO:
\titulo{Estudo de Impedâncias e Instabilidades Coletivas aplicadas aos aceleradores do LNLS.}
\autor{Fernando Henrique de Sá}
\local{Campinas}
\data{2016}
\orientador{Prof. Dr. Antônio Rubens Britto de Castro}
\coorientador[Co-orientador]{Prof. Dr. Sílvio}
\instituicao{%
    UNIVERSIDADE ESTADUAL DE CAMPINAS
    \par
    Instituto de Física Gleb Wataghin (IFGW)
    }
\tipotrabalho{Tese (Doutorado)}
% O preambulo deve conter o tipo do trabalho, o objetivo, o nome da instituição
% e a área de concentração
\preambulo{Tese apresentada ao Instituto de Física da Universidade Estadual
           de Campinas como parte dos requisitos exigidos para a obtenção do
		   título de Doutor em Física, com ênfase em Física de Aceleradores.}

\newcommand{\udefint}[2]{\int\!\!\text{d}#1 #2}                 % Integral indefinida
\newcommand{\udefoint}[2]{\oint\!\text{d}#1 #2}               % Integral fechada
\newcommand{\defint}[4]{\int_{#3}^{#4}\!\!\text{d}#1 #2}       % Integral definida
\newcommand{\infint}[2]{\defint{#1}{#2}{-\infty}{\infty}}      % Integra definida infinito
\newcommand{\dertot}[3][{}]{\frac{\mathrm{d}^{#1}#2}{\mathrm{d} #3^{#1}}} % Derivada total
\newcommand{\derpar}[3][{}]{\frac{\partial^{#1}#2}{\partial #3^{#1}}}     % Derivada parcial
\newcommand{\average}[2][{}]{\left\langle #2 \right\rangle_{#1}}
\newcommand{\vect}[1]{\overrightarrow{\boldsymbol{#1}}}
\newcommand{\versor}[1]{\boldsymbol{\hat#1}}
\newcommand{\tensor}[1]{\overleftrightarrow{\boldsymbol{#1}}} % Tensor
\newcommand{\fourier}[1]{\tilde{#1}}  % representation of the Fourier Transform
\newcommand{\real}[1]{\Re\left\{#1\right\}}
\newcommand{\imag}[1]{\Im\left\{#1\right\}}
\newcommand{\engw}[1]{\emph{#1}}        % Palavra em Língua Inglesa


\includeonly{
%content/Chap1-introducao,
content/Chap3-impedancias_e_wakes,
%content/App1-derivacoes,
}

\begin{document}
	% Retira espaço extra obsoleto entre as frases
	\frenchspacing

	%---------- Elementos pré-textuais --------------%
	\pretextual
	%Capa
\imprimircapa


%Folha de rosto sem número de página
\setcounter{page}{3}
\imprimirfolhaderosto*


% Ficha Catalográfica
% Não sei o que é ainda, mas parece que a universidade vai me fornecer uma
%após a defesa da minha tese. Quando ela fizer isso, tenho que comentar as
%linhas abaixo e descomentar o \includepdf:
\begin{fichacatalografica}
    \vspace*{\fill}
    \begin{center}
        \textsc{Inclua aqui o pdf com a ficha catalográfica fornecida pela BAE.}
    \end{center}
    \vspace*{\fill}
    %\includepdf{ficha-catalografica.pdf}
\end{fichacatalografica}


% Folha de aprovação
% Na versão final, tenho que excluir essas linhas e incluir o \includepdf
\newpage
\vspace*{\fill}
\begin{center}
    \textsc{Inclua aqui a folha de assinaturas.}
\end{center}
\vspace*{\fill}
\newpage
%\includepdf[pagecommand={\thispagestyle{plain}}]{folha-assinaturas.pdf}
\cleardoublepage


% Dedicatória
\begin{dedicatoria}
    \vspace*{\fill}
    \centering
    \noindent
    \textit{Dedico esta tese à todo mundo.}
    \vspace*{\fill}
\end{dedicatoria}

% Agradecimentos
\begin{agradecimentos}
    \lipsum[1-4]
\end{agradecimentos}

% Epígrafe
\begin{epigrafe}
    \vspace*{\fill}
    \begin{flushright}
        \textit{``Aqui jaz seis anos da minha vida.''\\
        (Fernando Henrique de Sá)}
    \end{flushright}
\end{epigrafe}

% Resumos em português
\begin{resumo}
    \lipsum[1-2]
    \vspace{\onelineskip}
    \noindent\textbf{Keywords}: keyword 1; keyword 2; keyword 3.
    \vspace{\fill}
\end{resumo}

% Resumo em Inglês
\begin{otherlanguage*}{brazil}
    \begin{center}{\ABNTEXchapterfont\huge Resumo}\end{center}

    \lipsum[1-2]
    \vspace{\onelineskip}
    \noindent\textbf{Palavras-chaves}: palavra-chave 1; palavra-chave 2; palavra-chave 3.
    \vspace{\fill}
\end{otherlanguage*}
\cleardoublepage


% Lista de ilustrações
\pdfbookmark[0]{\listfigurename}{lof}
\listoffigures*
\cleardoublepage


% Lista de tabelas
\pdfbookmark[0]{\listtablename}{lot}
\listoftables*
\cleardoublepage


% Lista de Acronimos e Abreviações
\renewcommand{\nomname}{Lista de Acrônimos e Abreviações}
\pdfbookmark[0]{\nomname}{las}
\printnomenclature
\cleardoublepage


% Sumário
\pdfbookmark[0]{\contentsname}{toc}
\tableofcontents*
\cleardoublepage

	
	%---------- Elementos textuais ------------------%
	\textual
	
%	\chapter{Introduction} 	\label{cap:intro}

In scientific facilities commomly known as synchrotron lights sources (SLS) the interaction between light and matter is used to study properties of a variaty of materials. Throught techniques involving absortion, reflection, refraction and scattering of light of different 'colors' by the materials under study, scientists can determine their atomic structure and composition.

The frequency of the light used in these facilities range from tera-hertz to hard X-rays and its origin is always related to synchrotron emittion of radiation by charged particles, hence the name of the facility. The light emitted by centripetal acceleration of ultra-relativistic particles has unique properties for the use in scientific investigation. Besides its broad spectrum, among the advantages in relation to other methods are the high total flux emitted and the strong collimation.


Every SLS depends on a particle accelerator to generate, accelerate and excite the charged particles, generally electrons, to emit the radiation. Most light sources uses synchrotron storage rings where subatomic charged particles, generally electrons, are extracted from materials and accelerated to relativistic energies in order to produce radiation. Among the several types of accelerators the synchrotron is a machine 

A Synchrotron Light Source is a class of Light Source where the light is generated in a circular storage ring which particles are stored for hours in almost circular stable orbits.


and depending on the type of accelerator the light may acquire some other important properties for its use as scientific tool. There are several types of accelerator, however, nowadays the most  synchrotron light sources are based in two types. The linear accelerators, used in Free Electron Lasers (FEL) and X-ray Free Electron Lasers (X-FELS), and the synchrotron storage ring accelerators.

In both types of accelerators the light is generated by magnetic devices called insertion devices (IDs) that generates an alternating magnetic field along the particle trajectory which makes them wiggle. 


In the latter, ultra-relativistic charged particles are stored in approximately round machines for hours, hence the name storage rings and they generate radiation 


Ao projetar um acelerador de partículas, primeiramente é feito um estudo da interação entre uma única partícula e os campos externos, gerados pelos elementos magnéticos e elétricos do acelerador. Nesse estudo, aspectos complexos como a dinâmica não linear, erros de campo, multipolos e erros de alinhamento são considerados.

Nos aceleradores atuais, como por exemplo os anéis de armazenamento de fontes de Luz Síncrotron, há uma necessidade crescente de se obter feixes cada vez mais intensos e colimados e menores nas direções transversais . Essas características fazem com que o estudo de outro tipo de interação seja crucial para o
\textit{design} de um acelerador. Essa interação é de origem coletiva, ou seja, está relacionada a efeitos que campos gerados pelas próprias partículas do feixe causam nas outras.

Efeitos coletivos geralmente são pequenos quando comparados com os dos campos externos e são tratados como uma perturbação. Contudo, conforme a intensidade do feixe aumenta, esses campos auto-induzidos se tornam cada vez mais intensos e podem gerar instabilidades coletivas, que fazem com que o feixe seja perdido
ou então sofra oscilações, coerentes ou incoerentes, que deturpam as características desejáveis para a radiação síncrotron gerada.

Desde Janeiro de 2011 está sendo desenvolvido um novo modo de operação para o anel de armazenamento de elétrons do LNLS, o UVX. O objetivo desse projeto é diminuir a emitância da máquina de 100~nm.rad para valores menores que 50~nm.rad.

Um modo de operação já foi desenvolvido teoricamente, simulado, e implementado no anel, como descrito em \cite{Fernando}. Testes com baixa corrente evidenciaram uma diminuição no tamanho horizontal do feixe compatível com o esperado teoricamente. Contudo, quando testado com alta corrente no anel (250~mA), o feixe apresentou uma redução de tamanho horizontal muito menor, além de demonstrar instabilidade e alto acoplamento transversal.

Tais indícios sugeriram que uma instabilidade coletiva tinha sido ativada devido ao aumento da densidade de elétrons no feixe. Tendo isso em mente, começamos a desenvolver um estudo de efeitos coletivos para determinar a instabilidade e tentar curá-la.

Além desse objetivo mais imediato, o estudo de efeitos coletivos também visa adquirir um conhecimento que será muito importante para o design da nova fonte de luz síncrotron do LNLS, o Sirius \cite{Sirius}, haja vista que esses efeitos são os principais limitantes de desempenho das fontes de terceira geração.


%%%%%%%%%%%%%%%%%%%%%%%%%%%%%%%%%%%%%%%%%%%%%%%%%%%%%%%%%%%%%%%%%%%%%%%%%%%%%%%%%%%%%%%%%%%%%%%%%%%%%%%%%
%%%%%%%%%%%%%%%%%%%%%%%%%%%%%%%%%%%%%%%%%%%%%%%%%%%%%%%%%%%%%%%%%%%%%%%%%%%%%%%%%%%%%%%%%%%%%%%%%%%%%%%%%
\section{Modelo do Acelerador}
Um anel de armazenamento de elétrons consiste em uma rede de elementos magnéticos e elétricos que confinam o movimento das partículas em uma órbita fechada para que elas realizem um número elevado de revoluções.

O estudo da dinâmica de interação entre as partículas armazenadas e os campos externos que condicionam seu movimento geralmente é feito em um sistema de coordenadas adequado que mede os desvios das trajetórias em relação à de uma partícula ideal, com energia nominal e posições transversais e longitudinal adequadas.

Esse estudo pode se tornar bastante complicado e por isso um modelo simplificado é usado quando abordamos o tema de instabilidades coletivas. Nesse modelo, apenas as características fundamentais da dinâmica são mantidas, para facilitar a interpretação física e a resolução matemática das equações.

Considera-se que o anel de armazenamento é circular, com raio $R = L/2\pi$, e que os elétrons sofrem oscilações harmônicas nas três direções: longitudinal, $s$, radial, $x$ e vertical $z$, de modo que a equação de movimento é dada por:
\begin{equation}\label{modelo}
 u'' + \omega_u^2 u = 0,
\end{equation}
onde $u = x, z, s$ e $\omega_u$ é a frequência fundamental de oscilação naquela direção, dada por:
\begin{equation}\nonumber
  \omega_u = \frac{\nu_u}{R},
\end{equation}
com $\nu_u$ sendo a sintonia bétatron para as oscilações transversais e síncrotron para as longitudinais.

Quando uma força, além daquelas provenientes dos elementos da rede, atua sobre as partículas, o movimento passa a ser forçado e \eqref{modelo} fica:
\begin{equation}\label{modeloforcado}
 u'' + \omega_u^2 u = \frac{F(t)}{m\gamma},
\end{equation}
supondo que a solução para a equação acima é oscilatória com frequência $\Omega \approx \omega_u$, veremos, por substituição na equação de movimento, que a variação na frequência de oscilação será dada por:
\begin{equation}
\Delta \omega_u \equiv \Omega - \omega_u \approx - \frac{F(t)}{2\omega_u \gamma m u_0}e^{i\Omega t}.
\end{equation}
onde a parte imaginária de $\Delta \omega$ representa um termo exponencial de amortecimento ou excitação no movimento oscilatório. Assim, podemos definir a taxa de crescimento da instabilidade como:
\begin{equation}
 \tau^{-1} \equiv \Im\{\Delta \omega_u\}
\end{equation}
onde taxas de crescimento positivas indicam a existência de instabilidade e taxas negativas implicam em maior amortecimento do feixe.

Os modelos de efeitos coletivos que serão discutidos terão a finalidade de descrever força $F(t)$, de forma que possamos analisar o desvio de frequência e determinar a ocorrência ou não da instabilidade.

Um modelo um pouco mais elaborado que o descrito aqui consideraria os mecanismos de amortecimento natural do feixe, de forma que a existência da instabilidade dependeria de sua taxa de crescimento ser maior ou menor que a taxa de amortecimento natural.

%%%%%%%%%%%%%%%%%%%%%%%%%%%%%%%%%%%%%%%%%%%%%%%%%%%%%%%%%%%%%%%%%%%%%%%%%%%%%%%%%%%%%%%%%%%%%%%%%%%%%%%%%
%%%%%%%%%%%%%%%%%%%%%%%%%%%%%%%%%%%%%%%%%%%%%%%%%%%%%%%%%%%%%%%%%%%%%%%%%%%%%%%%%%%%%%%%%%%%%%%%%%%%%%%%%
\section{Efeitos Coletivos}
A figura de mérito de uma Fonte de Luz Síncrotron é o brilho da luz gerada, que pode ser definido como:
\begin{equation}\label{eq:defbrilho}
 B=\frac{\dot{N}_\gamma}{4\pi^2\sigma_x\sigma_y\sigma_{x'}\sigma_{y'}\mathrm{d}E
/E_\gamma}\quad \left(\mathrm{\frac{\text{fótons} \cdot s^{-1}}{mm^2 \cdot
mrad^2 \cdot 0,1\% bandwidth}}\right)
\end{equation}
onde $\sigma_u$ e $\sigma_{u'}$, com $u= x, y$, são os desvios padrão das distribuições espaciais e angulares do feixe de elétrons, $\dot{N}_\gamma$ é o fluxo de elétrons integrado em um intervalo de 0.1\% de banda de energia, $\mathrm{d}E/E_\gamma$.

Assim, estamos interessados em determinar e evitar todos os efeitos coletivos que degradem o brilho da máquina, que podem ser
\begin{itemize}
 \item Oscilações transversais: que aumentam o tamanho e a divergência do feixe de elétrons.
 \item Oscilações longitudinais: que implicam em um aumento do espalhamento de energia, $\sigma_E$, dos elétrons e um consequente aumento do tamanho horizontal, além de alargamento das linhas de energia dos onduladores.
 \item Instabilidades: limitam a corrente máxima do anel, $I$, e consequentemente o fluxo de fótons, $\dot{N}_\gamma \sim I$.
\end{itemize}

Os efeitos coletivos se manifestam de duas maneiras: a primeira é por meio da interação eletromagnética direta entre as partículas, denominada efeito das cargas espaciais (\textit{space charge} em Inglês) e a outra é através de \textit{Wake Fields}, que são o resultado da interação dos campos eletromagnéticos gerados pelos elétrons com as paredes mais próximas e atuam em partículas subsequentes \cite{Khan}.

O efeito das cargas espaciais é o mais conhecido dentre os efeitos coletivos, mas é pequeno para anéis de armazenamento de elétrons, porque ele decresce com o aumento da energia relativística $\gamma$. Tal propriedade pode ser percebida calculando a força de atração que uma linha de corrente com densidade de carga $\lambda$ uniforme, se movendo com velocidade $\beta c$ exerce sobre uma carga $q$  a uma distância $r$ viajando paralelamente com mesma velocidade:
\begin{equation}
 F= q(E - \beta B)=\frac{q\lambda}{2\pi \epsilon_0 \gamma^2 r},
\end{equation}
onde fica evidente o comportamento decrescente da força de interação com o aumento da energia.

Por outro lado, os \textit{Wake Fields} não dependem da energia do feixe, mas sim da geometria da câmara de vácuo e das características dos materiais que a constituem.

Um dos tipos de \textit{Wake Fields} mais conhecidos, que possui uma solução analítica simples de ser obtida é o da parede resistiva. Este problema consiste em uma câmara de vácuo cilíndrica de raio $b$, infinitamente espessa e longa, formada por um material de condutividade $\sigma$. A uma distância $a<b$ do seu centro há uma carga $q$ viajando longitudinalmente com velocidade $c$ e a uma distância $z$ atrás dela há outra carga $q$, também viajando com velocidade $c$.

A resolução desse problema envolve expandir as densidades de carga e de corrente que geram os \textit{Wake Fields} em termos de anéis de carga  concêntricos ao eixo de simetria da câmara de vácuo:
\begin{align}
\rho(r,\theta,s,t)  = \sum_{n=0}^\infty \rho_m &=  \sum_{m=0}^\infty
\frac{I_m}{(1+\delta_{0m})\pi a^{m+1}} \delta(r-a) \delta(s-ct) \cos(m\theta)\\
\vec{j}(r,\theta,s,t) = \sum_{m=0}^\infty\vec{j}_m &=\sum_{m=0}^\infty c \rho_m
\hat{s}
\end{align}
onde $I_m = q a^m$ é o momento de multipolo de ordem $m$ da partícula. Usar essa base para expansão da carga facilita bastante os cálculos porque o problema se resume a resolver as Equações de Maxwell para a m-ésima componente dos campos.

Como a câmara de vácuo não é perfeitamente condutora, há campos não nulos nessa região também. Nesse caso, as fontes são dadas por:
\begin{equation}
 \rho = 0, \qquad \vec{j} = \sigma \vec{E}.
\end{equation}

Após escrever as Equações de Maxwell em coordenadas cilíndricas e substituir as densidades de carga e corrente nas expressões, é possível determinar a dependência em $\theta$ dos campos. Também, como há simetria longitudinal, os campos não dependem da posição relativa ao anel $s$, mas apenas da distância da partícula fonte, $z=s-ct$, de modo que $z>0$ corresponde a posições a frente da fonte.

Devido a essa propriedade, que acopla o tempo e a coordanada longitudinal, é interessante fazer uma Transformada de Fourier em $z$ para transformar a equação diferencial parcial em $r,t$ e $s$ em uma ordinária apenas em $r$. Assim, escrevemos:
\begin{align}
 (E_s,E_r,B_\theta)(r,\theta,z) &= \cos(m\theta)
\int_{-\infty}^\infty\!\!\frac{dk}{2\pi}(\tilde{E}_s,\tilde{E}_r,\tilde{B} _\theta)(r,k),\\
 (B_s,B_r,E_\theta)(r,\theta,z) &= \sin(m\theta)
\int_{-\infty}^\infty\!\!\frac{dk}{2\pi}(\tilde{B}_s,\tilde{B}_r,\tilde{E} _\theta)(r,k).
\end{align}

Aplicando as condições de contorno de que o campo não pode divergir em $r=0$ e de continuidade dos campos tangenciais na parede da câmara, obtém-se as expressões para todas as regiões do espaço. Dessa forma, é possível determinar a força que atua em uma partícula teste a uma distância $z$ atrás da fonte:
\begin{align}\nonumber \label{eq:res.wall}
(F_\lVert)_m = \frac{eI_m}{\pi b^{2m+1}(1+\delta_{0m})}
\sqrt{\frac{c}{\sigma}}r^m \cos(m\theta) \frac{1}{|z|^{3/2}}\\
(\vec{F}_\perp)_m = \frac{2eI_m}{\pi b^{2m+1}} \sqrt{\frac{c}{\sigma}}
m r^{m-1} \frac{1}{|z|^{1/2}} \left(\hat{r}\cos(m\theta) -\hat{\theta}
\sin(m\theta)\right).
\end{align}
A solução acima é uma aproximação válida para uma região
\begin{displaymath}
\eta^{1/3}b\ll|z|\ll \frac{b}{\eta}, \qquad \eta \equiv \frac{c}{4\pi \sigma b},
\end{displaymath}
sendo que para uma câmara de vácuo de alumínio com $5 cm$ de raio, temos que $6\times10^{-6} m\ll |z| \ll 3\times10^7 m$. A solução exata, assim como a resolução detalhada desse problema pode ser encontrada em \cite{Chao}.

A simplicidade de \eqref{eq:res.wall} e facilidade na resolução são consequências das simetrias longitudinal e axial do problema original, que permitiram o desacoplamento dos multipolos e a dependência apenas em $|z|$. Quando uma dessas simetrias é quebrada, a determinação dos campos e das forças que atuam sobre o feixe fica muito complicada, de modo que aproximações devem ser feitas.

Uma dessas aproximações é a de feixe rígido, que já foi usada implicitamente na solução da parede resistiva. Ela consiste em não considerar o efeito causado pela ação do \textit{Wake Field} no feixe, ou seja, tanto as posições transversais como a longitudinal das partículas do feixe são fixas. Apesar dessa aproximação não ser auto-consistente e não prever o surgimento de instabilidades coletivas, ela é valida para anéis de armazenamento de elétrons, em que $\beta \approx 1$, e permite obter expressões  que são usadas posteriormente em modelos de instabilidades.

%%%%%%%%%%%%%%%%%%%%%%%%%%%%%%%%%%%%%%%%%%%%%%%%%%%%%%%%%%%%%%%%%%%%%%%%%%%%%%%%%%%%%%%%%%%%%%%%%%%%%%%%%
\section{\textit{Wake Functions}}

Em geral, para seções da câmara de vácuo que não possuem simetria translacional, o \textit{Wake Field} e a \textit{Wake Force} que atua sobre uma partícula prova passam a depender das variáveis $s$ e $t$ separadamente e as equações se tornam muito complicadas para serem resolvidas.

Para tentar recuperar a dependência apenas de $z = s - \beta c t$, ao invés de tentarmos determinar a força instantânea que partícula sente, vamos olhar para o impulso recebido quando ela passa por toda a seção geradora do \textit{wake field}. Matematicamente, usando a aproximação de feixe rígido, temos:

\begin{equation}
\beta c \Delta\vec{ p}(x,y,z) = \int^{L/2}_{-L/2}\!\! ds
\vec{F}(x,y,s,(s-z)/\beta c)
\end{equation}
onde o $L/2$ nos limites de integração não representa o comprimento da seção que gera o \textit{Wake Field} mas sim um comprimento adequado para calcular a integral de modo que os campos elétrico e magnético nesse ponto sejam iguais, ou seja, não sofram mais o efeito da assimetria gerada pela estrutura. Por exemplo, se a estrutura em questão é periódica, o limite de integração pode ser o período da estrutura, se é do tipo cavidade, o limite de integração deve ser muito maior que o tamanho da cavidade.

As duas aproximações feitas acima (feixe rígido e limites de integração) impõem restrições sobre o momento que a partícula recebe dos \textit{Wake Field}. Tais restrições são evidenciadas pelo Teorema de \textit{Panofsky-Wenzel} \cite{Bio}:
\begin{equation}
 \vec{\nabla} \times \Delta\vec{ p} = \vec{0}
\quad \Rightarrow \left\lbrace 
\begin{array}{l}
(\vec{\nabla} \times \Delta \vec{p})_s = 0
\xrightarrow{coord.~cartesianas} \frac{\partial \Delta p_x}{\partial y}
= \frac{\partial \Delta p_y}{\partial x} \\
\frac{\partial \Delta \vec{ p}_\bot}{\partial z} =
\vec{\nabla}_\bot \Delta p_s
\end{array} \right.
\end{equation}
onde $\Delta\vec{ p}$ é o momento linear ganho por uma partícula de prova a uma distância $z$ atrás da fonte.

Na demonstração do teorema acima são usadas apenas as equações de Maxwell, a força de Lorentz e as duas aproximações. Portanto, ele não depende das condições de contorno e nem da fonte, além de não exigir que $\beta = 1$ (apenas que ele seja alto o suficiente para a aproximação de feixe rígido ser válida).

Há um corolário do Teorema de \textit{Panofsky-Wenzel} que afirma:
\begin{equation}
 \beta = 1\quad \Rightarrow \quad \vec{\nabla}_\bot \cdot
\Delta\vec{ p}_\bot = 0.
\end{equation}

Devido a sua generalidade e poder de simplificação, esse teorema é fundamental para o estudo de efeitos coletivos. Uma aplicação direta pode ser obtida o escrevendo em coordenadas cilíndricas:
\begin{align}
 \frac{\partial }{\partial r} (r\Delta p_\theta)& = \frac{\partial}{\partial
\theta}(\Delta p_r) \\
 \frac{\partial}{\partial z}(\Delta p_\theta) &= \frac{1}{r}
\frac{\partial}{\partial \theta}(\Delta p_s) \\
 \frac{\partial}{\partial z}(\Delta p_r) &= \frac{\partial}{\partial r}(\Delta
p_s) \\
(\beta = 1) \quad  \frac{\partial}{\partial r}(r\Delta p_r) &=
\frac{\partial}{\partial \theta}(\Delta p_\theta).
\end{align}

As equações acima podem ser facilmente resolvidas de modo que a solução pode ser convenientemente escrita da seguinte forma:
\begin{align}
\label{eq:W}
 \beta c \Delta \vec{p}_\bot &= -q \sum_{m=0}^\infty I_m W_m(z) m
r^{m-1}\left(\hat{r}\cos(m\theta) - \hat{\theta}\sin(m\theta)\right)\\
\label{eq:W'}
\beta c \Delta p_s &= -q \sum_{m=0}^\infty I_m W'_m(z) r^{m}\cos(m\theta)
\end{align}
onde $I_m$ é o m-ésimo multipolo da distribuição de carga\footnote{para uma carga pontual a uma distância $a$ do eixo de simetria $I_m=qa^m$.} que gera o \textit{Wake Field} e $q$ é a carga da partícula de prova. As funções $W_m$ e $W'_m$, que estão relacionadas entre si por
\begin{equation}
 W'_m(z)=\frac{d}{dz}W_m(z),
\end{equation}
são chamadas \textit{Wake Functions} longitudinal e transversal, respectivamente, e são determinadas pelas condições de contorno do problema, devendo ser resolvidas independentemente.

Apesar de a forma exata das \textit{Wake Functions} depender do problema específico a ser tratado, é possível inferir algumas de suas propriedades gerais, assim como algumas propriedades do impulso.

A primeira propriedade é devida à causalidade: $W_m(z) = 0$ e $W'_m(z)=0$ para $z>0$, ou seja, uma partícula que está a frente do feixe não sofre ação do \textit{Wake Field} gerado por ele. Apesar dessa característica ser restrita para $\beta = 1$ ela é sempre aplicada, por ser uma aproximação válida em anéis
reais.

Outras propriedades envolvem o fato delas serem reais e possuírem dimensão $[W'_k] = V/C/m^{2k}$ e $[W_k] = V/C/m^{2k-1}$, sendo que essa última propriedade é importante porque, para problemas com simetria rotacional, a m-ésima componente do impulso transversal é escalada por $(a/b)^{2m-1}$ e do longitudinal por $(a/b)^{2m}$, onde $a$ é o tamanho transversal do feixe e $b$ o raio da câmara de vácuo. Como $a\ll b$, os modos mais baixos dominam, de forma que a maioria das instabilidades longitudinais são geradas por $W'_0$ e a maioria das instabilidades transversais por $W_1$\footnote{Visto que $W_0$ não tem sentido físico por $(\Delta \vec{p}_\bot)_0 = \vec{0}$ (ver \eqref{eq:W}).} \cite{Chao, Bio}.

Ainda, por argumentos de conservação de energia \cite{Chao, Bio} é possível demonstrar que $ W'_m(0^-)>0$ e decresce conforme $|z|$ aumenta, pois uma partícula imediatamente atrás de uma outra deve sofrer uma força retardadora, caso contrário o sistema ganharia energia indefinidamente. Isso implica, pelo Teorema de \textit{Panofsky-Wenzel}, que $W_m(z)$ começa em $0$ (não demonstrável, mas a maioria das \textit{Wake Functions} seguem essa regra) e cresce negativamente e monotonicamente com $|z|$, para $|z|$ pequeno. Essa análise permite concluir que em anéis onde predominam \textit{Wake Fields} transversais, feixes curtos são preferíveis e em máquinas onde os longitudinais dominam feixes longos são melhores.

Uma característica importante da transferência de momento linear que podemos tirar de \eqref{eq:W'} ocorre quando $m=0$, ou seja, no caso da força monopolar. Nesse caso, $\Delta p_s$ é independente das coordenadas transversais da partícula prova, de modo que todas as partículas em um determinado plano sentem a mesma força, proporcional à $W'_0$. Esse tipo de impulso pode levar a um auto-empacotamento do feixe ou então à \textit{microwave instability}.

Ainda, para $m=1$ (dipolo), o impulso transversal é constante com as coordenadas transversais, enquanto o impulso longitudinal depende linearmente da distância $x$ (considerando $\hat{x}$ a direção em que a carga geradora foi posta), de modo que partículas em lados opostos do eixo sofrem impulsos opostos. Esse tipo
de característica distorce o feixe em um formato de banana e pode levar à instabilidade chamada \textit{Beam Breakup} em que o feixe é distorcido até atingir a câmara de vácuo.

\subsection{Impedâncias}

Assim como foi feito na resolução da \textit{resistive wall} é interessante descrever os \textit{wake fields} no domínio da frequência, pois o tratamento analítico das equações geralmente é simplificado quando estamos nesse espaço, além do que a interpretação das equações também se torna mais simples nesse caso, como é o caso das estruturas ressonantes, como a cavidade de RF, que permitem que apenas campos com frequências específicas ressoem por um longo tempo, e possam agir sobre o feixe, gerando instabilidades.

Desse modo, podemos definir impedância como a Transformada de Fourier das
\textit{Wake Function}:
\begin{align}
Z^\lVert_m (\omega)&=\int^\infty_\infty\!\!\frac{dz}{c}e^{-i\omega z/c}W'_m(z)\\
Z^\perp_m (\omega)&= i\int^\infty_\infty \!\! \frac{dz}{c}e^{-i\omega z/c}W_m(z)
\end{align}
onde $Z^\lVert_m$ é a impedância longitudinal e $Z^\perp_m$ a impedância transversal.

Algumas propriedades gerais da impedância são dadas abaixo:
\begin{itemize}
 \item $Z_m^\lVert(-\omega) = [ Z_m^\lVert(-\omega)]^*$ e $Z_m^\perp(\omega) = - [Z_m^\perp(\omega)]^*$ : devido ao fato de $W_m(z)$ ser real;
\item $Z_m^\Vert(\omega) = \frac{\omega}{c}Z_m^\perp(\omega)$, para coordenadas cilíndricas: devido ao Teorema de \textit{Panofsky-Wenzel};
\item $\Re\{Z_m^\Vert(\omega)\}\geq 0$ e $\Re\{Z_m^\perp(\omega)\}\geq 0$, se $\omega\geq 0$: porque a energia do feixe não pode aumentar sem a existência de forças externas.
\end{itemize}

A interpretação física das impedâncias é análoga à da impedância de um circuito elétrico. Para ilustrar a validade dessa analogia, vejamos o seguinte exemplo: considere um feixe com momento m-polar que gera um \textit{Wake Field} dado por
\begin{equation}
 P_m = \hat{P}_m e^{-i\omega(t-s/v)} \nonumber
\end{equation}
e uma partícula com carga $q$ a uma distância $z$ da fonte. O potencial elétrico longitudinal sentido por essa partícula será (ver\eqref{eq:W'}) \cite{Bio}:
\begin{equation}\label{eq:pot.imp}
\hat{V} =-\frac{\hat{P}_m I_m}{q} Z_m^\lVert
\end{equation}
onde $I_m$ é o m-ésimo multipolo da partícula teste. Para a componente monopolar \eqref{eq:pot.imp} se reduz à $V = Z^\lVert_0 P_0$, que é idêntica à expressão para um circuito elétrico.

Em referência a essa analogia, impedâncias de cavidades, que possuem picos de ressonância para determinadas frequências são modeladas como um circuito RLC em paralelo:
\begin{equation}
 Z_m^\lVert(\omega)= \frac{R_s}{1+iQ
\left(\frac{\omega_R}{\omega}-\frac{\omega}{\omega_R}\right)}
\end{equation}
onde $\omega_R = 1/\sqrt{LC}$ é a frequência de ressonância e $Q=R_s \sqrt{C/L}$ é o fator de qualidade da cavidade. Mesmo em casos em que a geometria não é do tipo estrutura ressonante, é comum classificar as impedâncias em capacitiva ($\Im\{Z_m^\lVert\}>0$) ou indutiva ($\Im\{Z_m^\lVert\}<0$), dependo do sinal de sua parte imaginária \cite{Chao}.

Outra propriedade da impedância está ligada a perda de energia. Existe três meios pelos quais um feixe pode perder energia em um anel de armazenamento: por emissão de radiação, colisão com gases residuais e por \textit{Wake Fields}; sendo que essa última é chamada de perda parasita.

Para mostrar a relação da perda parasita com a impedância, consideremos que o feixe possui uma distribuição de carga 
\begin{equation}\nonumber
\rho(\tau)= e N\lambda(\tau),\qquad \text{com} \quad\int_{-T_0/2}^{T_0/2} \!\! d \tau \lambda(\tau) = 1,
\end{equation}
onde $\tau$ é o avanço das partículas em relação à síncrona, $N$ é o número de partículas no feixe, $e$ é a carga elétrica de cada partícula e $T_0$ é o período de revolução no anel. Segundo \eqref{eq:W'}, a energia ganha por uma partícula do feixe devido a interação com os \textit{Wake Field} gerados por toda a distribuição em uma volta é
\begin{equation}\nonumber
\Delta \epsilon(\tau) = -e^2 N \int_{-T_0/2}^{\tau} \!\! d\tau'
W'_0(\tau-\tau')\lambda(\tau') = -e^2 N \omega_0\sum_{n=-\infty}^{\infty}
Z_0^\lVert(n\omega_0) \tilde{\lambda}_n e^{in\omega_0\tau}
\end{equation}
onde na última igualdade foi usado o Teorema da Convolução, e
\begin{displaymath}
 \tilde{\lambda}_n =
\frac{1}{2\pi}\int_{-T_0/2}^{T_0/2}\!\!d\tau\lambda(\tau)e^{-in\omega_0\tau}
\end{displaymath}
com $\omega_0 = 2\pi/T_0$. Assim, a energia média ganha por partícula por volta é
\begin{equation}\nonumber
\overline{\Delta \epsilon} = \int_{-T_0/2}^{T_0/2}\!\! d\tau \lambda(\tau)
\Delta \epsilon(\tau) = -e^2 N\omega_0 \sum_{n=-\infty}^{\infty}
Z_0^\lVert(n\omega_0) |\tilde{\lambda}_n|^2.
\end{equation}
Contudo, como $\lambda(\tau)$ é real, $|\tilde{\lambda}_n|^2$ é par; e como $ Z_0^\lVert(-\omega) = [Z_0^\lVert(\omega)]^*$, apenas a parte real da impedância contribui para a perda de energia:
\begin{equation}\label{eq:energiaperdida}
\overline{\Delta \epsilon} = -e^2 N\omega_0 \sum_{n=-\infty}^{\infty}
\Re\{Z_0^\lVert(n\omega_0)\} |\tilde{\lambda}_n|^2.
\end{equation}

Analisando \eqref{eq:energiaperdida} fica mais evidente a analogia da impedância com àquela de circuitos elétricos, pois apenas a parte real contribui com a dissipação de energia pelo feixe, enquanto a parte imaginária vai gerar alterações na fase deste. Também, nota-se que a energia perdida por partícula cresce linearmente com o número de elétrons no feixe.

%%%%%%%%%%%%%%%%%%%%%%%%%%%%%%%%%%%%%%%%%%%%%%%%%%%%%%%%%%%%%%%%%%%%%%%%%%%%%%%%%%%%%%%%%%%%%%%%%%%%%%%%%
%%%%%%%%%%%%%%%%%%%%%%%%%%%%%%%%%%%%%%%%%%%%%%%%%%%%%%%%%%%%%%%%%%%%%%%%%%%%%%%%%%%%%%%%%%%%%%%%%%%%%%%%%
\section{Modelos de Instabilidades Coletivas}

Podemos classificar os \textit{Wake Fields} de acordo com seu alcance. Campos que ressoam e decaem lentamente, correspondentes a bandas finas de impedância, são de longo alcance, enquanto aqueles que decaem rápido, correspondentes a bandas largas de impedância, são chamados de campos de curto alcance.

Considerando essa classificação,as instabilidades coletivas também podem ser dividas em duas classes:
\begin{itemize}
 \item Instabilidades tipo Robinson ou inter-pacotes: resultantes da ação de \textit{Wake Fields} de longo alcance, geram oscilações dos pacotes, com amplitude proporcional à corrente do feixe. Apesar de existirem para todos os valores de corrente, na prática elas são visíveis para correntes nas quais a taxa de crescimento das amplitudes de oscilação é maior que a dos mecanismos de amortecimento natural do feixe. 
 \item Instabilidades de modo acoplado ou intra-pacotes: são fenômenos que relacionam partículas de um mesmo pacote. Neste caso, a instabilidade ocorre acima de um determinado valor de corrente do feixe.
\end{itemize}

A solução exata do comportamento dos elétrons estocados, sujeitos a forças de interação entre si, só é possível através da resolução da equação de Vlasov \cite{Chao, Bio}, que considera uma distribuição contínua de partículas no espaço de fase. Todavia, modelos de macro-partículas, que consideram  que os
pacotes do feixe consistem em poucas partículas interagindo, geralmente uma ou duas, são fáceis de resolver, fornecem resultados satisfatórios para a maioria dos casos e possibilitam uma interpretação física dos fenômenos envolvidos. 

%%%%%%%%%%%%%%%%%%%%%%%%%%%%%%%%%%%%%%%%%%%%%%%%%%%%%%%%%%%%%%%%%%%%%%%%%%%%%%%%%%%%%%%%%%%%%%%%%%%%%%%%%
\subsection{Modelos de uma partícula}

Considera-se que cada pacote do feixe é formado por apenas uma macro-partícula, não apresentando estrutura interna. Esses modelos são bons para descrever as instabilidades tipo Robinson, pois a consideração feita é válida quando as partículas de um mesmo pacote sofrem oscilações coerentes. Tendo isso em mente, vamos analisar as direções longitudinal e transversal separadamente.

%%%%%%%%%%%%%%%%%%%%%%%%%%%%%%%%%%%%%%%%%%%%%%%%%%%%%%%%%%%%%%%%%%%%%%%%%%%%%%%%%%%%%%%%%%%%%%%%%%%%%%%%%
\subsubsection{Longitudinal}

Primeiramente, consideremos que apenas um pacote esteja circulando no anel. Consideremos também que apenas a \textit{Wake Function} monopolar ($m=0$) contribui. Para esse caso, a força que atua sobre o pacote é dada por \cite{Chao}:
\begin{equation}\label{eq:1partforcalongit}
 \frac{F(t)}{z_0} \approx \frac{N^2 e^2 \eta}{z_0cT_0}\sum^\infty_{k=-\infty}
 W'_0 (kT_0) + (z(t-kT_0)-z(t))(W'_0)'(kT_0)
\end{equation}
onde $\eta$ é aproximadamente o fator compactação de momento da rede magnética, $N$ é o número de partículas no pacote e a soma foi estendida até infinito pela propriedade de causalidade da \textit{Wake Function}. Desse modo, a variação da frequência é dada por:
\begin{equation}
 \Delta \omega = -\frac{Ne^2 \eta}{2\omega_s\gamma m_e c T_0}
\sum^\infty_{k=-\infty} \left(e^{i\Omega k T_0}-1\right)(W'_0)'(kT_0).
\end{equation}
onde nota-se que o primeiro termo de \eqref{eq:1partforcalongit} não foi considerado, por se tratar de um efeito estático, que apenas desloca a posição de equilíbrio do feixe.

Em termos da impedância, a taxa de crescimento da instabilidade pode ser escrita da seguinte maneira:
\begin{equation}\label{long1b}
 \tau^{-1}=\frac{Ne^2 \eta}{2\omega_s E T_0^2} \sum^\infty_{p=-\infty}
(p\omega_0+\omega_s)\Re \{Z^\lVert_0(p\omega_0+\omega_s)\}
\end{equation}

Em um anel de armazenamento, grande parte da impedância longitudinal é oriunda da cavidade de radio-frequência, ou seja, é do tipo banda fina. Para esse tipo de impedância, nota-se que apenas dois termos da somatória, os valores positivo e negativo de $p$ que mais se aproximam da ressonância, contribuem significativamente para a taxa de crescimento, ver Figura \ref{fig:narrow}, de modo que \eqref{long1b} pode ser aproximada para: 
\begin{equation} \label{narrowband}
 \tau^{-1}=\frac{Ne^2 \eta}{2\omega_s E T_0^2}|p|\omega_0
\Re\{-Z^\lVert_0(-|p|\omega_0 +\omega_s) + Z^\lVert_0(|p|\omega_0 +\omega_s)\}.
\end{equation}
Assim, se o pico da impedância for maior que $p\omega$ a soma será positiva e haverá instabilidade, caso contrário haverá amortecimento.

% \begin{figure}[!t]
% \center
%  \includegraphics[scale=0.5]{Imagens/narrowband.png}
%  \caption{Ilustração de \eqref{long1b} aplicada a uma impedância de banda fina, em cinza. Cada ponto corresponde a um termo da somatória. Nesse exemplo, a soma dos termos é positiva e, consequentemente a taxa de crescimento. (retirado de \cite{Khan}).}
%  \label{fig:narrow}
% \end{figure}

Especificamente para o caso da cavidade RF, sabemos que seu modo fundamental está próximo de $h\omega$. Assim, se o harmônico fundamental da cavidade for um pouco maior que o produto acima, haverá uma instabilidade longitudinal. Essa instabilidade foi a primeira a ser estudada e é chamada de Instabilidade de Robinson.

Considerando que $h$ pacotes compõem o feixe, eles oscilarão em um de seus auto-modos, em que a diferença de fase entre sucessivos pacotes é dada por $2\pi\mu/h$, com $\mu=0,...,h-1$. assim, as taxas de crescimento devem ser calculadas para cada modo:
\begin{equation}
\tau^{-1}_\mu=\frac{h N e^2 \eta}{2\omega_s E T_0^2}
\sum^\infty_{p=-\infty} (ph\omega_0+\mu\omega_0+\omega_s)
\Re\{Z^\lVert_0(ph\omega_0+\mu\omega_0+\omega_s)\}
\end{equation}

Esses modos podem interagir, por exemplo, com os HOMs das cavidades de radio-frequência e produzir instabilidades que vão ou não surgir, dependendo do auto-modo de ressonância do feixe.

%%%%%%%%%%%%%%%%%%%%%%%%%%%%%%%%%%%%%%%%%%%%%%%%%%%%%%%%%%%%%%%%%%%%%%%%%%%%%%%%%%%%%%%%%%%%%%%%%%%%%%%%%
\subsection{Transversal}

Considerando novamente a situação \textit{single-bunch} e que apenas a \textit{wake function} transversal dipolar ($m=1$) contribui, a força que atua sobre o pacote será:
\begin{equation}\label{eq:trans1b}
 \frac{F(t)}{y_0}=\frac{N^2 e^2}{y_0 c T_0} \sum^\infty_{k=-\infty}
y(t-kT_0)W_1(kT_0)
\end{equation}
o que implica em uma variação de frequência dada por:
\begin{equation}
 \Delta \omega = - i\frac{Ne^2}{2\omega_\beta \gamma m_e c T^2_0}
\sum^\infty_{p=-\infty} Z^\perp_1 (p \omega_0 + \Omega)
\end{equation}
de modo que a taxa de crescimento é:
\begin{equation}\label{resistive}
\tau^{-1} = -\frac{Ne^2 c}{2\omega_\beta E T^2_0} \sum^\infty_{p=-\infty}
\Re\{Z^\perp_ 1 (p \omega_0 + \omega_\beta)\}
\end{equation}

Para a expressão acima, é importante analisar dois tipos de impedâncias. As de banda fina, que podem gerar uma versão transversal da instabilidade de Robinson, e as de parede resistiva, que além de causarem instabilidades entre-pacotes, cobrem um longo campo de frequências, caindo com $1/\sqrt{|\omega|}$, o que exige a consideração de vários termos na soma para obter a taxa de crescimento com precisão (ver Figura \ref{fig:resistive}).

% \begin{figure}[!h]
% \center
%  \includegraphics[scale=0.5]{Imagens/resistive.png}
%  \caption{Ilustração de \eqref{resistive} assumindo uma impedância de parede resistiva, em cinza. Cada ponto corresponde a um termo da somatória. Nesse exemplo, a soma dos termos é negativa e a taxa de crescimento positiva.  (retirado de \cite{Khan}).}
%  \label{fig:resistive}
% \end{figure}

Para o caso da impedância de parede resistiva, a soma de termos positivos e negativos correspondentes, leva a conclusão de que para valores de sintonia cuja parte fracionária é menor que $0.5$ o movimento é estável, enquanto valores maiores são instáveis \cite{Khan}.

De modo similar à direção longitudinal, $h$ pacotes equidistantes oscilarão em seus auto-modos. Também, de acordo com \eqref{eq:trans1b}, a taxa de crescimento transversal é dada por:
\begin{equation}
\tau^{-1}_\mu = \frac{h Ne^2 c}{2\omega_\beta E T^2_0} \sum^\infty_{p=-\infty}
\Re\{Z^\perp_ 1 (p h\omega_0 +\mu\omega_0 + \omega_\beta)\}.
\end{equation}

Neste caso, a taxa de crescimento é sempre positiva para impedâncias de parede resistiva, ou seja, sempre haverá instabilidade de parede resistiva para modos de operação \textit{multi-bunch}. Ainda, simulações mostram que o sistema tente a evoluir para o modo de oscilação que maximiza a taxa de crescimento da instabilidade \cite{Khan}.

%%%%%%%%%%%%%%%%%%%%%%%%%%%%%%%%%%%%%%%%%%%%%%%%%%%%%%%%%%%%%%%%%%%%%%%%%%%%%%%%%%%%%%%%%%%%%%%%%%%%%%%%%
\subsection{Modelos de poucas partículas}

Neste tipo de modelos, considera-se que os pacotes são formados por duas ou mais partículas distribuídas longitudinalmente que interagem entre si. São aplicados, principalmente, no estudo de instabilidades do tipo intra-pacotes, geradas por \textit{wake fields} de curto alcance. Aqui serão considerados apenas modelos de duas macro-partículas, uma na parte da frente do pacote, chamada de cabeça , e outra na parte de trás, chamada cauda.

O movimento transversal acoplado dessas duas macro-partículas é modelado da seguinte forma:

\begin{align}
 \ddot{y}_1(t)+\omega^2_\beta(\delta_E)y_1(t)&=0& \\
 \ddot{y}_2(t)+\omega^2_\beta(\delta_E)y_2(t)&=\frac{Ne^2 W_1}{2\gamma c
T_0m_e}y_1.&
\end{align}
onde $y_1$ se refere à cabeça e $W_1$ é assumido constante no comprimento do pacote. As equações acima são válidas para a primeira metade do movimento síncrotron, sendo que na segunda metade ($t>T_s/2$) as partículas trocam de posição e as equações também devem ter seus índices trocados.

A variação da sintonia bétatron com a energia pode ser expressa por:
\begin{equation}
 \omega_\beta(\delta_E)=\omega_\beta(0)+\omega_0 \xi \delta_E =
\omega_\beta(0)+\omega_0 \xi \frac{\omega_s\hat{z}}{c \eta} \cos(\omega_st)
\end{equation}
onde $\omega_s$ é a frequência síncrotron e $\xi$ é a cromaticidade.

Resolvendo o sistema de equações, chega-se a:
\begin{equation}
\left(\begin{array}{c}\tilde{y}_1 \\ \tilde{y}_2 \end{array} \right)_{t=T_s/2}
=  e^{-i\omega_\beta T_s/2} 
\left(\begin{array}{cc} 1 & 0 \\ ia & 1 \end{array}\right)
\left(\begin{array}{c}\tilde{y}_1 \\ \tilde{y}_2 \end{array} \right)_{t=0}
\end{equation}
onde
\begin{equation}
 a = \frac{\pi N e~2 W_1}{4\gamma c T_0 m_e \omega_\beta \omega_s}\left(
1+i\frac{4 \xi \omega_0 \hat{z}}{\pi c \eta}\right).
\end{equation}

Após uma revolução síncrotron completa teremos:
\begin{equation}
\left(\begin{array}{c}\tilde{y}_1 \\ \tilde{y}_2 \end{array} \right)_{t=T_s}
=  e^{-i\omega_\beta T_s} 
\left(\begin{array}{cc} 1-a^2 & ia \\ ia & 1 \end{array}\right)
\left(\begin{array}{c}\tilde{y}_1 \\ \tilde{y}_2 \end{array} \right)_{t=0}.
\end{equation}

Os autovalores da matriz de transferência acima são dados por:
\begin{align}
 \lambda &= e^{\pm i\phi}\quad \text{com} \quad \cos(\phi) \equiv
1-\frac{a^2}{2}, \quad \text{se} \quad a\leq 2 \\
 \lambda &= e^{\pm i\mu}\quad \text{com} \quad \cosh(\mu) \equiv \frac{a^2}{2}
- 1, \quad \text{se} \quad a > 2.
\end{align}

Com esses resultados é possível analisar qualitativamente dois tipos de instabilidades. A primeira é chamada de efeito \textit{head-tail} (cabeça-cauda em português). No caso em que $|a| << 1$, que é válido para fontes de radiação síncrotron operando no modo \textit{multi-bunch}, a aproximação $\phi \approx a$ é válida. Os auto-vetores correspondentes aos autovalores determinados acima descrevem dois modos de oscilação das macro-partículas com diferenças de fases relativas de 0 e $\pi$.

Considerando $\eta$ positivo, vemos que se $\xi<0$ o modo 0 cresce com uma taxa $\Im\{a\}$ enquanto o modo $\pi$ é amortecido com a mesma taxa. Daí vem a necessidade de introduzir sextupolos nos anéis de armazenamento. O objetivo é manter a cromaticidade nula ou com um valor positivo pequeno, para amortecer o modo 0. Nesse caso, o modo $\pi$ seria excitado, contudo o modelo de duas partículas superestima sua taxa de crescimento, sendo que na prática ela é pequena em comparação com a do modo 0.

Nota-se que a instabilidade \textit{head-tail} ocorre para todos os valores de corrente. Ainda, aplicando-se um alto valor positivo de cromaticidade é possível amortecer oscilações de multi-pacotes, a custa de uma abertura dinâmica e tempo de vida reduzidos.

A outra instabilidade é conhecida por vários nomes na literatura:  \textit{head-tail turbulence, strong head-tail instability, transverse microwave instability, transverse mode coupling}. Quando a condição $|a| << 1$ não é mais válida, o movimento não pode ser mais visto como dois modos separados, de modo que para $\xi = 0$ a amplitude se mantém confinada, se $a<2$. Contudo, para valores maiores que esse, há instabilidade até mesmo com a cromaticidade nula e as taxas de crescimento são da ordem do amortecimento natural por emissão de radiação síncrotron \cite{Khan}.

%%%%%%%%%%%%%%%%%%%%%%%%%%%%%%%%%%%%%%%%%%%%%%%%%%%%%%%%%%%%%%%%%%%%%%%%%%%%%%%%%%%%%%%%%%%%%%%%%%%%%%%%%
%%%%%%%%%%%%%%%%%%%%%%%%%%%%%%%%%%%%%%%%%%%%%%%%%%%%%%%%%%%%%%%%%%%%%%%%%%%%%%%%%%%%%%%%%%%%%%%%%%%%%%%%%
\section{Medidas contra Efeitos Coletivos}

Após ter feito uma análise dos mecanismos que geram instabilidades coleticas em feixes não-contínuos, é possível identificar algumas medidas úteis para evitar esse tipo de problema em anéis de armazenamento.

Uma delas é a minimização da impedância da câmara de vácuo. Como foi estudado, a impedância é determinada pela geometria e condutividade da parede da câmara de vácuo. 

Em anéis de armazenamento há muitos locais em que a câmara é interrompida ou então sofre alterações em sua seção transversal. Geralmente a impedância introduzida por essas descontinuidades pode ser consideravelmente minimizadas se os seguintes critérios forem obedecidos \cite{Khan}:
\begin{itemize}
 \item Interrupções da câmara, como em flanges ou \textit{bellows} devem possuir pontes metálicas para que a corrente imagem flua;
 \item Mudanças inevitáveis no formato da câmara, como em \textit{tapers} nos finais de onduladores, devem ser projetados com ângulos pequenos (tipicamente da ordem de \SI{10}{\degree} ou menos);
 \item Buracos ou fendas na câmara, por exemplo portas de bombas de vácuo, portas de radiação síncrotron, ou o canal de injeção do feixe, devem ser projetados de modo que uma parede lisa fique mais perto do feixe que a interrupção.
\end{itemize}

No caso de cavidades de radio-frequência a maior contribuição para a impedância vem dos modos harmônicos de ordem mais altas, chamados HOMs (\textit{higher-order modes}). Para combater esses efeitos, geralmente emprega-se absorvedores de frequência de bandas largas ou então antenas para extrair modos particulares \cite{YellowCERN95}. Além desses métodos, muito esforço é investido no desenvolvimento de geometrias que possuam HOMs menos intensos.

Para as impedâncias de parede resistiva, geralmente as medidas mais comuns a se tomar são fazer câmaras largas e com material de boa condutividade elétrica. Se a seção transversal tiver que ser pequena, como no caso dos onduladores, a escolha do material da câmara é de importância primordial \cite{Khan}.

As medidas discutidas até agora não se aplicam para os casos em que tem-se uma dada câmara de vácuo, ou seja, uma máquina que era estável, mas devido a alguma mudança no modo operação ou em algum elemento do anel, assim como na corrente estocada, passou a demonstrar instabilidades. Para esses casos, as variáveis
que podem ser alteradas são reduzidas, pois parâmetros como circunferência, energia do feixe, perdas por radiação e frequências síncrotron e bétatron geralmente são dedicadas a outras necessidades.

Um modo eficiente de suprimir instabilidades é por meio do aumento da cromaticidade da máquina, que desloca o espectro do feixe. Contudo, esse método envolve aumentar a força dos sextupolos em regiões dispersivas e, consequentemente, diminuir a abertura dinâmica e o tempo de vida \cite{Khan}.

O efeito de impedâncias de banda fina podem ser minimizados alterando sua frequência, para diminuir a interação com o espectro do feixe. Para cavidades de RF, isso pode ser feito deformando-as mecanicamente, por meio do aumento de sua temperatura, por exemplo \cite{112Khan}. Todavia, deve-se fazer essas alterações tendo sempre mais de um grau de liberdade, para que apenas os HOMs sejam alterados, mantendo o modo fundamental constante.

\textit{Landau damping} é um tipo de amortecimento gerado pelos componentes não lineares dos potenciais definidos pelos campos magnéticos guia e pela voltagem RF. Nas direções transversais ele é introduzido por sextupolos, e na longitudinal pela forma senoidal do potencial. Um amortecimento adicional pode ser introduzido na direção longitudinal adicionando uma voltagem de RF com um múltiplo do modo fundamental, por meio das cavidades Landau \cite{114Khan}.

Além desses modos passivos de amortecimento das instabilidades, há os ativos. Entre eles, estão inclusos a modulação de fase da cavidade RF e o uso de  sistemas de \textit{Feedback}. O primeiro método já é empregado no UVX desde 2003, quando foi instalada uma nova cavidade de RF no anel, para a instalação de dispositivos de inserção, que possuia um HOM muito elevado (ver \cite{rfUVX}).

Os Sistemas de \textit{Feedback}, cujo princípio de funcionamento será discutido na seção \ref{feedback}, são muito empregados nas máquinas atuais para amortecimento de instabilidades, estando presentes na maioria das máquinas de terceira geração. Contudo, eles são úteis em apenas em uma banda estreita de taxas de crescimento, podendo não ser capazes de controlar taxas duas ordens de grandeza maiores que as de amortecimento do feixe \cite{Khan}, além de combaterem apenas instabilidades do tipo Robison. Por isso, os métodos passivos continuam sendo muito importantes para a estabilidade do feixe.

\chapter{Introduction} \label{chap:intro}




Most light sources uses synchrotron storage rings where subatomic charged particles, generally electrons, are extracted from materials and accelerated to relativistic energies in order to produce radiation. Among the several types of accelerators the synchrotron is a machine 


and depending on the type of accelerator the light may acquire some other important properties for its use as scientific tool. There are several types of accelerator, however, nowadays the most  synchrotron light sources are based in two types. The linear accelerators, used in Free Electron Lasers (FEL) and X-ray Free Electron Lasers (X-FELS), and the synchrotron storage ring accelerators.

In both types of accelerators the light is generated by magnetic devices called insertion devices (IDs) that generates an alternating magnetic field along the particle trajectory which makes them wiggle. 


In the latter, ultra-relativistic charged particles are stored in approximately round machines for hours, hence the name storage rings and they generate radiation 

  \section{Synchrotron light sources}
  
  In scientific facilities commomly known as synchrotron lights sources (SLS) the interaction between light and matter is used to study properties of a variaty of materials. Throught techniques involving absortion, reflection, refraction and scattering of light of different 'colors' by the materials under study, scientists can determine their atomic structure and composition.

The frequency of the light used in these facilities ranges from tera-hertz to hard X-rays and its origin is always related to synchrotron emittion of radiation by charged particles, hence the name of the facility. The light emitted by centripetal acceleration of ultra-relativistic particles has unique properties for use in scientific investigation. Besides its broad spectrum, among the advantages in relation to other methods are the high total flux emitted and the strong collimation.
  
    \subsection{Types of Light Sources}

   In general we can group syncrotron light facilities in two groups depending on the topology of the accelerators involved: linear and circular. In the linear sources, denominated Free Electron Lasers (FEL), the light is generated by special devices called Insertion Devices (ID) that are put along the straight trajectory of the beam and generates a transverse magnetostatic field with an amplitude that varies sinusoidally with respect to the longitudinal direction. When the beam passes throught this field it wiggles, and synchrotron emittion of radiation happens due to its deflection at each wiggle. The light emitted from successive wiggles interferes in such a way that only photons with the right frequency survive and the resulting radiation has a spectrum where all the energy is concentrated at very thin peaks around the resonant frequency. The intensity of these peaks is proportional to the number of particles in the beam and the number of wiggles of the ID field and their bandwidth is proportional to the inverse of the number of wiggles, hence these devices are built with the maximum number of periods that are technically possible. Additionally, the polarization of the radiation is defined by the direction of the magnetic field of the ID. For example, if the field is vertical in relation to the ground, the electrons will oscilate horizontally and the radiation will be horizontally polarized. Circular and elliptical polarizations can also be achieved by changing not only the intensity of the field but also its direction as a function of the longitudinal position. In a real insertion device all these properties of the light can be tuned according to the needs of the experiment to be carried out at the beamline, which makes this devices a very powerful tool for scientific experiments.
   
   Besides all the amazing properties of the insertion devices, in Free Electrons Lasers it is possible to excite the beam to emit light coherently, hence the name of these facilities. There are several techniques to achieve this and their enumeration or explanation is beyond the scope of this work. The important fact is that besides the high level of coherence of the light in these facilities, the intensity of the photon beam is increased by orders of magnitude with these techniques because it becomes proportional to the square of the number of particles in the beam. The excitation of coherent emittion becomes increas 
   
    
    \subsection{Light Source Generations}
    
    Synchrotron Light Sources can be classified in five generations. The first generation is composed by colliders storage ring's 
  \section{Synchrotron Light and Applications}
    \subsection{Experimental Techniques}
  \section{Brazilian Synchrotron Light Laboratory}
  
  The Centro Nacional de Pesquisa em Energia e Materiais (CNPEM) is a brazilian institution located in Campinas-SP that gathers four national laboratories, being the Brazilian Synchrotron Light Laboratory (LNLS) one of them. This laboratory was created in 1987 with the objectives of project, construct and operate a synchrotron light source.
  
Such objective was achieved successfully with the creation of UVX, a second generation light source which begun operation with external users in 1997. 
  
    \subsection{UVX}
    \subsection{Sirius}
  \section{Collective Effects}
    \subsection{Direct Interaction}
    \subsection{Interaction throught medium}
  \section{Description of the work}
  
  The main objective of this work was to study the subject of collective effects in electron storage rings, gathering the knowledge generated by the community in the last years, and apply this knoledge to study these effects on Sirius storage ring, with estimatives of instabilities thresholds among other collective phenomena commom to this type of accelerators.
  
  
\chapter{Concepts of Single Particle Dynamics}

Before the study of collective effects in storage ring is necessary to understand some concepts of single particle dynamics.

  \section{Storage Ring Main Devices}
  
  The first and most important concept is that of a storage ring. As its name suggests, it is a machine that keeps charged particles in almost circular revolution trajectories for millions of turns with almost constant energy. How these particles are injected in this machine or accelerated to its nominal energy is beyond the scope of this work.
  
  Connected to the concept of a storage ring is the one of the reference orbit. This special closed orbit is the one that an idealized particle with the storage ring nominal energy would follow if it had the correct initial conditions.In practice the trajectories of all particles stored in the machine will be close to it. For this reason the reference orbit is chosen to be the origin of the reference frame, defining a curved coordinate system that moves along the ring, with one longitudinal coordinate, parallel to the local orbit, and two transverse coordinates, perpendicular to it.
  
  Besides, all the components of a storage ring are aligned according to this special orbit in such a way that their center, symmetry points or axis coincide with transverse sections of the reference orbit.
  
  The storage ring task of keeping particles confined for such long times is achieved with the aim of electromagnetic fields to guide and replenish the energy lost by the particles due to synchrotron radiation emittion and an integrated ultra high vacuum system to increase their mean free path, minimizing scattering by gas molecules and consequent loss.
  
  Below we describe some of the main components of a storage ring.
    
    \subsection{Dipoles}
  
  Dipoles are devices that generate a static magnetic field that is non-zero at the reference orbit under ideal conditions. This is the only magnet that have such characteristic, in other words, this magnet is the only type of magnet that must afect the trajectory of the idealized particle that defines the reference orbit. It acts on the particle throught the Lorentz force, such as other magnets, bending its trajectory. Ideally, the local curvature that a dipole introduces in the reference particle path must be equal to the curvature of the reference orbit and the reference orbit must be curved only at places where there is a dipole magnetic fields. Shortly, dipoles are the responsible for the ring topoly of a storage ring.
  
    \subsection{Multipoles}
    
  Apart from dipoles, all the other magnets are placed at positions where the reference orbit is straight. Therefore, they do not act on idealized reference orbit and must generate a null magnetic field at the positions of the reference orbit.Their function is to confine the movement of the realistic particles in trajectories close to the reference orbit.
  
  Multipoles are grouped in several types, according to the characteristics of the field they gerenerate in the vicinity of the reference orbit.For example, an ideal quadrupole generates fiels whose transverse components grows linearly with the transverse coordinates of the local reference system and have no longitudinal fields, Sextupoles generetes quadratic fields, Octupoles, cubic, and so on. These properties are not exactly true for real magnets, but in gerenal are a good approximation.
  
  The quadrupoles are the most important multipoles in a storage ring. Because of the linear relation between the field transverse components and the particles position in relation to the reference orbit, they act as magnetic lenses, focusing and defocusing the particles beam. Together with the dipoles, they define the main properties of a storage ring, as the particles average energy, the transverse beam emittance, and beam sizes along the ring. These three factors combined define the brilliance of the photons, the figure of merit of a light source.
  
  It is important to mention that dipoles also work as lenses for the beam. As the quadrupoles, they may have transverse field components that vary linearly with the displacement from the reference orbit, being the only differences the fact that this orbit is curved and the field is non-zero at the reference orbit. Actually, even a dipole with uniform field may focus particles, due to the curvature of the reference system that introduces differences in path length in non-zero field regions for different transverse displacements, or due to edge fields generated by the finite length of the magnet.
  
  Dipoles work as spectrometers, deflecting more or less particles with less or more energy than the reference particle, which means they could not sustain these particles stored because the total deflection angle would not be $2\pi$ rad and they would spiral in or out, hitting the vaccum chamber. Quadrupoles are arranged to correct this intrinsic limitation of the dipoles, adding or subtracting liquid deflection in one turn for particles with more or less energy. This force particles to have different closed orbits depending on their energy, but at least maintain the chance of them to remain stable.
However, quadrupoles also suffer from chromatic aberrations, focusing more or less the particles, depending on their energy. This difference in focusing makes particles oscillate more or less around their closed orbit, changing their fundamental resonance frequency. This is not a fundamental problem, but in almost all modern storage rings it is impossible to store particles with only dipoles and quadrupoles. Sextupoles can correct that effect if placed at the right positions along the ring because of their non-linear magnetic field. They also introduce chromatic effects, but for the modern storage rings this does not affect the capability of storaging particles.

  Sextupoles are needed, but they introduce several complications for the design of a storage ring, because their non-linear fields introduce chaos in some regions of the particles phase space. More sextupoles or higher order multipoles can be introduced to avoid chaos as much as possible and also to help correcting higher order chromatic effects, but this analysis is beyond the scope of this work.
  
    \subsection{RF Cavity}

  Each turn the electrons lose energy due to synchrotron radiation which must be replenished periodically in order for them to remain in stable orbits, with energy close to the nominal energy of the storage ring. The magnetostatic components described above cannot perform such a task neither any other component or method relying on static electromagnetic fields of any kind, because according to Maxwell equations and the Lorentz equation, the liquid energy transfered to charged particle by static fields in one turn over the ring must be zero. Then the laws of physics constraint that it is necessary to rely on time dependent electromagnetic fields to replenish the energy of the electrons. The way this is accomplished in a storage ring is through the use of devices called RF cavities.
  
  RF cavity is a jargon for a cylindrical electromagnetic cavity with the lowest Transverse Magnetic (TM01) mode in the range of radiofrequency. Cavities for use in storage rings must have at least two small holes in its axis for the beam passage and one other hole to couple the cavity with an external source to feed the mode TM01 with energy. This energy is transfered to the particles in the storage ring throught the almost homogeneous longitudinal eletric field of TM01 when the particles passes throught the cavity.
  
    \subsection{Vacuum System}
  
  The vacuum system is responsible for creating a compact region around the reference orbit in the whole machine with very low pressure, which minimize collisions of the stored charged particles with gas molecules and, consequently, increase the average time particles can be stored with stable movement. Quantitatively, the average pressure of a storage ring must be lower than 1 nTorr for the average stored time of the particles to be of the order of a few dozens of hours.
  
  The vacuum system is composed of two main subsystems, the vacuum vessel, which defines the boundaries of the electrons atmosfere with the environment, and vacuum pumps to maintain the desired difference in pressure between the two environments. Most of the extension of the vacuum vessel is composed of straight and long chambers with a specific cross section, constant along the extension of the chamber. They are made of metals due to several desirable properties these materials presents, such as high heat and electrical conductivity, maleability, high acceptance to yelding and brasing and high resistence to pressure. Among them we highlight implications of the high electrical conductivity due to its importance for this work.
  
   
  \section{Transverse Dynamics}
  
  After the introduction of the main components, it is convenient to describe in general terms how is the movement of the stored particles. These particles are ultra-relativistic electrons with energy of the order of a few \si{GeV}, and most of their velocity is always in the direction of the tangent of the ideal orbit. Gererally there are approximately hundreds of billions electrons grouped in several bunches along the reference orbit, each bunch having a few milimiters length and transverse sizes of the order of dozens of microns. Each electron is under the influence of a variaty of electromagnetic fields (gravity can be neglected), coming from the static magnetic fields of the dipoles and multipoles, the radiofrequency field of the RF cavity, the direct fields of other electrons in the same bunch and the fields scattered by the vaccum vessel, generated by other electrons in the same bunch, in other bunches or even by themselves in previous turns. Also, they emit synchrotron radiation which makes them lose energy.
  
  The path of understanding and describing the dynamics of the stored particles begin with an approximation that neglect the effects of their self-fields, i.e. their interaction with each other, with the vaccum chamber and with the residual molecules in their atmosphere. In this framework the only forces acting on the particles are the magnetic fields of the dipoles and multipoles and the longitudinal electric field of the RF Cavity and the only way they can lose energy is throught synchrotron radiation emittion.
    
    \subsection{Hamiltonean Approximation}
  
  When such calculations are performed for current synchrotrons it is noticeable that the dynamics of the longitudinal motion is much slower than the dynamics of the transverse motion. As an example in the Sirius storage ring, particles take approximately 131 revolutions around the ring to complete one turn around the fixed point in the longitudinal direction while they oscillate 49 times per revolution around the fixed point in the transverse plane. This property makes it possible to separate the study of the longitudinal plane from the transverse one, considering the energy deviation one particle as a constant parameter in the transverse equations of motion. The effects of the radiation energy loss are even slower than the longidudinal motion, taking a few thousands of turns in the ring to significantly change the transverse motion.

  Neglecting the energy variations of the particles, and the randonnes of the radiation emittion, the motion of the particles can be described by a constant hamiltonean in the Frenet-Serret coordinate system defined by the ideal orbit. Procceding in the approximations, considering the paraxial motion of the particles around the closed orbit, this hamiltonean can be simplified to a quadratic form in the momenta coordinates.
  
  This Hamiltonean generates two second order coupled and non-linear equations of motion for the transverse coordinates of motion that acurately describes the short and mid term stability of the particles. Most of the non-linearities and coupling in these equations come from the magnetic fields of the components along the ring, being the other contributor the curvature of the reference orbit and the energy deviation of each particle.
  
  
    \subsection{Linear Equations of Motion}
    \subsection{Betatron Function and Phase Advance}
    \subsection{Dispersion Function}
    \subsection{Envelope Equations}
    \subsection{Chromaticity and Action Dependent Tune-shift}
  \section{Longitudinal Dynamics}
    \subsection{Phase Stability Principle}
    \subsection{The Potential-Well}
    \subsection{Linear equations of motion}
  \section{Radiation Damping and Equilibrium Parameters}
    \subsection{Longitudinal Damping}
    \subsection{Transverse Damping}
    \subsection{Quantum Excitation and Equilibrium parameters}
    \subsection{Fokker-Planck Equation}

\chapter{Wake-Fields and Impedances}
  \section{Wake-Fields}
    \subsection{Interaction Mechanisms}
    \subsection{Causality and Catch-up Distance}
    \subsection{Wake-function Definition}
    \subsection{Panofsky-Wenzel Theorem}
    \subsection{Symmetry Analysis}
    \subsection{Potential of Bunch of particles}
  \section{Impedances}
    \subsection{Impedance Definition}
    \subsection{Impedance Properties}
      \subsubsection{Causality}
      \subsubsection{Passivity}
      \subsubsection{Energy Loss}
  \section{Classification of Impedances and Wake-functions}
    \subsection{As Resonators}
      \subsubsection{Broad-band}
      \subsubsection{Narrow-Band}
    \subsection{As circuit components}
      \subsubsection{Inductive}
      \subsubsection{Capacitive}
      \subsubsection{Resistive}
    \subsection{According to its source}
      \subsubsection{Resistive Wall}
      \subsubsection{Geometric}
      \subsubsection{Coherent Synchrotron Radiation}
  \section{Impedance Calculation}
    \subsection{Analytic Methods}
      \subsubsection{Field Matching Technique}
      \subsubsection{Perturbative Method}
      \subsubsection{Parabolic Equation}
    \subsection{Numeric Methods}
      \subsubsection{Frequency Domain Solvers}
      \subsubsection{2D Time domain Solvers}
      \subsubsection{3D Time domain Solvers}

\chapter{Collective Effects}
  \section{Energy Loss}
  \section{Potential-Well Distortion}
  \section{Tune-shifts}
    \subsection{Coherent Tune-shifts}
    \subsection{Incoherent Tune-shifts}
  \section{Instabilities}
    \subsection{Multi-Turn Instabilities}
    \subsection{Single-Turn Instabilities}
  \section{Landau damping}
  \section{Analytical Treatment}
    \subsection{The Linearized Fokker-Planck Equation}
      \subsubsection{Stationary Solution: Haissinski Equation}
      \subsubsection{Modal expantion}
        \paragraph{Head-tail modes}
      \subsubsection{Solution for Gaussian Bunches}
      \subsubsection{Low Current Limit: Multi-Turn Instabilities}
      \subsubsection{High Current Limit: Mode Coupling Instabilities}
    \subsection{The Microwave Instability}
    \subsection{Strong Head-tail instability}
  \section{Numerical Treatment}
    \subsection{Tracking Codes}
      \subsubsection{Lump of Impedance Kicks}
      \subsubsection{Slicing and particle deposition}
      \subsubsection{Single-Bunch tracking codes}
      \subsubsection{Multi-Bunch tracking codes}


\chapter{Experimental Methods}
  \section{Impedance Measurement}
    \subsection{Direct Measurement}
      \subsubsection{Wire technique}
      \subsubsection{Double wire technique}
    \subsection{Beam Measument}
      \subsubsection{Single-bunch Current Dependent Tune-shifts}
      \subsubsection{Multi-bunch Current Dependent Tune-shifts}
      \subsubsection{Single-bunch Turn-by-Turn measurements}
      \subsubsection{Orbit bump method}
      \subsubsection{response matrix fitting}
      \subsubsection{Instability Threshold Measurements}
  \section{Instability Cures}
    \subsection{Landau Cavity}
    \subsection{Chromaticity}
    \subsection{Cavity power absorbers}
    \subsection{Cavity temperature adjustments}
    \subsection{Bunch-by-Bunch Feedback System}
      \subsubsection{How it works}
      \subsubsection{Possible Experiments}
        \paragraph{Grow-Damp Experiments}
        \paragraph{Tune measurement}
        \paragraph{Beam Response Function Measurement}
      \subsubsection{Main Limitations}
    
\chapter{Methodology}
  \section{}

\chapter{Results}
  \section{Components Impedance Modelling}
    \subsection{Multi-Layer Resistive Wall}
      \subsubsection{Pipe}
      \subsubsection{Ceramic Chambers}
      \subsubsection{Thin Chambers}
    \subsection{Geometric Transitions}
      \subsubsection{2D-Calculations}
        \paragraph{BC Chamber}
        \paragraph{RF Cavity taper}
        \paragraph{Undulators Tapers}
      \subsubsection{3D-Calculations}
        \paragraph{BPMs}
        \paragraph{Bellows}
        \paragraph{Radiation Masks}
    \subsection{Coherent Synchrotron Radiation (CSR)}
  \section{Impedance Budget}
    \subsection{Longitudinal Impedance}
      \subsubsection{Effective Z/n}
      \subsubsection{Multi-bunch Kloss and Dissipated Power}
    \subsection{Vertical Impedance}
    \subsection{Horizontal Impedance}
  \section{Simulations and Instabilities Thresholds}
    \subsection{Frequency Domain Calculations}
      \subsubsection{Vertical Plane}
        \paragraph{Single-bunch Tune-Shifts}
        \paragraph{Multi-bunch Tune-Shifts}
        \paragraph{Coupled-Bunch Instabilities}
        \paragraph{Single-Bunch Instabilities}
      \subsubsection{Horizontal Plane}
        \paragraph{Single-bunch Tune-Shifts}
        \paragraph{Multi-bunch Tune-Shifts}
        \paragraph{Coupled-Bunch Instabilities}
        \paragraph{Single-Bunch Instabilities}
      \subsubsection{Longitudinal Plane}
        \paragraph{Multi-Bunch Instabilities}
        \paragraph{Single-Bunch Instabilities}
    \subsection{Time Domain Calculations}
      \subsubsection{Longitudinal Plane}
      \subsubsection{Vertical Plane}
      \subsubsection{Horizontal Plane}


	%\part{Fundamentos}
    %%%%%%%%%%%%%%%%%%%%%%%%%%%%%%%%%%%%%%%%%%%%%%%%%%%%%%%%%%%%%%%%%%%%%%%%%%%%%%%%%%%%%%%%%%%%%%%%%%%%%%%%%
%%%%%%%%%%%%%%%%%%%%%%%%%%%%%%%%%%%%%%%%%%%%%%%%%%%%%%%%%%%%%%%%%%%%%%%%%%%%%%%%%%%%%%%%%%%%%%%%%%%%%%%%%
\chapter{Física de Aceleradores}

O propósito dessa seção é definir os aspectos fundamentais da dinâmica dos elétrons em um anel de armazenamento de elétrons em uma fonte de luz síncrotron dentro da aproximação de não interação entre partículas. Não há a intenção de ser completo ou rigoroso na definição das quantidades envolvidas, mas apenas familiarizar o leitor com a nomenclatura e principais propriedades do feixe. Referências serão indicadas para os leitores mais curiosos.

Um anel de armazenamento ideal possuiria as seguintes características:
\begin{itemize}
 \item Tempo de vida infinito: Os elétrons armazenados assim continuariam por um tempo infinito, sem colidirem com a câmara de vácuo.
 \item Tamanho transversal muito menor que o limite de difração: ou em outras palavras, emitância do feixe de elétrons muito menor que a emitância do feixe de fótons. Dessa forma, o tamanho do feixe de radiação na amostra seria definido apenas pelo princípio de incerteza dos fótons,
\end{itemize}

\todo{parte tirada do relatório de estágio}
Nesta parte do relatório, será estudada a dinâmina linear do movimento dos elétrons no anel de armazenamento. Em suma, essa abordagem considera o efeito dos campos produzidos por dipolos e quadrupolos, que são lineares com as coordenadas espaciais do movimento.

%%%%%%%%%%%%%%%%%%%%%%%%%%%%%%%%%%%%%%%%%%%%%%%%%%%%%%%%%%%%%%%%%%%%%%%%%%%%%%%%%%%%%%%%%%%%%%%%%%%%%%%%%
%%%%%%%%%%%%%%%%%%%%%%%%%%%%%%%%%%%%%%%%%%%%%%%%%%%%%%%%%%%%%%%%%%%%%%%%%%%%%%%%%%%%%%%%%%%%%%%%%%%%%%%%%
\section{Dinâmica Linear Transversal}

%%%%%%%%%%%%%%%%%%%%%%%%%%%%%%%%%%%%%%%%%%%%%%%%%%%%%%%%%%%%%%%%%%%%%%%%%%%%%%%%%%%%%%%%%%%%%%%%%%%%%%%%%
\subsection{Equações de Movimento}

A primeira etapa do estudo consiste em definir um sistema de coordenadas adequado, representado na \mbox{Figura~\ref{fig:coord}}. Trata-se de um sistema de Frenet-Serret com torção nula (órbita plana), acoplado a órbita ideal da partícula.
As equações que definem esse sistema são \cite{frenet}:
\begin{align} \label{eq:coord}
 \hat{\mathbf{s}}(s) &= \frac{d\mathbf{r}(s)}{d s} \nonumber\\
 \hat{\mathbf{x}}(s) &= -\rho(s)\,\hat{\mathbf{s}}'(s) \qquad
\hat{\mathbf{x}}'(s)=\frac{1}{\rho(s)}\hat{\mathbf{s}}(s) \\
 \hat{\mathbf{z}}(s) &= \hat{\mathbf{s}}(s)\times\hat{\mathbf{x}}(s)
\nonumber
\end{align}
onde $\rho(s)$ é o raio de curvatura da órbita, o apóstrofo indica derivação com relação a coordenada longitudinal, $s$, e os vetores unitários $\hat{\mathbf{s}}$, longitudinal, $\hat{\mathbf{x}}$, radial,
$\hat{\mathbf{z}}$, vertical, nessa ordem, formam uma base ortonormal dextrógira.

% \begin{figure}[!ht]
%  \center
%  %\includegraphics[scale=1.2]{Imagens/coord.png}
%  \caption{Sistema de coordenadas de Frenet-Serret usado no estudo de dinâmica de feixes em anéis de armazenamento.}
%  \label{fig:coord}
% \end{figure}

Os imãs de um anel de armazenamento são construídos e dispostos de forma que uma partícula com energia nominal, e tendo as condições iniciais corretas, se mova ao longo de uma órbita fechada ideal, enquanto as outras partículas apresentam
desvios em relação a essa órbita. Dessa forma, torna-se matematicamente mais fácil estudar o problema usando o sistema de coordenadas descrito acima, pois é possível truncar expansões dos campos magnéticos, além de que as expressões obtidas são de fácil interpretação.

Como já foi dito, os campos magnéticos tratados no estudo da dinâmica linear são aqueles gerados por dipolos e quadrupolos. Os dipolos são componentes magnéticos construídos para gerar um campo que não dependa das coordenadas transversais na proximidade da órbita ideal e perpendicular ao plano dessa
órbita. São eles que, juntamente com a energia nominal dos elétrons, definem a órbita fechada ideal e, consequentemente, seu raio de curvatura:
\begin{equation}
 \frac{1}{\rho(s)}=G(s)=\frac{ecB_0(s)}{E_0}
\end{equation}
em que $G(s)$ é a função curvatura, $e$ é a carga do elétron, $c$ a velocidade da luz, $B_0$ o campo gerado pelo dipolo e $E_0$ a energia nominal. Na equação acima fica clara a natureza puramente geométrica da função curvatura e sua periodicidade, além do fato de sua integral ao longo da órbita
ideal em todo o anel ser $2\pi$.

Já os quadrupolos, tem a função de focalizar ou defocalizar o feixe de elétrons, sendo projetados para gerar um campo nulo na órbita ideal, e um gradiente da seguinte forma:
\begin{equation}
 B_z(s,x,z)=\left(\frac{\partial B}{\partial x}\right)_{0,s}x \qquad \mathrm{e}
\qquad B_x(s,x,z)=\left(\frac{\partial B}{\partial x}\right)_{0,s}z
\end{equation}
onde a segunda equação é consequência de $\nabla \times B = 0$. A inexistência do termo $\partial B / \partial z$ é imposta na especificação do quadrupolo, devido ao fato dessa componente gerar um acoplamento entre os movimentos radial e vertical.
Alguns quadrupolos, chamados quadrupolos skew, são produzidos para terem apenas esse termo, e são inseridos no anel para reduzir o acoplamento gerado por campos espúrios, ou para aumentá-lo, dada a estreita relação desse fator com o tempo de vida do feixe de elétrons.
Também nesse caso pode-se definir uma função geométrica, a focalização:
\begin{equation}
  K_1(s)=\frac{ec}{E_0}\left(\frac{\partial B}{\partial x}\right)_{0,s}.
\end{equation}
que também possui periodicidade, como a função curvatura.

Os anéis de armazenamento são constituidos de vários superperíodos, ou seja, células idênticas repetidas para gerar uma órbita fechada. Nesses casos, a periodicidade das funções de curvatura e focalização são iguais ao comprimento de um superperíodo.

Com as definições acima, é possível obter a equação de movimento para um elétron. Há várias ferramentas matemáticas para isso. Dentre elas, destaca-se a formulação Hamiltoniana da mecânica clássica.

O procedimento a seguir nesse caso é, resumidamente, o seguinte: monta-se a Lagrangiana da partícula, que é o produto escalar do 4-vetor potencial eletromagnético com o 4-vetor velocidade e aplica-se a ela uma transformada de Legendre para substituir a variável velocidade pelo momento canônico. Após isso, faz-se uma transformação canônica na Hamiltoniana, para escrevê-la no sistema de coordenadas de Frenet-Serret, definido pela \mbox{equação~\eqref{eqcoord}}, e muda-se a variável independente, de $t$
(tempo)
para $s$.

Uma das vantagens deste modo de obtenção da equação de movimento é que até este ponto do desenvolvimento as equações são exatas, de modo que caso se queira estudar efeitos de ordem mais alta, tanto no desvio de energia da partícula, como de campo magnético, basta expandir a Hamiltoniana até tais ordens \cite{Wiedemann2 II}, além de também ser possível estudar o movimento longitudinal e seu acoplamento com a dinâmica transversal \cite{Lee}.

Contudo, como estamos interessados nas contribuições lineares, devemos considerar um desvio de energia dos elétrons até primeira ordem, $E=E_0+\boldsymbol{\epsilon}$, fazer a aproximação paraxial, em que
$\frac{\partial x}{\partial s}\ll 1$ e $\frac{\partial z}{\partial s} \ll 1$, ou seja, $p\approx p_s$ e considerar apenas os campos dipolares e quadrupolares discutidos acima. Com estas considerações, chega-se à seguinte equação \cite{Wiedemann3, Lee, Sands}:
\begin{align}
\label{eq:movhor}
 & x'' = K_x(s)x+G(s)\frac{\boldsymbol{\epsilon}}{E_0} \qquad
K_x(s)=-G^2(s)-K_1(s) &\\
\label{eq:movvert}
 & z'' = K_z(s)z \qquad \qquad \qquad \,\,\,\, K_z = K_1(s) &
\end{align}
onde o apóstrofo indica derivação com relação a coordenada longitudinal $s$. Nota-se na equação acima, pelas definições de $K_x(s)$ e $K_z(s)$, uma propriedade fundamental dos quadrupolos: quando um quadrupolo focaliza o feixe na direção radial (horizontal), ele defocaliza na direção vertical e vice-versa.

A derivação completa das equações de movimento expostas acima pela formulação Hamiltoniana pode ser encontrada nas referências \cite{Lee, Wiedemann3}, sendo que em \cite{Sands} e \cite{Wiedemann3} uma derivação diferente, baseada em argumentos geométricos sobre o movimento de uma partícula com energia
$E=E_0+\boldsymbol{\epsilon}$, é realizada.


%%%%%%%%%%%%%%%%%%%%%%%%%%%%%%%%%%%%%%%%%%%%%%%%%%%%%%%%%%%%%%%%%%%%%%%%%%%%%%%%%%%%%%%%%%%%%%%%%%%%%%%%%
\subsection{Oscilações Betatron}

Considerando apenas partículas com energia igual à nominal, ou seja, fazendo $\boldsymbol{\epsilon} = 0$, as \mbox{equações~\ref{eqmovhor} e \ref{eqmovvert}}, possuem a mesma forma, e podem ser estudadas simultaneamente.

Na maioria dos anéis, as funções de focalização e curvatura são constantes nos trechos com magnetos e nulas nos trechos sem magneto (livres). Dessa forma, a equação do movimento é a equação de Hill:

\begin{equation}\label{eqoscbet}
 y''+K(s)y=0
\end{equation}
onde $y$ representa as coordenadas transversais do movimento, $x$ e $z$. A solução dessa equação é dada por:
\begin{equation}
 y=\left\{\begin{array}{ll}
           a \cos\left(\sqrt{K}s+b\right) & 	K>0 \\
	   a s +b                      &	K=0 \\
	   a \cosh\left(\sqrt{-K}s+b\right) &	K<0
          \end{array}\right.
\end{equation}
onde as constantes $a$ e $b$ são determinadas pelas condições iniciais do problema. De forma que podemos escrever:
\begin{equation}
 \mathbf{y}(s)=M(s|s_0)\mathbf{y}(s_0)
\qquad \mathrm{com}\qquad
\mathbf{y}(s) = \left(\begin{array}{c}
			y(s) \\
			y'(s)
		       \end{array}\right)
\end{equation}
onde $M(s|s_0)$ é a matriz de transferência, que transporta a partícula do ponto inicial $s_0$ para o ponto final $s$.

Como a solução geral da equação de Hill é uma composição de duas soluções linearmente independentes, e as condições iniciais
\begin{equation} \label{eqcondinihill}
\mathbf{y_1}(s_0)=\left(\begin{array}{c}
			1 \\
			0
		       \end{array}\right)
\qquad \mathrm{e} \qquad
\mathbf{y_2}(s_0)=\left(\begin{array}{c}
			0 \\
			1
		       \end{array}\right)
\end{equation}
geram tais soluções, a forma geral da matriz de transferência é:
\begin{equation}\label{eqmatriztransf}
 M(s|s_0)=\left(\begin{array}{cc}
                  C(s,s_0)  &  S(s,s_0) \\
	          C'(s,s_0) &  S'(s,s_0)
		\end{array}\right).
\end{equation}
sendo que $C(s,s_0)$ corresponde à solução da primeira condição inicial de \mbox{\eqref{eqcondinihill}} e $S(s,s_0)$ é solução da segunda. Essas funções são análogas às funções seno e cosseno na resolução do oscilador harmônico.

% \begin{figure}[t]
%  \center
% % \includegraphics[scale=0.8]{Imagens/linha_de_transporte.png}
%  \caption{Linha de transporte genérica onde QF representa um quadrupolo focalizador, QD um defocalizador e B um dipolo \cite{Wiedemann3}.}
%  \label{fig:linhatransporte}
% \end{figure}

Uma propriedade da evolução de uma condição inicial para a final nesse tipo de problema é que a matriz de qualquer intervalo é o produto da matriz dos subintervalos. Assim, a matriz de transferência da linha de transporte representada na \mbox{Figura~\ref{fig:linhatransporte}} seria
\begin{equation}
M(s_8|s_0)=M_8M_7M_6M_5M_4M_3M_2M_1.
\end{equation}

De um modo geral, a matriz de uma volta no anel é dada por:
\begin{displaymath}
 M(s + P|s)=M^n(s)\qquad \mathrm{com} \qquad M(s)=M(s+L|s)
\end{displaymath}
onde $L$ é o comprimento de um superperíodo, $n$ é número de superperiodos e $P=n L$ é o perímetro (comprimento) do anel.

Devido ao grande número de revoluções que os elétrons fazem no anel, as matrizes devem ser limitadas, ou seja, todos os elementos de $M(s)$ devem ser finitos conforme o número de voltas cresce indefinidamente. Para definir a condição de estabilidade, analisaremos os autovalores da matriz de
transferência, que obedecem a seguinte equação:
\begin{equation}
 \lambda^2+\mathrm{Trace}\left(M(s)\right) \lambda +
\mathrm{det}\left(M(s)\right).
\end{equation}
Contudo, $\mathrm{det}\left(M(s)\right)=1$, por uma propriedade da equação de Hill \cite{Lee,Wiedemann3}, e $\mathrm{Trace}\left(M(s)\right)$ não depende de $s$, pois há uma relação de similaridade entre $M(s_1)$ e $M(s_2)$:
\begin{equation} \label{eqsimilaridade}
 M(s_2+L | s_1) = M(s_2) M(s_2 | s_1) = M(s_2 | s_1) M(s_1)
\end{equation}
Assim, fazendo $\mathrm{Trace}\left(M(s)\right)=2 \cos(\Phi)$, onde $\Phi$ é chamado de avanço de fase betatron do superperiodo, descobre-se que $\lambda_\pm = e^{\pm i \Phi}$, com $\Phi$ real se  $\mathrm{Trace}\left(M(s)\right)\leq 2$.

O critério de estabilidade descrito acima é fundamental para a escolha de novas redes magnéticas ou ópticas para um anel, sendo o primeiro vínculo a ser testado neste projeto.

Uma matriz $2\times2$ com as propriedades da matriz de transferência, $\mathrm{det}M=1$ e $\mathrm{Trace}M\leq 2$, pode ser parametrizada sem perda de generalidade da seguinte forma:

\begin{equation}
 M = \left(
\begin{array}{cc}
 \cos(\Phi) + \alpha \sin(\Phi)  & \beta \sin(\Phi) \\
 -\gamma \sin(\Phi) & \cos(\Phi) -\alpha \sin(\Phi)
\end{array} \right) = I \cos(\Phi) + J\sin(\Phi)
\end{equation}
onde $\alpha$, $\beta$ e $\gamma$ são os chamados parâmetros de Courant-Snyder, $\Phi$ é denominado avanço de fase betatron, $I$ é a matriz identidade e
\begin{equation}
 J= \left(\begin{array}{cc}
          \alpha  & \beta \\
          -\gamma & -\alpha
         \end{array} \right)
\quad \mathrm{com} \quad J^2=-I \, \Leftrightarrow
\, \beta \gamma = 1+ \alpha^2
\end{equation}

Usando as propriedades da matriz $J$ e a relação de similaridade  \eqref{eqsimilaridade}, podemos obter a evolução dos parâmetros de Courant-Snyder de um ponto a outro do anel:
\begin{equation}\label{eqevolparcouder}
\left(\begin{array}{c}
       \beta \\ \alpha \\ \gamma
      \end{array}\right) =
\left(\begin{array}{ccc}
       C^2    &   -2 C S      &  S^2   \\
       -C C'  &  C S' + C' S  &  -S S' \\
       C'^2   & -2 C' S'      &  S'^2
      \end{array}\right)
\left(\begin{array}{c}
       \beta_0 \\ \alpha_0 \\ \gamma_0
      \end{array}\right)
\end{equation}

Essa representação matricial das soluções da equação de Hill é muito importante para a simulação e construção de modelos matemáticos de anéis, devido à simplicidade dos cálculos e facilidade de implementação em uma linguagem computacional.
A maioria dos softwares de dinâmica de feixes existentes utiliza esse formalismo para calcular os parâmetros da óptica linear, como os parâmetros de Courant-Snyder e o avanço de fase.

Contudo, para compreendermos o significado físico dos parâmetros $\beta$, $\alpha$ e $\Phi$ e para obtermos outros resultados, é interessante analisarmos a natureza pseudo-harmônica do movimento dos elétrons. Para isso, utilizaremos o teorema de Floquet aplicado à equação de Hill.

Dada a equação \ref{eqoscbet}, o Teorema de Floquet afirma que existem duas soluções linearmente independentes da forma
\begin{equation}
 y_1(s)= w(s) e^{i \psi (s)}, \qquad y_2(s)= w(s) e^{-i \psi (s)}
\qquad \mathrm{com}\quad w(s+L)=w(s).
\end{equation}
Substituindo essas soluções na equação de Hill, obtemos:
\begin{align}
 & & w''= - K w + 1/w^3 \\
 & & \psi' = 1/w^2
\end{align}
onde, pela primeira equação, nota-se que $w$ não muda de sinal, pois uma vez próximo de $0$, o termo $1/w^3$ cresce indefinidamente, fazendo com que $w''$ cresça e $w$ aumente. Dessa forma, definimos $w$ positivo. Como as soluções $y_1$ e $y_2$ são linearmente independentes, podemos escrever a matriz
de um período em função delas:
\begin{equation}
 M(s)=\left(
\begin{array}{cc}
 \cos(\phi)-w w'\sin(\phi)    & w^2\sin(\phi) \\
 -\frac{1+w^2 w'^2}{w^2}\sin(\phi)  & \cos(\phi) + w w' \sin(\phi)
\end{array}\right)
\end{equation}
onde $\phi = \psi(s+L) - \psi(s)$. Comparando a equação acima, com a parametrização de Courant-Snyder, nota-se que:
\begin{equation}
 w^2 \equiv \beta(s) , \qquad \alpha \equiv -w w' = -\frac{\beta'(s)}{2},\qquad
\Phi \equiv \phi = \int_0^L \frac{d s}{\beta(s)}.
\end{equation}

Através das relações acima, da periodicidade de $\beta$, e da \mbox{equação \ref{eqevolparcouder}}, os parâmetros de Courant-Snyder e o avanço de fase ao longo do anel ficam univocamente definidos dada uma função $K(s)$, ou seja, dada uma rede magnética.

Agora, podemos escrever a solução geral da equação de Hill em sua forma pseudo-harmônica
\begin{equation}\label{eqpseudoharm}
 y(s)=\sqrt{\varepsilon} \sqrt{\beta_y(s)}\cos(\psi_y(s) + \phi_0),\quad
\psi_y(s)=\int_0^s \frac{d x}{\beta_y(x)}
\end{equation}
onde $\sqrt{\varepsilon}$ e $\phi_0$ são constantes determinadas pelas condições iniciais, sendo que definimos a dimensão de $\sqrt{\varepsilon}$ como m$^{1/2}$.

Analisando a equação acima, notamos que o movimento oscilatório dos elétrons em torno da órbita de referência é muito parecido com o movimento harmônico. As diferenças estão no fato de que amplitude do movimento é modulada pelo parâmetro $\beta$, chamado função betatron, e que o avanço de fase não é linear com a coordenada $s$ e também não é, em geral, multiplo de $2 \pi$ para uma volta completa, o que faz com que o movimento não seja periódico. A Figura \ref{fig:oscilacao_betatron} ilustra os argumentos acima.

% \begin{figure}[t]
%  \center
% % \includegraphics[scale=0.8]{Imagens/oscilacao_betatron.png}
%  \caption{Oscilações betatron de um elétron em várias revoluções. Linha tracejada representa a função betatron e linha sólida a trajetória. Adaptado de
%  \cite{Sands}.}
%  \label{fig:oscilacao_betatron}
% \end{figure}

A analogia com o movimento harmônico fica ainda mais clara quando determinamos o valor da constante $\sqrt{\varepsilon}$. Derivando a \mbox{equação \ref{eqpseudoharm}} e isolando a constante, obtemos:
\begin{equation}
 \varepsilon = \gamma x^2 + 2 \alpha x x' + \beta x'^2
\end{equation}
que é a equação de uma elipse no espaço de fase, \mbox{Figura \ref{fig:elipse_espacofase}}. Isso implica que, dado um ponto da coordenada longitudinal, o espaço de fase de todos os elétrons serão elipses de mesma orientação e excentricidade, determinadas pelos parâmetros de Courant-Snyder, mas com áreas $\pi \varepsilon$ diferentes, definidas pelas condições iniciais. Ainda, dado um elétron, a área de sua elipse do espaço de fase permanece constante ao longo de todo o anel.

% \begin{figure}[!b]
%  \center
% % \includegraphics[scale=0.5]{Imagens/elipse_espacofase.png}
%  \caption{Elipse do movimento de um elétron no espaço de fase para $s$ fixo~\cite{Wiedemann3}.}
%  \label{fig:elipse_espacofase}
% \end{figure}

Esta última propriedade é uma consequência direta do teorema de Liouville, que afirma que a densidade de pontos no espaço de fase não se altera para sistemas sob a ação de forças conservativas e em que a posição da partícula não dependa do seu momento linear \cite{Wiedemann3}. Também,ela é a principal motivação para se definir a emitância natural do feixe, visto que podemos escolher uma elipse que represente todo o feixe de elétrons e essa elipse apresentará a mesma área em todo o anel (a definição de emitância será dada na \mbox{seção \ref{efeitosemissao}}).

Um outro parâmetro fundamental das oscilações betatron é a normalização do avanço de fase de uma volta no anel de armazenamento. Define-se sintonia como:
\begin{equation}
 \nu_y = \frac{1}{2 \pi}\int^{s+P}_s \frac{d x}{\beta_y(x)}.
\end{equation}

O valor da sintonia influencia fortemente características como tempo de vida, abertura dinâmica e acoplamento, pois pequenos desvios nos valores de campo magnético dos dipolos e quadrupolos, a existência de elementos não lineares, como sextupolos e dispositivos de inserção, entre outras fontes, geram interferências que, quando entram em ressonância com as oscilações betatron, causam desvios arbitrariamente grandes na órbita, causando a a perda dos elétrons.

É possível mostrar \cite{Lee,Wiedemann3, Huth} que ressonâncias podem ser excitadas quando a seguinte relação é satisfeita
\begin{equation}
 m \nu_x + n \nu_z = l,\qquad m, n, l \in \mathbf{Z},
\end{equation}
onde $\nu_x$ e $\nu_z$ são as sintonias horizontal e vertical, respectivamente. Contudo, nem todas as ressonâncias são excitadas, de modo que as que devem ser evitadas são as de ordem, $r=|m|+|n|$, mais baixas.

Uma ressonância com fácil interpretação física é a de ordem 1, ou seja, quando pelo menos uma das duas sintonias é um número inteiro. Suponha que exista um erro dipolar em um ponto do anel. Se a sintonia é um número inteiro, os elétrons passarão por aquele ponto com a mesma fase e, portanto, as distorções causadas na órbita devido àquele erro se somarão, fazendo com que a amplitude da oscilação sempre aumente, até que o elétron colida com a câmara de vácuo do anel.

%%%%%%%%%%%%%%%%%%%%%%%%%%%%%%%%%%%%%%%%%%%%%%%%%%%%%%%%%%%%%%%%%%%%%%%%%%%%%%%%%%%%%%%%%%%%%%%%%%%%%%%%%
\subsection{Função Dispersão}

Os elétrons de um feixe em um anel de armazenamento possuem uma distribuição de energia em torno da nominal, resultado de um equilíbrio entre emissão de radiação e aceleração na cavidade de radio-frequência, além de outros fatores, como espalhamentos. Essa distribuição altera valores e gera novos parâmetros do movimento do feixe. Nessa parte do relatório serão definidas três consequências geradas por desvios pequenos (de primeira ordem), sendo a primeira a função dispersão.

Voltando à equação \ref{eqmovhor}, vemos que o desvio de energia $\boldsymbol{\epsilon}$ não influenciará o movimento vertical, mas dará origem a uma nova órbita fechada horizontal, $x_{\boldsymbol{\epsilon}}$, de modo que o deslocamento horizontal total será:
\begin{equation}
 x = x_\beta + x_{\boldsymbol{\epsilon}},\quad\mathrm{com}\quad
x_{\boldsymbol{\epsilon}}(s+P) = x_{\boldsymbol{\epsilon}}(s)
\end{equation}
onde $x_\beta$ denota as oscilações betatron estudadas no ítem anterior.

Devido à linearidade da equação de movimento, esta órbita será linear com o desvio de energia da partícula. Assim, definimos a função dispersão:

\begin{equation}
 x_{\boldsymbol{\epsilon}} (s) = \eta(s) \frac{\boldsymbol{\epsilon}}{E_0},\quad
\eta'' + K(s)\eta=G(s),
\quad \eta(s+P)=\eta(s).
\end{equation}

A solução completa da equação acima é a soma da solução da equação homogênea com uma solução particular. Como as funções $G(s)$ e $K(s)$ são constantes por trechos, podemos evoluir os valores da função dispersão de um ponto inicial para um ponto final usando a formulação matricial:
\begin{equation}
 \left(\begin{array}{c}
        \eta(s_2) \\
        \eta'(s_2)
       \end{array}\right)
 = M(s_2|s_1)
 \left(\begin{array}{c}
        \eta(s_1) \\
        \eta'(s_1)
       \end{array}\right)
 +
 \left(\begin{array}{c}
        d \\
        d'
       \end{array}\right)
\end{equation}
onde $M(s_2|s_1)$ é a matriz de transferência definida na \mbox{equação \ref{eqmatriztransf}} e $d$ é a solução particular, que pode ser obtida através da função de Green construída através das soluções principais da equação homogênea, $C(s_2,s_1)$ e $S(s_2,s_1)$, \cite{Wiedemann3}:
\begin{align}
 F(s_2,\tilde{s}) &= S(s_2,s_1) C(\tilde{s},s_1) - C(s_2,s_1) S(\tilde{s},s_1)
\\
d &= \int^{s_2}_{s_1} G(\tilde{s}) F(s_2,\tilde{s}) d \tilde{s}. &
\end{align}

Apesar deste formalismo matricial não gerar, inicialmente, funções periódicas, este é o método mais usado por softwares de dinâmica de feixe para a determinação de $\eta$ em anéis de armazenamento, devido à facilidade de programação e precisão dos cálculos.

Uma outra forma de obtenção da função dispersão, que satisfaz diretamente a condição de periodicidade, se dá pela utilização da Função de Green periódica da equação de Hill \cite{Lee}:
\begin{equation}
 \eta(s) =  \frac{\sqrt{\beta(s)}}{2\sin(\pi \nu)}\oint G(\tilde{s})
\sqrt{\beta(\tilde{s})} \cos \left(\pi \nu +|\psi(s) - \psi(\tilde{s})|\right)d
\tilde{s}.
\end{equation}
onde nota-se que $\eta$ diverge para sintonias inteiras.

Outro parâmetro fundamental para a óptica de um anel de armazenamento é o fator compactação de momento. De acordo com a métrica do sistema de coordenadas definido na \mbox{equação \ref{eqcoord}}, o comprimento de uma curva infinitesimal é dado por:
\begin{equation}
 (d l)^2 = (1+G x)^2(d s)^2+(d x)^2+(d z)^2 = \left((1+G x)^2+x'^2+
z'^2\right)\,(ds)^2.
\end{equation}
Fazendo $x=x_{\boldsymbol{\epsilon}}$, visto que em primeira ordem as oscilações betatron não alteram o comprimento da órbita, e considerando a aproximação paraxial, obtemos:
\begin{equation}
 P_{\boldsymbol{\epsilon}} = \oint (1+G(s) \eta(s) \delta)ds \quad \Rightarrow
\quad
\Delta P = \delta \oint G(s) \eta(s)\, d s
\end{equation}
onde $P_{\boldsymbol{\epsilon}}$ é o comprimento da nova órbita e $\delta = \boldsymbol{\epsilon} / E_0$ é o desvio relativo de energia. Define-se o fator compactação de momento como \cite{Lee}:
\begin{equation}
 \alpha_c \leq \frac{1}{P} \frac{d \Delta P}{d \delta} = \frac{1}{P}
\oint G(s) \eta(s) \, d s.
\end{equation}
onde observamos que $\alpha_c$ mede a magnitude da função dispersão nos trechos curvos do anel.

A importância do fator compactação de momento está no fato de que ele acopla os movimentos longitudinal e horizontal, sendo determinante para a estabilidade e tempo de amortecimento das oscilações de energia.

Por fim, vamos definir o conceito de cromaticidade. Quando um elétron passa por um quadrupolo com energia diferente da nominal, ele experimenta uma focalização diferente:
\begin{equation}
  K_\delta(s)=\frac{ec}{E_0(1+\delta)}\left(\frac{\partial B}{\partial
x}\right)_{0,s} \approx \frac{ec}{E_0}\left(\frac{\partial B}{\partial
x}\right)_{0,s}(1-\delta) = K_1(s)(1 - \delta)
\end{equation}
Inserindo esse termo na equação do movimento betatron \mbox{(equação \ref{eqoscbet})}, é possível demonstrar \cite{Lee, Wiedemann3} que haverá uma alteração na sintonia dada por
\begin{align}
 \Delta \nu_x \approx&-\left(\frac{1}{4\pi}\oint\beta_x K_1\,ds\right)\delta \\
 \Delta \nu_z \approx&\left(\frac{1}{4\pi}\oint\beta_z K_1\,ds\right)\delta
\end{align}

Define-se cromaticidade como:
\begin{equation}\label{eqcromaticidade}
 \xi_y \leq \frac{d (\Delta \nu_y)}{d \delta}
\end{equation}
sendo que a cromaticidade gerada apenas pelas focalizações da óptica linear é chamada cromaticidade natural, dada por:
\begin{align}
 \xi_x &\approx -\frac{1}{4 \pi} \oint \beta_x K_1\, d s  \\
 \xi_z &\approx \frac{1}{4 \pi} \oint \beta_z K_1\, d s.
\end{align}
A cromaticidade natural é sempre negativa em ambos os planos, pois a focalização é menos/mais efetiva para partículas com energia maior/menor, fazendo com que haja menos/mais oscilações bétatron, o que diminui/aumenta o valor das sintonias.

Valores negativos de cromaticidade pioram a performance de um anel de armazenamento devido a perda de partículas induzida pela mudança da sintonia e por excitarem uma instabilidade do feixe chamada \textit{head-tail} \cite{Wiedemann3}. Por isso, surge a necessidade de inserir sextupolos no anel para corrigir a cromaticidade, sendo que pode ser demonstrado que \cite{Wiedemann3}:
\begin{align}
 \xi_x &\approx\frac{1}{4 \pi} \oint(- \beta_x K_1+m\eta)\, d s \\
 \xi_z &\approx\frac{1}{4 \pi} \oint (\beta_z K_1-m\eta)\, d s.
\end{align}
onde $m=(ec/E_0)(\partial^2B/\partial x^2)$ é a ``força'' do sextupolo.

Pelo fato dos sextupolos serem elementos não lineares, sua inserção no anel altera toda a dinâmica do sistema, afetando a abertura dinâmica\footnote{Neste caso, entende-se por abertura dinâmica o conjunto de condições iniciais que uma partícula deve possuir para fazer um número elevado de revoluções de modo que sua trajetória permaneça limitada}, antes infinita, e excitando ressonâncias betatron de ordem mais alta. Por isso, ao construir uma óptica, deve-se levar em conta tais fatores para instalar os sextupolos em locais da rede que minimizem a força necessária para anular a cromaticidade e que tenham avanços de fase adequados \cite{Huth}.


%%%%%%%%%%%%%%%%%%%%%%%%%%%%%%%%%%%%%%%%%%%%%%%%%%%%%%%%%%%%%%%%%%%%%%%%%%%%%%%%%%%%%%%%%%%%%%%%%%%%%%%%%
%%%%%%%%%%%%%%%%%%%%%%%%%%%%%%%%%%%%%%%%%%%%%%%%%%%%%%%%%%%%%%%%%%%%%%%%%%%%%%%%%%%%%%%%%%%%%%%%%%%%%%%%%
\section{Dinâmica Longitudinal}

No desenvolvimento das equações de movimento da seção anterior não consideramos efeitos de perda ou ganho de energia pelos elétrons. Contudo, a emissão de radiação gera consequências no movimento em todas as direções.

Nessa seção estudaremos os principais efeitos que resultam na definição de parâmetros importantes para o desenvolvimento do projeto.

\subsection{Dinâmica longitudinal e oscilações de energia}

A quantidade de radiação emitida por um elétron depende do seu desvio de energia, de modo que para $\boldsymbol{\epsilon}$ pequeno podemos expandir:

\begin{equation}
 U_{\mathrm{rad}}(\boldsymbol{\epsilon}) = U_0 + Q \boldsymbol{\epsilon}
\end{equation}
onde $U_0$ é a energia perdida por partículas com energia igual a nominal e \cite{Sands}

\begin{equation}
 Q=\left(\frac{d U_\mathrm{rad}}{d \boldsymbol{\epsilon}}\right)_0 =
\frac{U_0}{E_0}\left[2+\mathcal{D}\right]\quad \text{com} \quad\mathcal{D} =
\frac{\oint \eta G(G^2+2 K_1) \, ds}{\oint G^2\,ds}.
\end{equation}

Toda energia perdida por radiação deve ser reposta para que os elétrons mantenham, em média, sua energia constante. Como campos magnéticos não fornecem energia para as partículas, essa reposição deve ser feita por campos elétricos.

O modo mais eficiente de fornecer energia para os elétrons e que caracteriza as fontes luz síncrotron é por meio de cavidades ressonantes que geram campo elétrico variável e tangencial à trajetória dos elétrons, as cavidades RF.

A frequência do modo ressonante dessas cavidades deve ser um múltiplo inteiro da frequência de revolução dos elétrons:

\begin{equation}
 f_\mathrm{rf}=kf_\mathrm{rev}
\end{equation}
onde $k$ é chamado de número harmônico, de modo que eles sempre passem pela cavidade quando o campo elétrico estiver em fase correta.

A primeira condição para se definir a fase do campo elétrico que gera um equilíbrio estável é que ela deve ser tal que a energia fornecida seja igual a perdida em uma volta por um elétron com energia nominal:

\begin{equation}
 e V_\mathrm{rf}(\phi)=U_0 ,
\end{equation}
onde $\phi$ é a fase do campo elétrico e $e$ a carga elétrica do elétron. Assim, se um elétron com energia $E_0$ passa pela cavidade quando ela se encontra nessa fase, ele repetirá esse processo indefinidamente. Esses elétrons são denominados elétrons síncronos.

A segunda condição exige que a derivada do potencial elétrico seja negativa, para que elétrons com energia um pouco diferente da nominal oscilem em torno da energia nominal com frequência $\Omega$, sofrendo um amortecimento em direção a região síncrona com constante $\alpha_{\boldsymbol{\epsilon}}$, sendo estes valores dados por \cite{Sands}:

\begin{equation}
 \alpha_{\boldsymbol{\epsilon}}=\frac{1}{2T_0}\frac{d U_\mathrm{rad}}{d E} =
 \frac{U_0}{2 T_0 E_0}J_{\boldsymbol{\epsilon}} \quad \text{com} \quad
 J_{\boldsymbol{\epsilon}} = 2+\mathcal{D}
\end{equation}

\begin{equation}
 \Omega^2 = \frac{\alpha_c e \dot{V}_0}{T_0 E_0}
\end{equation}
onde $T_0$ é o período de revolução dos elétrons, $J_{\boldsymbol{\epsilon}}$ é número de partição de amortecimento de energia e $\dot{V}$ é o módulo da derivada da voltagem de r.f. na fase síncrona. Geralmente, o amortecimento de energia é um processo lento comparado ao período de revolução, sendo que o tempo de amortecimento, inverso da constante $\alpha_{\boldsymbol{\epsilon}}$, é da ordem de milisegundos.

Juntamente com o amortecimento, há um termo de excitação que é gerado pelo processo de emissão de fótons. Para entender essa excitação, suponha que um elétron possua uma oscilação de energia da forma
\begin{equation}
 \boldsymbol{\epsilon}=A_0 e^{i \Omega(t-t_0)}
\end{equation}
onde desprezamos o amortecimento, devido à escala de tempo que queremos analisar ser muito menor.

% \begin{figure}[b]
%  \center
%  \includegraphics[scale=0.5]{Imagens/emissao_energia.png}
%  \caption{Variação da energia do elétron pela emissão de um fóton \cite{Sands}.}
%  \label{fig:emissaoenergia}
% \end{figure}

Agora suponha que este elétron emita um fóton no tempo $t_i$ \mbox{(Figura \ref{fig:emissaoenergia})}. Ele apresentará uma descontinuidade na energia, e passará a oscilar segundo a equação

\begin{equation}
 \boldsymbol{\epsilon}=A_0 e^{i \Omega(t-t_0)}-u e^{i \Omega(t-t_i)}
\end{equation}
de modo que sua nova amplitude de oscilação será dada por:

\begin{equation}
 A_1^2 = A_0^2 + u^2 - 2 A_0 u \cos(\Omega(t_i-t_0)).
\end{equation}

Fazendo a média temporal da equação acima, notamos que a amplitude aumenta devido ao processo de emissão, visto que devido à aleatoriedade dos tempos de emissão, a média da função cosseno é nula.

É possível demonstrar que sempre há um equilíbrio entre os efeitos de excitação por emissão e de amortecimento e que, nessa situação o desvio de energia dos elétrons têm uma Distribuição Normal com média nula e variância dada por \cite{Wiedemann3}:
\begin{equation}
 \sigma^2_{\boldsymbol{\epsilon}} = \frac{C_q \gamma^2_0
E_0^2}{J_{\boldsymbol{\epsilon}}} \frac{\oint G^3\, ds}{\oint G^2\, ds}
\end{equation}
onde $C_q = \mathrm{3,84\cdot 10^{-13}\, m}$ e $\gamma_0$ é a energia relativística.

O movimento longitudinal está acoplado com as oscilações de energia, pois, devido à função dispersão, partículas com energia diferente da nominal percorrerão o anel em diferentes órbitas fechadas. Isto fará com que o tempo de um ciclo e, consequentemente, sua posição longitudinal relativa à posição síncrona se alterem.

Esse acoplamento faz com que o movimento longitudinal também seja oscilatório e amortecido em torno da posição síncrona, com mesma frequência e amortecimento das oscilações de energia. Assim, os elétrons não se distribuem uniformemente ao longo do anel, mas se acumulam em pacotes, sendo que o número de pacotes é igual ao número harmônico.

O tamanho longitudinal dos pacotes é dado em função das características da cavidade r.f., do fator compactação de momento e da distribuição de energia \cite{Sands}:

\begin{equation}
 \sigma_s= \frac{\alpha_c}{\Omega E_0}\sigma_{\boldsymbol{\epsilon}}
\end{equation}

%%%%%%%%%%%%%%%%%%%%%%%%%%%%%%%%%%%%%%%%%%%%%%%%%%%%%%%%%%%%%%%%%%%%%%%%%%%%%%%%%%%%%%%%%%%%%%%%%%%%%%%%%
\subsection{Tamanho Transversal}

Na direção transversal também há efeitos de amortecimento e excitação do feixe relacionados à cavidade r.f.~e à emissão de radiação, efeitos estes que definem o tamanho do feixe nessas direções. Primeiramente, vamos analisar a direção horizontal.

% \begin{figure}[!b]
%  \center
%  \includegraphics[scale=0.4]{Imagens/emissao_horizontal.png}
%  \caption{Efeito do processo de emissão nas oscilações betatron \cite{Sands}.}
%  \label{fig:emissaohorizontal}
% \end{figure}

O cone de radiação emitida por um elétron tem um ângulo de divergência em relação à trajetória da partícula da ordem de $1/\gamma$ \cite{Lee}. Desse modo, pode-se considerar que o momento linear perdido por esse processo está na direção do movimento.

Se não fosse pela existência da função dispersão esse processo não alteraria a trajetória da partícula. Contudo, quando o elétron emite um fóton, ele perde energia e sua órbita fechada se altera (vide \mbox{(Figura \ref{fig:emissaohorizontal})}, fazendo com que as oscilações betatron também sofram um deslocamento e, como o movimento é contínuo:

\begin{equation}
 \delta x = \delta x_\beta + \eta \frac{d E}{E_0} = 0 \quad
\Rightarrow \quad \delta x_\beta =- \eta \frac{d E}{E_0}.
\end{equation}

É possível demonstrar que esta alteração no movimento gera um amortecimento na amplitude dado por:

\begin{equation}
 \frac{\delta a}{T_0a} = \frac{U_0}{2T_0 E}\mathcal{D}.
\end{equation}
Como o valor de $\mathcal{D}$ é em geral próximo da unidade, esse termo é positivo, ou seja, há um aumento na amplitude do movimento devido ao efeito da emissão de radiação.

% \begin{figure}[t]
%  \center
%  \includegraphics[scale=0.8]{Imagens/amortecimento_cavrf.png}
%  \caption{Amortecimento das oscilações betatron causado pela aceleração na cavidade r.f.~\cite{Sands}.}
%  \label{fig:amortecimentocavrf}
% \end{figure}

As cavidades r.f. não introduzem efeitos de amortecimento do tipo descrito para o processo de emissão porque elas são inseridas em trechos retos do anel, onde a perda ou ganho de energia não implicam em uma mudança imediata na órbita seguida pelo elétron.

A cavidade r.f.~também introduz um termo de amortecimento nas oscilações betatron: quando o elétron passa pela cavidade de r.f.~o momento linear que ele recebe, $\delta \textbf{P}$, está apenas na direção longitudinal, o que altera o ângulo do movimento betatron \mbox{(Figura \ref{fig:amortecimentocavrf})}, gerando um amortecimento das oscilações dado por \cite{Sands}:

\begin{equation}
  \frac{\delta a}{T_0a} = \frac{U_0}{2T_0 E}.
\end{equation}
Somando as duas contribuições, obtemos a constante de amortecimento das oscilações betatron horizontais:

\begin{equation}\label{constanteamorthor}
\alpha_x =  \frac{U_0}{2T_0 E}J_x , \quad \text{com} \quad J_x =(1-\mathcal{D}).
\end{equation}
onde $J_x$ é o número de partição de amortecimento da direção horizontal.

A constante de amortecimento descrita na \mbox{equação \ref{constanteamorthor}}, abrange apenas efeitos médios, relativos à perda de energia pelo processo emissão de radiação. Contudo, há também efeitos de excitação nas oscilações betatron, devido à efeitos quânticos (emissão quantizada e aleatória), que equilibram o efeito do amortecimento, gerando um valor de equilíbrio para a amplitude do movimento, $\sqrt{\varepsilon}$, dado por:

\begin{equation}
 \langle\varepsilon\rangle = C_q\frac{2 \gamma^2 \langle
\mathcal{H}/|\rho^3|\rangle}{J_x\langle 1/\rho^2 \rangle} ,\quad \text{com}
\quad \mathcal{H}=  \gamma \eta^2 + 2 \alpha \eta \eta' + \beta \eta'^2
\end{equation}
onde $C_q = \mathrm{3,83 \times 10^-13}$ m e $\langle \rangle$ representa a média ao longo da órbita de referência.

Também é possível mostrar que a distribuição de partículas na direção transversal segue uma distribuição normal \cite{Lee}, sendo que o valor $\langle\varepsilon\rangle$ corresponte a um sigma da distribuição de
amplitudes. Dessa forma, sabendo que o tamanho do feixe devido apenas às oscilações betatron é dado por

\begin{equation}
 \sigma_{x,\beta}^2(s) = \frac{1}{2}\beta_x(s)\langle\varepsilon\rangle,
\end{equation}
podemos definir a emitância natural do anel de armazenamento como:

\begin{equation}\label{eqemitancia}
 \varepsilon_0 \leq \frac{\sigma_{x,\beta}^2(s)}{\beta_x(s)} =
C_q\frac{\gamma^2 \langle \mathcal{H}/|\rho^3|\rangle}{J_x\langle 1/\rho^2
\rangle}
\end{equation}

O tamanho total do feixe horizontal é dado pela soma do tamanho devido às oscilações betatron com o introduzido pelas oscilações de energia, através da função dispersão. Matematicamente temos:
\begin{equation}
 x = x_\beta + \eta\frac{\boldsymbol{\epsilon}}{E_0} \quad \Rightarrow \quad
\sigma_x = \sqrt{\varepsilon_0 \beta(s)
+\eta^2 \left(\frac{\sigma_{\boldsymbol{\epsilon}}}{E_0}\right)^2}
\end{equation}

Na direção vertical a função dispersão é nula, fazendo com que tanto $\mathcal{D}$ como $\mathcal{H}$ sejam nulos. De acordo com os cálculos feitos acima, isso implicaria em uma emitância nula na direção vertical, pois haveria apenas um amortecimento causado pela cavidade r.f. Contudo, há um outro efeito de excitação quântica, que não foi considerado na derivação acima por ser $1/\gamma^2$ vezes menor que aquele analisado, gerado pela projeção da componente do momento da radiação emitida perpendicular à trajetória do elétron na direção vertical (matematicamente: $\mathbf{P_\perp} \cdot \mathbf{\hat{z}}$).

A constante de amortecimento para a direção vertical é dada por \cite{Sands}:
\begin{equation}
 \alpha_z = \frac{U_0}{2T_0 E_0}J_z , \quad \text{com} \quad J_z = 1
\end{equation}
de modo que os números de partição satisfazem o teorema de Robinson \cite{Robinson}:
\begin{equation}
 J_x + J_{\boldsymbol{\epsilon}} + J_z = 4 \quad \text{ou} \quad J_x +
J_{\boldsymbol{\epsilon}} = 3.
\end{equation}

Quando há campos magnéticos que acoplam os movimentos horizontal e vertical, a emitância natural é distribuida entre as direções $x$ e $z$ proporcionalmente ao campo, de modo que:
\begin{eqnarray}
& \varepsilon_x + \varepsilon_z = \varepsilon_0 & \\
& \varepsilon_z = \kappa \varepsilon_x
\end{eqnarray}
onde $\kappa$ é denominado coeficiente de acoplamento.

Quando há acoplamento, o tamanho vertical deixa de ser desprezível em relação ao horizontal, o que faz aumentar o volume dos pacotes, diminuindo a densidade de carga em cada pacote. Por sua vez, o tempo de vida do feixe é aumentado e o brilho da luz síncrotron diminuído.

    \chapter{Wakes and Impedances}\label{cap:wake_impedances}

%%%%%%%%%%%%%%%%%%%%%%%%%%%%%%%%%%%%%%%%%%%%%%%%%%%%%%%%%%%%%%%%%%%%%%%%%%%%%%%%%%%%%%%%%%%%%%%%%%%%%%%%%
%%%%%%%%%%%%%%%%%%%%%%%%%%%%%%%%%%%%%%%%%%%%%%%%%%%%%%%%%%%%%%%%%%%%%%%%%%%%%%%%%%%%%%%%%%%%%%%%%%%%%%%%%
\section{\engw{Wake Fields}}

In this section we are going to introduce the concept of wake field in particle accelerators and try to understand its foundations and main properties for the subsequent analysis of its influence over the movement of the charged particles in the accelerator. There are several approaches in the literature to explain this subject, the most appreciated by the author is the one presented in reference \cite{Stupakov2000a}, which will be reproduced in parts here.

%%%%%%%%%%%%%%%%%%%%%%%%%%%%%%%%%%%%%%%%%%%%%%%%%%%%%%%%%%%%%%%%%%%%%%%%%%%%%%%%%%%%%%%%%%%%%%%%%%%%%%%%%
\subsection{Interaction Mechanisms}\label{ssec:interation_mechanisms}

Intuitively, we tend to think the direct interaction between charged particles, such as electric repulsion, is the responsible for the collective effects observed in storage rings, however, as we will see in the subsequent analysis, this is not the main mechanism for ultra-relativistic particles.

\begin{figure}[hb!]
\centering
\label{fig:wake1}
\begin{tikzpicture}[scale=1]
\def\d{1cm}
\draw[<->] (1,0) node[below]{$z$}
		-- ++(-\d,0) node[below left] {$S$}
        -- ++(0,\d) node[left] {$\rho$}; %coord sys S
\draw[<->] (6*\d,0) node[below]{$z'$}
		-- ++(-\d,0) node[below left] {$S'$}
        -- ++(0,\d) node[left] {$\rho'$}; % coord sys S'
\coordinate (V) at (0.5,0);
\coordinate (Q1) at (4cm,1.5cm);
\coordinate (Q2) at (0.5cm,2.5cm);
\draw[->] (5*\d,0.5*\d) -- ++(V) node[above] {$\boldsymbol{v}$};
\filldraw[fill=black] (Q1) circle[radius=0.05] node[above] {$q$}; % source particle
\draw[->] (Q1) -- ++(V) node[above] {$\boldsymbol{v}$}; % velocity vector
\filldraw[fill=black] (Q2)  circle[radius=0.05]; % test particle
\draw[->] (Q2) -- ++(V) node[above] {$\boldsymbol{v}$}; % velocity vector
\draw[dashed,|-|] ($(Q1)-(0,0.2)$)
				   let \p1 = ($(Q2) - (Q1)$)
                   in -- ++(\x1,0) node[midway,below] {$s$}; %horizontal distance
\draw[dashed,|-|] ($(Q2)-(0.2,0)$)
				   let \p1 = ($(Q1) - (Q2)$)
                   in -- ++(0,\y1) node[midway,left] {$\rho$}; % vertical distance
\draw[->] (Q1) -- ++($(Q2) - (Q1)$) node[midway,above] {$\boldsymbol{R}$}; %vector
\end{tikzpicture}
\caption{Duas partículas interagindo via campo direto.}
\end{figure}

To see this lets consider the interaction of a source particle $Q$ moving with velocity $\vect{v}=v{\vect{\hat{z}}}$ with a witness particle $q$ moving with the same velocity (parallel path) at a distance $s$ in the direction parallel to the movement and at a transverse distance $\rho$, as shown in Figure\,\ref{fig:wake1}. We want to determine the force that the source particle exerts on the witness particle. One way to do this is by calculating the electric field of the source particle in the co-moving frame of reference, $S'$, and Lorentz transforming it back to the laboratory's frame. After the math we obtain:

\begin{align}
 \label{eq:fields_free_particle}
 \vect{E} = \frac{q}{4\pi\epsilon_0}\frac{\vect{R}}{\gamma^2 R^{*3}}, & & \vect{B} = \frac{1}{c^2}\vect{v} \times \vect{E}
\end{align}
where $\vect{R}$ is the vector which connects both particles, going from the source to the witness and  $R^{*2} = s^2 + x^2/\gamma^2$, e $\gamma = 1/(1-v^2/c^2)$.

Combining equation\,\ref{eq:fields_free_particle} with the Lorentz force, we get the longitudinal and transverse force over the witness particle:
\begin{align}\label{eq:space_charge_force}
 F_l &= E_z = -\frac{q}{4\pi\epsilon_0}\frac{s}{\gamma^2\left(s^2+x^2/\gamma^2\right)^{3/2}}, \\
 F_t &= E_x - vB_y = -\frac{q}{4\pi\epsilon_0}\frac{x}{\gamma^4\left(s^2+x^2/\gamma^2\right)^{3/2}}
\end{align}

In accelerator physics the force $\vect{F}$ is known as space charge force. We can infer from equation\,\ref{eq:space_charge_force} that for any position $s$ and $x$, the longitudinal force is proportional to $\gamma^{-2}$ and $F_t \sim \gamma^{-4}$ if $s \gg x/\gamma$ and  $F_t \sim \gamma^{-1}$ if $s \approx 0$. This way, in the ultra-relativistic limit, $\gamma \to \infty$, the electromagnetic interaction between particles moving parallel to each other in free space is zero. It is easy to show that in this limit, if the movement of the particles is not parallel, there is an interaction force only for $s=0$, but, as their speed is the same, this situation can only happen for an infinitesimal time.


In this work we are interested in the the interaction between particles in the ultra-relativistic limit, $v \to c$. The space charge effects discussed above are despicable in this limit and the interaction between the particles is due to the presence of the walls of the vaccum chamber. Note that taking the limit $v \to c$ in the equation\,\ref{eq:fields_free_particle} and remembering that $s = vt - z$, we can write the electromagnetic field of a ultra-relativistic charge as
\begin{align}
\vect{E} = \frac{q}{2\pi\epsilon_0}\frac{\vect{\hat{r}}}{r}\delta(z-ct), & & \vect{B} =\frac{1}{c}\vect{\hat{z}}\times\vect{E},
\end{align}
where $\vect{r} = \vect{\hat{x}}x + \vect{\hat{y}}y$ is a bidimensional vector in cylindrical coordinates ($\vect{\hat{x}}$ and $\vect{\hat{y}}$ are unit vectors in the $x$ e $y$ directions, respectively). The equations above show that the field is pancake-like and follow the beam as it travels througth the empty space. It is important to notice that this solution is steady-state, it was necessary an infinite amount of time before $t$ to build it and that's why there is no causal paradoxes in it.

\begin{figure}[hb!]
\centering
\label{fig:wake2}
\begin{tikzpicture}
\draw[very thick] (0,0) -- ++(10,0) (0,4) -- ++(10,0); %vacuum chamber
\draw[dashed] (0,2) -- ++(10,0); %eixo de simetria
\coordinate (V) at (0.5,0);
\coordinate (Q1) at (4cm,2.2cm);
\coordinate (Q2) at (0.5cm,2.5cm);
\filldraw[fill=black] (Q1) circle[radius=0.05] node[left] {$q$}; % source particle
\draw[->] (Q1) -- ++(V) node[above] {$\boldsymbol{v}$}; % velocity vector
\draw[-{Stealth[length=10pt]}] (Q1) let \p1 = (Q1) in --(\x1,4);
\draw[-{Stealth[length=10pt]}] (Q1) let \p1 = (Q1) in --(\x1,0) node[midway,right] {$\boldsymbol{E}$};
\draw ($(Q1)+(0,1)$) circle[radius=0.2] node[right=0.2] {$\boldsymbol{B}$};
\filldraw ($(Q1)+(0,1)$) circle[radius=0.07];
\end{tikzpicture}
\caption{Particles interacting in a perfectly conducting cylindrical tube.}
\end{figure}

Lets consider now a pipe with cylindrical symmetry\footnote{do not confuse cylindrical symmetry with cylinder. By cylindrical symmetry we mean a system with translational symmetry in one direction.}, hollow and with absolute vacuum in its interior, made of a perfect electric conductor material and with arbitrary cross section. If we put the particles of the previous example inside this pipe, moving parallel to symmetry axis, they will induce image charges in the surface of the wall which cancel the electromagnetic field inside the metal.

The image charges travel with the same velocity $\vect{v}$ of the particles (see Figure\,\ref{fig:wake2}). As they move in parallel paths with constant velocity, in the limit $v \to c$, according to the previous results, they do not interact, independently of how close they are from each other.

From this analysis we conclude that the interaction between particles in the ultra-relativistic limit can occur only for two reasons:
\begin{itemize}
    \item The wall is not perfectly conducting, or
    \item The pipe does not have cylindrical symmetry (which generally is due to the presence of RF cavities, flanges, bellows, beam position monitors, vacuum pumps, among other elements in the vacuum chamber of an accelerator).
\end{itemize}

%%%%%%%%%%%%%%%%%%%%%%%%%%%%%%%%%%%%%%%%%%%%%%%%%%%%%%%%%%%%%%%%%%%%%%%%%%%%%%%%%%%%%%%%%%%%%%%%%%%%%%%%%
\subsection{Causality and catch up distance}

If one particle moves in a straight line at light speed, the electromagnetic field scattered by the discontinuities of the chamber will not catch up with it and will not affect the charges travelling ahead of it. The field will only interact with the charges moving behind of the source particle. Such property is known as causality.

Even though this property is not strictly true in the real world, because particles always travel at speeds lower than the light's, it is true in most practical cases, as we will se bellow.

Lets try to calculate the distance $z$ where the field generated by some discontinuity in the vacuum chamber will catch up with a witness particle at a distance $s$ behind the source particle. At the time $t=0$ the source particle passes through the discontinuity and an electromagnetic wave is generated with its wave front travelling at the speed of light in all directions, forming a sphere of radius $R$, see FigureXX. At any given time after this, the following relation holds:
\begin{align}
ct = R \quad vt = z && \Rightarrow && R = \frac{z}{\beta} \quad \text{where} \,\, \beta = \frac{v}{c}
\end{align}
where $z$ is the distance travelled by the source particle. Besides that, at the specific time when the wake catchs up with the witness particle, the following relation is valid:
\begin{align}
R^2 = b^2 + (z-s)^2 && \Rightarrow &&
z^2(\frac{1}{\beta^2}-1) + 2sz - (b^2 + s^2) = 0 && \Rightarrow \\
z = -\gamma^2 \beta^2 s + \sqrt{s^2\gamma^4\beta^4 + \gamma^2\beta^2\left(b^2 + s^2\right)} && = \gamma^2 \beta^2 s\left(-1 + \sqrt{1 + \frac{1}{\gamma^2\beta^2}\left(1 + \frac{b^2}{s^2}\right)}\right) &&
\end{align}
where $b$ is the distance from the discontinuity to the trajectory of the particles and $\gamma = 1/\sqrt{1-\beta^2}$ is the relativistic energy. FigureXX shows a graphic of this function, normalized by the distance $b$. We notice that, for $s=0$, which means the field catching up with the source particle, $z = \gamma\beta b$. For the case of Sirius, $\gamma \approx 5870$ and $\beta \approx 1$, if $ b = 2$mm, $z \approx 12$m.

\begin{figure}[hb!]
\centering
\label{fig:catch_up}
\begin{tikzpicture}
\draw[very thick] (0,0) -- ++(10,0) (0,4) -- ++(10,0); %vacuum chamber
\draw[very thick] (4.8,4) to [out=-90,in=0] (5,3.8) to [out=180,in=-90] (5.2,4);
\draw[dashed] (0,2) -- ++(10,0); %eixo de simetria
\coordinate (V) at (0.5,0);
\coordinate (Q1) at (4cm,2.2cm);
\coordinate (Q2) at (0.5cm,2.5cm);
\filldraw[fill=black] (Q1) circle[radius=0.05] node[left] {$q$}; % source particle
\draw[->] (Q1) -- ++(V) node[above] {$\boldsymbol{v}$}; % velocity vector
\draw[-{Stealth[length=10pt]}] (Q1) let \p1 = (Q1) in --(\x1,4);
\draw[-{Stealth[length=10pt]}] (Q1) let \p1 = (Q1) in --(\x1,0) node[midway,right] {$\boldsymbol{E}$};
\draw ($(Q1)+(0,1)$) circle[radius=0.2] node[right=0.2] {$\boldsymbol{B}$};
\filldraw ($(Q1)+(0,1)$) circle[radius=0.07];
\end{tikzpicture}
\caption{The catch up distance.}
\end{figure}


%%%%%%%%%%%%%%%%%%%%%%%%%%%%%%%%%%%%%%%%%%%%%%%%%%%%%%%%%%%%%%%%%%%%%%%%%%%%%%%%%%%%%%%%%%%%%%%%%%%%%%%%%
\subsection{Definição de Wake}\label{ssec:wake_definition}

The Electromagnetic interaction of charged particles with the environment generally has a small effect when compared with the effect of the guiding electric and magnetic fields of the accelerators components and can be treated as a perturbation. In an zeroth order approximation we can assume the beam moves with constant speed in a straight line and solve Maxwell equations

%%%%%%%%%%%%%%%%%%%%%%%%%%%%%%%%%%%%%%%%%%%%%%%%%%%%%%%%%%%%%%%%%%%%%%%%%%%%%%%%
%%%%%%%%%%%%%%%%%%%%%%%%%%%%%%%%%%%%%%%%%%%%%%%%%%%%%%%%%%%%%%%%%%%%%%%%%%%%%%%%
%%%%%%%%%%%%%%%%%%%%%%%%%%%%%%%%%%%%%%%%%%%%%%%%%%%%%%%%%%%%%%%%%%%%%%%%%%%%%%%%
A interação eletromagnética de partículas carregadas com o ambiente normalmente tem um efeito pequeno quando comparado com o efeito de campos elétricos e magnéticos externos dos aceleradores e pode ser considerada com uma perturbação. Em uma aproximação de ordem zero, podemos assumir que o feixe se move com velocidade constante em uma linha reta, e então resolvemos as equações de Maxwell, encontramos os campos e computamos o efeitos desses campos no movimento das partículas. Com essa abordagem negligenciamos efeitos de segunda ordem porque o movimento em uma órbita perturbada podem gerar apenas uma pequena mudança nos campos computados pela aproximação de ordem zero. Essas correções são geralmente pequenas, especialmente para partículas ultra-relativísticas.

Outra importante característica da interação entre o campo eletromagnético gerado e as partículas é que em muitos casos de importância prática eles estão localizados em uma região pequena comparada com o tamanho da órbita do feixe. Ela também ocorre uma escala de tempo muito menor que a de oscilação do feixe no acelerador (como os períodos bétatron e síncrotron). Isso nos permite considerar essa interação dentro da aproximação de impulso e caracterizá-la pelo momento transferido para a partícula.

Dessa forma, podemos introduzir a noção de \engw{wake} da seguinte maneira. Considere a partícula 1, com carga $q$ se movendo ao longo do eixo $z$ com uma velocidade próxima à da luz, $v\approx c$, de modo que $z=ct$ (veja a figura \ref{fig:4}). Uma partícula 2 com carga unitária se move paralelamente à partícula 1, com a mesma velocidade, a uma distância $s$ com deslocamento transversal $\vect{\rho}$ relativo ao eixo $z$. O vetor $\vect{\rho}$ é um vetor bi-dimensional perpendicular ao eixo $z$, $\vect{\rho} = (x,y)$. Apesar de as partículas viajarem no vácuo, há contornos materiais no problema que espalham o campo eletromagnético que gera uma interação entre as partículas.

Assumindo que as equações de Maxwell foram resolvidas e que os campos gerados pela partícula 1 foram encontrados, podemos calcular a mudança no momento $\Delta \vect{p}$ da segunda partícula causada por esse campo como uma função do deslocamento $\vect{\rho}$ e da distância $s$,

\begin{equation}
 \Delta \vect{p}(\vect{\rho},s) = \defint{t}{\left[\vect{E}(\vect{\rho},z,t) + \vect{\hat{z}}\times \vect{B}(\vect{\rho},z,t)\right]_{z=ct-s}}{-\infty}{\infty}.
\end{equation}

Note que a integral é feita sobre uma linha reta --- a órbita não perturbada da segunda partícula. Os limites de integração são estendidos de menos para mais infinito assumindo que a integral converge.

Como a dinâmica do feixe é diferente nas direções longitudinal e transversal, é útil separar o momento longitudinal $\Delta p_z$ da componente transversal $\vect{\Delta p}_\perp$. Dessa forma, com uma convenção de sinal e um fator de normalização $c/q$, podemos definir as chamadas \engw{wake functions}, ou simplesmente \engw{wakes} longitudinal e transversal,

\begin{equation}\label{eq:wake_definition}\begin{aligned}
    w_l(\vect{\rho},s) &= -\frac{c}{q} \Delta p_z = -\frac{c}{q} \udefint{t}{E_z|_{z=ct-s}}, \\
    \vect{w}_t(\vect{\rho},s) &= \frac{c}{q} \vect{\Delta p}_\perp = \frac{c}{q} \udefint{t}{\left[\vect{E}_\perp + \vect{\hat{z}}\times \vect{B}\right]_{z=ct-s}}
\end{aligned}\end{equation}

Note o sinal de menos na definição de $w_l$ --- ele é introduzido para que um \engw{wake} longitudinal positivo corresponda a uma perda de energia da partícula teste (caso ambas a partícula fonte e teste tenham o mesmo sinal de carga). Os \engw{wakes} definidos tem dimensão de \si{\volt\per\coulomb} no Sistema Internacional de Unidades.

Por causa do princípio de causalidade o \engw{wakefield} não se propaga a frente da partícula fonte, então
\begin{equation}
    w_l(\vect{\rho},s) = 0, \qquad \vect{w}_t(\vect{\rho},s) = \vect{0}, \qquad \mathrm{para } \quad s < 0.
\end{equation}

Na definição acima foi assumido que o campo eletromagnético estava localizado no espaço e no tempo e que a integral na equação \ref{eq:wake_definition} converge. Contudo, há casos em que isso não é verdade e a fonte do \engw{wake} é distribuida uniformemente em um longo caminho, como é o caso do \engw{wake} de parede resistiva de uma câmara de vácuo. Nesse caso é mais conveniente introduzir o \engw{wake} por unidade de comprimento, descartando a integração na equação \ref{eq:wake_definition}:

\begin{equation}\begin{aligned}
    w_l(\vect{\rho},s) &= -\frac1q E_z|_{z=ct-s}, \\
    \vect{w}_t(\vect{\rho},s) &= \frac1q\left[\vect{E}_\perp + \vect{\hat{z}}\times \vect{B}\right]_{z=ct-s}.
\end{aligned}\end{equation}

Nessa definição os \engw{wakes} adquirem uma dimensão adicional de inverso de comprimento e tem dimensão \si{\volt\per\coulomb\per\meter} no Sistema Internacional de Unidades.

%%%%%%%%%%%%%%%%%%%%%%%%%%%%%%%%%%%%%%%%%%%%%%%%%%%%%%%%%%%%%%%%%%%%%%%%%%%%%%%%%%%%%%%%%%%%%%%%%%%%%%%%%
\subsection{Teorema de Panofsky-Wenzel}

Várias relações gerais entre os \engw{wakes} longitudinal e transversal podem ser obtidas das equações de Maxwell sem que seja necessário especificar as condições de contorno para os campos.

Vamos introduzir o vetor $\vect{R} =(\vect{\rho},-s)$ (o sinal de menos na frente do $s$ é devido ao fato de $s$ ser positivo para posições atrás da partícula fonte) e considerar o momento $\vect{\Delta p}$ na equação \ref{eq:wake_definition} como uma função de $\vect{R}$. Vamos assumir que o campo eletromagnético é especificado através do potencial vetor $\vect{A}(\vect{r},t)$ e o potencial escalar $\phi(\vect{r},t)$, e computar
$\vect{\Delta p}$ para os dados campos. É conveniente usar a formulação Lagrangiana para as equações de movimento,

\begin{equation}\label{eq:euler_lagrange}
    \dertot{}{t}\derpar{L}{\vect{v}} = \derpar{L}{\vect{r}} = \vect{\nabla}L,
\end{equation}
com a Lagrangiana para a partícula teste com carga unitária é
\begin{equation}\label{eq:lagrangiana_charged_part}
   L = -mc^2 \sqrt{1 - \frac{v^2}{c^2}} + \frac1c \vect{Av} - \phi
\end{equation}
substituindo a equação \ref{eq:lagrangiana_charged_part} na equação \ref{eq:euler_lagrange} obtemos:

\begin{equation}
 \dertot{}{t} \left(\vect{p} + \frac1c \vect{A}\right) = \vect{\nabla}\left(\frac1c \vect{Av} - \phi\right).
\end{equation}
onde $\vect{p} = m\gamma\vect{v}$.

Agora, integrando esta equação ao longo da órbita da partícula teste, $x=\mathrm{const}$, $y=\mathrm{const}$ e $z = ct-s$, e assumindo que os campos $\vect{A}$ e $\phi$ vão a zero no infinito, encontramos

\begin{equation}
 \vect{\Delta p}(\vect{R}) = \udefint{t}{\vect{\nabla}\left(\frac1c\vect{Av} - \phi \right) = \frac{q}{c} \vect{\nabla_R}W(\mathrm{R})}.
\end{equation}
onde introduzimos o \engw{wake potential} $W$,
\begin{equation}
 W(\vect{R}) = \frac{c}{q}\udefint{t}{\left(\frac1c \vect{Av} -\phi\right)}
      \overset{\vect{v} \approx c\vect{\hat{z}}}{=}
                 \frac{c}{q}\udefint{t}{\left(A_z -\phi\right)}.
\end{equation}

Assim, provamos uma relação que estabelece que todas as três componentes do vetor $\vect{\Delta p}$ podem ser obtidas derivando uma única função escalar $W$. Relembrando a relação entre os componentes de $\vect{\Delta p}$ e os \engw{wakes}, \ref{eq:wake_definition}, descobrimos que

\begin{equation}\label{eq:wake_function_definition}
 w_l = - \derpar{W}{(-s)} = \derpar{W}{s},\qquad \vect{w}_l = \vect{\nabla_\rho}W,
\end{equation}
e, consequentemente
\begin{equation}\label{eq:panofsky_wenzel_theorem}
 \derpar{\vect{w}_t}{s} = \vect{\nabla_\rho}w_l.
\end{equation}

Esta relação é comumente chamada de teorema de Panofsky-Wenzel. Note que $\nabla_\rho$ é um gradiente bidimensional com respeito às coordenadas $x$ e $y$.

Uma das aplicações computacionais mais importantes do teorema de Panofsky-Wenzel é que o conhecimento do \engw{wake function}, $w_l$, longitudinal nos permite encontrar o \engw{wake} transversal,$\vect{w}_t$ por meio de uma integração simples da equação \ref{eq:panofsky_wenzel_theorem}.

Outra propriedade importante de $W$ é que ele é uma função harmônica das variáveis $x$ e $y$,
\begin{equation}\label{eq:wake_potential_harmonic}
 \Delta_\perp W \equiv \derpar[2]{W}{x} + \derpar[2]{W}{y} = 0.
\end{equation}
Para provar isso vamos usar o fato que ambos $\vect{A}$ e $\phi$ satisfazem a equação de onda no espaço livre, $(\partial^2/\partial t^2 - c^2 \Delta)\vect{A} = \vect{0}$ e $(\partial^2/\partial t^2 - c^2 \Delta)\phi = 0$. Dessa forma,

\begin{equation}\begin{aligned}
0
&=\frac{c}{q}\udefint{t}{\left(\derpar[2]{}{t}-c^2\Delta\right)\!\!(A_z-\phi)}\\
&=\frac{c}{q}\left[\udefint{t}{\left(\derpar[2]{}{t} - c^2\derpar[2]{}{z}\right)} -
	            c^2\udefint{t}{\left(\derpar[2]{}{x} + \derpar[2]{}{y}\right)}\right]\!\!(A_z-\phi)\\
&=\frac{c}{q}\udefint{t}{\left(\derpar{}{t}+c\derpar{}{z}\right)
				     \!\!\left(\derpar{}{t}-c\derpar{}{z}\right)\!\!(A_z -\phi)}
   -c^2\!\left(\derpar[2]{W}{x} + \derpar[2]{W}{y}\right)
\end{aligned}\end{equation}

A última integral nessa equação é nula porque
\begin{equation}
    \derpar{}{t} + c\derpar{}{z} \approx \derpar{}{t} + \vect{v\nabla} = \dertot{}{t}
\end{equation}
e
\begin{equation}
  \udefint{t}{\left(\derpar{}{t} + c\derpar{}{z}\right)\!\!
             \left(\derpar{}{t} - c\derpar{}{z}\right)\!\!(A_z - \phi)}
  = \udefint{t}{\dertot{}{t}\!\!\left(\derpar{}{t}-c\derpar{}{z}\right)\!\!(A_z - \phi)}
  =  0.
\end{equation}

%%%%%%%%%%%%%%%%%%%%%%%%%%%%%%%%%%%%%%%%%%%%%%%%%%%%%%%%%%%%%%%%%%%%%%%%%%%%%%%%%%%%%%%%%%%%%%%%%%%%%%%%%
\subsection{Sistemas Com Um Eixo de Simetria}
Na seção \ref{ssec:wake_definition} nós definimos o \engw{wake} como uma função do deslocamento da partícula teste relativo ao caminho da partícula fonte. Em aplicações práticas nós também estamos interessados em saber como o \engw{wake} depende da trajetória da partícula fonte. Assumiremos que o sistema em consideração tem um eixo de simetria e vamos escolhe-lo como o eixo $z$ do sistema de coordenadas, veja \ref{fig:5}. Agora a partícula fonte, $1$, se move na direção $z$ com um deslocamento dado pelo vetor $\vect{\rho'}$, e a partícula teste viaja paralelamente à partícula fonte, com a mesma velocidade, a uma distância $s$ atrás da fonte com um deslocamento $\vect{\rho}$ relativo ao eixo. Os vetores $\vect{\rho'}$ e $\vect{\rho}$ são os vetores bidimensionais perpendiculares ao eixo $z$. O \engw{wake} ainda é definido pela equação \eqref{eq:wake_definition} mas agora ele será considerado como uma função de $\vect{\rho'}$, $\vect{\rho}$ e $s$
\begin{equation}\begin{aligned}
w_l &= w_l(\vect{\rho},\vect{\rho'},s), \\
\vect{w}_t &= \vect{w}_t(\vect{\rho},\vect{\rho'},s).
\end{aligned}\end{equation}

Normalmente a câmara de vácuo é projetada de forma que o eixo do sistema serve como uma órbita ideal para o feixe. Assim, desvios desse eixo são relativamente pequenos e ambos os vetores $\vect{\rho'}$ e $\vect{\rho}$ são tipicamente muito menores que a câmara de vácuo, de forma que em $w_l$ podemos negligenciá-los e introduzir um \engw{wake} longitudinal que só depende de $s$,

\begin{equation}
	w_l(s) = w_l(\vect{0},\vect{0},s).
\end{equation}

Se os elementos que formam a câmara de vácuo também tem alguma simetria (por exemplo, se eles tem seção transversal circular, elíptica ou retangular), o \engw{wake} transverso no eixo, onde $(\vect{\rho},\vect{\rho'}) = (\vect{0},\vect{0})$, é nulo, $\vect{w}_t(\vect{0},\vect{0},s)=\vect{0}$. Para pequenos valores de $(\vect{\rho},\vect{\rho'})$ nós podemos expandir $\vect{w}_t(\vect{\rho},\vect{\rho'},s)$ mantendo apenas os termos lineares. Dessa forma, obtemos uma relação tensorial entre os \engw{wakes} transversos e os deslocamentos,

\begin{equation}
	\vect{w}_t(\vect{\rho},\vect{\rho'},s) = \overleftrightarrow{\vect{W}_1}(s)\vect{\rho} +
    										 \overleftrightarrow{\vect{W}_2}(s)\vect{\rho'},
\end{equation}
onde $\overleftrightarrow{\vect{W}_1}$ e $\overleftrightarrow{\vect{W}_2}$ são tensores bidimensionais de ordem 2.

\subsection{Sistemas com Simetria axial}

Em um sistema com simetria axial, o \engw{wake potential}, $W$, depende apenas dos módulos de $\vect{\rho}$ e $\vect{\rho'}$ e do ângulo, $\theta$ entre eles. Sempre podemos escolher um sistema de coordenadas tal que o vetor $\vect{\rho'}$ fique no plano $xz$, veja a figura \ref{fig:6}, de forma que $W$ será uma função periódica e par \todo{entender porque é par} do ângulo $\theta$ em um sistema de coordenadas cilíndrico. Decompondo $W$ em séries de Fourier em $\theta$ temos:

\begin{equation}
	W(\rho,\rho',\theta,s) = \sum_{m=0}^\infty W_m (\rho,\rho',s) \cos(m\theta).
\end{equation}

Inserindo essa equação na equação \eqref{eq:wake_potential_harmonic}, temos

\begin{equation}
	\sum_{m=0}^\infty \left(\frac1\rho\derpar{}{\rho}\rho\derpar{W_m}{\rho} -
    					    \frac{m^2}{\rho^2}W_m\right)\cos(m\theta) = 0
\end{equation}
de onde podemos encontrar a dependência explícita em $\rho$ de $W$,
\begin{equation}\label{eq:wake_potential_of_rho}
	W_m(\rho,\rho',s) = A_m(\rho',s)\rho^m.
\end{equation}
Na equação \eqref{eq:wake_potential_of_rho} a solução singular na origem, $W_m \propto \rho^{-m}$ foi descartada.

Também é possível encontrar a dependência de $W_m$ em função de $\rho'$, \todo{Achar essa derivação e incluir aqui} ver \cite{Bane_PAC1983}, que é
\begin{equation}
	A_m(\rho',s) = F_m(s)\rho'^m.
\end{equation}

Usando a equação \eqref{eq:wake_function_definition} podemos calcular os \engw{wake functions}
\begin{equation}
	w_l = \sum w_l^{(m)}, \qquad \vect{w}_t = \sum \vect{w}_t^{(m)}
\end{equation}
onde
\begin{equation}\label{eq:wake_function_cylindrical}\begin{aligned}
w_l^{(m)} &= \rho'^m\rho^mF'_m(s)\cos(m\theta),\\
\vect{w}_t^{(m)} &= m\rho'^m\rho^{m-1} F_m(s)\left[\vect{\hat{r}}\cos(m\theta) -
												  \vect{\hat{\theta}}\sin(m\theta)\right]
\end{aligned}\end{equation}
onde $\vect{\hat{r}}$ e $\vect{\hat{\theta}}$ são vetores unitários nas direções radial e azimutal no sistema cilíndrico de coordenadas e $F'_m$ é a derivada de $F_m$ em relação à $s$. Lembre que nessas equações nós assumimos que a partícula fonte está em $\theta = 0$.

As equações \eqref{eq:wake_function_cylindrical} são válidas para valores arbitrários de $\rho$ e $\rho'$. Próximo ao eixo, onde os deslocamentos são pequenos, os termos de ordem mais alta, com valores grandes de $m$, também ficam pequenos. Nesse caso podemos manter apenas os termos não nulos de mais baixa ordem,
\begin{equation}\label{eq:wake_function_expanded}\begin{aligned}
	w_l & \equiv w_l^{(0)} = F'_0(s), \\
    \vect{w}_t & \equiv \vect{w}_t^{(1)} = F_1(s)\rho'\left(\vect{\hat{r}}\cos(\theta) - \vect{\hat{\theta}}\sin(\theta)\right) = \vect{\rho'} F_1(s)
\end{aligned}\end{equation}
onde a última igualdade da última equação é justificada porque $\vect{\hat{r}}\cos(\theta) - \vect{\hat{\theta}}\sin(\theta) = \vect{\hat{x}}$, que é a direção em que está a partícula fonte, de acordo com nossa definição inicial.

Geralmente o \engw{wake} transversal definido na equação \eqref{eq:wake_function_expanded} é redefinido como
\begin{equation}
	w_t(s) \equiv \frac{|\vect{w}_t|}{|\vect{\rho'}|} = F_1(s)
\end{equation}
e adquire a unidade \si{\volt\per\coulomb\per\meter}. De acordo com essa definição, um \engw{wake} transversal positivo significa um impulso na direção do deslocamento da partícula fonte (caso ambas cargas tenham o mesmo sinal).


%%%%%%%%%%%%%%%%%%%%%%%%%%%%%%%%%%%%%%%%%%%%%%%%%%%%%%%%%%%%%%%%%%%%%%%%%%%%%%%%%%%%%%%%%%%%%%%%%%%%%%%%%
%%%%%%%%%%%%%%%%%%%%%%%%%%%%%%%%%%%%%%%%%%%%%%%%%%%%%%%%%%%%%%%%%%%%%%%%%%%%%%%%%%%%%%%%%%%%%%%%%%%%%%%%%
\section{Impedâncias}

Nesta seção vamos definir o conceito de impedância e derivar algumas de suas propriedades.

%%%%%%%%%%%%%%%%%%%%%%%%%%%%%%%%%%%%%%%%%%%%%%%%%%%%%%%%%%%%%%%%%%%%%%%%%%%%%%%%%%%%%%%%%%%%%%%%%%%%%%%%%
\subsection{Definição de Impedância}
O conhecimento dos \engw{wake functions} longitudinal e transversal nos dá uma informação completa, dentro da aproximação de feixe rígido, sobre a interação eletromagnética do feixe com o seu ambiente. Contudo, em muitos casos, especialmente no estudo de instabilidades do feixe, é mais conveniente usar a Transformada de Fourier dos \engw{wake functions} ou as impedâncias. Também, geralmente é mais fácil calcula a impedância para uma dada geometria da câmara de vácuo ao invés da \engw{wake function}.

Por razões históricas a impedâncias longitudinal, $Z_l$, e transversal, $Z_t$, são definidas como a Transformada de Fourier dos \engw{wakes} com fatores multiplicativos diferentes,
\begin{equation}\label{eq:impedances_definition}\begin{aligned}
Z_l(\omega) &= \frac1c \defint{s}{w_l(s)e^{i\omega s/c}}{0}{\infty},\\
Z_t(\omega) &= -\frac{i}{c} \defint{s}{w_t(s)e^{i\omega s/c}}{0}{\infty}.
\end{aligned}\end{equation}
Note que a integração na equação \eqref{eq:impedances_definition} pode ser estendida para a região de valores negativos de $s$, porque $w_l$ e $w_t$ são nulos naquela região. Além disso, a impedância pode ser definida para valores complexos de $\omega$, desde que $\Im(\omega) > 0 $ para que a integral convirja. Dessa forma, a impedância é uma função analítica no plano superior da variável complexa $\omega$.

Um aspecto importante a respeito da definição de impedância é que há divergências em sua definição na literatura. Por exemplo, as referências \cite{Zotter1993} e \cite{Wilson1987} definem a impedância longitudinal como o complexo conjugado da equação \eqref{eq:impedances_definition}. Nesse trabalho estamos seguindo a definição das referências \cite{CHao1993,Stupakov2000a,Heifets1991}.

{\huge Até aqui}

%%%%%%%%%%%%%%%%%%%%%%%%%%%%%%%%%%%%%%%%%%%%%%%%%%%%%%%%%%%%%%%%%%%%%%%%%%%%%%%%%%%%%%%%%%%%%%%%%%%%%%%%%
%%%%%%%%%%%%%%%%%%%%%%%%%%%%%%%%%%%%%%%%%%%%%%%%%%%%%%%%%%%%%%%%%%%%%%%%%%%%%%%%%%%%%%%%%%%%%%%%%%%%%%%%%
\section{Cálculo dos wake-potential a partir do ECHOzR}

De acordo com a referência, temos que:

\begin{align}
W_{||}(x_0,y_0,x,y,s) &= \frac1w\sum_{m=1}^\infty W_m(y_0,y,s)\sin(k_{x,m}x_0)\sin(k_{x,m}x), \\
W_y(x_0,y_0,x,y,s) &= \frac1w\sum_{m=1}^\infty k_{x,m}W_{y,m}(y_0,y,s)\sin(k_{x,m}x_0)\sin(k_{x,m}x), \\
W_x(x_0,y_0,x,y,s) &= \frac1w\sum_{m=1}^\infty k_{x,m}W_{x,m}(y_0,y,s)\sin(k_{x,m}x_0)\cos(k_{x,m}x),
\end{align}
where $w$ is the half-width of the structure, $0<x<2w$ is the horizontal position of the trailing particle, $0<x_0<2w$ is the horizontal position of the source particle, $y$ is vertical position of the trailing particle, $y_0$ is vertical position of the source particle, $s=z-ct$ is the position of the trailing particle relative to the source particle and
\begin{equation}
k_{x,m} = \frac{\pi}{2w}m
\end{equation}
The other terms are given by
\begin{align}
W_m(y_0,y,s)     &= W^{cc}_m(s)\cosh(k_{x,m}y_0)\cosh(k_{x,m}y) + W^{ss}_m(s)\sinh(k_{x,m}y_0)\sinh(k_{x,m}y), \\
W_{y,m}(y_0,y,s) &= S^{cc}_m(s)\cosh(k_{x,m}y_0)\sinh(k_{x,m}y) + S^{ss}_m(s)\sinh(k_{x,m}y_0)\cosh(k_{x,m}y), \\
W_{x,m}(y_0,y,s) &= S^{cc}_m(s)\cosh(k_{x,m}y_0)\cosh(k_{x,m}y) + S^{ss}_m(s)\sinh(k_{x,m}y_0)\sinh(k_{x,m}y),
\end{align}
where
\begin{equation}
S^{cc}_m = \int_{-\infty}^s W^{cc}_m(s')\mathrm{d}s', \qquad S^{ss}_m = \int_{-\infty}^s W^{ss}_m(s')\mathrm{d}s'
\end{equation}
and the quantities $W^{cc}_m(s')$ and $W^{ss}_m(s')$ are related to the output of the softwares ECHOzR and ECHO2D by:
\begin{align}
W^{cc}_m(s') = \frac{W_m^M(y_0,y,s)}{\cosh(k_{x,m}y_0)\cosh(k_{x,m}y)} \\
W^{ss}_m(s') = \frac{W_m^E(y_0,y,s)}{\sinh(k_{x,m}y_0)\sinh(k_{x,m}y)}
\end{align}
where $W_m^M(y_0,y,s)$ and $W_m^M(y_0,y,s)$ are the results of the simulation with magnetic and electric boundary conditions, respectively.


Our objective is to obtain the formulas for the monopole longitudinal, dipole and quarupolar transverse wake functions at the point $\vec{v} = (x_0=w,y_0=0,x=w,y=0)$ from these results. The monopolar wakes are readly obtained:
\begin{align}
W_m(0,0,s) = W^{cc}_m(s) &\Rightarrow W_{||}(\vec{v}) = \frac1w\sum^\infty_{m=1,\mathrm{odd}} W^{cc}_m(s), \\
W_{y,m}(0,0,s) = 0 &\Rightarrow W_y(\vec{v}) = 0, \\
W_{x,m}(0,0,s) = S^{cc}_m(s) &\Rightarrow W_x(\vec{v}) = \frac1w\sum^\infty_{m=1} S^{cc}_m(s)\frac{\sin(m\pi)}{2} = 0
\end{align}
To calculate the dipolar and quadrupolar wakes, we need to take the first derivative of the transverse wakes at the point of interest, $\vec{v}$. It is easy to see that the first derivatives of the longitudinal wake are zero. For the transverse:

\begin{align}
W_{y,d}(s) &= \left.\frac{\mathrm{d}}{\mathrm{d}y_0}W_y\right|_{\vec{v}} = \frac1w\sum^\infty_{m=1,\mathrm{odd}} k_{x,m}^2 S^{ss}_m(s) = \frac1w\int_{-\infty}^s\sum^\infty_{m=1,\mathrm{odd}} k_{x,m}^2 W^{ss}_m(s') \mathrm{d}s',\\
W_{y,q}(s) &= \left.\frac{\mathrm{d}}{\mathrm{d}y}W_y\right|_{\vec{v}} = \frac1w\sum^\infty_{m=1,\mathrm{odd}} k_{x,m}^2 S^{cc}_m(s) = \frac1w\int_{-\infty}^s\sum^\infty_{m=1,\mathrm{odd}} k_{x,m}^2 W^{cc}_m(s') \mathrm{d}s', \\
W_{x,d}(s) &= \left.\frac{\mathrm{d}}{\mathrm{d}x_0}W_x\right|_{\vec{v}} = \frac1w\sum^\infty_{m=1,\mathrm{even}} k_{x,m}^2 S^{cc}_m(s) = \frac1w\int_{-\infty}^s\sum^\infty_{m=1,\mathrm{even}}k_{x,m}^2 W^{cc}_m(s') \mathrm{d}s', \\
W_{x,q}(s) &= \left.\frac{\mathrm{d}}{\mathrm{d}x}W_x\right|_{\vec{v}} = -\frac1w\sum^\infty_{m=1,\mathrm{odd}} k_{x,m}^2 S^{cc}_m(s) =-\frac1w\int_{-\infty}^s\sum^\infty_{m=1,\mathrm{odd}} k_{x,m}^2 W^{cc}_m(s') \mathrm{d}s'
\end{align}
and all the skew terms are zero.

%%%%%%%%%%%%%%%%%%%%%%%%%%%%%%%%%%%%%%%%%%%%%%%%%%%%%%%%%%%%%%%%%%%%%%%%%%%%%%%%%%%%%%%%%%%%%%%%%%%%%%%%%
%%%%%%%%%%%%%%%%%%%%%%%%%%%%%%%%%%%%%%%%%%%%%%%%%%%%%%%%%%%%%%%%%%%%%%%%%%%%%%%%%%%%%%%%%%%%%%%%%%%%%%%%%
\section{Wake Calculation From GdfiDL Simulations}

First lets consider the general expansion of the Wake potential of a bunch in the transverse coordinates of the centroid of the source $(x_s,y_s)$ and the integration path $(x,y)$ up to third order:

\begin{align}
W(\vect{x},s) &= W_0(s) + \sum_{i=1}^4 M_i(s) x_i + \frac12\sum_{i,j=1}^4 D_{ij}(s) x_i x_j + \frac13\sum_{i,j,k=1}^4 Q_{ijk}(s) x_i x_j x_k
\end{align}
where we considered $\vect{x} = (x_1,x_2,x_3,x_4) = (x_s,y_s,x,y)$ and because of the commutative property of multiplication, we can set $D_{ij} = D_{ji}$ and $Q_{ijk}=Q_{kij}=Q_{jki}=Q_{ikj}=Q_{jik}=Q_{kji}$ without loss of generality. With these considerations, number of independent components of $D$ is 10 and $Q$ is 20. Besides that, the fact that $W$ is an harmonic function of the transverse coordinates of the integration path, imposes that
\begin{align}
D_{33} &= - D_{44} \\
Q_{33i} &= - Q_{44i}, \quad \text{with} \quad i=1,2,3,4
\end{align}
which leaves only nine independent components of $D$ and sixteen of $Q$.

Thus, the wake forces become:
\begin{align}
F_L(\vec{x},s) = W'(\vec{x},s) &= W_0' + \sum_{i=1}^4 M_i' x_i + \frac12\sum_{i,j=1}^4 D_{ij}' x_i x_j + \frac13\sum_{i,j,k=1}^4 Q_{ijk}' x_i x_j x_k \\
F_x(\vec{x},s) = \derpar{W(\vect{x},s)}{x} &= M_3 + \sum_{i=1}^4 D_{3i} x_i + \sum_{i,j=1}^4 Q_{3ij} x_i x_j \\
F_y(\vec{x},s) = \derpar{W(\vect{x},s)}{y} &= M_4 + \sum_{i=1}^4 D_{4i} x_i + \sum_{i,j=1}^4 Q_{4ij} x_i x_j
\end{align}

Generally we are interested in obtaining the linear terms as function of the transverse coordinates correct up to second order. It means we want to isolate the linear from the quadract terms in the simulations. Thus, lets keep only the second order terms in the wake forces above.
\begin{align}
F_L(\vect{x},s) &= W_0' + \vect{M'}^T \cdot \vect{x} + \frac12\vect{x}^T \cdot \tensor{D'}  \cdot \vect{x} \\
F_x(\vect{x},s) &= M_x + \vect{D_x}^T \cdot \vect{x} +        \vect{x}^T \cdot \tensor{Q_x} \cdot \vect{x} \\
F_y(\vect{x},s) &= M_y + \vect{D_y}^T \cdot \vect{x} +        \vect{x}^T \cdot \tensor{Q_y} \cdot \vect{x}
\end{align}
where
\begin{align}
\vect{M} &= \begin{pmatrix*}[r] M_{1}\\ M_{2}\\ M_{3}\\ M_{4}\end{pmatrix*} \\
\tensor{D}  &= \begin{pmatrix*}[r] D_{11} & D_{12} & D_{13} & D_{14} \\
                                   D_{21} & D_{22} & D_{23} & D_{24} \\
                                   D_{31} & D_{32} & D_{33} & D_{34} \\
                                   D_{41} & D_{42} & D_{43} & D_{44}
               \end{pmatrix*} =
               \begin{pmatrix*}[r] D_{11} & D_{12} & D_{13} & D_{14} \\
                                   D_{12} & D_{22} & D_{23} & D_{24} \\
                                   D_{13} & D_{23} & D_{33} & D_{34} \\
                                   D_{14} & D_{24} & D_{34} & -D_{33}
               \end{pmatrix*}\\
\vect{D_x} &= \tensor{D}\cdot \vect{\hat{x}} \\
\vect{D_y} &= \tensor{D}\cdot \vect{\hat{y}} \\
\tensor{Q_x}&= \begin{pmatrix*}[r] Q_{311} & Q_{312} & Q_{313} & Q_{314} \\
                                   Q_{321} & Q_{322} & Q_{323} & Q_{324} \\
                                   Q_{331} & Q_{332} & Q_{333} & Q_{334} \\
                                   Q_{341} & Q_{342} & Q_{343} & Q_{344}
               \end{pmatrix*} =
               \begin{pmatrix*}[r] Q_{113} & Q_{123} & Q_{133} & Q_{134} \\
                                   Q_{123} & Q_{223} & Q_{233} & Q_{234} \\
                                   Q_{133} & Q_{233} & Q_{333} & Q_{334} \\
                                   Q_{134} & Q_{234} & Q_{334} & -Q_{333}
               \end{pmatrix*}\\
\tensor{Q_y}&= \begin{pmatrix*}[r] Q_{411} & Q_{412} & Q_{413} & Q_{414} \\
                                   Q_{421} & Q_{422} & Q_{423} & Q_{424} \\
                                   Q_{431} & Q_{432} & Q_{433} & Q_{434} \\
                                   Q_{441} & Q_{442} & Q_{443} & Q_{444}
               \end{pmatrix*} =
               \begin{pmatrix*}[r] Q_{114} & Q_{124} & Q_{134} & Q_{133} \\
                                   Q_{124} & Q_{224} & Q_{234} & Q_{233} \\
                                   Q_{134} & Q_{234} & Q_{334} & -Q_{333} \\
                                   Q_{133} & Q_{233} &-Q_{333} & -Q_{334}
               \end{pmatrix*}\\
\end{align}



Due to their importance, some components of the $D$ tensor have a name: the $D_{31}$ and $D_{42}$ are called dipolar wakes; and $D_{33}$ and $D_{44}$ are the quadrupolar wakes. While the first generates coherent tune-shifts and instabilities, the later is the responsible for incoherent tune-shifts of the beam. The other terms are the skew components, which are zero for most practical cases, as we will see below.

\subsection{Symmetry analysis}

When the geometry has symmetry the number of independent compononts in $M$, $D$ and $Q$ are reduced even more.
When the symmetry occurs in one plane, the number of independent components is two, five and eight, respectivelly.Below we list examples for the most practical cases:

\begin{itemize}
\item symmetry in the $yz$ plane, or $x=0$.
\begin{align}
W(x_s,y_s,x,y,s) = & W(-x_s,y_s,-x,y,s) \Rightarrow \\
M_2= & M_4=0\\
D_{12}=D_{14}= & D_{23}=D_{34}=0\\
Q_{111}=Q_{122}= & Q_{113}=Q_{223}=0\\
Q_{133}=Q_{333}= & Q_{124}=Q_{234}=0
\end{align}
\begin{align}
\vect{M}&= \begin{pmatrix} M_{1}\\ 0\\ M_{3}\\ 0\end{pmatrix} &
\tensor{D} &= \begin{pmatrix} D_{11} &    0   & D_{13} &    0    \\
                                  0   & D_{22} &    0   &  D_{24} \\
                               D_{13} &    0   & D_{33} &    0    \\
                                  0   & D_{24} &    0   & -D_{33}
               \end{pmatrix}\\
\tensor{Q_x}&=\begin{pmatrix}    0    & Q_{123} &    0    & Q_{134} \\
                               Q_{123} &    0    & Q_{233} &    0    \\
                                  0    & Q_{233} &    0    & Q_{334} \\
                               Q_{134} &    0    & Q_{334} &    0
               \end{pmatrix} &
\tensor{Q_y}&=\begin{pmatrix} Q_{114} &    0    & Q_{134} &    0    \\
                                  0    & Q_{224} &    0    & Q_{233} \\
                               Q_{134} &    0    & Q_{334} &    0    \\
                                  0    & Q_{233} &    0    & -Q_{334}
               \end{pmatrix}
\end{align}

\item symmetry in the $xz$ plane, or $y=0$.
\begin{align}
W(x_s,y_s,x,y,s) = & W(x_s,-y_s,x,-y,s) \Rightarrow \\
M_1= & M_3=0\\
D_{12}=D_{14}= & D_{23}=D_{34}=0\\
Q_{112}=Q_{222}= & Q_{123}=Q_{233}=0\\
Q_{114}=Q_{224}= & Q_{134}=Q_{334}=0
\end{align}
\begin{align}
\vect{M}    &= \begin{pmatrix}   0  \\ M_{2}\\  0  \\ M_{4}\end{pmatrix} &
\tensor{D}  &= \begin{pmatrix} D_{11} &    0   & D_{13} &    0   \\
                                  0   & D_{22} &    0   & D_{24} \\
                               D_{13} &    0   & D_{33} &    0   \\
                                  0   & D_{24} &    0   & -D_{33}
               \end{pmatrix}\\
\tensor{Q_x}&= \begin{pmatrix} Q_{113} &    0    & Q_{133} &    0    \\
                                  0    & Q_{223} &    0    & Q_{234} \\
                               Q_{133} &    0    & Q_{333} &    0    \\
                                  0    & Q_{234} &    0    & -Q_{333}
               \end{pmatrix} &
\tensor{Q_y}&= \begin{pmatrix}    0    & Q_{124} &    0    & Q_{133} \\
                               Q_{124} &    0    & Q_{234} &    0    \\
                                  0    & Q_{234} &    0    & -Q_{333} \\
                               Q_{133} &    0    &-Q_{333} &    0
               \end{pmatrix}
\end{align}

\item symmetry in the plane $y=x$.
\begin{align}
W(x_s,y_s,x,y,s) =& W(y_s,x_s,y,x,s) \Rightarrow \\
M_1=M_2, \quad & M_3=M_4\\
D_{11}=D_{22}, \quad D_{13}=D_{24}, \quad & D_{14}=D_{23}, \quad D_{33}=0 \\
Q_{111}=Q_{222}, \quad Q_{112}= Q_{122}, \quad & Q_{113}=Q_{224}, \quad Q_{114}= Q_{223} \\
Q_{123}=Q_{124}, \quad Q_{133}=-Q_{233}, \quad & Q_{134}=Q_{234}, \quad Q_{333}=-Q_{334}
\end{align}
\begin{align}
\vect{M}  &= \begin{pmatrix} M_{1}\\ M_{1}\\ M_{3}\\ M_{3}\end{pmatrix} &
\tensor{D}   &=\begin{pmatrix} D_{11} & D_{12} & D_{13} & D_{14} \\
                               D_{12} & D_{11} & D_{14} & D_{13} \\
                               D_{13} & D_{14} &    0   & D_{34} \\
                               D_{14} & D_{13} & D_{34} &    0
               \end{pmatrix}\\
\tensor{Q_x}&= \begin{pmatrix} Q_{113} & Q_{123} & Q_{133} & Q_{134} \\
                               Q_{123} & Q_{114} &-Q_{133} & Q_{134} \\
                               Q_{133} &-Q_{133} & Q_{333} &-Q_{333} \\
                               Q_{134} & Q_{134} &-Q_{333} &-Q_{333}
               \end{pmatrix} &
\tensor{Q_y} &=\begin{pmatrix} Q_{114} & Q_{123} & Q_{134} & Q_{133} \\
                               Q_{123} & Q_{113} & Q_{134} &-Q_{133} \\
                               Q_{134} & Q_{134} &-Q_{333} &-Q_{333} \\
                               Q_{133} &-Q_{133} &-Q_{333} & Q_{333}
               \end{pmatrix}
\end{align}

\item symmetry in the plane $y=-x$.
\begin{align}
W(x_s,y_s,x,y,s) =& W(-y_s,-x_s,-y,-x,s) \Rightarrow \\
M_1=-M_2, \quad & M_3=-M_4\\
D_{11}=D_{22}, \quad D_{13}=D_{24}, \quad & D_{14}=D_{23}, \quad D_{33}=0 \\
Q_{111}=-Q_{222}, \quad Q_{112}=-Q_{122}, \quad & Q_{113}=-Q_{224}, \quad Q_{114}=-Q_{223} \\
Q_{123}=-Q_{124}, \quad Q_{133}= Q_{233}, \quad & Q_{134}=-Q_{234}, \quad Q_{333}= Q_{334}
\end{align}
\begin{align}
\vect{M}  &= \begin{pmatrix} M_{1}\\ -M_{1}\\ M_{3}\\ -M_{3}\end{pmatrix} &
\tensor{D}   &=\begin{pmatrix} D_{11} & D_{12} & D_{13} & D_{14} \\
                               D_{12} & D_{11} & D_{14} & D_{13} \\
                               D_{13} & D_{14} &    0   & D_{34} \\
                               D_{14} & D_{13} & D_{34} &    0
               \end{pmatrix}\\
\tensor{Q_x}&= \begin{pmatrix} Q_{113} & Q_{123} & Q_{133} & Q_{134} \\
                               Q_{123} &-Q_{114} & Q_{133} &-Q_{134} \\
                               Q_{133} & Q_{133} & Q_{333} & Q_{333} \\
                               Q_{134} &-Q_{134} & Q_{333} &-Q_{333}
               \end{pmatrix} &
\tensor{Q_y} &=\begin{pmatrix} Q_{114} &-Q_{123} & Q_{134} & Q_{133} \\
                              -Q_{123} &-Q_{113} &-Q_{134} & Q_{133} \\
                               Q_{134} &-Q_{134} & Q_{333} &-Q_{333} \\
                               Q_{133} & Q_{133} &-Q_{333} &-Q_{333}
               \end{pmatrix}
\end{align}
\end{itemize}

Below we combine some of the above mentioned symmetries which are very commom in the simulations performed for accelerators elements:

\begin{itemize}
\item x=0 and y=0.
\begin{align}
\vect{M}= \vect{0} & &
\tensor{D} = \begin{pmatrix} D_{11} &    0   & D_{13} &    0    \\
                                  0   & D_{22} &    0   &  D_{24} \\
                               D_{13} &    0   & D_{33} &    0    \\
                                  0   & D_{24} &    0   & -D_{33}
               \end{pmatrix} & &
\tensor{Q_x}=\tensor{0} & &
\tensor{Q_y}=\tensor{0}
\end{align}

\item $y=-x$ and $y=x$.
\begin{align}
\vect{M}= \vect{0} & &
\tensor{D}   &=\begin{pmatrix} D_{11} & D_{12} & D_{13} & D_{14} \\
                               D_{12} & D_{11} & D_{14} & D_{13} \\
                               D_{13} & D_{14} &    0   & D_{34} \\
                               D_{14} & D_{13} & D_{34} &    0
               \end{pmatrix}& &
\tensor{Q_x}=\tensor{0} & &
\tensor{Q_y}=\tensor{0}
\end{align}

\item $x=0$, $y=0$, $y=-x$ and $y=x$.
\begin{align}
\vect{M}= \vect{0} & &
\tensor{D}   &=\begin{pmatrix} D_{11} &   0    & D_{13} &   0    \\
                                 0    & D_{11} &    0   & D_{13} \\
                               D_{13} &   0    &    0   &   0    \\
                                 0    & D_{13} &    0   &   0
               \end{pmatrix} & &
\tensor{Q_x}=\tensor{0} & &
\tensor{Q_y}=\tensor{0}
\end{align}
\end{itemize}
where we notice that when there is at least 2 planes of symmetry, all the odd order terms are zero.

\subsection{GDFIDL Simulations}

One simulation in GDFIDL consists of passing a linear gaussian bunch with the velocity of light in the longitudinal direction and with a specific transverse position, say $(x_s,y_s)=(d_x,d_y)$ through the simulated structure solving Maxwell equations in time. While doing this, the code saves in memory the wake potential $W(d_x,d_y,x,y,s)$ for all transverse positions $(x,y)$ of integration and all $s$. This procedure is very time consuming and we gerally try to perform the mininum amount of simulations possible to get the results we need.

Lets suppose a simulation was performed with the position of the source in $(x_s,y_s)=(d,0)$. If we calculate the Horizontal Wake potential in a path where $y=0$ then, up to second order in the transverse coordinates we can write:

\begin{align}
F_L(d,0,x,0) &= W_0  +  M_1d  +  M_3x + D_{11}d^2 + D_{13}dx + D_{33}x^2 \\
F_x(d,0,x,0) &= M_x + D_{x3} x + D_{x1} d + Q_{x33} x^2 + Q_{x11} d^2 + Q_{x13} x d\\
F_y(d,0,x,0) &= M_y + D_{y3} x + D_{y1} d + Q_{y33} x^2 + Q_{y11} d^2 + Q_{y13} x d
\end{align}

Now, integrating the total wake in 4 different $x$ points we get:
\begin{align}
	F_1 = F_x(d,x_1) &= M_x  +  D_{x3} x_1  +  D_{x1} d  +  Q_{x33} x_1^2  +  Q_{x11} d^2  +  Q_{x13} x_1 d \\
    F_2 = F_x(d,x_2) &= M_x  +  D_{x3} x_2  +  D_{x1} d  +  Q_{x33} x_2^2  +  Q_{x11} d^2  +  Q_{x13} x_2 d \\
    F_3 = F_x(d,x_3) &= M_x  +  D_{x3} x_3  +  D_{x1} d  +  Q_{x33} x_3^2  +  Q_{x11} d^2  +  Q_{x13} x_3 d \\
    F_4 = F_x(d,x_4) &= M_x  +  D_{x3} x_4  +  D_{x1} d  +  Q_{x33} x_4^2  +  Q_{x11} d^2  +  Q_{x13} x_4 d
\end{align}

To extract the component $D_x$ from the total wake we do:
\begin{align}
\Delta_1 = \frac{F_1}{x_1} - \frac{F_2}{x_2} &= \left(M_x  +  D_{x1}d  +  Q_{x11}d^2\right)\left(\frac{1}{x_1}-\frac{1}{x_2}\right) + Q_{x33}(x_1-x_2) \\
\Delta_2 = \frac{F_3}{x_3} - \frac{F_4}{x_4} &= \left(M_x  +  D_{x1}d  +  Q_{x11}d^2\right)\left(\frac{1}{x_3}-\frac{1}{x_4}\right) + Q_{x33}(x_3-x_4)
\end{align}
then, remembering $(1/a-1/b)/(a-b) = -1/ab$ we get:
\begin{align}
	\frac{\Delta_1}{x_1-x_2} - \frac{\Delta_2}{x_3-x_4} &= \left(M_x  +  D_{x1} d  +  Q_{x11} d^2\right)\left(\frac{1}{x_3x_4} - \frac{1}{x_1x_2}\right)
\end{align}
\begin{align}
	M_x  +  D_{x1} d +  Q_{x11} d^2 &= \left(\frac{x_1x_2x_3x_4}{x_3x_4 - x_1x_2}\right)\left(\frac{\Delta_1}{x_1-x_2} - \frac{\Delta_2}{x_3-x_4}\right)
\end{align}

Note that in general to isolate the dipolar component $D_x$, it is necessary to perform another two simulations. If another simulation is performed with the source at $(x_s,y_s) = (-d,0)$ it is also possible to isolate the dipolar component.

Now, lets try to extract the quadrupolar component
\begin{align}
\Delta_1 = \frac{F_1 - F_2}{x_1-x_2} &= D_{x3} + Q_{x13}d + Q_{x33}(x_1 + x_2)
\end{align}
Notice that, if we choose $x_2=-x_1$ the contribution from $Q_{x33}$ is canceled and
\begin{align}
	D_{x3} + Q_{x13}d = \Delta_1
\end{align}
where it is clear that is necessary other simulation to extract the quadrupolar wake.

%%%%%%%%%%%%%%%%%%%%%%%%%%%%%%%%%%%%%%%%%%%%%%%%%%%%%%%%%%%%%%%%%%%%%%%%%%%%%%%%%%%%%%%%%%%%%%%%%%%%%%%%%
%%%%%%%%%%%%%%%%%%%%%%%%%%%%%%%%%%%%%%%%%%%%%%%%%%%%%%%%%%%%%%%%%%%%%%%%%%%%%%%%%%%%%%%%%%%%%%%%%%%%%%%%%
\section{Panofski-Wenzel}

O conceito de wake-function e impedância tem seu surgimento motivado pelo teorema de Panofski-Wenzel. Esse teorema é resultado de duas aproximações feitas na análise do problema de duas partículas relativísticas interagindo entre si por meio de campos eletromagnéticos espalhados pela câmara de vácuo do anel de armazenamento. Por sua central importância para o subsequente desenvolvimento desse trabalho, vamos olhar esse teorema com um pouco mais de detalhe.

Primeiro, vamos considerar a imagem apresentada na figura X, ou seja, consideremos uma situação em que temos uma partícula de carga $Q$ atravessando a estrutura ilustrada com velocidade $\vec{v}$ ao longo da câmara de vácuo de um anel de armazenamento. A uma distância longitudinal $z$ dela, uma partícula de prova com carga $q$ também atravessa a estrutura na mesma direção. Estamos interessados em saber qual é a força que essa carga de prova sentirá ao longo de sua passagem por toda a estrutura. Pela equação de Lorentz, temos a todo instante:
\begin{equation}
 \vec{F} = q\left(\vec{E} + \vec{v} \times \vec{B}\right)
\end{equation}
onde $\vec{E}$ e $\vec{B}$ é o campo eletromagnético na posição da partícula que está sentindo a força. Esse campo pode ser decomposto como a soma de vários campos com origem e interpretação física diferentes: O campo externo, gerado pelos ímãs e cavidades de RF que guiam o feixe e o campo de interação, que foi gerado pela partícula fonte, de carga $Q$. O campo de interação ainda pode ser decomposto em duas contribuições distintas: o campo direto, que existiria mesmo sem a presença da câmara de vácuo e o campo indireto, resultado da interação dessa carga com a câmara de vácuo. Para obter a transferência de momento total sentida pela carga de prova ao longo de toda estrutura, temos que integrar a equação acima na trajetória da partícula. Ao realizarmos esse procedimento, logo vemos que a solução auto-consistente desse problema é muito complexa, pois a trajetória da partícula prova depende da força que ela sentiu nos tempos anteriores, assim como da trajetória da carga fonte, devido à dependência dos campos eletromagnéticos que são gerados por ela.

Assim, para conseguir seguir mais adiante na análise desse problema devemos fazer aproximações. A primeira delas é considerar que ambas as partículas são rígidas e possuem velocidades paralelas uma com a outra e paralelas ao eixo de simetria da câmara de vácuo (caso a câmara de vácuo não tenha simetria, considera-se a velocidade das partículas paralelas à direção em que o feixe de elétrons deverá passar). Essa consideração simplifica drasticamente a análise do problema, como veremos logo a seguir.

Antes de seguir com a demonstração do teorema, vamos justificar essa aproximação. Na maioria dos aceleradores de partículas as partículas estão no limite ultra-relativístico, em que sua velocidade é aproximadamente a da luz e varia muito pouco com variação de momento. Isso implica que nesse limite as partículas são bastante rígidas

	%\part{Contribuições}
	\chapter{Impedance Calculations}

The Sirius impedance budget model is based on analytical calculations and numeric simulations with time and frequency solver codes. In the next sections we will describe how we have modeled some of the subsystems' impedance.

\section{Vacuum Chamber Impedance}

We can separate Sirius vacuum chambers into three groups:
\begin{description}
 \item[Straight Sections:] circular cross-section, $\Phi = \,$\SI{24}{\milli\meter};
 \item[Dipoles without radiation exit:] circular cross-section, $\Phi = \,$\SI{24}{\milli\meter};
 \item[Dipoles with radiation exit:] circular cross-section, $\Phi = \,$\SI{24}{\milli\meter} + a \SI{5}{\milli\meter} height ``nose'';
\end{description}

All chambers are made of cooper with thickness of \SI{1}{\milli\meter}. Thus, if we do not take into account curvature effects and the radiation exit, we can model the dipole chambers the same way as the straight section chambers.

Mounet and Métral \cite{mounet_metral2009} solved Maxwell equations analytically for the case of a macro-particle traveling longitudinally through an axisymmetric chamber composed by an arbitrary number of layers of different materials with generic electric permeability and magnetic permissivity. The authors presented general formulas for electromagnetic fields and impedances.

We have implemented on Matlab\textregistered Mounet-Métral formulas for the first order longitudinal, horizontal and vertical impedances. To validate the implementation, we compared with Chao's book calculation, as shown in figure \ref{fig:compare_mounet_chao}. In order for the results to be equivalent, we had to change the sign of the impedances calculated through Mounet-Métral formulas. This difference is due to the definition of the longitudinal coordinate: in Chao's book the distance $z$ between the source particle and the point where the impedance is being calculated is positive and in Mounet-Métral's report it is negative.

% \begin{figure}[!t]
%  \centering
%  \subfigure[\label{fig:compare_mounet_chao1}]{\includegraphics[width=0.45\textwidth]{figures/compare_mounet_chao1.png}}
%  \subfigure[\label{fig:compare_mounet_chao2}]{\includegraphics[width=0.45\textwidth]{figures/compare_mounet_chao2.png}}
%  \caption{Comparison between the implemented formulas from \cite{mounet_metral2009} and Chao's book \cite{Chao1993} calculation for an aluminum chamber, $\sigma =$ \SI{33.4}{\mega\siemens\per\meter}, with a radius of 5 cm.}
%  \label{fig:compare_mounet_chao}
% \end{figure}

The advantage of using Mounet-Métral formalism is that we can estimate the effect of NEG Coating on the impedance. Sirius will have about 90\% of the whole ring coated with NEG to improve the vacuum, i.e., all chambers described above will be coated. Figure \ref{fig:rw_with_neg} shows the comparison of the chamber wall impedance with and without the NEG coating and table \ref{tab:rw_with_neg} presents the main parameters values used for this calculation. 

\begin{table}[!b]
 \centering
 \caption{Main parameters used on the chamber wall impedance.}
 \label{tab:rw_with_neg}
 \begin{tabular}{lcl}\hline
  Chamber radius      & 12               & \si{\milli\meter} \\\hline
  Copper conductivity\cite{matwebsite} & 59 & \si{\mega\siemens\per\meter} \\\hline
  NEG thickness       & 1                & \si{\micro\meter}   \\\hline
  NEG conductivity    & 4.0              & \si{\mega\siemens\per\meter} \\\hline
 Length               & 480              & \si{\meter}   \\\hline
 \end{tabular}
\end{table}

We notice that the NEG coating has a similar effect on both planes, it begins to affect the imaginary impedance at lower frequencies than it does for the real one. These results are compatible with a previous study made by Nagaoka \cite{nagaoka2004}, and can be explained if we think that the wall impedance of a single thick metal layer is purely reactive in the limit of low frequencies. Also, we can infer that for very high frequencies the impedance with NEG coating tends to the impedance of a chamber with a infinite layer of NEG, which is very reasonable.

% \begin{figure}[!t]
%  \centering
%  \includegraphics[width=\textwidth]{figures/rw_with_neg.png}
%  \caption{NEG coating effect on transverse and longitudinal impedances. The ``Thick NEG'' curve is the unrealistic impedance of an infinitely thick layer of NEG, calculated to show how that the beam ``does not see'' the copper for high frequencies.}
%  \label{fig:rw_with_neg}
% \end{figure}

There is an uncertainty about NEG's conductivity which is presented in Table \ref{tab:neg_salad}. With this in mind, we calculated the impedance for each value in this table. The results are shown in figure \ref{fig:neg_salad}. We notice the value used to compose our model is lower than the others and gives a lower imaginary and higher real impedance.

\begin{table}[!t]
 \centering
 \caption{Several values for NEG conductivity}
 \label{tab:neg_salad}
\begin{tabular}{ccc}
Author                 & $\sigma$ [\si{\mega\siemens\per\meter}] & Explanation \\\hline
Nagaoka \cite{nagaoka2004}  & 4.0    & \begin{minipage}{0.6\textwidth}
                                        \vspace{1mm}
                                        Used this value to estimate the effect of NEG on Soleil wall impedance. Also, this is the conductivity of Vanadium, main component of the alloy.
                                        \vspace{1mm}
                                       \end{minipage} \\\hline
Nagaoka \cite{nagaoka2004}  & 0.0625 & \begin{minipage}{0.6\textwidth}
                                        \vspace{1mm}
                                        Said E. Plouviez measured this value, but I did not find the indicated reference.
                                        \vspace{1mm}
                                       \end{minipage} \\\hline
Métral \cite{metral_talk2011}& 0.04  & \begin{minipage}{0.6\textwidth}
                                        \vspace{1mm}
                                        Made the following reference: David Seebacher, F. Caspers, NEG properties in the microwave range, SPSU Meeting, 17th February, CERN. I couldn't find the file either.
                                        \vspace{1mm}
                                       \end{minipage} \\\hline  
Kersevan \cite{kersevan2002} & 0.2   & \begin{minipage}{0.6\textwidth}
                                        \vspace{1mm}
                                        Said Plouviez measured this value at \SI{14}{\giga\hertz}, but haven't given any reference.
                                        \vspace{1mm}
                                       \end{minipage} \\\hline                                                    
\end{tabular}
\end{table}


% \begin{figure}[!t]
%  \centering
%  \includegraphics[width=\textwidth]{figures/neg_salad.png}
%  \caption{Wall impedance for several values of NEG conductivity. The coating thickness is \SI{1}{\micro\meter} in all calculations.}
%  \label{fig:neg_salad}
% \end{figure}


\section{Small Gap Undulators}

By small gap undulators we mean the out of vacuum insertion devices that, on Sirius, will have a thick elliptical cooper vacuum chamber. We have modeled this impedance as a flat chamber, using the expression of the round chamber multiplied by Yokoya factors \cite{yokoya1993, gluckstern1993}. The relevant parameters of this modeling are presented in table \ref{tab:small_gap_undulators} and the impedance is plotted in figure \ref{fig:small_gap_undulators}.

\begin{table}[!t]
 \centering
 \caption{Main parameters used on the small gap undulators model.}
 \label{tab:small_gap_undulators}
 \begin{tabular}{lcl}\hline
  Chamber full gap      & 10               & \si{\milli\meter} \\\hline
  Copper conductivity\cite{matwebsite} & 59& \si{\mega\siemens\per\meter} \\\hline
  Length                & 3                & \si{\meter}   \\\hline
 \end{tabular}
\end{table}

% \begin{figure}[!t]
%  \centering
%  \includegraphics[width=\textwidth]{figures/small_gap_undulator.png}
%  \caption{Small gap undulator impedance.}
%  \label{fig:small_gap_undulators}
% \end{figure}


\section{In-vacuum Undulators}\label{sec:in-vacuum_undulators}

The in-vacuum undulators for Sirius will be made of NdFeB covered by a thin sheet of copper to diminish heating of the magnets and other wake fields issues, such as instabilities.

A good model for this element's impedance would be a flat multi-layer geometry. Even though Mounet and Métral \cite{mounet_metral2010} have developed general theory for this geometry we still have not successfully implemented their formulas.

The same authors, in another work \cite{mounet_metralhb2010} have demonstrated that application of Yokoya factors \cite{yokoya1993} are not valid for all frequencies but hold very well in their example for the range from \SI{1}{\mega\hertz} to \SI{1}{\tera\hertz}. 

Yokoya's theory was developed under the assumption of metallic chamber walls of constant electromagnetic properties, then, in our case, it is reasonable to expect that for the frequency range where the beam sees only the copper sheet we could use Mounet and Métral formulas for the round chamber multiplied by Yokoya's factors. Figure \ref{fig:compare_sheet_thick_flat} shows the impedance for several copper sheet thicknesses. We notice that down to \SI{100}{\mega\hertz} all curves agree with the flat chamber result and then begin to deviate in increasing order of thickness.

% \begin{figure}[!t]
%  \centering
%  \includegraphics[width=\textwidth]{figures/compare_sheet_thick_flat.png}
%  \caption{Model of in-vacuum undulator impedance using Mounet-Métral multi-layer round chamber formulas for several copper sheet thickness plus impedance calculated with thick copper layer flat chamber formula.}
%  \label{fig:compare_sheet_thick_flat}
% \end{figure}

The choice of the sheet thickness is mainly based on the heat deposition on the magnets. We want to minimize the total power loss by the beam and assure this power goes to the copper sheet and not to the magnets. In figure \ref{fig:heat_load_invac_magnets} we can notice that even for a \SI{50}{\micro\meter} thick sheet no heat goes to the magnets, at least not by radiation.

% \begin{figure}[!b]
%  \centering
%  \includegraphics[width=0.5\textwidth]{figures/heat_load_invac_magnets.png}
%  \caption{Beam power loss in function of copper sheet thickness for several bunch lengths.}
%  \label{fig:heat_load_invac_magnets}
% \end{figure}

Table \ref{tab:invac_undulators} shows the main material and geometry parameters used in the undulators modeling and figure \ref{fig:in_vacuum_undulators} shows the two types of in-vacuum undulators that will be present in the ring. We varied some orders of magnitude the values of the NdFeB parameters but it did not influenced the impedance significantly.

\begin{table}[!t]
 \centering
 \caption{Main parameters used on the in-vacuum undulators model.}
 \label{tab:invac_undulators}
 \begin{tabular}{lcl}\hline
  Chamber full gap      & 4 and 5.3            & \si{\milli\meter} \\\hline
  Copper conductivity\cite{matwebsite} & 59& \si{\mega\siemens\per\meter} \\\hline
  Copper sheet thickness &     50          & \si{\micro\meter} \\\hline
  NdFeB relative magnetic permeability & 10    & \\\hline
  NdFeB conductivity\cite{matwebsite} & 0.625& \si{\mega\siemens\per\meter}\\\hline
  Length                & 2                & \si{\meter}   \\\hline
 \end{tabular}
\end{table}

% \begin{figure}[!t]
%  \centering
%  \includegraphics[width=\textwidth]{figures/in_vacuum_undulators.png}
%  \caption{Two types of in-vacuum undulators that will be used on Sirius ring. The magnets installed at the high $\beta_x$ straight sections will have a gap of \SI{5.3}{\milli\meter}, while the ones placed at the low $\beta_x$ sections will have \SI{4}{\milli\meter}}
%  \label{fig:in_vacuum_undulators}
% \end{figure}


\section{Ferrite Kickers}

The impedance of this element is very tricky to model. We discuss the process adopted with more details in other report. Summarizing, it has two contributions: the coupled flux, which couples the beam, through the window frame, to the external circuit of the pulse generator, and the uncoupled flux, due to losses on the chamber walls. Figure \ref{fig:kicker_window_frame} shows the geometry of the kicker's window frame.

% \begin{figure}[!t]
%  \centering
%  \includegraphics[width=0.7\textwidth]{figures/kicker_window_frame.png}
%  \caption{Transverse cross section of the ferrite kickers that will be used for injection on Sirius.}
%  \label{fig:kicker_window_frame}
% \end{figure}

Regarding the coupled flux \cite{NassibianSacherer1979, VoelkerLambertson1989, DavinoHahn2003}, we ignored the effect of the Ti coating and the ceramics, because they're not important at low frequencies, below \SI{100}{\mega\hertz}, since the generator circuit does not have resonances at higher frequencies. The impedance we used for the generator is:
\begin{equation}
 Z_g = \frac{1}{1/R + i\omega C_p}
\end{equation}
which represents a RC circuit in parallel. Together with the inductance of the window frame, this impedance generates a peak centered at $\sim$\SI{50}{\mega\hertz}.


There are four possible models for the uncoupled flux:
\begin{description}
 \item[Tsutsui's Model:] Was created by Tsutsui \cite{tsutsui2000, tsutsui_vos2000} to model an in vacuum kicker, i.e. without the ceramic vacuum chamber and the Ti coating. For our case it is interesting to look at this model because it reproduces the fact that the beam sees both, the ferrite and the copper plates;
 \item[Worst Case:] Following the discussion carried out in section \ref{sec:in-vacuum_undulators}, we can use the Mounet-Métral formulas  multiplied by Yokoya factors with the following layers: Ti, ceramic, ferrite. This is considered worst case because the beam does not see the copper plates, that have high conductivity;
 \item[Best Case:] In this case we could use Mounet-Métral formulas with the layers: Ti, ceramic, copper. Now its the opposite, the beam does not see the low conductivity of the ferrite;
 \item[Average Case:] We can take the mean value of the last two models pondered by the ratio of the angles of direct exposure to the beam.
\end{description}

Figure \ref{fig:kicker_model} shows the results for the four cases with coupled flux already computed, plus an unrealistic case without Ti coating. We can perceive that the coating damp all the ceramic and ferrite resonances present in the ``Without Coating'' curve. This make us believe that it would also damp the uncoupled flux peak in the ``Tsutsui Model'', eliminating almost completely the longitudinal impedance and the above \SI{1}{\giga\hertz} transverse impedance even for the ``Worst Case'' scenario. However, the coating introduces an uncoupled flux peak in low frequencies that sum up with the originated by the coupled flux. This peak will introduce coupled bunch oscillations in the transverse plan, which are not very harmful since we will already need a transverse feedback system due to resistive wall instability.

% \begin{figure}[!t]
%  \centering
%  \includegraphics[width=\textwidth]{figures/kicker_model.png}
%  \caption{Impedances for four different models of the ferrite kicker magnets.}
%  \label{fig:kicker_model}
% \end{figure}

Below \SI{10}{\mega\hertz} we lose numeric precision on the calculation of the uncoupled flux and we decided to set this contribution to zero. This is not very important because the beam only have two or three lines below this frequency and the peak of the impedance is above this value.

Given the discussion above, we decided to use the ``Mean Case'' as the kicker impedance model in our budget. We understand it is a ``chimera'' that does not represent mathematically the problem, but if we think physically, it has all the important aspects of the geometry. This modeling is yet under study and improvements are being pursued, mainly by the implementation of the multi-layer flat chamber formulas.

Table \ref{tab:kicker_model_data} presents the values of the main parameters used in this model. Figure \ref{fig:permeability} shows the comparison between the magnetic permeability used in the model and the data taken from the CMD5005 data-sheet \cite{datasheetcmd5005}.

\begin{table}[!t]
 \centering
 \caption{Main parameters used on the kicker model.}
 \label{tab:kicker_model_data}
 \begin{tabular}{lcl}\hline
  Inner radius      &    10                & \si{\milli\meter} \\\hline
  Ti Conductivity\cite{matwebsite} & 1.8& \si{\mega\siemens\per\meter} \\\hline
  Ti thickness &     2             & \si{\micro\meter} \\\hline
  Ceramic electric permissivity & 9.3    & \\\hline
  Ceramic thickness & 7.5             & \si{\milli\meter}\\\hline
  Ferrite thickness & 20            & \si{\milli\meter} \\\hline
  Ferrite electric permissivity & 12 & \\\hline
  $R$               &  50           & \si{\ohm} \\\hline
  $C_p$             &  30          & \si{\pico\farad} \\\hline
  Length                & 0.6                & \si{\meter}   \\\hline
 \end{tabular}
\end{table}


% \begin{figure}[!t]
%  \centering
%  \subfigure[\label{fig:fitted_permeability}]{\includegraphics[width=0.45\textwidth]{figures/fitted_permeability.png}}
%  \subfigure[\label{fig:data_permeability}]{\includegraphics[width=0.45\textwidth]{figures/data_permeability.png}}
%  \caption{\subref{fig:fitted_permeability} Relative magnetic permeability used in the model. \subref{fig:data_permeability} Relative magnetic permeability taken from \cite{datasheetcmd5005}.}
%  \label{fig:permeability}
% \end{figure}


\section{Broad Band Impedance}

There are several components whose impedance were not calculated yet. For this reason, we have complemented our budget with a broad band resonator based on calculations and measurements of other synchrotron laboratories.

In the Table \ref{tab:broad_band_data} there are the parameters of the broad band resonators used in the impedance budget model and figure \ref{fig:broad_band} shows the impedance. For the longitudinal plane, we used the circumference scaled impedance from Soleil storage ring \cite{nagaoka2006} and for the transverse plane we used the circumference and beta scaled ESRF data, taken from \cite{nagaoka1999}, where the beta function scaling were made considering a medium value of \SI{20}{\meter} for ESRF.

\begin{table}[!hb]
 \centering
 \caption{Main parameters used on the broad-band model.}
 \label{tab:broad_band_data}
 \begin{tabular}{lccc}\hline
  Plane &  $R$ &  $f_R$  & $Q$\\\hline
  Longitudinal & \SI{5.3}{\kilo\ohm}& \SI{20}{\giga\hertz} &1\\\hline
  Horizontal & \SI{0.42}{\mega\ohm\per\meter} &\SI{22}{\giga\hertz} & 1 \\\hline
  Vertical   & \SI{0.42}{\mega\ohm\per\meter} &\SI{22}{\giga\hertz} & 1 \\\hline
 \end{tabular}
\end{table}


% \begin{figure}[!hb]
%  \centering
%  \includegraphics[width=\textwidth]{figures/broad_band.png}
%  \caption{Broad band impedance model used in our budget.}
%  \label{fig:broad_band}
% \end{figure}

\section{RF Cavity}

Even though Sirius RF cavity will be superconducting, we will commission the storage ring with a 5-Cells PETRA cavity and we must include its higher order modes in our budget.

to be done\dots


    \chapter{Collective Effects Related to Wake Fields}

The storage ring's global properties relevant to the calculation of collective instabilities are presented in table \ref{tab:sr_main_properties}

\begin{table}[!ht]
 \centering
 \caption{Storage ring's main properties for instabilities estimations.}
 \label{tab:sr_main_properties}
 \begin{tabular}{lcl}
  Circumference & 518.25 & \si{\meter}\\\hline
  Revolution Period & 1.73 & \si{\micro\second}\\\hline
  Revolution Frequency & 0.578 & \si{\mega\hertz}\\\hline
  Harmonic Number      & 864   & \\\hline
  Momentum compaction  & \SI{1.7e-4}{}\\\hline
  Energy               & 3      & \si{\giga\electronvolt}\\\hline
  Horizontal Tune      & 46.171 & \\\hline
  Vertical Tune        & 14.147 & \\\hline
 \end{tabular}
\end{table}

In our studies of collective effects on Sirius, we identified five stages in the operation of the ring since the commissioning until the full-current mode. The main characteristics of these stages are presented in table \ref{tab:sirius_stages}.

\begin{table}[!hb]
 \centering
 \caption{Sirius stages of operation}
 \label{tab:sirius_stages}
 \begin{tabular}{lccccc}
 Stage  & Comm.  & IDs  & IDS + Landau & Full-Current & SB \\\hline
 Current [\si{\milli\ampere}] & 10     & 100  &  100        & 300  & 2    \\\hline
 Bunch Length [\si{\milli\meter}] & 3.3 & 3.4 &  13.4       & 13.4 & 3.8    \\\hline
 Hor. damp. time [\si{\milli\second}]  & 16  & 9.4 &  9.4   & 9.4  &9.4 \\\hline
 Ver. damp. time [\si{\milli\second}]  & 21  & 12.5&  12.5  & 12.5 &12.5   \\\hline
 Long. damp. time [\si{\milli\second}] & 12.5& 7.5 &  7.5   & 7.5  &7.5  \\\hline
 \end{tabular}
\end{table}

The impedance budget model we are using is presented in table \ref{tab:impedance_budget} and the total impedances are plotted in figure \ref{fig:impedance_budget}.

\begin{table}[!ht]
 \centering
 \caption{Impedance Budget Model}
 \label{tab:impedance_budget}
 \begin{tabular}{lccc}
 Element                  &  Number  & $<\beta_x>$ & $<\beta_y>$ \\\hline
Wall With Coating         & 1        &   7         &   11        \\\hline  
In-vac. Und. @ low Betax  & 4        &       4     &       1.5   \\\hline      
In-vac. Und. @ high betax & 1        &         16  &         4   \\\hline        
Small Gap Undulators      & 2        &        16   &        4.5 \\\hline       
Ferrite Kickers           &   4      &       17    &       4   \\\hline        
Broad Band                &    1     &      6.8    &      11   \\\hline       
\end{tabular}
\end{table}

% \begin{figure}[!ht]
%  \centering
%  \includegraphics[width=0.95\textwidth]{figures/impedance_budget.png}
%  \caption{Impedance budget model for Sirius, with transverse impedance multiplied by the average beta functions in the location of each element. The kicker impedance seems discontinuous because it is negative at some frequencies.}
%  \label{fig:impedance_budget}
% \end{figure}

In the subsequent subsections we will estimate the main instabilities thresholds and strengths for each operation stage. The calculations are based on the solution of the linearized Vlasov equation for equally spaced and populated Gaussian bunches \cite{Chao1993, chin1985, cai2011}.

So far we have implemented four codes on Matlab\textregistered to calculate instabilities:
 \paragraph{Transverse Mode Coupling:} It is based on the development made by Chin \cite{chin1985} and it does basically what MOSES \cite{moses1986} does, except that:
 \begin{description}
  \item It handles arbitrary impedances;
  \item It handles the multi-bunch case;
  \item It cannot handle tune spread.
 \end{description}
To check the reliability of the code we compared it with MOSES for a broad band impedance for different chromaticities. Figure \ref{fig:comp_moses_matlab} shows the results.

% \begin{figure}[!ht]
%  \centering
%  \subfigure[\label{fig:tmci_chrom0}]{
%  \begin{minipage}{0.49\textwidth}
%   \includegraphics[width=\textwidth]{figures/tmci_matlab_chrom0.png}
%  \end{minipage}
%  \begin{minipage}{0.415\textwidth}
%   \includegraphics[width=\textwidth]{figures/tmci_moses_imag_chrom0.png}
% \includegraphics[width=\textwidth]{figures/tmci_moses_real_chrom0.png}
% \end{minipage}}
 
%  \subfigure[\label{fig:tmci_chrom01}]{
%  \begin{minipage}{0.49\textwidth}
%   \includegraphics[width=\textwidth]{figures/tmci_matlab_chrom01.png}
%  \end{minipage}
%  \begin{minipage}{0.415\textwidth}
%   \includegraphics[width=\textwidth]{figures/tmci_moses_imag_chrom01.png}  \includegraphics[width=\textwidth]{figures/tmci_moses_real_chrom01.png}
% \end{minipage}}
%  \caption{Comparison between MOSES (right) and our code (left) for:  \subref{fig:tmci_chrom0} zero chromaticity and \subref{fig:tmci_chrom01} 0.1 normalized chromaticity.}
%  \label{fig:comp_moses_matlab}
% \end{figure}

 \paragraph{Longitudinal Mode Coupling:} It does the same as the previous code, but for the longitudinal plane. We still have to systematically check this implementation with tracking codes, but preliminary results are positive.
 
 \paragraph{Transverse Coupled Bunch Instabilities:} Based on the effective impedance approximation \cite{Chao1993}. It were checked by comparisons with the transverse mode coupling code, since the results for coupled bunch instabilities are contained on it.
 
 \paragraph{Longitudinal Coupled Bunch Instabilities:} Idem to the previous one.
 
 
\section{Coupled Bunch Instabilities}\label{sec:coupled_bunch_instabilities}

It is important to know the coupled bunch instability threshold at zero chromaticity during the commissioning. In this phase the in-vacuum undulators' gap will be opened and we must calculate the instability without these elements. Figure \ref{fig:cbi_thres} shows the results. We notice that we can run the machine in the nominal current of the commissioning stage without the feedback system.

% \begin{figure}[!ht]
%  \centering
%  \subfigure[\label{fig:cbi_h_thres}]{\includegraphics[width=\textwidth]{figures/cbi_h_thres.png}}
%  \subfigure[\label{fig:cbi_v_thres}]{\includegraphics[width=\textwidth]{figures/cbi_v_thres.png}}
%  \caption{Coupled Bunch instability thresholds at zero chromaticity during commissioning for the:  \subref{fig:cbi_h_thres} Horizontal and \subref{fig:cbi_v_thres} Vertical Planes.}
%  \label{fig:cbi_thres}
% \end{figure}

Figure \ref{fig:cbi_h}/\ref{fig:cbi_v} shows the characterization of the coupled bunch instabilities for the horizontal/vertical plane for the operational modes of interest. We notice that both tunes being above integer contribute to diminish the estimated growth rates. This way, operating the ring at a normalized chromaticity of 0.05/0.1 in the horizontal/vertical plane, which corresponds to an unnormalized chromaticity of 2.3/1.4, we can avoid this kind of instabilities.

% \begin{figure}[H]
%  \centering
%  \subfigure[\label{fig:cbi_h_id}]{\includegraphics[width=\textwidth]{figures/cbi_h_id.png}}
%  \subfigure[\label{fig:cbi_h_idlan}]{\includegraphics[width=\textwidth]{figures/cbi_h_idlan.png}}
%  \subfigure[\label{fig:cbi_h_fc}]{\includegraphics[width=\textwidth]{figures/cbi_h_fc.png}}
%  \caption{Coupled Bunch instabilities for the horizontal plane at stages:  \subref{fig:cbi_h_id} IDs. \subref{fig:cbi_h_idlan} \hbox{IDs + Landau cavity}. \subref{fig:cbi_h_fc} Full current.}
%  \label{fig:cbi_h}
% \end{figure}

% \begin{figure}[H]
%  \centering
%  \subfigure[\label{fig:cbi_v_id}]{\includegraphics[width=\textwidth]{figures/cbi_v_id.png}}
%  \subfigure[\label{fig:cbi_v_idlan}]{\includegraphics[width=\textwidth]{figures/cbi_v_idlan.png}}
%  \subfigure[\label{fig:cbi_v_fc}]{\includegraphics[width=\textwidth]{figures/cbi_v_fc.png}}
%  \caption{Coupled Bunch instabilities for the vertical plane at:  \subref{fig:cbi_v_id} IDs stage. \subref{fig:cbi_v_idlan} IDs + Landau cavity stage. \subref{fig:cbi_v_fc} Full current stage.}
%  \label{fig:cbi_v}
% \end{figure}



\section{Single Bunch Instabilities}


In this section we will analyze the single bunch instabilities, they will be important not only for the characterization of the single bunch operational mode, but also to have a worst case scenario for the hybrid-filling pattern. Table \ref{tab:impedance_effect_sb} and Figures \ref{fig:lmci} and \ref{fig:tmci} show some results.

We notice that the broad band impedance overcomes all others, including the resistive wall for the three planes, being the transverse ones the most prejudiced. From this observation we can infer two things:
\begin{enumerate}
 \item The broad band used is over estimated and we need to vary some of the parameters, mainly the shunt resistance to obtain some specification for the ring impedance;
 \item The stability of the ring relies on the yet unknown, and we need to evaluate other element's impedances in order to have a more realistic view.
\end{enumerate}


\begin{table}[!ht]
 \centering
 \caption{Impedance Budget effects on the SB operational mode}
 \label{tab:impedance_effect_sb}
 \begin{tabular}{lccc}
\multirow{2}{*}{Element}&$\kappa_L$&$\beta_x \kappa_x$& $\beta_y \kappa_y$\\
 &[\si{\volt\per\pico\coulomb}] &[\si{\kilo\volt\per\pico\coulomb}] &     [\si{\kilo\volt\per\pico\coulomb}]\\\hline 
Wall With Coating           & 4          &-7.4    &   -12    \\\hline     
In-vac. Und. @ low Betax    & 0.28       &-2.7    &   -2     \\\hline
In-vac. Und. @ high Betax   & 0.053      &-1.2    &   -0.58  \\\hline     
Small Gap Undulators        & 0.084      &-0.5    &   -0.28  \\\hline     
Ferrite Kickers             & 0.24       &-0.29   &   -0.1   \\\hline     
Broad Band                  & 26         &-58     &   -94    \\\hline
Total                       & 30.4       &-69.9   &   -108   \\\hline
\end{tabular}
\end{table}


% \begin{figure}[H]
%  \centering
%  \subfigure[\label{fig:lmci_all-imped}]{\includegraphics[width=0.49\textwidth]{figures/lmci_all-imped.png}}
%  \subfigure[\label{fig:lmci_sem-bbr}]{\includegraphics[width=0.49\textwidth]{figures/lmci_sem-bbr.png}}
%  \caption{\subref{fig:lmci_all-imped} Longitudinal mode coupling instability with the whole impedance budget taken into account. \subref{fig:lmci_sem-bbr} Without broad band resonator.}
%  \label{fig:lmci}
% \end{figure}

% \begin{figure}[!t]
%  \centering
%  \subfigure[\label{fig:tmci_all-imped}]{\includegraphics[width=0.49\textwidth]{figures/tmci_h_all-imped.png} 
%  \includegraphics[width=0.49\textwidth]{figures/tmci_v_all-imped.png}}
%  \subfigure[\label{fig:tmci_sem-bbr}]{\includegraphics[width=0.49\textwidth]{figures/tmci_h_sem-bbr.png}
%  \includegraphics[width=0.49\textwidth]{figures/tmci_v_sem-bbr.png}}
%  \caption{\subref{fig:tmci_all-imped} Transverse mode coupling instability with the whole impedance budget taken into account. \subref{fig:tmci_sem-bbr} Without broad band resonator.}
%  \label{fig:tmci}
% \end{figure}


Based on the results of section \ref{sec:coupled_bunch_instabilities} we investigated the effect of positive chromaticity on single bunch instabilities. As can be noticed in figure \ref{fig:tmci_chrom} the instability is not completely damped, but its growth rates do not increase as fast as they would with zero chromaticity. 

It seems that with positive chromaticity it is easier for a conventional feedback system to control the beam, however, in order to understand better the behavior of this instability we need to look at the evolution of the distribution function calculated through the eigenvector decomposition.

% \begin{figure}
%  \centering
%  \subfigure[\label{fig:tmci_h_chrom}]{\includegraphics[width=0.49\textwidth]{figures/tmci_h_chrom.png}}
%  \subfigure[\label{fig:tmci_v_chrom}]{\includegraphics[width=0.49\textwidth]{figures/tmci_v_chrom.png}}
%  \caption{\subref{fig:tmci_h_chrom} Transverse mode coupling instability with the whole impedance budget taken into account. \subref{fig:tmci_v_chrom} Without broad band resonator.}
%  \label{fig:tmci_chrom}
% \end{figure}

    
	% Finaliza a parte no bookmark do PDF, para que se inicie o bookmark na raiz
	\bookmarksetup{startatroot}%

	%\chapter*[Conclusão]{Conclusão}
	%\addcontentsline{toc}{chapter}{Conclusão}
	%\lipsum[1-5]


	% --------- Elementos pós-textuais --------------&
	\postextual
	% Referências bibliográficas
%\bibliographystyle{unsrt}
\bibliography{library}


% Apêndices
\begin{apendicesenv}
% Imprime uma página indicando o início dos apêndices
\partapendices
\chapter{Derivações}

\section{Duas partículas interagindo no vácuo}

\begin{figure}[hb!]
\centering
\begin{tikzpicture}[scale=1]
\def\d{1cm}
\draw[<->] (1,0) node[below]{$z$} 
		-- ++(-\d,0) node[below left] {$S$} 
        -- ++(0,\d) node[left] {$\rho$}; %coord sys S
\draw[<->] (6*\d,0) node[below]{$z'$}
		-- ++(-\d,0) node[below left] {$S'$} 
        -- ++(0,\d) node[left] {$\rho'$}; % coord sys S'
\coordinate (V) at (0.5,0);
\coordinate (Q1) at (4cm,1.5cm);
\coordinate (Q2) at (0.5cm,2.5cm);
\draw[->] (5*\d,0.5*\d) -- ++(V) node[above] {$\boldsymbol{v}$};
\filldraw[fill=black] (Q1) circle[radius=0.05] node[above] {$q$}; % source particle
\draw[->] (Q1) -- ++(V) node[above] {$\boldsymbol{v}$}; % velocity vector
\filldraw[fill=black] (Q2)  circle[radius=0.05]; % test particle
\draw[->] (Q2) -- ++(V) node[above] {$\boldsymbol{v}$}; % velocity vector
\draw[dashed,|-|] ($(Q1)-(0,0.2)$)
				   let \p1 = ($(Q2) - (Q1)$)
                   in -- ++(\x1,0) node[midway,below] {$s$}; %horizontal distance
\draw[dashed,|-|] ($(Q2)-(0.2,0)$)
				   let \p1 = ($(Q1) - (Q2)$)
                   in -- ++(0,\y1) node[midway,left] {$\rho$}; % vertical distance
\draw[->] (Q1) -- ++($(Q2) - (Q1)$) node[midway,above] {$\boldsymbol{R}$}; %vector
\end{tikzpicture}
\caption{Duas partículas interagindo via campo direto.}
\end{figure}

Primeiro, campos gerados por $q_1$. No referencial $S'$:
\begin{align}
\vect{E'} &= \frac{q}{4\pi\epsilon_0} \frac{\vect{\hat{r'}}}{r'^2} \\
\vect{B'} &= \vect{0'}
\end{align}
assumindo que a partícula 1 está na origem do sistema de coordenadas $S'$.

Lembrando que a transformação entre coordenadas esféricas para cilíndricas são:
\begin{align}
r' &= \sqrt{s'^2 + \rho'^2} \\
\vect{\hat{r'}} &= \cos\theta'\vect{\hat{z'}} + \sin\theta'\vect{\hat{\rho'}} =
                  -\frac{s'}{r'}\vect{\hat{z'}} + \frac{\rho'}{r'}\vect{\hat{\rho'}}
\end{align}
onde a coordenada $\phi$ fica inalterada.

Assim podemos reescrever o campo elétrico em suas partes longitudinal e transversal:
\begin{align}
\vect{E'}_{||} &= \frac{q}{4\pi\epsilon_0} \frac{\cos\theta'\vect{\hat{z'}}}{r'^2} = 
                 -\frac{q}{4\pi\epsilon_0} \frac{s'\vect{\hat{z'}}}{(\rho'^2+s'^2)^{3/2}} \\
\vect{E'}_{\perp} &= \frac{q}{4\pi\epsilon_0} \frac{\sin\theta'\vect{\hat{\rho'}}}{r'^2} = 
                     \frac{q}{4\pi\epsilon_0} \frac{x'\vect{\hat{\rho'}}}{(\rho'^2+s'^2)^{3/2}}
\end{align}

Lembrando as equações de transformação de Lorentz para campos elétricos e magnéticos, para esse problema:

\begin{align}
 \vect{E}_{||} &= \vect{E'}_{||}\\
 \vect{B}_{||} &= \vect{B'}_{||}\\
 \vect{E}_\perp &= \gamma\left(\vect{E'}_\perp - \vect{v}\times\vect{B'}\right)\\
 \vect{B}_\perp &= \gamma\left(\vect{B'}_\perp + \frac{1}{c^2}\vect{v}\times\vect{E'}\right)
\end{align}

Ainda, as coordenadas espaciais são transformadas da seguinte maneira:

\begin{align}
\vect{\hat{\rho}} &=\vect{\hat{\rho'}}, \quad \vect{\hat{z}} = \vect{\hat{z'}}\\
\rho &= \rho', \quad z = \frac{z'}{\gamma}
\end{align}

Assim, podemos notar que:

\begin{align}
\vect{E}_{||} &= -\frac{q}{4\pi\epsilon_0} \frac{s'\vect{\hat{z'}}}{(\rho'^2+s'^2)^{3/2}} = 
				 -\frac{q}{4\pi\epsilon_0} \frac{\gamma s\vect{\hat{z}}}{((\gamma s)^2+\rho^2)^{3/2}}=
                 -\frac{q}{4\pi\epsilon_0} \frac{s\vect{\hat{z}}}{\gamma^2R^{*3}} \\
\vect{E}_\perp&= \frac{q}{4\pi\epsilon_0}\frac{\gamma \rho'\vect{\hat{\rho'}}}{(\rho'^2+s'^2)^{3/2}} = 
				 \frac{q}{4\pi\epsilon_0}\frac{\gamma \rho\vect{\hat{\rho}}}{((\gamma s)^2+\rho^2)^{3/2}}=
                 \frac{q}{4\pi\epsilon_0} \frac{\rho\vect{\hat{\rho}}}{\gamma^2R^{*3}} \\
\vect{B}_\perp&= -\frac{\gamma   vE'_\perp}{c^2}\vect{\hat{\phi'}} =
		         -\frac{vE_\perp}{c^2}\vect{\hat{\phi}}
\end{align}
onde $R^* = \sqrt{s^2+\left(\rho/\gamma\right)^2}$

Agora podemos analisar a força exercida pela partícula fonte sobre a partícula teste usando a força de Lorentz
\begin{equation}
\vect{F} = \left(\vect{E} + \vect{v}\times\vect{B}\right)
\end{equation}
onde foi assumida carga unitária para a partícula teste, e os campos calculados anteriormente.

Assumindo que a velocidade da partícula teste é a mesma da partícula fonte (mesmo módulo e direção), as componentes longitudinal e transversal da força ficam:

\begin{align}
\vect{F}_{||} &= \vect{E}_{||} = -\frac{q}{4\pi\epsilon_0} \frac{s\vect{\hat{z}}}{\gamma^2R^{*3}}\\
\vect{F}_\perp&= \left(1+\frac{\vect{v}\times\vect{v}\times}{c^2}\right)\vect{E}_\perp = 
                 \left(1-\frac{v^2}{c^2}\right)\vect{E}_\perp = 
                 \frac{q}{4\pi\epsilon_0} \frac{\rho\vect{\hat{z}}}{\gamma^4R^{*3}}
\end{align}
onde vemos que a força longitudinal tende a zero proporcionalmente a $\gamma^{-2}$ quando $v \to c$ e que a força longitudinal tende a zero com $\gamma^{-4}$, se $s>\rho/\gamma$ e com $\gamma^{-1}$ se $s<\rho/\gamma$.

Agora, vamos assumir que a velocidade da partícula teste não é paralela à velocidade da partícula fonte
\begin{equation}
\vect{v_2} = v(\cos\delta \vect{\hat{z}} + \sin\delta\vect{\hat{\rho}})
\end{equation}

Assim, a força sofrida por essa partícula fica

\begin{equation}\begin{aligned}
\vect{F} &= \vect{E}_{||} + \vect{E}_\perp - v(\cos\delta \vect{\hat{z}} + \sin\delta\vect{\hat{\rho}}) \times \vect{\hat{\phi}}\frac{vE_\perp}{c^2} \\
&= \left(1-\frac{v^2}{c^2}\cos\delta\right)\vect{E}_\perp + \vect{E}_{||} + \frac{v^2}{c^2}E_\perp\sin\delta\vect{\hat{z}}
\end{aligned}\end{equation}

Olhando essa expressão, vemos que, conforme $v \to c$, a força pode ser expressa como:

\begin{equation}
\vect{F} = \left\{
\begin{aligned}
\left((1-\cos\delta)\vect{\hat{\rho}} + \sin\delta\vect{\hat{z}}\right) 
\frac{q}{4\pi\epsilon_0} \frac{\gamma\vect{\hat{\rho}}}{\rho^2} & &|s|<\rho/\gamma\\
0 & &|s|>\rho/\gamma
\end{aligned}\right.
\end{equation}






\chapter{Apêndice 2}
\lipsum[55-57]
\end{apendicesenv}


% Anexos
\begin{anexosenv}
% Imprime uma página indicando o início dos anexos
\partanexos
\chapter{Anexo 1}
\lipsum[30]
\chapter{Anexo 2}
\lipsum[31]
\chapter{Anexo 3}
\lipsum[32]
\end{anexosenv}


% Índice remissivo
\printindex

	
\end{document}