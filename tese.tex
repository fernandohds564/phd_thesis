\documentclass[
	% -- opções da classe memoir --
	12pt,				% tamanho da fonte
	openright,			% capítulos começam em páginas ímpar (insere página vazia caso preciso)
	oneside,			% para impressão em verso e anverso. Oposto a oneside
	a4paper,		% tamanho do papel.
	% -- opções da classe abntex2 --
	chapter=TITLE,		% títulos de capítulos convertidos em letras maiúsculas
	section=TITLE,		% títulos de seções convertidos em letras maiúsculas
	%subsection=TITLE,	% títulos de subseções convertidos em letras maiúsculas
	%subsubsection=TITLE,% títulos de subsubseções convertidos em letras maiúsculas
	% -- opções do pacote babel --
    brazil,				% o último idioma é o principal do documento
	english,			% idioma adicional para hifenização
	%french,			% idioma adicional para hifenização
	%spanish,			% idioma adicional para hifenização
	sumario=tradicional,
	]{abntex2}

%%%%%% IMPORTAÇÃO DOS PACOTES %%%%%%
\documentclass[
	% -- opções da classe memoir --
	12pt,				% tamanho da fonte
	openright,			% capítulos começam em páginas ímpar (insere página vazia caso preciso)
	oneside,			% para impressão em verso e anverso. Oposto a oneside
	a4paper,		% tamanho do papel.
	% -- opções da classe abntex2 --
	chapter=TITLE,		% títulos de capítulos convertidos em letras maiúsculas
	section=TITLE,		% títulos de seções convertidos em letras maiúsculas
	%subsection=TITLE,	% títulos de subseções convertidos em letras maiúsculas
	%subsubsection=TITLE,% títulos de subsubseções convertidos em letras maiúsculas
	% -- opções do pacote babel --
    brazil,				% o último idioma é o principal do documento
	english,			% idioma adicional para hifenização
	%french,			% idioma adicional para hifenização
	%spanish,			% idioma adicional para hifenização
	sumario=tradicional,
	]{abntex2}

% --------------- PACOTES -----------------%
%%%%% conflict with float package, loaded by one of the packages bellow.
% \usepackage{floatrow} % pacote para criar ambientes float genéricos
% \floatsetup[table]{style=plaintop} % poe legendas de tabelas em cima.

% Pacotes fundamentais
\usepackage{cmap}				% Mapear caracteres especiais no PDF
\usepackage{lmodern}			% Usa a fonte Latin Modern
\usepackage[T1]{fontenc}		% Selecão de códigos de fonte. hifenação correta.
\usepackage[utf8]{inputenc}		% Codificacao do documento (conversão automática dos acentos)
\usepackage{lastpage}			% Usado pela Ficha catalográfica
\usepackage{indentfirst}		% Indenta o primeiro parágrafo de cada seção.
\usepackage{xcolor}				% Controle das cores
\usepackage[pdftex]{graphicx}	% Inclusão de gráficos
\graphicspath{{figures/}}
\usepackage{subcaption}
\usepackage{epstopdf}           % Pacote que converte as figuras em eps para pdf


% Pacotes adicionais, usados apenas no âmbito do Modelo Canônico do abnteX2
\usepackage{nomencl}
\usepackage{amsmath}   % usado para ter o environment pmatrix, align
\usepackage{mathtools} % usado para ter o environment pmatrix*
\usepackage{mathrsfs}  % usado para ter o curly H
\usepackage{xfrac}     % usado para ter o comando \sfrac
\usepackage{bbm}
\usepackage[chapter]{algorithm}
\usepackage{algorithmic}
\usepackage{multirow}
\usepackage{rotating}
\usepackage{pdfpages}       % insere páginas pdf no arquivo
\usepackage{siunitx}        % pacote para padronizar display de números e unidades

\usepackage[brazilian,hyperpageref]{backref}	 % Paginas com as citacões na bibl

\usepackage[toc,acronym,nomain,nogroupskip,nonumberlist]{glossaries}%nonumberlist
\newacronym{linac}{LINAC}{Linear Accelerator}
\newacronym{cnpem}{CNPEM}{National Center for Research in Energy and Materials}
\newacronym{lnls}{LNLS}{Brazilian Synchrotron Light Laboratory}
\newacronym{3gls}{3$^\text{rd}$ GLS}{Third Generation Light Sources}
\newacronym{4gls}{4$^\text{th}$ GLS}{Fourth Generation Light Sources}
\newacronym{mac}{MAC}{Machine Advisory Committee}
\newacronym{bsc}{BSC}{Beam Stay Clear}
\newacronym{supcond}{SC-RF}{Superconducting RF Cavity}
\newacronym{csr}{CSR}{coherent synchrotron radiation}
\newacronym{dsp}{DSP}{direct space charge}
\newacronym{isp}{ISP}{indirect space charge}
\newacronym{maxeq}{ME}{Maxwell Equations}
\newacronym{lhs}{l.h.s.}{left hand side}
\newacronym{rhs}{r.h.s.}{right hand side}
\newacronym{xfel}{FEL}{X-Ray Free Electron Lasers}
\newacronym{als}{ALS}{Advanced Light Source}
\newacronym{cern}{CERN}{European Organization for Nuclear Research}
\newacronym{neg}{NEG}{Non--Evaporable Getter}
\newacronym{pec}{PEC}{Perfect Electric Conductor}
\newacronym{esrf}{ESRF}{European Synchrotron Radiation Facility}
\newacronym{mba}{MBA}{Multi-Bend-Achromat}
\newacronym{tmci}{TMCI}{Transverse Mode-Coupling Instability}
\newacronym{lmci}{LMCI}{Longitudinal Mode-Coupling Instability}
\newacronym{fofb}{FOFB}{Fast Orbit Feedback System}
\newacronym{apu}{APU}{Adjustable Phase Undulator}
\newacronym{rms}{rms}{root-mean-square}
\newacronym{vfp}{VFP}{Vlasov Fokker Planck}
\newacronym{si}{SI}{International System of Units}
\newacronym{ibs}{IBS}{intrabeam scattering}
\newacronym{pic}{PIC}{particle in cell}
\newacronym{dc}{DC}{direct current}

\newglossaryentry{bpm}
{
  name={BPM},
  description={Beam Position Monitor},
  first={\glsentrydesc{bpm} (\glsentrytext{bpm})},
  plural={BPMs},
  descriptionplural={Beam Position Monitors},
  firstplural={\glsentrydescplural{bpm} (\glsentryplural{bpm})}
}
\newglossaryentry{sls}
{
  name={SLS},
  description={Synchrotron Light Source},
  first={\glsentrydesc{sls} (\glsentrytext{sls})},
  plural={SLSs},
  descriptionplural={Synchrotron Light Sources},
  firstplural={\glsentrydescplural{sls} (\glsentryplural{sls})}
}
\newglossaryentry{id}
{
  name={ID},
  description={Insertion Device},
  first={\glsentrydesc{id} (\glsentrytext{id})},
  plural={IDs},
  descriptionplural={Insertion Devices},
  firstplural={\glsentrydescplural{id} (\glsentryplural{id})}
}
\newglossaryentry{bbr}
{
  name={BBR},
  description={broad band resonator},
  first={\glsentrydesc{bbr} (\glsentrytext{bbr})},
  plural={BBRs},
  descriptionplural={broad band resonators},
  firstplural={\glsentrydescplural{bbr} (\glsentryplural{bbr})}
}
\newglossaryentry{hom}
{
  name={HOM},
  description={higher order mode},
  first={\glsentrydesc{hom} (\glsentrytext{hom})},
  plural={HOMs},
  descriptionplural={higher order modes},
  firstplural={\glsentrydescplural{hom} (\glsentryplural{hom})}
}


% Pacote que faz Citações padrão ABNT
\usepackage[
alf,
versalete,
abnt-and-type=&,
abnt-etal-cite=2,
abnt-etal-list=3,
abnt-doi=link,
% abnt-repeated-author-omit=yes,
abnt-etal-text=emph]{abntex2cite}
% \citebrackets[]
% \usepackage[style=numeric-comp]{biblatex}
\renewcommand{\authorcapstyle}{\small}
\renewcommand{\authorstyle}{\relax}

% Pacote de customização - Unicamp
\usepackage{unicamp}

% Pacote para fazer desenhos
\usepackage{tikz}
%\usetikzlibrary{⟨list of libraries separated by commas⟩} % carrega bibliotecas adicionais
\usetikzlibrary{calc,arrows.meta}


% Pacotes de Debbuging:
\usepackage{lipsum}     % Pacote que gera texto dummy
%\usepackage{blindtext}  % Pacote que gera texto dummy
% \usepackage{showlabels} % Pacote que mostra os labels das equações no pdf
\usepackage{todonotes}  % Pacote que insere notas no pdf

% Posso criar definições de acrônimos e ele automaticamente coloca o nome completo
% no texto na primeira ocorrência ou o acrônimo nas referências subsequentes
% \frenchspacing
% \renewcommand{\finalnamedelim}{\ \&\ }
% \makenoidxglossaries
% \newacronym{linac}{LINAC}{Linear Accelerator}
\newacronym{cnpem}{CNPEM}{National Center for Research in Energy and Materials}
\newacronym{lnls}{LNLS}{Brazilian Synchrotron Light Laboratory}
\newacronym{3gls}{3$^\text{rd}$ GLS}{Third Generation Light Sources}
\newacronym{4gls}{4$^\text{th}$ GLS}{Fourth Generation Light Sources}
\newacronym{mac}{MAC}{Machine Advisory Committee}
\newacronym{bsc}{BSC}{Beam Stay Clear}
\newacronym{supcond}{SC-RF}{Superconducting RF Cavity}
\newacronym{csr}{CSR}{coherent synchrotron radiation}
\newacronym{dsp}{DSP}{direct space charge}
\newacronym{isp}{ISP}{indirect space charge}
\newacronym{maxeq}{ME}{Maxwell Equations}
\newacronym{lhs}{l.h.s.}{left hand side}
\newacronym{rhs}{r.h.s.}{right hand side}
\newacronym{xfel}{FEL}{X-Ray Free Electron Lasers}
\newacronym{als}{ALS}{Advanced Light Source}
\newacronym{cern}{CERN}{European Organization for Nuclear Research}
\newacronym{neg}{NEG}{Non--Evaporable Getter}
\newacronym{pec}{PEC}{Perfect Electric Conductor}
\newacronym{esrf}{ESRF}{European Synchrotron Radiation Facility}
\newacronym{mba}{MBA}{Multi-Bend-Achromat}
\newacronym{tmci}{TMCI}{Transverse Mode-Coupling Instability}
\newacronym{lmci}{LMCI}{Longitudinal Mode-Coupling Instability}
\newacronym{fofb}{FOFB}{Fast Orbit Feedback System}
\newacronym{apu}{APU}{Adjustable Phase Undulator}
\newacronym{rms}{rms}{root-mean-square}
\newacronym{vfp}{VFP}{Vlasov Fokker Planck}
\newacronym{si}{SI}{International System of Units}
\newacronym{ibs}{IBS}{intrabeam scattering}
\newacronym{pic}{PIC}{particle in cell}
\newacronym{dc}{DC}{direct current}

\newglossaryentry{bpm}
{
  name={BPM},
  description={Beam Position Monitor},
  first={\glsentrydesc{bpm} (\glsentrytext{bpm})},
  plural={BPMs},
  descriptionplural={Beam Position Monitors},
  firstplural={\glsentrydescplural{bpm} (\glsentryplural{bpm})}
}
\newglossaryentry{sls}
{
  name={SLS},
  description={Synchrotron Light Source},
  first={\glsentrydesc{sls} (\glsentrytext{sls})},
  plural={SLSs},
  descriptionplural={Synchrotron Light Sources},
  firstplural={\glsentrydescplural{sls} (\glsentryplural{sls})}
}
\newglossaryentry{id}
{
  name={ID},
  description={Insertion Device},
  first={\glsentrydesc{id} (\glsentrytext{id})},
  plural={IDs},
  descriptionplural={Insertion Devices},
  firstplural={\glsentrydescplural{id} (\glsentryplural{id})}
}
\newglossaryentry{bbr}
{
  name={BBR},
  description={broad band resonator},
  first={\glsentrydesc{bbr} (\glsentrytext{bbr})},
  plural={BBRs},
  descriptionplural={broad band resonators},
  firstplural={\glsentrydescplural{bbr} (\glsentryplural{bbr})}
}
\newglossaryentry{hom}
{
  name={HOM},
  description={higher order mode},
  first={\glsentrydesc{hom} (\glsentrytext{hom})},
  plural={HOMs},
  descriptionplural={higher order modes},
  firstplural={\glsentrydescplural{hom} (\glsentryplural{hom})}
}

% comando \gls


%---------------CONFIGURAÇÕES------------%

% informações do PDF. O pacote hyperref já foi incluido em abntex2, eu acho
\makeatletter
\hypersetup{
     	%pagebackref=true,
		pdftitle={\@title},
		pdfauthor={\@author},
    	pdfsubject={\imprimirpreambulo},
	    pdfcreator={LaTeX with abnTeX2},
		pdfkeywords={abnt}{latex}{abntex}{abntex2}{trabalho acadêmico},
		hidelinks,					% desabilita as bordas dos links
		colorlinks=false,       	% false: boxed links; true: colored links
    	linkcolor=blue,          	% color of internal links
    	citecolor=blue,        		% color of links to bibliography
    	filecolor=magenta,      	% color of file links
		urlcolor=blue,
%		linkbordercolor={1 1 1},	% set to white
		bookmarksdepth=4
}
\makeatother

% Espaçamentos entre linhas e parágrafos
\setlength{\parindent}{1.3cm} % Tamanho da identação do parágrafo
% Controle do espaçamento entre um parágrafo e outro:
\setlength{\parskip}{0.2cm}  % tente também \onelineskip

%\setlength{\mathindent}{0cm}

% Compila o índice
\makeindex
\makenomenclature


% Informacoes de dados para CAPA e FOLHA DE ROSTO:
\titulo{Estudo de Impedâncias e Instabilidades Coletivas aplicadas ao Sirius.}
\autor{Fernando Henrique de Sá}
\local{Campinas}
\data{2016}
\orientador{Prof. Dr. Antônio Rubens Britto de Castro}
\coorientador[Co-orientador]{Prof. Dr. Sílvio}
\instituicao{%
    UNIVERSIDADE ESTADUAL DE CAMPINAS
    \par
    Instituto de Física Gleb Wataghin (IFGW)
    }
\tipotrabalho{Tese (Doutorado)}
% O preambulo deve conter o tipo do trabalho, o objetivo, o nome da instituição
% e a área de concentração
\preambulo{Tese apresentada ao Instituto de Física da Universidade Estadual
           de Campinas como parte dos requisitos exigidos para a obtenção do
		   título de Doutor em Física, com ênfase em Física de Aceleradores.}

\newcommand{\udefint}[2]{\int\!\!\text{d}#1 #2}                 % Integral indefinida
\newcommand{\udefoint}[2]{\oint\!\text{d}#1 #2}               % Integral fechada
\newcommand{\defint}[4]{\int_{#3}^{#4}\!\!\text{d}#1 #2}       % Integral definida
\newcommand{\infint}[2]{\defint{#1}{#2}{-\infty}{\infty}}      % Integra definida infinito
\newcommand{\dertot}[3][{}]{\frac{\mathrm{d}^{#1}#2}{\mathrm{d} #3^{#1}}} % Derivada total
\newcommand{\derpar}[3][{}]{\frac{\partial^{#1}#2}{\partial #3^{#1}}}     % Derivada parcial
\newcommand{\average}[2][{}]{\left\langle #2 \right\rangle_{#1}}
\newcommand{\vect}[1]{\overrightarrow{\boldsymbol{#1}}}
\newcommand{\versor}[1]{\boldsymbol{\hat#1}}
\newcommand{\tensor}[1]{\overleftrightarrow{\boldsymbol{#1}}} % Tensor
\newcommand{\fourier}[1]{\tilde{#1}}  % representation of the Fourier Transform
\newcommand{\real}[1]{\Re\left\{#1\right\}}
\newcommand{\imag}[1]{\Im\left\{#1\right\}}
\newcommand{\engw}[1]{\emph{#1}}        % Palavra em Língua Inglesa


% \includeonly{
% % %content/Chap1-introducao,
% % content/Chap2-fisica_aceleradores,
% % content/Chap3-impedancias_e_wakes,
% % %content/App1-derivacoes,
% }

\begin{document}
	% Retira espaço extra obsoleto entre as frases
	\frenchspacing

	%---------- Elementos pré-textuais --------------%
	\pretextual
	%Capa
\imprimircapa


%Folha de rosto sem número de página
\setcounter{page}{3}
\imprimirfolhaderosto*


% Ficha Catalográfica
% Não sei o que é ainda, mas parece que a universidade vai me fornecer uma
%após a defesa da minha tese. Quando ela fizer isso, tenho que comentar as
%linhas abaixo e descomentar o \includepdf:
\begin{fichacatalografica}
    \vspace*{\fill}
    \begin{center}
        \textsc{Inclua aqui o pdf com a ficha catalográfica fornecida pela BAE.}
    \end{center}
    \vspace*{\fill}
    %\includepdf{ficha-catalografica.pdf}
\end{fichacatalografica}


% Folha de aprovação
% Na versão final, tenho que excluir essas linhas e incluir o \includepdf
\newpage
\vspace*{\fill}
\begin{center}
    \textsc{Inclua aqui a folha de assinaturas.}
\end{center}
\vspace*{\fill}
\newpage
%\includepdf[pagecommand={\thispagestyle{plain}}]{folha-assinaturas.pdf}
\cleardoublepage


% Dedicatória
\begin{dedicatoria}
    \vspace*{\fill}
    \centering
    \noindent
    \textit{Dedico esta tese à todo mundo.}
    \vspace*{\fill}
\end{dedicatoria}

% Agradecimentos
\begin{agradecimentos}
    \lipsum[1-4]
\end{agradecimentos}

% Epígrafe
\begin{epigrafe}
    \vspace*{\fill}
    \begin{flushright}
        \textit{``Aqui jaz seis anos da minha vida.''\\
        (Fernando Henrique de Sá)}
    \end{flushright}
\end{epigrafe}

% Resumos em português
\begin{resumo}
    \lipsum[1-2]
    \vspace{\onelineskip}
    \noindent\textbf{Keywords}: keyword 1; keyword 2; keyword 3.
    \vspace{\fill}
\end{resumo}

% Resumo em Inglês
\begin{otherlanguage*}{brazil}
    \begin{center}{\ABNTEXchapterfont\huge Resumo}\end{center}

    \lipsum[1-2]
    \vspace{\onelineskip}
    \noindent\textbf{Palavras-chaves}: palavra-chave 1; palavra-chave 2; palavra-chave 3.
    \vspace{\fill}
\end{otherlanguage*}
\cleardoublepage


% Lista de ilustrações
\pdfbookmark[0]{\listfigurename}{lof}
\listoffigures*
\cleardoublepage


% Lista de tabelas
\pdfbookmark[0]{\listtablename}{lot}
\listoftables*
\cleardoublepage


% Lista de Acronimos e Abreviações
\renewcommand{\nomname}{Lista de Acrônimos e Abreviações}
\pdfbookmark[0]{\nomname}{las}
\printnomenclature
\cleardoublepage


% Sumário
\pdfbookmark[0]{\contentsname}{toc}
\tableofcontents*
\cleardoublepage


	%---------- Elementos textuais ------------------%
	\textual

%	\chapter{Introduction} 	\label{cap:intro}

In scientific facilities commomly known as synchrotron lights sources (SLS) the interaction between light and matter is used to study properties of a variaty of materials. Throught techniques involving absortion, reflection, refraction and scattering of light of different 'colors' by the materials under study, scientists can determine their atomic structure and composition.

The frequency of the light used in these facilities range from tera-hertz to hard X-rays and its origin is always related to synchrotron emittion of radiation by charged particles, hence the name of the facility. The light emitted by centripetal acceleration of ultra-relativistic particles has unique properties for the use in scientific investigation. Besides its broad spectrum, among the advantages in relation to other methods are the high total flux emitted and the strong collimation.


Every SLS depends on a particle accelerator to generate, accelerate and excite the charged particles, generally electrons, to emit the radiation. Most light sources uses synchrotron storage rings where subatomic charged particles, generally electrons, are extracted from materials and accelerated to relativistic energies in order to produce radiation. Among the several types of accelerators the synchrotron is a machine 

A Synchrotron Light Source is a class of Light Source where the light is generated in a circular storage ring which particles are stored for hours in almost circular stable orbits.


and depending on the type of accelerator the light may acquire some other important properties for its use as scientific tool. There are several types of accelerator, however, nowadays the most  synchrotron light sources are based in two types. The linear accelerators, used in Free Electron Lasers (FEL) and X-ray Free Electron Lasers (X-FELS), and the synchrotron storage ring accelerators.

In both types of accelerators the light is generated by magnetic devices called insertion devices (IDs) that generates an alternating magnetic field along the particle trajectory which makes them wiggle. 


In the latter, ultra-relativistic charged particles are stored in approximately round machines for hours, hence the name storage rings and they generate radiation 


Ao projetar um acelerador de partículas, primeiramente é feito um estudo da interação entre uma única partícula e os campos externos, gerados pelos elementos magnéticos e elétricos do acelerador. Nesse estudo, aspectos complexos como a dinâmica não linear, erros de campo, multipolos e erros de alinhamento são considerados.

Nos aceleradores atuais, como por exemplo os anéis de armazenamento de fontes de Luz Síncrotron, há uma necessidade crescente de se obter feixes cada vez mais intensos e colimados e menores nas direções transversais . Essas características fazem com que o estudo de outro tipo de interação seja crucial para o
\textit{design} de um acelerador. Essa interação é de origem coletiva, ou seja, está relacionada a efeitos que campos gerados pelas próprias partículas do feixe causam nas outras.

Efeitos coletivos geralmente são pequenos quando comparados com os dos campos externos e são tratados como uma perturbação. Contudo, conforme a intensidade do feixe aumenta, esses campos auto-induzidos se tornam cada vez mais intensos e podem gerar instabilidades coletivas, que fazem com que o feixe seja perdido
ou então sofra oscilações, coerentes ou incoerentes, que deturpam as características desejáveis para a radiação síncrotron gerada.

Desde Janeiro de 2011 está sendo desenvolvido um novo modo de operação para o anel de armazenamento de elétrons do LNLS, o UVX. O objetivo desse projeto é diminuir a emitância da máquina de 100~nm.rad para valores menores que 50~nm.rad.

Um modo de operação já foi desenvolvido teoricamente, simulado, e implementado no anel, como descrito em \cite{Fernando}. Testes com baixa corrente evidenciaram uma diminuição no tamanho horizontal do feixe compatível com o esperado teoricamente. Contudo, quando testado com alta corrente no anel (250~mA), o feixe apresentou uma redução de tamanho horizontal muito menor, além de demonstrar instabilidade e alto acoplamento transversal.

Tais indícios sugeriram que uma instabilidade coletiva tinha sido ativada devido ao aumento da densidade de elétrons no feixe. Tendo isso em mente, começamos a desenvolver um estudo de efeitos coletivos para determinar a instabilidade e tentar curá-la.

Além desse objetivo mais imediato, o estudo de efeitos coletivos também visa adquirir um conhecimento que será muito importante para o design da nova fonte de luz síncrotron do LNLS, o Sirius \cite{Sirius}, haja vista que esses efeitos são os principais limitantes de desempenho das fontes de terceira geração.


%%%%%%%%%%%%%%%%%%%%%%%%%%%%%%%%%%%%%%%%%%%%%%%%%%%%%%%%%%%%%%%%%%%%%%%%%%%%%%%%%%%%%%%%%%%%%%%%%%%%%%%%%
%%%%%%%%%%%%%%%%%%%%%%%%%%%%%%%%%%%%%%%%%%%%%%%%%%%%%%%%%%%%%%%%%%%%%%%%%%%%%%%%%%%%%%%%%%%%%%%%%%%%%%%%%
\section{Modelo do Acelerador}
Um anel de armazenamento de elétrons consiste em uma rede de elementos magnéticos e elétricos que confinam o movimento das partículas em uma órbita fechada para que elas realizem um número elevado de revoluções.

O estudo da dinâmica de interação entre as partículas armazenadas e os campos externos que condicionam seu movimento geralmente é feito em um sistema de coordenadas adequado que mede os desvios das trajetórias em relação à de uma partícula ideal, com energia nominal e posições transversais e longitudinal adequadas.

Esse estudo pode se tornar bastante complicado e por isso um modelo simplificado é usado quando abordamos o tema de instabilidades coletivas. Nesse modelo, apenas as características fundamentais da dinâmica são mantidas, para facilitar a interpretação física e a resolução matemática das equações.

Considera-se que o anel de armazenamento é circular, com raio $R = L/2\pi$, e que os elétrons sofrem oscilações harmônicas nas três direções: longitudinal, $s$, radial, $x$ e vertical $z$, de modo que a equação de movimento é dada por:
\begin{equation}\label{modelo}
 u'' + \omega_u^2 u = 0,
\end{equation}
onde $u = x, z, s$ e $\omega_u$ é a frequência fundamental de oscilação naquela direção, dada por:
\begin{equation}\nonumber
  \omega_u = \frac{\nu_u}{R},
\end{equation}
com $\nu_u$ sendo a sintonia bétatron para as oscilações transversais e síncrotron para as longitudinais.

Quando uma força, além daquelas provenientes dos elementos da rede, atua sobre as partículas, o movimento passa a ser forçado e \eqref{modelo} fica:
\begin{equation}\label{modeloforcado}
 u'' + \omega_u^2 u = \frac{F(t)}{m\gamma},
\end{equation}
supondo que a solução para a equação acima é oscilatória com frequência $\Omega \approx \omega_u$, veremos, por substituição na equação de movimento, que a variação na frequência de oscilação será dada por:
\begin{equation}
\Delta \omega_u \equiv \Omega - \omega_u \approx - \frac{F(t)}{2\omega_u \gamma m u_0}e^{i\Omega t}.
\end{equation}
onde a parte imaginária de $\Delta \omega$ representa um termo exponencial de amortecimento ou excitação no movimento oscilatório. Assim, podemos definir a taxa de crescimento da instabilidade como:
\begin{equation}
 \tau^{-1} \equiv \Im\{\Delta \omega_u\}
\end{equation}
onde taxas de crescimento positivas indicam a existência de instabilidade e taxas negativas implicam em maior amortecimento do feixe.

Os modelos de efeitos coletivos que serão discutidos terão a finalidade de descrever força $F(t)$, de forma que possamos analisar o desvio de frequência e determinar a ocorrência ou não da instabilidade.

Um modelo um pouco mais elaborado que o descrito aqui consideraria os mecanismos de amortecimento natural do feixe, de forma que a existência da instabilidade dependeria de sua taxa de crescimento ser maior ou menor que a taxa de amortecimento natural.

%%%%%%%%%%%%%%%%%%%%%%%%%%%%%%%%%%%%%%%%%%%%%%%%%%%%%%%%%%%%%%%%%%%%%%%%%%%%%%%%%%%%%%%%%%%%%%%%%%%%%%%%%
%%%%%%%%%%%%%%%%%%%%%%%%%%%%%%%%%%%%%%%%%%%%%%%%%%%%%%%%%%%%%%%%%%%%%%%%%%%%%%%%%%%%%%%%%%%%%%%%%%%%%%%%%
\section{Efeitos Coletivos}
A figura de mérito de uma Fonte de Luz Síncrotron é o brilho da luz gerada, que pode ser definido como:
\begin{equation}\label{eq:defbrilho}
 B=\frac{\dot{N}_\gamma}{4\pi^2\sigma_x\sigma_y\sigma_{x'}\sigma_{y'}\mathrm{d}E
/E_\gamma}\quad \left(\mathrm{\frac{\text{fótons} \cdot s^{-1}}{mm^2 \cdot
mrad^2 \cdot 0,1\% bandwidth}}\right)
\end{equation}
onde $\sigma_u$ e $\sigma_{u'}$, com $u= x, y$, são os desvios padrão das distribuições espaciais e angulares do feixe de elétrons, $\dot{N}_\gamma$ é o fluxo de elétrons integrado em um intervalo de 0.1\% de banda de energia, $\mathrm{d}E/E_\gamma$.

Assim, estamos interessados em determinar e evitar todos os efeitos coletivos que degradem o brilho da máquina, que podem ser
\begin{itemize}
 \item Oscilações transversais: que aumentam o tamanho e a divergência do feixe de elétrons.
 \item Oscilações longitudinais: que implicam em um aumento do espalhamento de energia, $\sigma_E$, dos elétrons e um consequente aumento do tamanho horizontal, além de alargamento das linhas de energia dos onduladores.
 \item Instabilidades: limitam a corrente máxima do anel, $I$, e consequentemente o fluxo de fótons, $\dot{N}_\gamma \sim I$.
\end{itemize}

Os efeitos coletivos se manifestam de duas maneiras: a primeira é por meio da interação eletromagnética direta entre as partículas, denominada efeito das cargas espaciais (\textit{space charge} em Inglês) e a outra é através de \textit{Wake Fields}, que são o resultado da interação dos campos eletromagnéticos gerados pelos elétrons com as paredes mais próximas e atuam em partículas subsequentes \cite{Khan}.

O efeito das cargas espaciais é o mais conhecido dentre os efeitos coletivos, mas é pequeno para anéis de armazenamento de elétrons, porque ele decresce com o aumento da energia relativística $\gamma$. Tal propriedade pode ser percebida calculando a força de atração que uma linha de corrente com densidade de carga $\lambda$ uniforme, se movendo com velocidade $\beta c$ exerce sobre uma carga $q$  a uma distância $r$ viajando paralelamente com mesma velocidade:
\begin{equation}
 F= q(E - \beta B)=\frac{q\lambda}{2\pi \epsilon_0 \gamma^2 r},
\end{equation}
onde fica evidente o comportamento decrescente da força de interação com o aumento da energia.

Por outro lado, os \textit{Wake Fields} não dependem da energia do feixe, mas sim da geometria da câmara de vácuo e das características dos materiais que a constituem.

Um dos tipos de \textit{Wake Fields} mais conhecidos, que possui uma solução analítica simples de ser obtida é o da parede resistiva. Este problema consiste em uma câmara de vácuo cilíndrica de raio $b$, infinitamente espessa e longa, formada por um material de condutividade $\sigma$. A uma distância $a<b$ do seu centro há uma carga $q$ viajando longitudinalmente com velocidade $c$ e a uma distância $z$ atrás dela há outra carga $q$, também viajando com velocidade $c$.

A resolução desse problema envolve expandir as densidades de carga e de corrente que geram os \textit{Wake Fields} em termos de anéis de carga  concêntricos ao eixo de simetria da câmara de vácuo:
\begin{align}
\rho(r,\theta,s,t)  = \sum_{n=0}^\infty \rho_m &=  \sum_{m=0}^\infty
\frac{I_m}{(1+\delta_{0m})\pi a^{m+1}} \delta(r-a) \delta(s-ct) \cos(m\theta)\\
\vec{j}(r,\theta,s,t) = \sum_{m=0}^\infty\vec{j}_m &=\sum_{m=0}^\infty c \rho_m
\hat{s}
\end{align}
onde $I_m = q a^m$ é o momento de multipolo de ordem $m$ da partícula. Usar essa base para expansão da carga facilita bastante os cálculos porque o problema se resume a resolver as Equações de Maxwell para a m-ésima componente dos campos.

Como a câmara de vácuo não é perfeitamente condutora, há campos não nulos nessa região também. Nesse caso, as fontes são dadas por:
\begin{equation}
 \rho = 0, \qquad \vec{j} = \sigma \vec{E}.
\end{equation}

Após escrever as Equações de Maxwell em coordenadas cilíndricas e substituir as densidades de carga e corrente nas expressões, é possível determinar a dependência em $\theta$ dos campos. Também, como há simetria longitudinal, os campos não dependem da posição relativa ao anel $s$, mas apenas da distância da partícula fonte, $z=s-ct$, de modo que $z>0$ corresponde a posições a frente da fonte.

Devido a essa propriedade, que acopla o tempo e a coordanada longitudinal, é interessante fazer uma Transformada de Fourier em $z$ para transformar a equação diferencial parcial em $r,t$ e $s$ em uma ordinária apenas em $r$. Assim, escrevemos:
\begin{align}
 (E_s,E_r,B_\theta)(r,\theta,z) &= \cos(m\theta)
\int_{-\infty}^\infty\!\!\frac{dk}{2\pi}(\tilde{E}_s,\tilde{E}_r,\tilde{B} _\theta)(r,k),\\
 (B_s,B_r,E_\theta)(r,\theta,z) &= \sin(m\theta)
\int_{-\infty}^\infty\!\!\frac{dk}{2\pi}(\tilde{B}_s,\tilde{B}_r,\tilde{E} _\theta)(r,k).
\end{align}

Aplicando as condições de contorno de que o campo não pode divergir em $r=0$ e de continuidade dos campos tangenciais na parede da câmara, obtém-se as expressões para todas as regiões do espaço. Dessa forma, é possível determinar a força que atua em uma partícula teste a uma distância $z$ atrás da fonte:
\begin{align}\nonumber \label{eq:res.wall}
(F_\lVert)_m = \frac{eI_m}{\pi b^{2m+1}(1+\delta_{0m})}
\sqrt{\frac{c}{\sigma}}r^m \cos(m\theta) \frac{1}{|z|^{3/2}}\\
(\vec{F}_\perp)_m = \frac{2eI_m}{\pi b^{2m+1}} \sqrt{\frac{c}{\sigma}}
m r^{m-1} \frac{1}{|z|^{1/2}} \left(\hat{r}\cos(m\theta) -\hat{\theta}
\sin(m\theta)\right).
\end{align}
A solução acima é uma aproximação válida para uma região
\begin{displaymath}
\eta^{1/3}b\ll|z|\ll \frac{b}{\eta}, \qquad \eta \equiv \frac{c}{4\pi \sigma b},
\end{displaymath}
sendo que para uma câmara de vácuo de alumínio com $5 cm$ de raio, temos que $6\times10^{-6} m\ll |z| \ll 3\times10^7 m$. A solução exata, assim como a resolução detalhada desse problema pode ser encontrada em \cite{Chao}.

A simplicidade de \eqref{eq:res.wall} e facilidade na resolução são consequências das simetrias longitudinal e axial do problema original, que permitiram o desacoplamento dos multipolos e a dependência apenas em $|z|$. Quando uma dessas simetrias é quebrada, a determinação dos campos e das forças que atuam sobre o feixe fica muito complicada, de modo que aproximações devem ser feitas.

Uma dessas aproximações é a de feixe rígido, que já foi usada implicitamente na solução da parede resistiva. Ela consiste em não considerar o efeito causado pela ação do \textit{Wake Field} no feixe, ou seja, tanto as posições transversais como a longitudinal das partículas do feixe são fixas. Apesar dessa aproximação não ser auto-consistente e não prever o surgimento de instabilidades coletivas, ela é valida para anéis de armazenamento de elétrons, em que $\beta \approx 1$, e permite obter expressões  que são usadas posteriormente em modelos de instabilidades.

%%%%%%%%%%%%%%%%%%%%%%%%%%%%%%%%%%%%%%%%%%%%%%%%%%%%%%%%%%%%%%%%%%%%%%%%%%%%%%%%%%%%%%%%%%%%%%%%%%%%%%%%%
\section{\textit{Wake Functions}}

Em geral, para seções da câmara de vácuo que não possuem simetria translacional, o \textit{Wake Field} e a \textit{Wake Force} que atua sobre uma partícula prova passam a depender das variáveis $s$ e $t$ separadamente e as equações se tornam muito complicadas para serem resolvidas.

Para tentar recuperar a dependência apenas de $z = s - \beta c t$, ao invés de tentarmos determinar a força instantânea que partícula sente, vamos olhar para o impulso recebido quando ela passa por toda a seção geradora do \textit{wake field}. Matematicamente, usando a aproximação de feixe rígido, temos:

\begin{equation}
\beta c \Delta\vec{ p}(x,y,z) = \int^{L/2}_{-L/2}\!\! ds
\vec{F}(x,y,s,(s-z)/\beta c)
\end{equation}
onde o $L/2$ nos limites de integração não representa o comprimento da seção que gera o \textit{Wake Field} mas sim um comprimento adequado para calcular a integral de modo que os campos elétrico e magnético nesse ponto sejam iguais, ou seja, não sofram mais o efeito da assimetria gerada pela estrutura. Por exemplo, se a estrutura em questão é periódica, o limite de integração pode ser o período da estrutura, se é do tipo cavidade, o limite de integração deve ser muito maior que o tamanho da cavidade.

As duas aproximações feitas acima (feixe rígido e limites de integração) impõem restrições sobre o momento que a partícula recebe dos \textit{Wake Field}. Tais restrições são evidenciadas pelo Teorema de \textit{Panofsky-Wenzel} \cite{Bio}:
\begin{equation}
 \vec{\nabla} \times \Delta\vec{ p} = \vec{0}
\quad \Rightarrow \left\lbrace 
\begin{array}{l}
(\vec{\nabla} \times \Delta \vec{p})_s = 0
\xrightarrow{coord.~cartesianas} \frac{\partial \Delta p_x}{\partial y}
= \frac{\partial \Delta p_y}{\partial x} \\
\frac{\partial \Delta \vec{ p}_\bot}{\partial z} =
\vec{\nabla}_\bot \Delta p_s
\end{array} \right.
\end{equation}
onde $\Delta\vec{ p}$ é o momento linear ganho por uma partícula de prova a uma distância $z$ atrás da fonte.

Na demonstração do teorema acima são usadas apenas as equações de Maxwell, a força de Lorentz e as duas aproximações. Portanto, ele não depende das condições de contorno e nem da fonte, além de não exigir que $\beta = 1$ (apenas que ele seja alto o suficiente para a aproximação de feixe rígido ser válida).

Há um corolário do Teorema de \textit{Panofsky-Wenzel} que afirma:
\begin{equation}
 \beta = 1\quad \Rightarrow \quad \vec{\nabla}_\bot \cdot
\Delta\vec{ p}_\bot = 0.
\end{equation}

Devido a sua generalidade e poder de simplificação, esse teorema é fundamental para o estudo de efeitos coletivos. Uma aplicação direta pode ser obtida o escrevendo em coordenadas cilíndricas:
\begin{align}
 \frac{\partial }{\partial r} (r\Delta p_\theta)& = \frac{\partial}{\partial
\theta}(\Delta p_r) \\
 \frac{\partial}{\partial z}(\Delta p_\theta) &= \frac{1}{r}
\frac{\partial}{\partial \theta}(\Delta p_s) \\
 \frac{\partial}{\partial z}(\Delta p_r) &= \frac{\partial}{\partial r}(\Delta
p_s) \\
(\beta = 1) \quad  \frac{\partial}{\partial r}(r\Delta p_r) &=
\frac{\partial}{\partial \theta}(\Delta p_\theta).
\end{align}

As equações acima podem ser facilmente resolvidas de modo que a solução pode ser convenientemente escrita da seguinte forma:
\begin{align}
\label{eq:W}
 \beta c \Delta \vec{p}_\bot &= -q \sum_{m=0}^\infty I_m W_m(z) m
r^{m-1}\left(\hat{r}\cos(m\theta) - \hat{\theta}\sin(m\theta)\right)\\
\label{eq:W'}
\beta c \Delta p_s &= -q \sum_{m=0}^\infty I_m W'_m(z) r^{m}\cos(m\theta)
\end{align}
onde $I_m$ é o m-ésimo multipolo da distribuição de carga\footnote{para uma carga pontual a uma distância $a$ do eixo de simetria $I_m=qa^m$.} que gera o \textit{Wake Field} e $q$ é a carga da partícula de prova. As funções $W_m$ e $W'_m$, que estão relacionadas entre si por
\begin{equation}
 W'_m(z)=\frac{d}{dz}W_m(z),
\end{equation}
são chamadas \textit{Wake Functions} longitudinal e transversal, respectivamente, e são determinadas pelas condições de contorno do problema, devendo ser resolvidas independentemente.

Apesar de a forma exata das \textit{Wake Functions} depender do problema específico a ser tratado, é possível inferir algumas de suas propriedades gerais, assim como algumas propriedades do impulso.

A primeira propriedade é devida à causalidade: $W_m(z) = 0$ e $W'_m(z)=0$ para $z>0$, ou seja, uma partícula que está a frente do feixe não sofre ação do \textit{Wake Field} gerado por ele. Apesar dessa característica ser restrita para $\beta = 1$ ela é sempre aplicada, por ser uma aproximação válida em anéis
reais.

Outras propriedades envolvem o fato delas serem reais e possuírem dimensão $[W'_k] = V/C/m^{2k}$ e $[W_k] = V/C/m^{2k-1}$, sendo que essa última propriedade é importante porque, para problemas com simetria rotacional, a m-ésima componente do impulso transversal é escalada por $(a/b)^{2m-1}$ e do longitudinal por $(a/b)^{2m}$, onde $a$ é o tamanho transversal do feixe e $b$ o raio da câmara de vácuo. Como $a\ll b$, os modos mais baixos dominam, de forma que a maioria das instabilidades longitudinais são geradas por $W'_0$ e a maioria das instabilidades transversais por $W_1$\footnote{Visto que $W_0$ não tem sentido físico por $(\Delta \vec{p}_\bot)_0 = \vec{0}$ (ver \eqref{eq:W}).} \cite{Chao, Bio}.

Ainda, por argumentos de conservação de energia \cite{Chao, Bio} é possível demonstrar que $ W'_m(0^-)>0$ e decresce conforme $|z|$ aumenta, pois uma partícula imediatamente atrás de uma outra deve sofrer uma força retardadora, caso contrário o sistema ganharia energia indefinidamente. Isso implica, pelo Teorema de \textit{Panofsky-Wenzel}, que $W_m(z)$ começa em $0$ (não demonstrável, mas a maioria das \textit{Wake Functions} seguem essa regra) e cresce negativamente e monotonicamente com $|z|$, para $|z|$ pequeno. Essa análise permite concluir que em anéis onde predominam \textit{Wake Fields} transversais, feixes curtos são preferíveis e em máquinas onde os longitudinais dominam feixes longos são melhores.

Uma característica importante da transferência de momento linear que podemos tirar de \eqref{eq:W'} ocorre quando $m=0$, ou seja, no caso da força monopolar. Nesse caso, $\Delta p_s$ é independente das coordenadas transversais da partícula prova, de modo que todas as partículas em um determinado plano sentem a mesma força, proporcional à $W'_0$. Esse tipo de impulso pode levar a um auto-empacotamento do feixe ou então à \textit{microwave instability}.

Ainda, para $m=1$ (dipolo), o impulso transversal é constante com as coordenadas transversais, enquanto o impulso longitudinal depende linearmente da distância $x$ (considerando $\hat{x}$ a direção em que a carga geradora foi posta), de modo que partículas em lados opostos do eixo sofrem impulsos opostos. Esse tipo
de característica distorce o feixe em um formato de banana e pode levar à instabilidade chamada \textit{Beam Breakup} em que o feixe é distorcido até atingir a câmara de vácuo.

\subsection{Impedâncias}

Assim como foi feito na resolução da \textit{resistive wall} é interessante descrever os \textit{wake fields} no domínio da frequência, pois o tratamento analítico das equações geralmente é simplificado quando estamos nesse espaço, além do que a interpretação das equações também se torna mais simples nesse caso, como é o caso das estruturas ressonantes, como a cavidade de RF, que permitem que apenas campos com frequências específicas ressoem por um longo tempo, e possam agir sobre o feixe, gerando instabilidades.

Desse modo, podemos definir impedância como a Transformada de Fourier das
\textit{Wake Function}:
\begin{align}
Z^\lVert_m (\omega)&=\int^\infty_\infty\!\!\frac{dz}{c}e^{-i\omega z/c}W'_m(z)\\
Z^\perp_m (\omega)&= i\int^\infty_\infty \!\! \frac{dz}{c}e^{-i\omega z/c}W_m(z)
\end{align}
onde $Z^\lVert_m$ é a impedância longitudinal e $Z^\perp_m$ a impedância transversal.

Algumas propriedades gerais da impedância são dadas abaixo:
\begin{itemize}
 \item $Z_m^\lVert(-\omega) = [ Z_m^\lVert(-\omega)]^*$ e $Z_m^\perp(\omega) = - [Z_m^\perp(\omega)]^*$ : devido ao fato de $W_m(z)$ ser real;
\item $Z_m^\Vert(\omega) = \frac{\omega}{c}Z_m^\perp(\omega)$, para coordenadas cilíndricas: devido ao Teorema de \textit{Panofsky-Wenzel};
\item $\Re\{Z_m^\Vert(\omega)\}\geq 0$ e $\Re\{Z_m^\perp(\omega)\}\geq 0$, se $\omega\geq 0$: porque a energia do feixe não pode aumentar sem a existência de forças externas.
\end{itemize}

A interpretação física das impedâncias é análoga à da impedância de um circuito elétrico. Para ilustrar a validade dessa analogia, vejamos o seguinte exemplo: considere um feixe com momento m-polar que gera um \textit{Wake Field} dado por
\begin{equation}
 P_m = \hat{P}_m e^{-i\omega(t-s/v)} \nonumber
\end{equation}
e uma partícula com carga $q$ a uma distância $z$ da fonte. O potencial elétrico longitudinal sentido por essa partícula será (ver\eqref{eq:W'}) \cite{Bio}:
\begin{equation}\label{eq:pot.imp}
\hat{V} =-\frac{\hat{P}_m I_m}{q} Z_m^\lVert
\end{equation}
onde $I_m$ é o m-ésimo multipolo da partícula teste. Para a componente monopolar \eqref{eq:pot.imp} se reduz à $V = Z^\lVert_0 P_0$, que é idêntica à expressão para um circuito elétrico.

Em referência a essa analogia, impedâncias de cavidades, que possuem picos de ressonância para determinadas frequências são modeladas como um circuito RLC em paralelo:
\begin{equation}
 Z_m^\lVert(\omega)= \frac{R_s}{1+iQ
\left(\frac{\omega_R}{\omega}-\frac{\omega}{\omega_R}\right)}
\end{equation}
onde $\omega_R = 1/\sqrt{LC}$ é a frequência de ressonância e $Q=R_s \sqrt{C/L}$ é o fator de qualidade da cavidade. Mesmo em casos em que a geometria não é do tipo estrutura ressonante, é comum classificar as impedâncias em capacitiva ($\Im\{Z_m^\lVert\}>0$) ou indutiva ($\Im\{Z_m^\lVert\}<0$), dependo do sinal de sua parte imaginária \cite{Chao}.

Outra propriedade da impedância está ligada a perda de energia. Existe três meios pelos quais um feixe pode perder energia em um anel de armazenamento: por emissão de radiação, colisão com gases residuais e por \textit{Wake Fields}; sendo que essa última é chamada de perda parasita.

Para mostrar a relação da perda parasita com a impedância, consideremos que o feixe possui uma distribuição de carga 
\begin{equation}\nonumber
\rho(\tau)= e N\lambda(\tau),\qquad \text{com} \quad\int_{-T_0/2}^{T_0/2} \!\! d \tau \lambda(\tau) = 1,
\end{equation}
onde $\tau$ é o avanço das partículas em relação à síncrona, $N$ é o número de partículas no feixe, $e$ é a carga elétrica de cada partícula e $T_0$ é o período de revolução no anel. Segundo \eqref{eq:W'}, a energia ganha por uma partícula do feixe devido a interação com os \textit{Wake Field} gerados por toda a distribuição em uma volta é
\begin{equation}\nonumber
\Delta \epsilon(\tau) = -e^2 N \int_{-T_0/2}^{\tau} \!\! d\tau'
W'_0(\tau-\tau')\lambda(\tau') = -e^2 N \omega_0\sum_{n=-\infty}^{\infty}
Z_0^\lVert(n\omega_0) \tilde{\lambda}_n e^{in\omega_0\tau}
\end{equation}
onde na última igualdade foi usado o Teorema da Convolução, e
\begin{displaymath}
 \tilde{\lambda}_n =
\frac{1}{2\pi}\int_{-T_0/2}^{T_0/2}\!\!d\tau\lambda(\tau)e^{-in\omega_0\tau}
\end{displaymath}
com $\omega_0 = 2\pi/T_0$. Assim, a energia média ganha por partícula por volta é
\begin{equation}\nonumber
\overline{\Delta \epsilon} = \int_{-T_0/2}^{T_0/2}\!\! d\tau \lambda(\tau)
\Delta \epsilon(\tau) = -e^2 N\omega_0 \sum_{n=-\infty}^{\infty}
Z_0^\lVert(n\omega_0) |\tilde{\lambda}_n|^2.
\end{equation}
Contudo, como $\lambda(\tau)$ é real, $|\tilde{\lambda}_n|^2$ é par; e como $ Z_0^\lVert(-\omega) = [Z_0^\lVert(\omega)]^*$, apenas a parte real da impedância contribui para a perda de energia:
\begin{equation}\label{eq:energiaperdida}
\overline{\Delta \epsilon} = -e^2 N\omega_0 \sum_{n=-\infty}^{\infty}
\Re\{Z_0^\lVert(n\omega_0)\} |\tilde{\lambda}_n|^2.
\end{equation}

Analisando \eqref{eq:energiaperdida} fica mais evidente a analogia da impedância com àquela de circuitos elétricos, pois apenas a parte real contribui com a dissipação de energia pelo feixe, enquanto a parte imaginária vai gerar alterações na fase deste. Também, nota-se que a energia perdida por partícula cresce linearmente com o número de elétrons no feixe.

%%%%%%%%%%%%%%%%%%%%%%%%%%%%%%%%%%%%%%%%%%%%%%%%%%%%%%%%%%%%%%%%%%%%%%%%%%%%%%%%%%%%%%%%%%%%%%%%%%%%%%%%%
%%%%%%%%%%%%%%%%%%%%%%%%%%%%%%%%%%%%%%%%%%%%%%%%%%%%%%%%%%%%%%%%%%%%%%%%%%%%%%%%%%%%%%%%%%%%%%%%%%%%%%%%%
\section{Modelos de Instabilidades Coletivas}

Podemos classificar os \textit{Wake Fields} de acordo com seu alcance. Campos que ressoam e decaem lentamente, correspondentes a bandas finas de impedância, são de longo alcance, enquanto aqueles que decaem rápido, correspondentes a bandas largas de impedância, são chamados de campos de curto alcance.

Considerando essa classificação,as instabilidades coletivas também podem ser dividas em duas classes:
\begin{itemize}
 \item Instabilidades tipo Robinson ou inter-pacotes: resultantes da ação de \textit{Wake Fields} de longo alcance, geram oscilações dos pacotes, com amplitude proporcional à corrente do feixe. Apesar de existirem para todos os valores de corrente, na prática elas são visíveis para correntes nas quais a taxa de crescimento das amplitudes de oscilação é maior que a dos mecanismos de amortecimento natural do feixe. 
 \item Instabilidades de modo acoplado ou intra-pacotes: são fenômenos que relacionam partículas de um mesmo pacote. Neste caso, a instabilidade ocorre acima de um determinado valor de corrente do feixe.
\end{itemize}

A solução exata do comportamento dos elétrons estocados, sujeitos a forças de interação entre si, só é possível através da resolução da equação de Vlasov \cite{Chao, Bio}, que considera uma distribuição contínua de partículas no espaço de fase. Todavia, modelos de macro-partículas, que consideram  que os
pacotes do feixe consistem em poucas partículas interagindo, geralmente uma ou duas, são fáceis de resolver, fornecem resultados satisfatórios para a maioria dos casos e possibilitam uma interpretação física dos fenômenos envolvidos. 

%%%%%%%%%%%%%%%%%%%%%%%%%%%%%%%%%%%%%%%%%%%%%%%%%%%%%%%%%%%%%%%%%%%%%%%%%%%%%%%%%%%%%%%%%%%%%%%%%%%%%%%%%
\subsection{Modelos de uma partícula}

Considera-se que cada pacote do feixe é formado por apenas uma macro-partícula, não apresentando estrutura interna. Esses modelos são bons para descrever as instabilidades tipo Robinson, pois a consideração feita é válida quando as partículas de um mesmo pacote sofrem oscilações coerentes. Tendo isso em mente, vamos analisar as direções longitudinal e transversal separadamente.

%%%%%%%%%%%%%%%%%%%%%%%%%%%%%%%%%%%%%%%%%%%%%%%%%%%%%%%%%%%%%%%%%%%%%%%%%%%%%%%%%%%%%%%%%%%%%%%%%%%%%%%%%
\subsubsection{Longitudinal}

Primeiramente, consideremos que apenas um pacote esteja circulando no anel. Consideremos também que apenas a \textit{Wake Function} monopolar ($m=0$) contribui. Para esse caso, a força que atua sobre o pacote é dada por \cite{Chao}:
\begin{equation}\label{eq:1partforcalongit}
 \frac{F(t)}{z_0} \approx \frac{N^2 e^2 \eta}{z_0cT_0}\sum^\infty_{k=-\infty}
 W'_0 (kT_0) + (z(t-kT_0)-z(t))(W'_0)'(kT_0)
\end{equation}
onde $\eta$ é aproximadamente o fator compactação de momento da rede magnética, $N$ é o número de partículas no pacote e a soma foi estendida até infinito pela propriedade de causalidade da \textit{Wake Function}. Desse modo, a variação da frequência é dada por:
\begin{equation}
 \Delta \omega = -\frac{Ne^2 \eta}{2\omega_s\gamma m_e c T_0}
\sum^\infty_{k=-\infty} \left(e^{i\Omega k T_0}-1\right)(W'_0)'(kT_0).
\end{equation}
onde nota-se que o primeiro termo de \eqref{eq:1partforcalongit} não foi considerado, por se tratar de um efeito estático, que apenas desloca a posição de equilíbrio do feixe.

Em termos da impedância, a taxa de crescimento da instabilidade pode ser escrita da seguinte maneira:
\begin{equation}\label{long1b}
 \tau^{-1}=\frac{Ne^2 \eta}{2\omega_s E T_0^2} \sum^\infty_{p=-\infty}
(p\omega_0+\omega_s)\Re \{Z^\lVert_0(p\omega_0+\omega_s)\}
\end{equation}

Em um anel de armazenamento, grande parte da impedância longitudinal é oriunda da cavidade de radio-frequência, ou seja, é do tipo banda fina. Para esse tipo de impedância, nota-se que apenas dois termos da somatória, os valores positivo e negativo de $p$ que mais se aproximam da ressonância, contribuem significativamente para a taxa de crescimento, ver Figura \ref{fig:narrow}, de modo que \eqref{long1b} pode ser aproximada para: 
\begin{equation} \label{narrowband}
 \tau^{-1}=\frac{Ne^2 \eta}{2\omega_s E T_0^2}|p|\omega_0
\Re\{-Z^\lVert_0(-|p|\omega_0 +\omega_s) + Z^\lVert_0(|p|\omega_0 +\omega_s)\}.
\end{equation}
Assim, se o pico da impedância for maior que $p\omega$ a soma será positiva e haverá instabilidade, caso contrário haverá amortecimento.

% \begin{figure}[!t]
% \center
%  \includegraphics[scale=0.5]{Imagens/narrowband.png}
%  \caption{Ilustração de \eqref{long1b} aplicada a uma impedância de banda fina, em cinza. Cada ponto corresponde a um termo da somatória. Nesse exemplo, a soma dos termos é positiva e, consequentemente a taxa de crescimento. (retirado de \cite{Khan}).}
%  \label{fig:narrow}
% \end{figure}

Especificamente para o caso da cavidade RF, sabemos que seu modo fundamental está próximo de $h\omega$. Assim, se o harmônico fundamental da cavidade for um pouco maior que o produto acima, haverá uma instabilidade longitudinal. Essa instabilidade foi a primeira a ser estudada e é chamada de Instabilidade de Robinson.

Considerando que $h$ pacotes compõem o feixe, eles oscilarão em um de seus auto-modos, em que a diferença de fase entre sucessivos pacotes é dada por $2\pi\mu/h$, com $\mu=0,...,h-1$. assim, as taxas de crescimento devem ser calculadas para cada modo:
\begin{equation}
\tau^{-1}_\mu=\frac{h N e^2 \eta}{2\omega_s E T_0^2}
\sum^\infty_{p=-\infty} (ph\omega_0+\mu\omega_0+\omega_s)
\Re\{Z^\lVert_0(ph\omega_0+\mu\omega_0+\omega_s)\}
\end{equation}

Esses modos podem interagir, por exemplo, com os HOMs das cavidades de radio-frequência e produzir instabilidades que vão ou não surgir, dependendo do auto-modo de ressonância do feixe.

%%%%%%%%%%%%%%%%%%%%%%%%%%%%%%%%%%%%%%%%%%%%%%%%%%%%%%%%%%%%%%%%%%%%%%%%%%%%%%%%%%%%%%%%%%%%%%%%%%%%%%%%%
\subsection{Transversal}

Considerando novamente a situação \textit{single-bunch} e que apenas a \textit{wake function} transversal dipolar ($m=1$) contribui, a força que atua sobre o pacote será:
\begin{equation}\label{eq:trans1b}
 \frac{F(t)}{y_0}=\frac{N^2 e^2}{y_0 c T_0} \sum^\infty_{k=-\infty}
y(t-kT_0)W_1(kT_0)
\end{equation}
o que implica em uma variação de frequência dada por:
\begin{equation}
 \Delta \omega = - i\frac{Ne^2}{2\omega_\beta \gamma m_e c T^2_0}
\sum^\infty_{p=-\infty} Z^\perp_1 (p \omega_0 + \Omega)
\end{equation}
de modo que a taxa de crescimento é:
\begin{equation}\label{resistive}
\tau^{-1} = -\frac{Ne^2 c}{2\omega_\beta E T^2_0} \sum^\infty_{p=-\infty}
\Re\{Z^\perp_ 1 (p \omega_0 + \omega_\beta)\}
\end{equation}

Para a expressão acima, é importante analisar dois tipos de impedâncias. As de banda fina, que podem gerar uma versão transversal da instabilidade de Robinson, e as de parede resistiva, que além de causarem instabilidades entre-pacotes, cobrem um longo campo de frequências, caindo com $1/\sqrt{|\omega|}$, o que exige a consideração de vários termos na soma para obter a taxa de crescimento com precisão (ver Figura \ref{fig:resistive}).

% \begin{figure}[!h]
% \center
%  \includegraphics[scale=0.5]{Imagens/resistive.png}
%  \caption{Ilustração de \eqref{resistive} assumindo uma impedância de parede resistiva, em cinza. Cada ponto corresponde a um termo da somatória. Nesse exemplo, a soma dos termos é negativa e a taxa de crescimento positiva.  (retirado de \cite{Khan}).}
%  \label{fig:resistive}
% \end{figure}

Para o caso da impedância de parede resistiva, a soma de termos positivos e negativos correspondentes, leva a conclusão de que para valores de sintonia cuja parte fracionária é menor que $0.5$ o movimento é estável, enquanto valores maiores são instáveis \cite{Khan}.

De modo similar à direção longitudinal, $h$ pacotes equidistantes oscilarão em seus auto-modos. Também, de acordo com \eqref{eq:trans1b}, a taxa de crescimento transversal é dada por:
\begin{equation}
\tau^{-1}_\mu = \frac{h Ne^2 c}{2\omega_\beta E T^2_0} \sum^\infty_{p=-\infty}
\Re\{Z^\perp_ 1 (p h\omega_0 +\mu\omega_0 + \omega_\beta)\}.
\end{equation}

Neste caso, a taxa de crescimento é sempre positiva para impedâncias de parede resistiva, ou seja, sempre haverá instabilidade de parede resistiva para modos de operação \textit{multi-bunch}. Ainda, simulações mostram que o sistema tente a evoluir para o modo de oscilação que maximiza a taxa de crescimento da instabilidade \cite{Khan}.

%%%%%%%%%%%%%%%%%%%%%%%%%%%%%%%%%%%%%%%%%%%%%%%%%%%%%%%%%%%%%%%%%%%%%%%%%%%%%%%%%%%%%%%%%%%%%%%%%%%%%%%%%
\subsection{Modelos de poucas partículas}

Neste tipo de modelos, considera-se que os pacotes são formados por duas ou mais partículas distribuídas longitudinalmente que interagem entre si. São aplicados, principalmente, no estudo de instabilidades do tipo intra-pacotes, geradas por \textit{wake fields} de curto alcance. Aqui serão considerados apenas modelos de duas macro-partículas, uma na parte da frente do pacote, chamada de cabeça , e outra na parte de trás, chamada cauda.

O movimento transversal acoplado dessas duas macro-partículas é modelado da seguinte forma:

\begin{align}
 \ddot{y}_1(t)+\omega^2_\beta(\delta_E)y_1(t)&=0& \\
 \ddot{y}_2(t)+\omega^2_\beta(\delta_E)y_2(t)&=\frac{Ne^2 W_1}{2\gamma c
T_0m_e}y_1.&
\end{align}
onde $y_1$ se refere à cabeça e $W_1$ é assumido constante no comprimento do pacote. As equações acima são válidas para a primeira metade do movimento síncrotron, sendo que na segunda metade ($t>T_s/2$) as partículas trocam de posição e as equações também devem ter seus índices trocados.

A variação da sintonia bétatron com a energia pode ser expressa por:
\begin{equation}
 \omega_\beta(\delta_E)=\omega_\beta(0)+\omega_0 \xi \delta_E =
\omega_\beta(0)+\omega_0 \xi \frac{\omega_s\hat{z}}{c \eta} \cos(\omega_st)
\end{equation}
onde $\omega_s$ é a frequência síncrotron e $\xi$ é a cromaticidade.

Resolvendo o sistema de equações, chega-se a:
\begin{equation}
\left(\begin{array}{c}\tilde{y}_1 \\ \tilde{y}_2 \end{array} \right)_{t=T_s/2}
=  e^{-i\omega_\beta T_s/2} 
\left(\begin{array}{cc} 1 & 0 \\ ia & 1 \end{array}\right)
\left(\begin{array}{c}\tilde{y}_1 \\ \tilde{y}_2 \end{array} \right)_{t=0}
\end{equation}
onde
\begin{equation}
 a = \frac{\pi N e~2 W_1}{4\gamma c T_0 m_e \omega_\beta \omega_s}\left(
1+i\frac{4 \xi \omega_0 \hat{z}}{\pi c \eta}\right).
\end{equation}

Após uma revolução síncrotron completa teremos:
\begin{equation}
\left(\begin{array}{c}\tilde{y}_1 \\ \tilde{y}_2 \end{array} \right)_{t=T_s}
=  e^{-i\omega_\beta T_s} 
\left(\begin{array}{cc} 1-a^2 & ia \\ ia & 1 \end{array}\right)
\left(\begin{array}{c}\tilde{y}_1 \\ \tilde{y}_2 \end{array} \right)_{t=0}.
\end{equation}

Os autovalores da matriz de transferência acima são dados por:
\begin{align}
 \lambda &= e^{\pm i\phi}\quad \text{com} \quad \cos(\phi) \equiv
1-\frac{a^2}{2}, \quad \text{se} \quad a\leq 2 \\
 \lambda &= e^{\pm i\mu}\quad \text{com} \quad \cosh(\mu) \equiv \frac{a^2}{2}
- 1, \quad \text{se} \quad a > 2.
\end{align}

Com esses resultados é possível analisar qualitativamente dois tipos de instabilidades. A primeira é chamada de efeito \textit{head-tail} (cabeça-cauda em português). No caso em que $|a| << 1$, que é válido para fontes de radiação síncrotron operando no modo \textit{multi-bunch}, a aproximação $\phi \approx a$ é válida. Os auto-vetores correspondentes aos autovalores determinados acima descrevem dois modos de oscilação das macro-partículas com diferenças de fases relativas de 0 e $\pi$.

Considerando $\eta$ positivo, vemos que se $\xi<0$ o modo 0 cresce com uma taxa $\Im\{a\}$ enquanto o modo $\pi$ é amortecido com a mesma taxa. Daí vem a necessidade de introduzir sextupolos nos anéis de armazenamento. O objetivo é manter a cromaticidade nula ou com um valor positivo pequeno, para amortecer o modo 0. Nesse caso, o modo $\pi$ seria excitado, contudo o modelo de duas partículas superestima sua taxa de crescimento, sendo que na prática ela é pequena em comparação com a do modo 0.

Nota-se que a instabilidade \textit{head-tail} ocorre para todos os valores de corrente. Ainda, aplicando-se um alto valor positivo de cromaticidade é possível amortecer oscilações de multi-pacotes, a custa de uma abertura dinâmica e tempo de vida reduzidos.

A outra instabilidade é conhecida por vários nomes na literatura:  \textit{head-tail turbulence, strong head-tail instability, transverse microwave instability, transverse mode coupling}. Quando a condição $|a| << 1$ não é mais válida, o movimento não pode ser mais visto como dois modos separados, de modo que para $\xi = 0$ a amplitude se mantém confinada, se $a<2$. Contudo, para valores maiores que esse, há instabilidade até mesmo com a cromaticidade nula e as taxas de crescimento são da ordem do amortecimento natural por emissão de radiação síncrotron \cite{Khan}.

%%%%%%%%%%%%%%%%%%%%%%%%%%%%%%%%%%%%%%%%%%%%%%%%%%%%%%%%%%%%%%%%%%%%%%%%%%%%%%%%%%%%%%%%%%%%%%%%%%%%%%%%%
%%%%%%%%%%%%%%%%%%%%%%%%%%%%%%%%%%%%%%%%%%%%%%%%%%%%%%%%%%%%%%%%%%%%%%%%%%%%%%%%%%%%%%%%%%%%%%%%%%%%%%%%%
\section{Medidas contra Efeitos Coletivos}

Após ter feito uma análise dos mecanismos que geram instabilidades coleticas em feixes não-contínuos, é possível identificar algumas medidas úteis para evitar esse tipo de problema em anéis de armazenamento.

Uma delas é a minimização da impedância da câmara de vácuo. Como foi estudado, a impedância é determinada pela geometria e condutividade da parede da câmara de vácuo. 

Em anéis de armazenamento há muitos locais em que a câmara é interrompida ou então sofre alterações em sua seção transversal. Geralmente a impedância introduzida por essas descontinuidades pode ser consideravelmente minimizadas se os seguintes critérios forem obedecidos \cite{Khan}:
\begin{itemize}
 \item Interrupções da câmara, como em flanges ou \textit{bellows} devem possuir pontes metálicas para que a corrente imagem flua;
 \item Mudanças inevitáveis no formato da câmara, como em \textit{tapers} nos finais de onduladores, devem ser projetados com ângulos pequenos (tipicamente da ordem de \SI{10}{\degree} ou menos);
 \item Buracos ou fendas na câmara, por exemplo portas de bombas de vácuo, portas de radiação síncrotron, ou o canal de injeção do feixe, devem ser projetados de modo que uma parede lisa fique mais perto do feixe que a interrupção.
\end{itemize}

No caso de cavidades de radio-frequência a maior contribuição para a impedância vem dos modos harmônicos de ordem mais altas, chamados HOMs (\textit{higher-order modes}). Para combater esses efeitos, geralmente emprega-se absorvedores de frequência de bandas largas ou então antenas para extrair modos particulares \cite{YellowCERN95}. Além desses métodos, muito esforço é investido no desenvolvimento de geometrias que possuam HOMs menos intensos.

Para as impedâncias de parede resistiva, geralmente as medidas mais comuns a se tomar são fazer câmaras largas e com material de boa condutividade elétrica. Se a seção transversal tiver que ser pequena, como no caso dos onduladores, a escolha do material da câmara é de importância primordial \cite{Khan}.

As medidas discutidas até agora não se aplicam para os casos em que tem-se uma dada câmara de vácuo, ou seja, uma máquina que era estável, mas devido a alguma mudança no modo operação ou em algum elemento do anel, assim como na corrente estocada, passou a demonstrar instabilidades. Para esses casos, as variáveis
que podem ser alteradas são reduzidas, pois parâmetros como circunferência, energia do feixe, perdas por radiação e frequências síncrotron e bétatron geralmente são dedicadas a outras necessidades.

Um modo eficiente de suprimir instabilidades é por meio do aumento da cromaticidade da máquina, que desloca o espectro do feixe. Contudo, esse método envolve aumentar a força dos sextupolos em regiões dispersivas e, consequentemente, diminuir a abertura dinâmica e o tempo de vida \cite{Khan}.

O efeito de impedâncias de banda fina podem ser minimizados alterando sua frequência, para diminuir a interação com o espectro do feixe. Para cavidades de RF, isso pode ser feito deformando-as mecanicamente, por meio do aumento de sua temperatura, por exemplo \cite{112Khan}. Todavia, deve-se fazer essas alterações tendo sempre mais de um grau de liberdade, para que apenas os HOMs sejam alterados, mantendo o modo fundamental constante.

\textit{Landau damping} é um tipo de amortecimento gerado pelos componentes não lineares dos potenciais definidos pelos campos magnéticos guia e pela voltagem RF. Nas direções transversais ele é introduzido por sextupolos, e na longitudinal pela forma senoidal do potencial. Um amortecimento adicional pode ser introduzido na direção longitudinal adicionando uma voltagem de RF com um múltiplo do modo fundamental, por meio das cavidades Landau \cite{114Khan}.

Além desses modos passivos de amortecimento das instabilidades, há os ativos. Entre eles, estão inclusos a modulação de fase da cavidade RF e o uso de  sistemas de \textit{Feedback}. O primeiro método já é empregado no UVX desde 2003, quando foi instalada uma nova cavidade de RF no anel, para a instalação de dispositivos de inserção, que possuia um HOM muito elevado (ver \cite{rfUVX}).

Os Sistemas de \textit{Feedback}, cujo princípio de funcionamento será discutido na seção \ref{feedback}, são muito empregados nas máquinas atuais para amortecimento de instabilidades, estando presentes na maioria das máquinas de terceira geração. Contudo, eles são úteis em apenas em uma banda estreita de taxas de crescimento, podendo não ser capazes de controlar taxas duas ordens de grandeza maiores que as de amortecimento do feixe \cite{Khan}, além de combaterem apenas instabilidades do tipo Robison. Por isso, os métodos passivos continuam sendo muito importantes para a estabilidade do feixe.




\chapter{Introduction} \label{chap:intro}




Most light sources uses synchrotron storage rings where subatomic charged particles, generally electrons, are extracted from materials and accelerated to relativistic energies in order to produce radiation. Among the several types of accelerators the synchrotron is a machine


and depending on the type of accelerator the light may acquire some other important properties for its use as scientific tool. There are several types of accelerator, however, nowadays the most  synchrotron light sources are based in two types. The linear accelerators, used in Free Electron Lasers (FEL) and X-ray Free Electron Lasers (X-FELS), and the synchrotron storage ring accelerators.

In both types of accelerators the light is generated by magnetic devices called insertion devices (IDs) that generates an alternating magnetic field along the particle trajectory which makes them wiggle.


In the latter, ultra-relativistic charged particles are stored in approximately round machines for hours, hence the name storage rings and they generate radiation

  \section{Synchrotron light sources}

  In scientific facilities commomly known as synchrotron lights sources (SLS) the interaction between light and matter is used to study properties of a variaty of materials. Throught techniques involving absortion, reflection, refraction and scattering of light of different 'colors' by the materials under study, scientists can determine their atomic structure and composition.

  The frequency of the light used in these facilities ranges from tera-hertz to hard X-rays and its origin is always related to synchrotron emission of radiation by charged particles, hence the name of the facility. The light emitted by centripetal acceleration of ultra-relativistic particles has unique properties for use in scientific investigation. Besides its broad spectrum, among the advantages in relation to other methods are the high total flux emitted and the strong collimation.

    \subsection{Types of Light Sources}

    In general we can group syncrotron light facilities in two groups depending on the topology of the accelerators involved: linear and circular. In the linear sources, denominated Free Electron Lasers (FEL), the light is generated by special devices called Insertion Devices (ID) that are put along the straight trajectory of the beam and generates a transverse magnetostatic field with an amplitude that varies sinusoidally with respect to the longitudinal direction. When the beam passes throught this field it wiggles, and synchrotron emission of radiation happens due to its deflection at each wiggle. The light emitted from successive wiggles interferes in such a way that only photons with specific frequencies survive and the resulting radiation has a spectrum where all the energy is concentrated at very thin peaks around multiples of this resonant frequency. The intensity of these peaks is proportional to the number of particles in the beam and the number of wiggles of the ID field and their bandwidth is proportional to the inverse of the number of wiggles, hence these devices are built with the maximum number of periods that are technically possible. Additionally, the polarization of the radiation is defined by the direction of the magnetic field of the ID. For example, if the field is vertical in relation to the ground, the electrons will oscilate horizontally and the radiation will be horizontally polarized. Circular and elliptical polarizations can also be achieved by changing not only the intensity of the field but also its direction as a function of the longitudinal position. In a real insertion device all these properties of the light can be tuned according to the needs of the experiment to be carried out at the beamline, which makes this devices a very powerful tool for scientific experiments.

    Besides all the amazing properties of the insertion devices, in Free Electrons Lasers it is possible to excite the beam to emit light coherently, hence the name of these facilities. There are several techniques to achieve this and their enumeration or explanation is beyond the scope of this work. The important fact is that besides the high level of coherence of the light in these facilities, the intensity of the photon beam is increased by orders of magnitude with these techniques because it becomes proportional to the square of the number of particles in the beam. The excitation of coherent emission becomes increasingly difficult as the energy of the photons increase, because ever smaller errors in the fields or decoherences in the electron beam destroy the cascade of the stimulated emission. This task is so difficult that only in the last decade coherent X-rays where successfully generated by new machines that where built specially for that, the X-FELs. This achievement represented a revolution in the synchrotron radiation community and opened up exciting new possibilities for scientific research.

    Among the several types of circular accelerators we highlight the synchrotron storage rings. There is an abuse of use of the word synchrotron here, so far it meant the nature of emission of the light in these facilities, but when used to describe this type of accelerator it is related to the synchronicity of the revolution time of the electrons with the electromagnetic field that drives their motion, which is the main mechanism that makes these machines work. In this type of light source radiation the electrons are grouped in several bunches that fill the whole storage ring and are confined in close to circular orbits by deflecting and focusing magnetostatic fields for hours. The radiation used in experiments can be generated by the same fields that deflect the beam (dipoles) or by insertion devices put in empty sections of the ring, called straight sections, just like in FELs, except for the coherent emission which was never achieved in storage rings due to technical problems.

    While the experiments in X-FELs are performed with only one strong single pulse of radiation that can be generated at a repetition rate of a few hundreads of Hertz, in storage rings the ligth emitted by the bunchs that fill the machine continuously hit the samples and the interaction patterns are recorded for as long as needed to achieve the desired resolution for the experiment. Besides, while in X-FELs it is only possible to have one beamline operating, in storage rings dozens of them can work simultaneously, performing completely different experiments.


    \subsection{Light Source Generations}

    All experiments in light sources are perfomed with the assumption that the properties of the light that hit the samples are known. This means that all analysis assumes the photon beam have a specifc energy, polarization, size and divergence at the interaction point. It is also important to know how many photons will hit the sample per unit of time, in order to compute how long the experiment will take for the emergency of the desired pattern in the collected data. the light beam  almost monochromatic beams of light because in this situation the collection of data and analysis of the interaction with matter is simplified and conclusion can be taken from it. Insertion devices already provide this, however the radiation from dipoles must be filtered and most of the energy emitted is discarded. Besides, several experiments requiseres spatial resolution and this light is generally focused to the minimum spot possible,,   There is a figure of merit that is used to classify the quality of the light of a given light source, the Brightness. It is defined as the averare total flux of photons with energy that falls within a given interval, divided by this interval and by the characteristic volume this beam ocuppies in the transverse phase space.

  \section{Brazilian Synchrotron Light Laboratory}

  The Centro Nacional de Pesquisa em Energia e Materiais (CNPEM) is a brazilian institution located in Campinas-SP that gathers four national laboratories, being the Brazilian Synchrotron Light Laboratory (LNLS) one of them. This laboratory was created in 1987 with the objectives of project, construct and operate a synchrotron light source.

  Such objective was achieved successfully with the creation of UVX, a second generation light source which begun operation with external users in 1997. From them

    \subsection{UVX}
    \subsection{Sirius}
  \section{Collective Effects}
    \subsection{Direct Interaction}
    \subsection{Interaction throught medium}
  \section{Description of the work}

  The main objective of this work was to study the subject of collective effects in electron storage rings, gathering the knowledge generated by the community in the last years, and apply this knoledge to study these effects on Sirius storage ring, with estimatives of instabilities thresholds among other collective phenomena commom to this type of accelerators.


\chapter{Concepts of Single Particle Dynamics}

Before the study of collective effects in storage ring is necessary to understand some concepts of single particle dynamics.

  \section{Storage Ring Main Devices} \label{ssec:storage_ring_main_devices}

  The first and most important concept is that of a storage ring. As its name suggests, it is a machine that keeps charged particles in almost circular revolution trajectories for millions of turns with almost constant energy. How these particles are injected in this machine or accelerated to its nominal energy is beyond the scope of this work.

  Connected to the concept of a storage ring is the one of the reference orbit. This special closed orbit is the one that an idealized particle with the storage ring nominal energy would follow if it had the correct initial conditions. In practice the trajectories of all particles stored in the machine will be close to it. For this reason the reference orbit is chosen to be the origin of the reference frame, defining a curved coordinate system that moves along the ring, with one longitudinal coordinate tangent to the local orbit and two transverse coordinates perpendicular to it.

  Besides, all the components of a storage ring are aligned according to this special orbit in such a way that their center, symmetry points or axis coincide with transverse sections of the reference orbit.

  The storage ring task of keeping particles confined for such long times is achieved with the aim of electromagnetic fields to guide and replenish the energy lost by the particles due to synchrotron radiation emission and an integrated ultra high vacuum system to increase their mean free path, minimizing scattering by gas molecules and consequent loss.

  Below we describe some of the main components of a storage ring.

    \subsection{Dipoles}

    Dipoles are devices that generate a static magnetic field that is non-zero at the reference orbit under ideal conditions. This is the only magnet that have such characteristic, in other words, this magnet is the only type of magnet that must afect the trajectory of the idealized particle that defines the reference orbit. It acts on the particle throught the Lorentz force, such as other magnets, bending its trajectory. Ideally, the local curvature that a dipole introduces in the reference particle path must be equal to the curvature of the reference orbit and the reference orbit must be curved only at places where there is a dipole magnetic fields. Shortly, dipoles are the responsible for the ring topoly of a storage ring.

    \subsection{Multipoles}

    Apart from dipoles, all the other magnets are placed at positions where the reference orbit is straight. Therefore, they do not act on idealized reference orbit and must generate a null magnetic field at the positions of the reference orbit. Their function is to confine the movement of the realistic particles in trajectories close to the reference orbit.

    Multipoles are grouped in several types, according to the characteristics of the field they gerenerate in the vicinity of the reference orbit.For example, an ideal quadrupole generates fiels whose transverse components grows linearly with the transverse coordinates of the local reference system and have no longitudinal fields, Sextupoles generate quadratic fields, Octupoles, cubic, and so on. These properties are not exactly true for real magnets, but in gerenal are a good approximation.

    The quadrupoles are the most important multipoles in a storage ring. Because their magnetic field components vary linearly with the particles position in relation to the reference orbit, they act as magnetic lenses, focusing and defocusing the particles beam. Together with the dipoles, they define the main properties of a storage ring, as the particles average energy, the transverse beam emittance, and beam sizes along the ring. These three factors combined define the brilliance of the photons, the figure of merit of a light source.

    It is important to mention that dipoles also work as lenses for the beam. As the quadrupoles, they may have transverse field components that vary linearly with the displacement from the reference orbit, being the only differences the fact that this orbit is curved and the field is non-zero at the reference orbit. Actually, even a dipole with uniform field may focus particles, due to the curvature of the reference system that introduces differences in path length in non-zero field regions for different transverse displacements, or due to edge fields generated by the finite length of the magnet.

    Dipoles work as spectrometers, deflecting more or less particles with less or more energy than the reference particle, which means they could not sustain these particles stored because the total deflection angle would not be $2\pi$ rad and they would spiral in or out, hitting the vaccum chamber. Quadrupoles are arranged to correct this intrinsic limitation of the dipoles, adding or subtracting liquid deflection in one turn for particles with more or less energy. This force particles to have different closed orbits depending on their energy, but at least maintain the chance of them to remain stable. However, quadrupoles also suffer from chromatic aberrations, focusing more or less the particles, depending on their energy. This difference in focusing makes particles oscillate more or less around their closed orbit, changing their fundamental resonance frequency. This is not a fundamental problem, but in almost all modern storage rings it is impossible to store particles with only dipoles and quadrupoles. Sextupoles can correct that effect if placed at the right positions along the ring because of their non-linear magnetic field. They also introduce chromatic effects, but for the modern storage rings this does not affect the capability of storaging particles.

    Sextupoles are needed, but they introduce several complications for the design of a storage ring, because their non-linear fields introduce chaos in some regions of the particles phase space. More sextupoles or higher order multipoles can be introduced to avoid chaos as much as possible and also to help correcting higher order chromatic effects, but this analysis is beyond the scope of this work.

    \subsection{RF Cavity}

    Each turn the electrons lose energy due to synchrotron radiation which must be replenished periodically in order for them to remain in stable orbits, with energy close to the nominal energy of the storage ring. The magnetostatic components described above cannot perform such a task neither any other component or method relying on static electromagnetic fields of any kind, because according to Maxwell equations and the Lorentz equation, the liquid energy transfered to charged particle by static fields in one turn over the ring must be zero. Then the laws of physics constraint that it is necessary to rely on time dependent electromagnetic fields to replenish the energy of the electrons. The way this is accomplished in a storage ring is through the use of devices called RF cavities.

    RF cavity is a jargon for a cylindrical electromagnetic cavity with the lowest Transverse Magnetic (TM01) mode in the range of radiofrequency. Cavities for use in storage rings must have at least two small holes in its axis for the beam passage and one other hole to couple the cavity with an external source to feed the mode TM01 with energy. This energy is transfered to the particles in the storage ring throught the almost homogeneous longitudinal eletric field of TM01 when the particles passes throught the cavity.

    \subsection{Vacuum System}

    The vacuum system is responsible for creating a compact region around the reference orbit in the whole machine with very low pressure, which minimize collisions of the stored charged particles with gas molecules and, consequently, increase the average time particles can be stored with stable movement. Quantitatively, the average pressure of a storage ring must be lower than 1 nTorr for the average stored time of the particles to be of the order of a few dozens of hours.

    The vacuum system is composed of two main subsystems, the vacuum vessel, which defines the boundaries of the electrons atmosfere with the environment, and vacuum pumps to maintain the desired difference in pressure between the two environments. Most of the extension of the vacuum vessel is composed of straight and long chambers with a specific cross section, constant along the extension of the chamber. They are made of metals due to several desirable properties these materials presents, such as high heat and electrical conductivity, maleability, high acceptance to yelding and brasing and high resistence to pressure. Among them we highlight implications of the high electrical conductivity due to its importance for this work.


  \section{Transverse Dynamics}

  After the introduction of the main components, it is convenient to describe in general terms how is the movement of the stored particles. These particles are ultra-relativistic electrons with energy of the order of a few \si{GeV}, and most of their velocity is always in the direction of the tangent of the ideal orbit. Gererally there are approximately hundreds of billions electrons grouped in several bunches along the reference orbit, each bunch having a few milimiters length and transverse sizes of the order of dozens of microns. Each electron is under the influence of a variaty of electromagnetic fields (gravity can be neglected), coming from the static magnetic fields of the dipoles and multipoles, the radiofrequency field of the RF cavity, the direct fields of other electrons in the same bunch and the fields scattered by the vaccum vessel, generated by other electrons in the same bunch, in other bunches or even by themselves in previous turns. Also, they emit synchrotron radiation which makes them lose energy.

  The path of understanding and describing the dynamics of the stored particles begin with an approximation that neglect the effects of their self-fields, i.e. their interaction with each other, with the vaccum chamber and with the residual molecules in their atmosphere. In this framework the only forces acting on the particles are the magnetic fields of the dipoles and multipoles and the longitudinal electric field of the RF Cavity and the only way they can lose energy is throught synchrotron radiation emission.

    \subsection{Hamiltonean Approximation}

    When such calculations are performed for current synchrotrons it is noticeable that the dynamics of the longitudinal motion is much slower than the dynamics of the transverse motion. As an example, in the Sirius storage ring particles take approximately 131 revolutions around the ring to complete one turn around the fixed point in the longitudinal direction while they oscillate 49 times per revolution around the fixed point in the transverse plane. This property makes it possible to separate the study of the longitudinal plane from the transverse one, considering the energy deviation of one particle as a constant parameter in the transverse equations of motion. The effects of the radiation energy loss are even slower than the longidudinal motion, taking a few thousands of turns in the ring to significantly change the transverse motion.

    Neglecting the energy variations of the particles, and the randonnes of the radiation emission, the motion of the particles can be described by a constant hamiltonean in the Frenet-Serret coordinate system defined by the ideal orbit. Procceding in the approximations, considering the paraxial motion of the particles around the closed orbit, this hamiltonean can be simplified to a quadratic form in the momenta coordinates.

    This Hamiltonean generates two second order coupled and non-linear equations of motion for the transverse coordinates of motion that accurately describes the short and mid term stability of the particles. Most of the non-linearities and coupling in these equations come from the magnetic fields of the components along the ring, being the other contributor the curvature of the reference orbit and the energy deviation of each particle. At this stage of the simplification of the problem the dynamics of the electrons still is very complicated, mainly for storage rings such as Sirius, with several and very strong Sextupoles that introduces chaos in the system in regions of the phase-space that are close to the fixed point.

    \subsection{Linear Equations of Motion}

	With further approximations, considering only the terms of the equations of motion that are linear with the phase-space coordinates and eliminating all coupling between the two transverse directions of motion, the analysis of the dynamics becomes simple enough to allow its complete understanding, with analytical solutions to the equations of motion and physically significant parametrizations, that summarizes and simply describes a given machine. In this regime, the transverse movement of the electron is only dependent on the fields of dipoles and quadrupoles.

	Looking back at the start of this section and reviewing all the approximations, at first sight it seems that the region of validity of these linear equations is so limited that their understanding is useless. However that is not what is observed in practice, because storage rings are carefully designed to maximize the validity of this linear behavior: the position, number and strength of all magnets are so well tuned that nonlinearities cancel each other. This approach is used because the general nonlinear dynamics is so complicated that there is no theoretical background to understand stability and confidence to build a machine that is intrinsically non-linear. This is a subject that is being investigated for a long time in the synchrotron comunity, but so far none light source was ever built based in a non-linear storage ring.

	Another important point to justify the study of these linear equation is that, luckly, for synchrotron radiation generation the smaller the transverse size and divergence of the electron beam the better, which means most particles will stay for most of the time in a very small region of approximately hundreads of microns around the fixed point, where only quadrupoles have a significant effect. Besides, if a particle experience large transverse oscillations, damping effects that arise due to synchrotron radiation emission will bring them close to the fixed point in a few dozens of miliseconds.


    \subsection{Betatron Function and Phase Advance} \label{ssub:betatron_function}

	The solutions of the homogeneous part of the linear equations of motion can be parametrized in the such form:
	\begin{align} \label{eq:betatron_motion}
		u(s) &= \sqrt{J\beta_u(s)} \cos(\mu_u(s) - \phi)
	\end{align}
	where $u$ stands for both, $x$ or $y$, $\beta_u(s)$ is called the betatron function and depends only on the the magnetic lattice and $\mu_u(s)$ is called phase advance and its relation to the betatron function is given by
	\begin{align}
		\mu'_u(s) &= \frac{1}{\beta_u(s)}.
	\end{align}
	The constants $\phi$ and $J$ depend on the initial conditions. It can be shown that:
	\begin{align} \label{eq:linear_invariant}
		J &= \gamma_u(s)u^2(s) + 2\alpha_u(s)u(s)u'(s) + \beta_u(s)u'^2(s),
												\quad \text{with}& \\
        \alpha_u(s) &= -\frac{\beta_u'(s)}{2} \quad \text{and}& \nonumber \\ \nonumber
		\gamma_u(s) &= \frac{1+\alpha_u^2(s)}{\beta_u(s)},&
	\end{align}
	which represents the equation of an ellipse in phase space with $J$ being the area of such ellipse.

	The relevance of this parametrization is that there are several very important practical informations regarding the properties and responses to perturbations of the beam that can be extracted directly from the betatron function. It can be seen from equation \eqref{eq:betatron_motion} that the maximum excursion a given particle can experience in a fixed longitudinal point of the ring is proportional to the square root of the betatron function. Analogously, the beam size of a distribution in equilibrium will also be proportional to the square root of the local betatron function. Thus, the ratio between the amplitudes of movement in two different positions is proportional to the ratio of the square root of the betatron functions at the two locations, a property that is fundamental in the process of defining the transverse sizes of the vaccum chamber of a storage ring. For example, lets suppose the maximum betatron function of a lattice is \SI{16}{\meter} and at this place the vaccum chamber has an internal dimension of \SI{12}{\milli\meter}. This means that in another place where the betatron function is only \SI{1}{\meter} the vaccum chamber can be much smaller, only \SI{3}{\milli\meter}, without affecting the stored beam, which would allow the installation of devices that requires such smaller appertures. This strategy of focusing the betatron function is commonly used in storage rings in straight sections where insertion devices are installed, with smaller gaps they can generate stronger magnetic fields, which is desirable for radiation emission.

	Another important property that can be infered from the betatron function is the sensibility of the beam to spurious eletromagnetic fields. It can be shown that the larger the betatron function the larger the effect of such field in the beam dynamics. In the special case of dipolar fields, which are constant in the transverse plane, this dependency goes with $\sqrt{\beta(s_0)}$, for a quadrupolar field it is proportional to $\beta(s_0)$, for a sextupole $\beta^{3/2}(s_0)$ and so on.

	Besides the betatron function, another important advantage of the parametrization presented in \eqref{eq:betatron_motion} is the interpretation of the integral of the phase advance in one turn around the ring. This integral normalized by $2\pi$ defines the tune of the machine:
	\begin{align}
		\nu_u = \frac{1}{2\pi}\udefoint{s}{\frac{1}{\beta_u(s)}}.
	\end{align}
	The integer part of this number corresponds to the number of complete oscillations in the phase space the particles make in one turn in the ring. To interpret the fractional part it is important to notice in equation \eqref{eq:betatron_motion} that the movement of a particle in a fixed longitudinal position in sucessive turns is a perfect senoid, independently of the parametrization:
	\begin{align}
		u_i(s_0) = \sqrt{J\beta_u(s_0)}\cos(2\pi\nu i -\phi_0).
	\end{align}
	where the fractionary part of the tune is identified as the natural frequency of oscillation. This means that resonances can be excited by any electromagnetic field along the ring with a frequency equal to the tune times the revolution frequency. This is the main mechanism that drives the collective instabilities that will be studied in this work. The electromagnetic fields generated by a bunch of particles interacts with other bunches and, because they have the same oscillation frequency, which is the tune, a collective oscillation emerges due to resonance. This can even happen in a single bunch, where oscillations of the head of the bunch drives the tail to ever larger oscillations.

	Additionally, resonances can even be excited by static fields if the tune is a rational number as long as these fields have the correct transverse spatial dependency. For example, if the fractionary part of the tune is $\sfrac12$ and there is a spurius constant quadrupolar field in some point of the ring, the kicks received by the particles in successive turns would always sum constructively and a resonance behaviour would be excited. If both transverse planes of motion are considered it can be shown that if the tune satisfies the equation:
	\begin{align}\label{eq:betatron_resonance}
		m\nu_x + n\nu_u &= p
	\end{align}
	where $m$, $n$ and $p$ are integers, resonances can be excited depending on the static magnetic field around the ring. The number $r^2 = m^2 + n^2$ is called the order of the resonance, and the lower the order the higher its strength. If both, $m$ and $n$ must be non-zero for the equation \eqref{eq:betatron_resonance} to be true, then the resonance depends on the existence of coupling fields, where the position of the particle in one direction influences the kick in the other direction.


    \subsection{Dispersion Function}

	The inomogeneus solution of the liner equations of motion can be written in the following form:
	\begin{align}
		x(s) = \eta(s)\delta
	\end{align}
	where $\eta(s)$ is called dispersion function and, just like the betatron function, depends only on the magnetic lattice. Notice that this is a particular solution of the equation, where periodic conditions were applied. This choice has an advantage over other solutions because of the meaning of the dispersion function: its value along the ring give the shape of the averaged trajectory of a particle with non-zero energy deviation. In other words, this particular solution of the inomogeneus equation gives the closed orbit of off-momentum particles.

	The dispersion function is very important to determine the equilibrium emittance of a storage ring, the minimization of the functional
	\begin{align}
		\mathscr{H}(\eta(s), \eta'(s)) &= \gamma(s)\eta^2(s) +
										2\alpha(s)\eta(s)\eta'(s) +
										 \beta(s)\eta'^2(s)
	\end{align}
	in the dipoles of the ring is one of the main goals when designing a storage ring. Besides, the dispersion function is important to calculate the momentum compaction factor, $\alpha$, which is the relative difference in path length in one turn in the ring per unit of energy deviation:
	\begin{align}\label{eq:momentum_compaction}
		\frac{\Delta L}{L} &\approx \delta \alpha \coloneqq \delta \udefoint{s}{\frac{\eta(s)}{\rho(s)}}.
	\end{align}
	This parameter, which in general is positive, is of fundamental importance for the longitudinal dynamics, because it couples linearly the time it takes for the particles to complete one turn in the ring with their energy deviation. Notice that the only sections of the ring which contribute to the integral are where the reference orbit is curved (in other places the radius is infinity) and this is because particles with more/less energy generally follow paths with larger/smaller radius in the dipoles, which increases/decreases the path lenth. Negative momentum compaction are possible to be obtained in several storage rings by adjusting the quadrupoles of the lattice, but these mode of operations are rare and in the case of Sirius it will not be possible to operate under such condition.

    \subsection{Chromaticity and Action Dependent Tune-shift}

	The quadrupoles correct the chromatic effects in the orbit of the beam that are generated in the dipoles and the result of such a correction is the finite value of the dispersion function. However the quadrupoles also are dispersive components, which means that their focusing strength depends on the energy of the particles. This causes an effect on the beam where each particle has a different tune that depends on its relative energy offset. The linear part of this dependency is called linear chromaticity, which, if not corrected, reduces the lifetime of the beam to a few seconds in most storage rings. For example, for the case of Sirius the chromaticity of the ring without sextupoles is $\approx -130$ for the horizontal direction. This means that particles with an energy deviation of only \SI{0.15}{\percent} would have a tune that is \SI{0.2}{} smaller than the tune of a particle with zero energy deviation. As the energy deviation oscillates in the scale of hundredth's of turns, this particle would almost certainly cross a resonance that would induce its loss. On the other hand, the equilibrium energy spread of the Sirius storage ring is only half of this value, $\approx$ \SI{0.09}{\percent}, which means that without chromaticity correction all particles with energy deviation above two sigma of this distribution would be lost.

	The sextupoles are introduced in the machine to correct the linear chromaticity to values close to zero. In this processes of minimization they change the higher-order chromatic terms and also introduce geometric aberrations, non-linearities in the dynamics of all particles that reduces the region around the fixed point where the beam is stable. The main contribution of this effect for the dynamics in the vicinity of the fixed point is the generation of a linear dependency of the tune of each particle with its action variable, $J_u$. Writting the linear expansion of the tune as a function of the energy deviation and the action we get:
	\begin{align}
		\nu_x(\delta, J_x, J_y) \approx \xi_x \delta + A_{xx} J_x + A_{xy} J_y \\\nonumber
		\nu_y(\delta, J_y, J_x) \approx \xi_y \delta + A_{yy} J_y + A_{yx} J_x
	\end{align}
	where $\xi_x$ and $\xi_y$ are the horizontal and vertical chromaticities, and $A_{xx}$, $A_{yy}$ and $A_{xy}=A_{yx}$ are the action dependent tune-shifts.


  \section{Longitudinal Dynamics}

  In the study of the transverse dynamics the time scale envolved was of a few turns in the storare ring, which allowed us to treat the energy deviation of the particles as another constant of motion. In this section the dynamics of the electrons in a few hundreads of turns will be analysed. In this scale the transverse betatron oscillations are averaged out and the longitudinal dynamics, which describes the path length and energy oscillations around the fixed point, has its natural frequency. The main factors that influence in this framework are the small unballances between the energy loss of the particles and their energy gain in the RF cavity in one turn and the revolution time variation due to the energy offset of the particles.

    \subsection{Changes in Revolution Time}

	In the studies of the transverse dynamics the longitudinal absolute position of the particles was used as independent variable for all the equations. In the longitudinal plane it is important to work with a relative value for the longitudinal position, being the reference of this measure the position of the synchronous particle, the idealized particle described in subsection \ref{ssec:storage_ring_main_devices} that has the nominal energy of the storage ring and is always in the ideal orbit, taking the nominal revolution time to complete one turn. Under such assumptions, we defining $z$ as the distance a given particle is behind the synchronous particle:
	\begin{align}
		z(t) = s_\text{sync}(t) - s(t)
	\end{align}
	where $t$ is the wall-clock time and $s_\text{sync}(t)$ is the position of the synchronous particle. Notice if the longitudinal position of the particle is larger than the position of the synchronous particle, then $z$ will be negative. This convention is very important and will be consistently adopted throught this work.

    When the energy of a particle changes its velocity is also modified, which contributes to the change of the revolution time. For most storage rings of synchrotron light sources this effect is negligible when compared to the change in path length described by equation \eqref{eq:momentum_compaction}, due to the ultra-relativistic regime in which these machines operate. This allows us to approximate the left hand side of equation \eqref{eq:momentum_compaction} to the relative change in revolution time. Then, from one turn to another, the relative position of a particle will change by:
	\begin{align}\label{eq:revolution_time_variation}
		z_{n+1} = z_n + \delta_n\alpha L_0
	\end{align}
	where $n$ refers to the current turn and $n+1$ to the next and $L_0$ is the nominal circumference of the ring.

	\subsection{The Energy Balance}

	It can be shown that the rate of energy loss of a particle due to synchrotron radiation emission is proportial to the square of the magnetic field times the square of its total energy. Translating this dependency to a storage ring, the energy loss in one turn depends on the magnetic lattice and the particles energy deviation. For the ideal particle the energy loss is always the same and depends only on the magnetic field of the dipoles, but as the closed orbit for particles with non-zero energy deviation is different, its energy loss will be different not only because of the intrinsic dependence of the emission but also because of the slightly different magnetic fields it will experience in one turn. The combination of these effects can be modelled in the following linear approximation:
	\begin{align}\label{eq:radiation_loss}
		\Delta E_\text{Rad} \approx -U_0 - E_0D\delta_n
	\end{align}
	where $U_0$ is the energy loss of the ideal particle in one turn, $E_0$ is the nominal energy of the storage ring and the coefficient $D$ includes both effects, the intrinsic dependence on the energy and the different orbit fields.

	The energy gain of a particle in the RF cavity is given by the integral of the longitudinal electric field, $E_\parallel$, along the path of the particle and depends only on the initial phase of the field when it enters the cavity:
	\begin{align}
		\Delta E = V(t_0) = q\defint{s}{E_\parallel(s,t)|_{t=\sfrac{s}{c}+t_0}}{0}{L_c}
	\end{align}
	where $V(t_0)$ is called the gap voltage of the cavity and $L_c$ is its length.

	Now lets assume the frequency of oscillation of the electromagnetic field inside the cavity, $\omega_{RF}$, is exactly a multiple of the revolution frequency of the synchronous particle, $\omega_0$:
	\begin{align}\label{eq:harmonic_number}
		\omega_{RF} &= h\omega_0
	\end{align}
	where $h$ is called harmonic number. With this assumption, even though the fields are time dependent the synchronous particle will always see the same conditions as it enters in the cavity. Now lets make a further assumption that the synchronous particle reaches the cavity in the exact time where it gains $U_0$ energy of the cavity:
	\begin{align}
		V(0) = U_0
	\end{align}
	where the time reference on left hand side is relative to the position of the synchronous particle. Considering both assumptions and combining the energy gain in the cavity with the energy loss in one turn, given by equation \eqref{eq:radiation_loss}, we get the following one turn energy balance for a storage ring:
	\begin{align}\label{eq:energy_balance}
		\delta_{n+1} = \delta_n - D\delta_n + \frac{V(z_{n+1})-U_0}{E_0}.
	\end{align}
	where the subscript $n+1$ in the particles position means that it will go around the ring first and then pass through the cavity.

	\subsection{Phase Stability Principle}

	Combining equations \eqref{eq:revolution_time_variation} and \eqref{eq:energy_balance} we get the one turn map for the longitudinal plane for which the synchronous position and the nominal energy defines a fixed point. To analyse the stability of this fixed point let's linearize the map in its vicinity:
	\begin{align}
		\begin{bmatrix} z_{n+1} \\ \delta_{n+1}\end{bmatrix} = \overbrace{
			\begin{bmatrix} 1 & \alpha L_0 \\ -V'_0 & 1-D - V'_0\alpha L_0 \end{bmatrix}
		}^M \begin{bmatrix} z_n \\ \delta_n \end{bmatrix}.
	\end{align}
	where $V'_0$ is the derivative of the voltage gap in relation to the arrival time of the particles at the synchronous position normalized by the nominal energy of the ring, $E_0$. The stability condition, $\text{Tr}(M) <= 2$, requires the derivative of the gap voltage to be positive if $\alpha$ is positive. To ilustrate this condition consider that initially a particle arrives ahead of the synchronous particle, the positive derivative means it will gain more energy, which makes it take longer to go around the ring, diminishing the difference in arrival time for the next turn. This way all particles remain in an oscillatory movement around the fixed point with frequency given by
	\begin{align}
		\text{Tr}(M) = 2\cos(2\pi\nu_s) \approx 2-(2\pi\nu_s)^2 \implies
		\nu_s \approx \frac{1}{2\pi}\sqrt{D+V'_0\alpha L_0},
	\end{align}
	where $\nu_s$ is called synchrotron tune in analogy to the betatron tunes defined in subsection \ref{ssub:betatron_function}.

	Additionally, the determinant of the one turn matrix is $(1-D)$, which implies the oscillations are damped, with $D$ being the damping factor. In most storage rings this effect is small compared to the oscillation time, requiring thousands of turns to influence the dynamics.

	The voltage gap has the same harmonic composition as function of the arrival time as the electric field as a function of time. This means that the condition imposed in equation \eqref{eq:harmonic_number} implies that there are at least $h$ stable fixed point along the ring and, as in general the voltage is a pure senoid, these are the only stable points. This means that it is possible to store up to $h$ aglomerations of electrons, called bunches, in a storage ring.

    \subsection{The Potential Well}

	There is an important approximation to the map equations derived in the previous sections that consists in considering the turn by turn iterations as inifinitesimal transformations and taking the limit to the continuum, considering differences between turns as derivatives. With this considerations, the equations of motion becomes:
	\begin{align}
		\dertot{z}{t} &= \frac{\alpha \delta(t)}{c} \\\nonumber
		\dertot{\delta}{t} &= \frac{D}{T_0} + \frac{V(z(t))-U_0}{T_0E_0}
	\end{align}
	where $T_0$ is the nominal revolution time of the ring. If the damping term is not considered, the equations of motion can be derived from a static Hamiltonean, given by:
	\begin{align}
		H(z, \delta) = \frac12\left(\frac{\alpha\delta}{c}\right)^2 \quad \overbrace{- \frac{\alpha}{c}\defint{x}{\frac{V(x)-U_0}{T_0E_0}}{0}{z}}^{U(z)}
	\end{align}
	where the the first term of the right hand side is the kinetic term and $U(z)$ is called the potential well, in an analogy with a potential energy.

  \section{Radiation Damping and Equilibrium Parameters}

  The synchrotron radiation emission is a quantum process that happens uncorrelatedly among all the electrons in the beam. While the average emission has a well defined and smooth behavior, such as the spectra that are calculated with classical electrodynamics for dipoles and insertion devices, the individual emissions are random events. As expected the effect this process generates on the beam is also dual, the average emission is responsible for energy loss and damping of the longitudinal and transverse oscillations, while the single emission process generates uncorrelated motion in all planes that heat up the beam.

  Both effects have very different dependency on the parameters of the particles. For example, in the last section it was used that the energy loss depends linealy with the energy deviation of the particles, which caused an exponential damping of the oscillations, i.e. a damping proportional to the amplitude of the oscillation, while the heating in the longitudinal plane happens because the emissions are instantaneous and uncorrelated among electrons or in time, having no short term dependency on any parameter of the electrons, which generates random walks for the energy deviations. Because these effects have such different dependencies they always compete with each other, if the amplitude of oscillation is large the damping dominates, if it is small there is a blow up. It is this competition that generates the equilibrium energy distribution of the beam and consequently, due to the potential well, the longitudinal distribution. While each electron is in an endeless variation of its amplitude of oscillation around the fixed point, the average of all electrons in a bunch remains stationary, with both effects cancelling each other.

  In the transverse plane the effect of the radiation emission on the dynamics is not as direct as in the longitudinal plane. Damping in the transverse plane is a two fold effect: first the electrons lose transverse momentum due to radiation emission because the emission is majoritarely on the direction of motion. This does not change the normalized momentum of the electron, because the longitudinal momentum is also affected by this emission. In a second moment the electron passes throught the RF cavity where the longitudinal momentum is replenished but the transverse momentum is unchanged, but the normalized transverse momentum is decreased. The liquid effect is an exponential damping of the transverse oscillations, proportional to the betatron action of the movement. The excitations of oscillations happens because of the dispersion function in the dipoles, when an electron emits radiation its closed orbit abruptly changes, because its energy deviation has changed. However the position of the electron did not change, this implies that a new betatron oscillation around this new closed orbit must start. Again, both effects cancel out at the equilibrium distribution, that defines the natural emittance of the storage ring.

    \subsection{Fokker-Planck Equation}
	To do.



\chapter{Wakes and Impedances}\label{cap:wake_impedances}

In the last chapter the main aspects of the single particle dynamics, governed by the guiding electromagnetic fields generated by the magnets and the RF cavity were analysed. The effects of the radiation emission on the particle that generated this radiation were also considered, and concepts such as equilibrium distribution of particles, emittance and energy spread were introduced, however no interaction among the particles in this distribution was considered. Besides the external fields, the self-fields, fields induced by the stored particles, are important to characterize the dynamics when the intensity of the beam becomes large. These fields have different effects on the beam depending on how the interaction happens.

  \section{Interaction Mechanisms}

  The most important interaction for storage rings of light sources is the collision of particles, generally refered to as Coulomb scattering or intrabeam scattering, which is so unpredictable and chaotic that its effects on the beam resembles the properties of the emission of radiation, causing emittance and energy spread increase. Additionally, the collision process also leads to particle loss throuth a mechanism called Touscheck scattering, where the transverse energy of oscillation is transfered to the longitudinal plane and the particles gain an energy deviation so large that they are lost. All these effects are very detrimental to new light sources, being the Touscheck lifetime their main source of particle loss.

  The \gls{dsp} is another type of interaction among the particles, being the result of the action of the cloud of electromagnetic field existent inside the beam on individual particles. Each particle generates an electric and a magnetic field that, when averaged among all particles, result in a net potential dependent on the shape and sizes of the bunch. This potential acts like an external field on the movement of the particles, leading to tune-shifts with amplitude and possible excitation of resonances and its dependency with the beam parameters makes it is very difficult to self-consistenly evaluate its effects. However, for ultra-relativistic electron beams such as the ones of a light source storage ring, with energies of the order of a few \si{\giga\electronvolt}, this effect is very small and can be neglected. This happens because in this limit the non-radiating fields generated by each particle is concentrated in a plane transverse to its movement and the electric and magnetic forces that act on other particles moving parallel to it cancel each other out. In Appendix \ref{app:lorentz_cancel} a more detailed explanation for this property is given.

  The \gls{csr} is another type of direct interaction between particles in a beam. The radiation emitted by the particles travels forward with the velocity of light and, due to the fact that the particles are moving on a curved trajectory when they emit light, this radiation catches up with the particles ahead of the emitting particle. If the wavelength of the radiation is of the same order of or larger than the bunch length the average of this effect is non-zero and the head of the bunch feels a net force. As this effect depends on the radiated field, in contrast to the \gls{dsp}, it does not tend to zero as the energy of the particles increases and can be very harmful depending on the bunch length of the beam. However, this mechanism of interaction suffers from shielding of the vacuum chamber because, depending on the transverse separation of the walls, the low frequency radiation cannot propagate, which mitigates the effect.

  All mechanisms described above are examples of direct interactions among the particles, they do not depend on the environment in which they are imerse to happen. The wake fields and the \gls{isp} on the other hand, use the vacuum chamber as intermediary of the interaction. The contact of the non-radiating field of the beam with the metallic walls of the chamber induces currents in the surface of the metal that travels with the beam, these surface currents also generate an electromagnetic field that propagates to the center of the vaccum chamber and influences the movement of the particles. This electromagnetic field can have properties of non-radiating and radiating field, depending on the characteristics of the vacuum chamber. If we consider the chamber is perfectly conducting and with translational symmetry in the longitudinal direction, then the surface charges travel in straight lines with the beam speed, which means they will emit only non-radiating fields. This is the origin of the \gls{isp}, that for the same reasons as the \gls{dsp} is negligible for high energy storage rings.

  If any of the two conditions imposed on the vacuum chamber is broken, then the surface charges also generate radiating fields, namely wake fields. For example, if the vaccum chamber has no translational symmetry, then the surface charges must follow curved paths, which makes then suffer accelerations and hence, radiate electromagnetic fields. Notice that this is only one of the several possible ways of introducing this mechanism, it would be equivalent to say that the surface of the metal scatter the fields generated by the particles in the beam and when the surface of the vacuum chamber has longitudinal symmetry the refletion is specular but when it has corrugations, the scattering is difuse. Precisely what happens is that the walls of the vacuum chamber impose \gls{bcon} on the fields that exist inside the vaccum vessel, univocally defining its time and spatial dependency.


  \section{Wake Fields}

  Even though the mechanism behind the interaction among the stored particles througth wake fields is very simple to describe qualitatively, a quantitative self-consistent description of this effect is practically impossible. The main difficulty comes from the fact that \gls{maxeq} should be solved using the vacuum chamber of the whole ring being subjected to a source, the beam, that is varying due to the external fields and also due to the effect of the own fields that we want to find. In order to tackle this problem self-consistency must be forgotten and approximations must be done.

  The first approximation is to consider that all the properties of the materials that compose the vacuum chamber are linear in relation to the intensity of the fields. This linearity combined with the linearity of the \gls{maxeq} allow us to solve the electromagnetic fields for a single source particle and sum over the beam to get the desired result. Another approximation consists in breaking the storage ring in several small parts that do not interact with each other, which allow us to solve \gls{maxeq} for each part independently. This approximation is valid for storage rings because generally the irregulaties or transitions in the the vacuum chamber are far from each other in such a way that the fields generated in one of them cannot propagate to the other.

  The two approximations described above already greatly simplifies the problem but in order to make it treatable other to are needed: the rigid beam and the impulse approximation.

  % To do so, in the next section we will define quantities based on some approximations  In this section we are going to introduce the concept of wake field in particle accelerators and try to understand its foundations and main properties for the subsequent analysis of its influence over the movement of the charged particles in the accelerator. There are several approaches in the literature to explain this subject, the most appreciated by the author is the one presented in reference \cite{Stupakov2000a}, which will be reproduced in parts here.


%%%%%%%%%%%%%%%%%%%%%%%%%%%%%%%%%%%%%%%%%%%%%%%%%%%%%%%%%%%%%%%%%%%%%%%%%%%%%%%%%%%%%%%%%%%%%%%%%%%%%%%%%
\subsection{Causality and catch up distance}

If one particle moves in a straight line at light speed, the electromagnetic field scattered by the discontinuities of the chamber will not catch up with it and will not affect the charges travelling ahead of it. The field will only interact with the charges moving behind of the source particle. Such property is known as causality.

Even though this property is not strictly true in the real world, because particles always travel at speeds lower than the light's, it is true in most practical cases, as we will see bellow.

Lets try to calculate the distance $z$ where the field generated by some discontinuity in the vacuum chamber will catch up with a witness particle at a distance $s$ behind the source particle. At the time $t=0$ the source particle passes through the discontinuity and an electromagnetic wave is generated with its wave front travelling at the speed of light in all directions, forming a sphere of radius $R$, see FigureXX. At any given time after this, the following relation holds:
\begin{align}
ct = R \quad vt = z && \Rightarrow && R = \frac{z}{\beta} \quad \text{where} \,\, \beta = \frac{v}{c}
\end{align}
where $z$ is the distance travelled by the source particle. Besides that, at the specific time when the wake catchs up with the witness particle, the following relation is valid:
\begin{align}
R^2 = b^2 + (z-s)^2  \Rightarrow z^2(\frac{1}{\beta^2}-1) + 2sz - (b^2 + s^2) = 0  \Rightarrow \\
z = -\gamma^2 \beta^2 s + \sqrt{s^2\gamma^4\beta^4 + \gamma^2\beta^2\left(b^2 + s^2\right)}  = \gamma^2 \beta^2 s\left(-1 + \sqrt{1 + \frac{1}{\gamma^2\beta^2}\left(1 + \frac{b^2}{s^2}\right)}\right)
\end{align}
where $b$ is the distance from the discontinuity to the trajectory of the particles and $\gamma = 1/\sqrt{1-\beta^2}$ is the relativistic energy. FigureXX shows a graphic of this function, normalized by the distance $b$. We notice that, for $s=0$, which means the field catching up with the source particle, $z = \gamma\beta b$. For the case of Sirius, $\gamma \approx 5870$ and $\beta \approx 1$, if $ b = 2$mm, $z \approx 12$m.

\begin{figure}[hb!]
\centering
\label{fig:catch_up}
\begin{tikzpicture}
\draw[very thick] (0,0) -- ++(10,0) (0,4) -- ++(10,0); %vacuum chamber
\draw[very thick] (4.8,4) to [out=-90,in=0] (5,3.8) to [out=180,in=-90] (5.2,4);
\draw[dashed] (0,2) -- ++(10,0); %eixo de simetria
\coordinate (V) at (0.5,0);
\coordinate (Q1) at (4cm,2.2cm);
\coordinate (Q2) at (0.5cm,2.5cm);
\filldraw[fill=black] (Q1) circle[radius=0.05] node[left] {$q$}; % source particle
\draw[->] (Q1) -- ++(V) node[above] {$\boldsymbol{v}$}; % velocity vector
\draw[-{Stealth[length=10pt]}] (Q1) let \p1 = (Q1) in --(\x1,4);
\draw[-{Stealth[length=10pt]}] (Q1) let \p1 = (Q1) in --(\x1,0) node[midway,right] {$\boldsymbol{E}$};
\draw ($(Q1)+(0,1)$) circle[radius=0.2] node[right=0.2] {$\boldsymbol{B}$};
\filldraw ($(Q1)+(0,1)$) circle[radius=0.07];
\end{tikzpicture}
\caption{The catch up distance.}
\end{figure}


\subsection{Definição de Wake}\label{ssec:wake_definition}

The Electromagnetic interaction of charged particles with the environment generally has a small effect when compared with the effect of the guiding electric and magnetic fields of the accelerators components and can be treated as a perturbation. In an zeroth order approximation we can assume the beam moves with constant speed in a straight line and solve Maxwell equations

%%%%%%%%%%%%%%%%%%%%%%%%%%%%%%%%%%%%%%%%%%%%%%%%%%%%%%%%%%%%%%%%%%%%%%%%%%%%%%%%
%%%%%%%%%%%%%%%%%%%%%%%%%%%%%%%%%%%%%%%%%%%%%%%%%%%%%%%%%%%%%%%%%%%%%%%%%%%%%%%%
%%%%%%%%%%%%%%%%%%%%%%%%%%%%%%%%%%%%%%%%%%%%%%%%%%%%%%%%%%%%%%%%%%%%%%%%%%%%%%%%
A interação eletromagnética de partículas carregadas com o ambiente normalmente tem um efeito pequeno quando comparado com o efeito de campos elétricos e magnéticos externos dos aceleradores e pode ser considerada com uma perturbação. Em uma aproximação de ordem zero, podemos assumir que o feixe se move com velocidade constante em uma linha reta, e então resolvemos as equações de Maxwell, encontramos os campos e computamos o efeitos desses campos no movimento das partículas. Com essa abordagem negligenciamos efeitos de segunda ordem porque o movimento em uma órbita perturbada podem gerar apenas uma pequena mudança nos campos computados pela aproximação de ordem zero. Essas correções são geralmente pequenas, especialmente para partículas ultra-relativísticas.

Outra importante característica da interação entre o campo eletromagnético gerado e as partículas é que em muitos casos de importância prática eles estão localizados em uma região pequena comparada com o tamanho da órbita do feixe. Ela também ocorre uma escala de tempo muito menor que a de oscilação do feixe no acelerador (como os períodos bétatron e síncrotron). Isso nos permite considerar essa interação dentro da aproximação de impulso e caracterizá-la pelo momento transferido para a partícula.

Dessa forma, podemos introduzir a noção de \engw{wake} da seguinte maneira. Considere a partícula 1, com carga $q$ se movendo ao longo do eixo $z$ com uma velocidade próxima à da luz, $v\approx c$, de modo que $z=ct$ (veja a figura \ref{fig:4}). Uma partícula 2 com carga unitária se move paralelamente à partícula 1, com a mesma velocidade, a uma distância $s$ com deslocamento transversal $\vect{\rho}$ relativo ao eixo $z$. O vetor $\vect{\rho}$ é um vetor bi-dimensional perpendicular ao eixo $z$, $\vect{\rho} = (x,y)$. Apesar de as partículas viajarem no vácuo, há contornos materiais no problema que espalham o campo eletromagnético que gera uma interação entre as partículas.

Assumindo que as equações de Maxwell foram resolvidas e que os campos gerados pela partícula 1 foram encontrados, podemos calcular a mudança no momento $\Delta \vect{p}$ da segunda partícula causada por esse campo como uma função do deslocamento $\vect{\rho}$ e da distância $s$,

\begin{equation}
 \Delta \vect{p}(\vect{\rho},s) = \defint{t}{\left[\vect{E}(\vect{\rho},z,t) + \vect{\hat{z}}\times \vect{B}(\vect{\rho},z,t)\right]_{z=ct-s}}{-\infty}{\infty}.
\end{equation}

Note que a integral é feita sobre uma linha reta --- a órbita não perturbada da segunda partícula. Os limites de integração são estendidos de menos para mais infinito assumindo que a integral converge.

Como a dinâmica do feixe é diferente nas direções longitudinal e transversal, é útil separar o momento longitudinal $\Delta p_z$ da componente transversal $\vect{\Delta p}_\perp$. Dessa forma, com uma convenção de sinal e um fator de normalização $c/q$, podemos definir as chamadas \engw{wake functions}, ou simplesmente \engw{wakes} longitudinal e transversal,

\begin{equation}\label{eq:wake_definition}\begin{aligned}
    w_l(\vect{\rho},s) &= -\frac{c}{q} \Delta p_z = -\frac{c}{q} \udefint{t}{E_z|_{z=ct-s}}, \\
    \vect{w}_t(\vect{\rho},s) &= \frac{c}{q} \vect{\Delta p}_\perp = \frac{c}{q} \udefint{t}{\left[\vect{E}_\perp + \vect{\hat{z}}\times \vect{B}\right]_{z=ct-s}}
\end{aligned}\end{equation}

Note o sinal de menos na definição de $w_l$ --- ele é introduzido para que um \engw{wake} longitudinal positivo corresponda a uma perda de energia da partícula teste (caso ambas a partícula fonte e teste tenham o mesmo sinal de carga). Os \engw{wakes} definidos tem dimensão de \si{\volt\per\coulomb} no Sistema Internacional de Unidades.

Por causa do princípio de causalidade o \engw{wakefield} não se propaga a frente da partícula fonte, então
\begin{equation}
    w_l(\vect{\rho},s) = 0, \qquad \vect{w}_t(\vect{\rho},s) = \vect{0}, \qquad \mathrm{para } \quad s < 0.
\end{equation}

Na definição acima foi assumido que o campo eletromagnético estava localizado no espaço e no tempo e que a integral na equação \ref{eq:wake_definition} converge. Contudo, há casos em que isso não é verdade e a fonte do \engw{wake} é distribuida uniformemente em um longo caminho, como é o caso do \engw{wake} de parede resistiva de uma câmara de vácuo. Nesse caso é mais conveniente introduzir o \engw{wake} por unidade de comprimento, descartando a integração na equação \ref{eq:wake_definition}:

\begin{equation}\begin{aligned}
    w_l(\vect{\rho},s) &= -\frac1q E_z|_{z=ct-s}, \\
    \vect{w}_t(\vect{\rho},s) &= \frac1q\left[\vect{E}_\perp + \vect{\hat{z}}\times \vect{B}\right]_{z=ct-s}.
\end{aligned}\end{equation}

Nessa definição os \engw{wakes} adquirem uma dimensão adicional de inverso de comprimento e tem dimensão \si{\volt\per\coulomb\per\meter} no Sistema Internacional de Unidades.

%%%%%%%%%%%%%%%%%%%%%%%%%%%%%%%%%%%%%%%%%%%%%%%%%%%%%%%%%%%%%%%%%%%%%%%%%%%%%%%%%%%%%%%%%%%%%%%%%%%%%%%%%
\section{Panofski-Wenzel}

O conceito de wake-function e impedância tem seu surgimento motivado pelo teorema de Panofski-Wenzel. Esse teorema é resultado de duas aproximações feitas na análise do problema de duas partículas relativísticas interagindo entre si por meio de campos eletromagnéticos espalhados pela câmara de vácuo do anel de armazenamento. Por sua central importância para o subsequente desenvolvimento desse trabalho, vamos olhar esse teorema com um pouco mais de detalhe.

Primeiro, vamos considerar a imagem apresentada na figura X, ou seja, consideremos uma situação em que temos uma partícula de carga $Q$ atravessando a estrutura ilustrada com velocidade $\vec{v}$ ao longo da câmara de vácuo de um anel de armazenamento. A uma distância longitudinal $z$ dela, uma partícula de prova com carga $q$ também atravessa a estrutura na mesma direção. Estamos interessados em saber qual é a força que essa carga de prova sentirá ao longo de sua passagem por toda a estrutura. Pela equação de Lorentz, temos a todo instante:
\begin{equation}
 \vec{F} = q\left(\vec{E} + \vec{v} \times \vec{B}\right)
\end{equation}
onde $\vec{E}$ e $\vec{B}$ é o campo eletromagnético na posição da partícula que está sentindo a força. Esse campo pode ser decomposto como a soma de vários campos com origem e interpretação física diferentes: O campo externo, gerado pelos ímãs e cavidades de RF que guiam o feixe e o campo de interação, que foi gerado pela partícula fonte, de carga $Q$. O campo de interação ainda pode ser decomposto em duas contribuições distintas: o campo direto, que existiria mesmo sem a presença da câmara de vácuo e o campo indireto, resultado da interação dessa carga com a câmara de vácuo. Para obter a transferência de momento total sentida pela carga de prova ao longo de toda estrutura, temos que integrar a equação acima na trajetória da partícula. Ao realizarmos esse procedimento, logo vemos que a solução auto-consistente desse problema é muito complexa, pois a trajetória da partícula prova depende da força que ela sentiu nos tempos anteriores, assim como da trajetória da carga fonte, devido à dependência dos campos eletromagnéticos que são gerados por ela.

Assim, para conseguir seguir mais adiante na análise desse problema devemos fazer aproximações. A primeira delas é considerar que ambas as partículas são rígidas e possuem velocidades paralelas uma com a outra e paralelas ao eixo de simetria da câmara de vácuo (caso a câmara de vácuo não tenha simetria, considera-se a velocidade das partículas paralelas à direção em que o feixe de elétrons deverá passar). Essa consideração simplifica drasticamente a análise do problema, como veremos logo a seguir.

Antes de seguir com a demonstração do teorema, vamos justificar essa aproximação. Na maioria dos aceleradores de partículas as partículas estão no limite ultra-relativístico, em que sua velocidade é aproximadamente a da luz e varia muito pouco com variação de momento. Isso implica que nesse limite as partículas são bastante rígidas


\subsection{Teorema de Panofsky-Wenzel}

Várias relações gerais entre os \engw{wakes} longitudinal e transversal podem ser obtidas das equações de Maxwell sem que seja necessário especificar as condições de contorno para os campos.

Vamos introduzir o vetor $\vect{R} =(\vect{\rho},-s)$ (o sinal de menos na frente do $s$ é devido ao fato de $s$ ser positivo para posições atrás da partícula fonte) e considerar o momento $\vect{\Delta p}$ na equação \ref{eq:wake_definition} como uma função de $\vect{R}$. Vamos assumir que o campo eletromagnético é especificado através do potencial vetor $\vect{A}(\vect{r},t)$ e o potencial escalar $\phi(\vect{r},t)$, e computar
$\vect{\Delta p}$ para os dados campos. É conveniente usar a formulação Lagrangiana para as equações de movimento,

\begin{equation}\label{eq:euler_lagrange}
    \dertot{}{t}\derpar{L}{\vect{v}} = \derpar{L}{\vect{r}} = \vect{\nabla}L,
\end{equation}
com a Lagrangiana para a partícula teste com carga unitária é
\begin{equation}\label{eq:lagrangiana_charged_part}
   L = -mc^2 \sqrt{1 - \frac{v^2}{c^2}} + \frac1c \vect{Av} - \phi
\end{equation}
substituindo a equação \ref{eq:lagrangiana_charged_part} na equação \ref{eq:euler_lagrange} obtemos:

\begin{equation}
 \dertot{}{t} \left(\vect{p} + \frac1c \vect{A}\right) = \vect{\nabla}\left(\frac1c \vect{Av} - \phi\right).
\end{equation}
onde $\vect{p} = m\gamma\vect{v}$.

Agora, integrando esta equação ao longo da órbita da partícula teste, $x=\mathrm{const}$, $y=\mathrm{const}$ e $z = ct-s$, e assumindo que os campos $\vect{A}$ e $\phi$ vão a zero no infinito, encontramos

\begin{equation}
 \vect{\Delta p}(\vect{R}) = \udefint{t}{\vect{\nabla}\left(\frac1c\vect{Av} - \phi \right) = \frac{q}{c} \vect{\nabla_R}W(\mathrm{R})}.
\end{equation}
onde introduzimos o \engw{wake potential} $W$,
\begin{equation}
 W(\vect{R}) = \frac{c}{q}\udefint{t}{\left(\frac1c \vect{Av} -\phi\right)}
      \overset{\vect{v} \approx c\vect{\hat{z}}}{=}
                 \frac{c}{q}\udefint{t}{\left(A_z -\phi\right)}.
\end{equation}

Assim, provamos uma relação que estabelece que todas as três componentes do vetor $\vect{\Delta p}$ podem ser obtidas derivando uma única função escalar $W$. Relembrando a relação entre os componentes de $\vect{\Delta p}$ e os \engw{wakes}, \ref{eq:wake_definition}, descobrimos que

\begin{equation}\label{eq:wake_function_definition}
 w_l = - \derpar{W}{(-s)} = \derpar{W}{s},\qquad \vect{w}_l = \vect{\nabla_\rho}W,
\end{equation}
e, consequentemente
\begin{equation}\label{eq:panofsky_wenzel_theorem}
 \derpar{\vect{w}_t}{s} = \vect{\nabla_\rho}w_l.
\end{equation}

Esta relação é comumente chamada de teorema de Panofsky-Wenzel. Note que $\nabla_\rho$ é um gradiente bidimensional com respeito às coordenadas $x$ e $y$.

Uma das aplicações computacionais mais importantes do teorema de Panofsky-Wenzel é que o conhecimento do \engw{wake function}, $w_l$, longitudinal nos permite encontrar o \engw{wake} transversal,$\vect{w}_t$ por meio de uma integração simples da equação \ref{eq:panofsky_wenzel_theorem}.

Outra propriedade importante de $W$ é que ele é uma função harmônica das variáveis $x$ e $y$,
\begin{equation}\label{eq:wake_potential_harmonic}
 \Delta_\perp W \equiv \derpar[2]{W}{x} + \derpar[2]{W}{y} = 0.
\end{equation}
Para provar isso vamos usar o fato que ambos $\vect{A}$ e $\phi$ satisfazem a equação de onda no espaço livre, $(\partial^2/\partial t^2 - c^2 \Delta)\vect{A} = \vect{0}$ e $(\partial^2/\partial t^2 - c^2 \Delta)\phi = 0$. Dessa forma,

\begin{equation}\begin{aligned}
0
&=\frac{c}{q}\udefint{t}{\left(\derpar[2]{}{t}-c^2\Delta\right)\!\!(A_z-\phi)}\\
&=\frac{c}{q}\left[\udefint{t}{\left(\derpar[2]{}{t} - c^2\derpar[2]{}{z}\right)} -
	            c^2\udefint{t}{\left(\derpar[2]{}{x} + \derpar[2]{}{y}\right)}\right]\!\!(A_z-\phi)\\
&=\frac{c}{q}\udefint{t}{\left(\derpar{}{t}+c\derpar{}{z}\right)
				     \!\!\left(\derpar{}{t}-c\derpar{}{z}\right)\!\!(A_z -\phi)}
   -c^2\!\left(\derpar[2]{W}{x} + \derpar[2]{W}{y}\right)
\end{aligned}\end{equation}

A última integral nessa equação é nula porque
\begin{equation}
    \derpar{}{t} + c\derpar{}{z} \approx \derpar{}{t} + \vect{v\nabla} = \dertot{}{t}
\end{equation}
e
\begin{equation}
  \udefint{t}{\left(\derpar{}{t} + c\derpar{}{z}\right)\!\!
             \left(\derpar{}{t} - c\derpar{}{z}\right)\!\!(A_z - \phi)}
  = \udefint{t}{\dertot{}{t}\!\!\left(\derpar{}{t}-c\derpar{}{z}\right)\!\!(A_z - \phi)}
  =  0.
\end{equation}

%%%%%%%%%%%%%%%%%%%%%%%%%%%%%%%%%%%%%%%%%%%%%%%%%%%%%%%%%%%%%%%%%%%%%%%%%%%%%%%%%%%%%%%%%%%%%%%%%%%%%%%%%
\subsection{Sistemas Com Um Eixo de Simetria}
Na seção \ref{ssec:wake_definition} nós definimos o \engw{wake} como uma função do deslocamento da partícula teste relativo ao caminho da partícula fonte. Em aplicações práticas nós também estamos interessados em saber como o \engw{wake} depende da trajetória da partícula fonte. Assumiremos que o sistema em consideração tem um eixo de simetria e vamos escolhe-lo como o eixo $z$ do sistema de coordenadas, veja \ref{fig:5}. Agora a partícula fonte, $1$, se move na direção $z$ com um deslocamento dado pelo vetor $\vect{\rho'}$, e a partícula teste viaja paralelamente à partícula fonte, com a mesma velocidade, a uma distância $s$ atrás da fonte com um deslocamento $\vect{\rho}$ relativo ao eixo. Os vetores $\vect{\rho'}$ e $\vect{\rho}$ são os vetores bidimensionais perpendiculares ao eixo $z$. O \engw{wake} ainda é definido pela equação \eqref{eq:wake_definition} mas agora ele será considerado como uma função de $\vect{\rho'}$, $\vect{\rho}$ e $s$
\begin{equation}\begin{aligned}
w_l &= w_l(\vect{\rho},\vect{\rho'},s), \\
\vect{w}_t &= \vect{w}_t(\vect{\rho},\vect{\rho'},s).
\end{aligned}\end{equation}

Normalmente a câmara de vácuo é projetada de forma que o eixo do sistema serve como uma órbita ideal para o feixe. Assim, desvios desse eixo são relativamente pequenos e ambos os vetores $\vect{\rho'}$ e $\vect{\rho}$ são tipicamente muito menores que a câmara de vácuo, de forma que em $w_l$ podemos negligenciá-los e introduzir um \engw{wake} longitudinal que só depende de $s$,

\begin{equation}
	w_l(s) = w_l(\vect{0},\vect{0},s).
\end{equation}

Se os elementos que formam a câmara de vácuo também tem alguma simetria (por exemplo, se eles tem seção transversal circular, elíptica ou retangular), o \engw{wake} transverso no eixo, onde $(\vect{\rho},\vect{\rho'}) = (\vect{0},\vect{0})$, é nulo, $\vect{w}_t(\vect{0},\vect{0},s)=\vect{0}$. Para pequenos valores de $(\vect{\rho},\vect{\rho'})$ nós podemos expandir $\vect{w}_t(\vect{\rho},\vect{\rho'},s)$ mantendo apenas os termos lineares. Dessa forma, obtemos uma relação tensorial entre os \engw{wakes} transversos e os deslocamentos,

\begin{equation}
	\vect{w}_t(\vect{\rho},\vect{\rho'},s) = \overleftrightarrow{\vect{W}_1}(s)\vect{\rho} +
    										 \overleftrightarrow{\vect{W}_2}(s)\vect{\rho'},
\end{equation}
onde $\overleftrightarrow{\vect{W}_1}$ e $\overleftrightarrow{\vect{W}_2}$ são tensores bidimensionais de ordem 2.

\subsection{Sistemas com Simetria axial}

Em um sistema com simetria axial, o \engw{wake potential}, $W$, depende apenas dos módulos de $\vect{\rho}$ e $\vect{\rho'}$ e do ângulo, $\theta$ entre eles. Sempre podemos escolher um sistema de coordenadas tal que o vetor $\vect{\rho'}$ fique no plano $xz$, veja a figura \ref{fig:6}, de forma que $W$ será uma função periódica e par \todo{entender porque é par} do ângulo $\theta$ em um sistema de coordenadas cilíndrico. Decompondo $W$ em séries de Fourier em $\theta$ temos:

\begin{equation}
	W(\rho,\rho',\theta,s) = \sum_{m=0}^\infty W_m (\rho,\rho',s) \cos(m\theta).
\end{equation}

Inserindo essa equação na equação \eqref{eq:wake_potential_harmonic}, temos

\begin{equation}
	\sum_{m=0}^\infty \left(\frac1\rho\derpar{}{\rho}\rho\derpar{W_m}{\rho} -
    					    \frac{m^2}{\rho^2}W_m\right)\cos(m\theta) = 0
\end{equation}
de onde podemos encontrar a dependência explícita em $\rho$ de $W$,
\begin{equation}\label{eq:wake_potential_of_rho}
	W_m(\rho,\rho',s) = A_m(\rho',s)\rho^m.
\end{equation}
Na equação \eqref{eq:wake_potential_of_rho} a solução singular na origem, $W_m \propto \rho^{-m}$ foi descartada.

Também é possível encontrar a dependência de $W_m$ em função de $\rho'$, \todo{Achar essa derivação e incluir aqui} ver \cite{Bane_PAC1983}, que é
\begin{equation}
	A_m(\rho',s) = F_m(s)\rho'^m.
\end{equation}

Usando a equação \eqref{eq:wake_function_definition} podemos calcular os \engw{wake functions}
\begin{equation}
	w_l = \sum w_l^{(m)}, \qquad \vect{w}_t = \sum \vect{w}_t^{(m)}
\end{equation}
onde
\begin{equation}\label{eq:wake_function_cylindrical}\begin{aligned}
w_l^{(m)} &= \rho'^m\rho^mF'_m(s)\cos(m\theta),\\
\vect{w}_t^{(m)} &= m\rho'^m\rho^{m-1} F_m(s)\left[\vect{\hat{r}}\cos(m\theta) -
												  \vect{\hat{\theta}}\sin(m\theta)\right]
\end{aligned}\end{equation}
onde $\vect{\hat{r}}$ e $\vect{\hat{\theta}}$ são vetores unitários nas direções radial e azimutal no sistema cilíndrico de coordenadas e $F'_m$ é a derivada de $F_m$ em relação à $s$. Lembre que nessas equações nós assumimos que a partícula fonte está em $\theta = 0$.

As equações \eqref{eq:wake_function_cylindrical} são válidas para valores arbitrários de $\rho$ e $\rho'$. Próximo ao eixo, onde os deslocamentos são pequenos, os termos de ordem mais alta, com valores grandes de $m$, também ficam pequenos. Nesse caso podemos manter apenas os termos não nulos de mais baixa ordem,
\begin{equation}\label{eq:wake_function_expanded}\begin{aligned}
	w_l & \equiv w_l^{(0)} = F'_0(s), \\
    \vect{w}_t & \equiv \vect{w}_t^{(1)} = F_1(s)\rho'\left(\vect{\hat{r}}\cos(\theta) - \vect{\hat{\theta}}\sin(\theta)\right) = \vect{\rho'} F_1(s)
\end{aligned}\end{equation}
onde a última igualdade da última equação é justificada porque $\vect{\hat{r}}\cos(\theta) - \vect{\hat{\theta}}\sin(\theta) = \vect{\hat{x}}$, que é a direção em que está a partícula fonte, de acordo com nossa definição inicial.

Geralmente o \engw{wake} transversal definido na equação \eqref{eq:wake_function_expanded} é redefinido como
\begin{equation}
	w_t(s) \equiv \frac{|\vect{w}_t|}{|\vect{\rho'}|} = F_1(s)
\end{equation}
e adquire a unidade \si{\volt\per\coulomb\per\meter}. De acordo com essa definição, um \engw{wake} transversal positivo significa um impulso na direção do deslocamento da partícula fonte (caso ambas cargas tenham o mesmo sinal).


%%%%%%%%%%%%%%%%%%%%%%%%%%%%%%%%%%%%%%%%%%%%%%%%%%%%%%%%%%%%%%%%%%%%%%%%%%%%%%%%%%%%%%%%%%%%%%%%%%%%%%%%%
%%%%%%%%%%%%%%%%%%%%%%%%%%%%%%%%%%%%%%%%%%%%%%%%%%%%%%%%%%%%%%%%%%%%%%%%%%%%%%%%%%%%%%%%%%%%%%%%%%%%%%%%%
\section{Impedâncias}

Nesta seção vamos definir o conceito de impedância e derivar algumas de suas propriedades.

%%%%%%%%%%%%%%%%%%%%%%%%%%%%%%%%%%%%%%%%%%%%%%%%%%%%%%%%%%%%%%%%%%%%%%%%%%%%%%%%%%%%%%%%%%%%%%%%%%%%%%%%%
\subsection{Definição de Impedância}
O conhecimento dos \engw{wake functions} longitudinal e transversal nos dá uma informação completa, dentro da aproximação de feixe rígido, sobre a interação eletromagnética do feixe com o seu ambiente. Contudo, em muitos casos, especialmente no estudo de instabilidades do feixe, é mais conveniente usar a Transformada de Fourier dos \engw{wake functions} ou as impedâncias. Também, geralmente é mais fácil calcula a impedância para uma dada geometria da câmara de vácuo ao invés da \engw{wake function}.

Por razões históricas a impedâncias longitudinal, $Z_l$, e transversal, $Z_t$, são definidas como a Transformada de Fourier dos \engw{wakes} com fatores multiplicativos diferentes,
\begin{equation}\label{eq:impedances_definition}\begin{aligned}
Z_l(\omega) &= \frac1c \defint{s}{w_l(s)e^{i\omega s/c}}{0}{\infty},\\
Z_t(\omega) &= -\frac{i}{c} \defint{s}{w_t(s)e^{i\omega s/c}}{0}{\infty}.
\end{aligned}\end{equation}
Note que a integração na equação \eqref{eq:impedances_definition} pode ser estendida para a região de valores negativos de $s$, porque $w_l$ e $w_t$ são nulos naquela região. Além disso, a impedância pode ser definida para valores complexos de $\omega$, desde que $\Im(\omega) > 0 $ para que a integral convirja. Dessa forma, a impedância é uma função analítica no plano superior da variável complexa $\omega$.

Um aspecto importante a respeito da definição de impedância é que há divergências em sua definição na literatura. Por exemplo, as referências \cite{Zotter1993} e \cite{Wilson1987} definem a impedância longitudinal como o complexo conjugado da equação \eqref{eq:impedances_definition}. Nesse trabalho estamos seguindo a definição das referências \cite{CHao1993,Stupakov2000a,Heifets1991}.


    \subsection{Causality and Catch-up Distance}
    \subsection{Wake-function Definition}
    \subsection{Panofsky-Wenzel Theorem}
    \subsection{Symmetry Analysis}
    \subsection{Potential of Bunch of particles}
  \section{Impedances}
    \subsection{Impedance Definition}
    \subsection{Impedance Properties}
      \subsubsection{Causality}
      \subsubsection{Passivity}
      \subsubsection{Energy Loss}
  \section{Classification of Impedances and Wake-functions}
    \subsection{As Resonators}
      \subsubsection{Broad-band}
      \subsubsection{Narrow-Band}
    \subsection{As circuit components}
      \subsubsection{Inductive}
      \subsubsection{Capacitive}
      \subsubsection{Resistive}
    \subsection{According to its source}
      \subsubsection{Resistive Wall}
      \subsubsection{Geometric}
      \subsubsection{Coherent Synchrotron Radiation}
  \section{Impedance Calculation}
    \subsection{Analytic Methods}
      \subsubsection{Field Matching Technique}
      \subsubsection{Perturbative Method}
      \subsubsection{Parabolic Equation}
    \subsection{Numeric Methods}
      \subsubsection{Frequency Domain Solvers}
      \subsubsection{2D Time domain Solvers}
      \subsubsection{3D Time domain Solvers}

\chapter{Collective Effects}
  \section{Energy Loss}
  \section{Potential-Well Distortion}
  \section{Tune-shifts}
    \subsection{Coherent Tune-shifts}
    \subsection{Incoherent Tune-shifts}
  \section{Instabilities}
    \subsection{Multi-Turn Instabilities}
    \subsection{Single-Turn Instabilities}
  \section{Landau damping}
  \section{Analytical Treatment}
    \subsection{The Linearized Fokker-Planck Equation}
      \subsubsection{Stationary Solution: Haissinski Equation}
      \subsubsection{Modal expantion}
        \paragraph{Head-tail modes}
      \subsubsection{Solution for Gaussian Bunches}
      \subsubsection{Low Current Limit: Multi-Turn Instabilities}
      \subsubsection{High Current Limit: Mode Coupling Instabilities}
    \subsection{The Microwave Instability}
    \subsection{Strong Head-tail instability}
  \section{Numerical Treatment}
    \subsection{Tracking Codes}
      \subsubsection{Lump of Impedance Kicks}
      \subsubsection{Slicing and particle deposition}
      \subsubsection{Single-Bunch tracking codes}
      \subsubsection{Multi-Bunch tracking codes}


\chapter{Experimental Methods}
  \section{Impedance Measurement}
    \subsection{Direct Measurement}
      \subsubsection{Wire technique}
      \subsubsection{Double wire technique}
    \subsection{Beam Measument}
      \subsubsection{Single-bunch Current Dependent Tune-shifts}
      \subsubsection{Multi-bunch Current Dependent Tune-shifts}
      \subsubsection{Single-bunch Turn-by-Turn measurements}
      \subsubsection{Orbit bump method}
      \subsubsection{response matrix fitting}
      \subsubsection{Instability Threshold Measurements}
  \section{Instability Cures}
    \subsection{Landau Cavity}
    \subsection{Chromaticity}
    \subsection{Cavity power absorbers}
    \subsection{Cavity temperature adjustments}
    \subsection{Bunch-by-Bunch Feedback System}
      \subsubsection{How it works}
      \subsubsection{Possible Experiments}
        \paragraph{Grow-Damp Experiments}
        \paragraph{Tune measurement}
        \paragraph{Beam Response Function Measurement}
      \subsubsection{Main Limitations}

\chapter{Methodology}
  \section{}

\chapter{Results}
  \section{Components Impedance Modelling}
    \subsection{Multi-Layer Resistive Wall}
      \subsubsection{Pipe}
      \subsubsection{Ceramic Chambers}
      \subsubsection{Thin Chambers}
    \subsection{Geometric Transitions}
      \subsubsection{2D-Calculations}
        \paragraph{BC Chamber}
        \paragraph{RF Cavity taper}
        \paragraph{Undulators Tapers}
      \subsubsection{3D-Calculations}
        \paragraph{BPMs}
        \paragraph{Bellows}
        \paragraph{Radiation Masks}
    \subsection{\Glsentryfull{csr}}
  \section{Impedance Budget}
    \subsection{Longitudinal Impedance}
      \subsubsection{Effective Z/n}
      \subsubsection{Multi-bunch Kloss and Dissipated Power}
    \subsection{Vertical Impedance}
    \subsection{Horizontal Impedance}
  \section{Simulations and Instabilities Thresholds}
    \subsection{Frequency Domain Calculations}
      \subsubsection{Vertical Plane}
        \paragraph{Single-bunch Tune-Shifts}
        \paragraph{Multi-bunch Tune-Shifts}
        \paragraph{Coupled-Bunch Instabilities}
        \paragraph{Single-Bunch Instabilities}
      \subsubsection{Horizontal Plane}
        \paragraph{Single-bunch Tune-Shifts}
        \paragraph{Multi-bunch Tune-Shifts}
        \paragraph{Coupled-Bunch Instabilities}
        \paragraph{Single-Bunch Instabilities}
      \subsubsection{Longitudinal Plane}
        \paragraph{Multi-Bunch Instabilities}
        \paragraph{Single-Bunch Instabilities}
    \subsection{Time Domain Calculations}
      \subsubsection{Longitudinal Plane}
      \subsubsection{Vertical Plane}
      \subsubsection{Horizontal Plane}


	% Finaliza a parte no bookmark do PDF, para que se inicie o bookmark na raiz
	\bookmarksetup{startatroot}%

	\chapter*[Conclusão]{Conclusão}
	\addcontentsline{toc}{chapter}{Conclusão}
	\lipsum[1-5]


	% --------- Elementos pós-textuais --------------&
	\postextual
	% Referências bibliográficas
%\bibliographystyle{unsrt}
\bibliography{library}


% Apêndices
\begin{apendicesenv}
% Imprime uma página indicando o início dos apêndices
\partapendices
\chapter{Derivações}

\section{Duas partículas interagindo no vácuo}

\begin{figure}[hb!]
\centering
\begin{tikzpicture}[scale=1]
\def\d{1cm}
\draw[<->] (1,0) node[below]{$z$} 
		-- ++(-\d,0) node[below left] {$S$} 
        -- ++(0,\d) node[left] {$\rho$}; %coord sys S
\draw[<->] (6*\d,0) node[below]{$z'$}
		-- ++(-\d,0) node[below left] {$S'$} 
        -- ++(0,\d) node[left] {$\rho'$}; % coord sys S'
\coordinate (V) at (0.5,0);
\coordinate (Q1) at (4cm,1.5cm);
\coordinate (Q2) at (0.5cm,2.5cm);
\draw[->] (5*\d,0.5*\d) -- ++(V) node[above] {$\boldsymbol{v}$};
\filldraw[fill=black] (Q1) circle[radius=0.05] node[above] {$q$}; % source particle
\draw[->] (Q1) -- ++(V) node[above] {$\boldsymbol{v}$}; % velocity vector
\filldraw[fill=black] (Q2)  circle[radius=0.05]; % test particle
\draw[->] (Q2) -- ++(V) node[above] {$\boldsymbol{v}$}; % velocity vector
\draw[dashed,|-|] ($(Q1)-(0,0.2)$)
				   let \p1 = ($(Q2) - (Q1)$)
                   in -- ++(\x1,0) node[midway,below] {$s$}; %horizontal distance
\draw[dashed,|-|] ($(Q2)-(0.2,0)$)
				   let \p1 = ($(Q1) - (Q2)$)
                   in -- ++(0,\y1) node[midway,left] {$\rho$}; % vertical distance
\draw[->] (Q1) -- ++($(Q2) - (Q1)$) node[midway,above] {$\boldsymbol{R}$}; %vector
\end{tikzpicture}
\caption{Duas partículas interagindo via campo direto.}
\end{figure}

Primeiro, campos gerados por $q_1$. No referencial $S'$:
\begin{align}
\vect{E'} &= \frac{q}{4\pi\epsilon_0} \frac{\vect{\hat{r'}}}{r'^2} \\
\vect{B'} &= \vect{0'}
\end{align}
assumindo que a partícula 1 está na origem do sistema de coordenadas $S'$.

Lembrando que a transformação entre coordenadas esféricas para cilíndricas são:
\begin{align}
r' &= \sqrt{s'^2 + \rho'^2} \\
\vect{\hat{r'}} &= \cos\theta'\vect{\hat{z'}} + \sin\theta'\vect{\hat{\rho'}} =
                  -\frac{s'}{r'}\vect{\hat{z'}} + \frac{\rho'}{r'}\vect{\hat{\rho'}}
\end{align}
onde a coordenada $\phi$ fica inalterada.

Assim podemos reescrever o campo elétrico em suas partes longitudinal e transversal:
\begin{align}
\vect{E'}_{||} &= \frac{q}{4\pi\epsilon_0} \frac{\cos\theta'\vect{\hat{z'}}}{r'^2} = 
                 -\frac{q}{4\pi\epsilon_0} \frac{s'\vect{\hat{z'}}}{(\rho'^2+s'^2)^{3/2}} \\
\vect{E'}_{\perp} &= \frac{q}{4\pi\epsilon_0} \frac{\sin\theta'\vect{\hat{\rho'}}}{r'^2} = 
                     \frac{q}{4\pi\epsilon_0} \frac{x'\vect{\hat{\rho'}}}{(\rho'^2+s'^2)^{3/2}}
\end{align}

Lembrando as equações de transformação de Lorentz para campos elétricos e magnéticos, para esse problema:

\begin{align}
 \vect{E}_{||} &= \vect{E'}_{||}\\
 \vect{B}_{||} &= \vect{B'}_{||}\\
 \vect{E}_\perp &= \gamma\left(\vect{E'}_\perp - \vect{v}\times\vect{B'}\right)\\
 \vect{B}_\perp &= \gamma\left(\vect{B'}_\perp + \frac{1}{c^2}\vect{v}\times\vect{E'}\right)
\end{align}

Ainda, as coordenadas espaciais são transformadas da seguinte maneira:

\begin{align}
\vect{\hat{\rho}} &=\vect{\hat{\rho'}}, \quad \vect{\hat{z}} = \vect{\hat{z'}}\\
\rho &= \rho', \quad z = \frac{z'}{\gamma}
\end{align}

Assim, podemos notar que:

\begin{align}
\vect{E}_{||} &= -\frac{q}{4\pi\epsilon_0} \frac{s'\vect{\hat{z'}}}{(\rho'^2+s'^2)^{3/2}} = 
				 -\frac{q}{4\pi\epsilon_0} \frac{\gamma s\vect{\hat{z}}}{((\gamma s)^2+\rho^2)^{3/2}}=
                 -\frac{q}{4\pi\epsilon_0} \frac{s\vect{\hat{z}}}{\gamma^2R^{*3}} \\
\vect{E}_\perp&= \frac{q}{4\pi\epsilon_0}\frac{\gamma \rho'\vect{\hat{\rho'}}}{(\rho'^2+s'^2)^{3/2}} = 
				 \frac{q}{4\pi\epsilon_0}\frac{\gamma \rho\vect{\hat{\rho}}}{((\gamma s)^2+\rho^2)^{3/2}}=
                 \frac{q}{4\pi\epsilon_0} \frac{\rho\vect{\hat{\rho}}}{\gamma^2R^{*3}} \\
\vect{B}_\perp&= -\frac{\gamma   vE'_\perp}{c^2}\vect{\hat{\phi'}} =
		         -\frac{vE_\perp}{c^2}\vect{\hat{\phi}}
\end{align}
onde $R^* = \sqrt{s^2+\left(\rho/\gamma\right)^2}$

Agora podemos analisar a força exercida pela partícula fonte sobre a partícula teste usando a força de Lorentz
\begin{equation}
\vect{F} = \left(\vect{E} + \vect{v}\times\vect{B}\right)
\end{equation}
onde foi assumida carga unitária para a partícula teste, e os campos calculados anteriormente.

Assumindo que a velocidade da partícula teste é a mesma da partícula fonte (mesmo módulo e direção), as componentes longitudinal e transversal da força ficam:

\begin{align}
\vect{F}_{||} &= \vect{E}_{||} = -\frac{q}{4\pi\epsilon_0} \frac{s\vect{\hat{z}}}{\gamma^2R^{*3}}\\
\vect{F}_\perp&= \left(1+\frac{\vect{v}\times\vect{v}\times}{c^2}\right)\vect{E}_\perp = 
                 \left(1-\frac{v^2}{c^2}\right)\vect{E}_\perp = 
                 \frac{q}{4\pi\epsilon_0} \frac{\rho\vect{\hat{z}}}{\gamma^4R^{*3}}
\end{align}
onde vemos que a força longitudinal tende a zero proporcionalmente a $\gamma^{-2}$ quando $v \to c$ e que a força longitudinal tende a zero com $\gamma^{-4}$, se $s>\rho/\gamma$ e com $\gamma^{-1}$ se $s<\rho/\gamma$.

Agora, vamos assumir que a velocidade da partícula teste não é paralela à velocidade da partícula fonte
\begin{equation}
\vect{v_2} = v(\cos\delta \vect{\hat{z}} + \sin\delta\vect{\hat{\rho}})
\end{equation}

Assim, a força sofrida por essa partícula fica

\begin{equation}\begin{aligned}
\vect{F} &= \vect{E}_{||} + \vect{E}_\perp - v(\cos\delta \vect{\hat{z}} + \sin\delta\vect{\hat{\rho}}) \times \vect{\hat{\phi}}\frac{vE_\perp}{c^2} \\
&= \left(1-\frac{v^2}{c^2}\cos\delta\right)\vect{E}_\perp + \vect{E}_{||} + \frac{v^2}{c^2}E_\perp\sin\delta\vect{\hat{z}}
\end{aligned}\end{equation}

Olhando essa expressão, vemos que, conforme $v \to c$, a força pode ser expressa como:

\begin{equation}
\vect{F} = \left\{
\begin{aligned}
\left((1-\cos\delta)\vect{\hat{\rho}} + \sin\delta\vect{\hat{z}}\right) 
\frac{q}{4\pi\epsilon_0} \frac{\gamma\vect{\hat{\rho}}}{\rho^2} & &|s|<\rho/\gamma\\
0 & &|s|>\rho/\gamma
\end{aligned}\right.
\end{equation}






\chapter{Apêndice 2}
\lipsum[55-57]
\end{apendicesenv}


% Anexos
\begin{anexosenv}
% Imprime uma página indicando o início dos anexos
\partanexos
\chapter{Anexo 1}
\lipsum[30]
\chapter{Anexo 2}
\lipsum[31]
\chapter{Anexo 3}
\lipsum[32]
\end{anexosenv}


% Índice remissivo
\printindex


\end{document}
