\chapter{Derivações}

\section{Duas partículas interagindo no vácuo}

\begin{figure}[hb!]
\centering
\begin{tikzpicture}[scale=1]
\def\d{1cm}
\draw[<->] (1,0) node[below]{$z$} 
		-- ++(-\d,0) node[below left] {$S$} 
        -- ++(0,\d) node[left] {$\rho$}; %coord sys S
\draw[<->] (6*\d,0) node[below]{$z'$}
		-- ++(-\d,0) node[below left] {$S'$} 
        -- ++(0,\d) node[left] {$\rho'$}; % coord sys S'
\coordinate (V) at (0.5,0);
\coordinate (Q1) at (4cm,1.5cm);
\coordinate (Q2) at (0.5cm,2.5cm);
\draw[->] (5*\d,0.5*\d) -- ++(V) node[above] {$\boldsymbol{v}$};
\filldraw[fill=black] (Q1) circle[radius=0.05] node[above] {$q$}; % source particle
\draw[->] (Q1) -- ++(V) node[above] {$\boldsymbol{v}$}; % velocity vector
\filldraw[fill=black] (Q2)  circle[radius=0.05]; % test particle
\draw[->] (Q2) -- ++(V) node[above] {$\boldsymbol{v}$}; % velocity vector
\draw[dashed,|-|] ($(Q1)-(0,0.2)$)
				   let \p1 = ($(Q2) - (Q1)$)
                   in -- ++(\x1,0) node[midway,below] {$s$}; %horizontal distance
\draw[dashed,|-|] ($(Q2)-(0.2,0)$)
				   let \p1 = ($(Q1) - (Q2)$)
                   in -- ++(0,\y1) node[midway,left] {$\rho$}; % vertical distance
\draw[->] (Q1) -- ++($(Q2) - (Q1)$) node[midway,above] {$\boldsymbol{R}$}; %vector
\end{tikzpicture}
\caption{Duas partículas interagindo via campo direto.}
\end{figure}

Primeiro, campos gerados por $q_1$. No referencial $S'$:
\begin{align}
\vect{E'} &= \frac{q}{4\pi\epsilon_0} \frac{\vect{\hat{r'}}}{r'^2} \\
\vect{B'} &= \vect{0'}
\end{align}
assumindo que a partícula 1 está na origem do sistema de coordenadas $S'$.

Lembrando que a transformação entre coordenadas esféricas para cilíndricas são:
\begin{align}
r' &= \sqrt{s'^2 + \rho'^2} \\
\vect{\hat{r'}} &= \cos\theta'\vect{\hat{z'}} + \sin\theta'\vect{\hat{\rho'}} =
                  -\frac{s'}{r'}\vect{\hat{z'}} + \frac{\rho'}{r'}\vect{\hat{\rho'}}
\end{align}
onde a coordenada $\phi$ fica inalterada.

Assim podemos reescrever o campo elétrico em suas partes longitudinal e transversal:
\begin{align}
\vect{E'}_{||} &= \frac{q}{4\pi\epsilon_0} \frac{\cos\theta'\vect{\hat{z'}}}{r'^2} = 
                 -\frac{q}{4\pi\epsilon_0} \frac{s'\vect{\hat{z'}}}{(\rho'^2+s'^2)^{3/2}} \\
\vect{E'}_{\perp} &= \frac{q}{4\pi\epsilon_0} \frac{\sin\theta'\vect{\hat{\rho'}}}{r'^2} = 
                     \frac{q}{4\pi\epsilon_0} \frac{x'\vect{\hat{\rho'}}}{(\rho'^2+s'^2)^{3/2}}
\end{align}

Lembrando as equações de transformação de Lorentz para campos elétricos e magnéticos, para esse problema:

\begin{align}
 \vect{E}_{||} &= \vect{E'}_{||}\\
 \vect{B}_{||} &= \vect{B'}_{||}\\
 \vect{E}_\perp &= \gamma\left(\vect{E'}_\perp - \vect{v}\times\vect{B'}\right)\\
 \vect{B}_\perp &= \gamma\left(\vect{B'}_\perp + \frac{1}{c^2}\vect{v}\times\vect{E'}\right)
\end{align}

Ainda, as coordenadas espaciais são transformadas da seguinte maneira:

\begin{align}
\vect{\hat{\rho}} &=\vect{\hat{\rho'}}, \quad \vect{\hat{z}} = \vect{\hat{z'}}\\
\rho &= \rho', \quad z = \frac{z'}{\gamma}
\end{align}

Assim, podemos notar que:

\begin{align}
\vect{E}_{||} &= -\frac{q}{4\pi\epsilon_0} \frac{s'\vect{\hat{z'}}}{(\rho'^2+s'^2)^{3/2}} = 
				 -\frac{q}{4\pi\epsilon_0} \frac{\gamma s\vect{\hat{z}}}{((\gamma s)^2+\rho^2)^{3/2}}=
                 -\frac{q}{4\pi\epsilon_0} \frac{s\vect{\hat{z}}}{\gamma^2R^{*3}} \\
\vect{E}_\perp&= \frac{q}{4\pi\epsilon_0}\frac{\gamma \rho'\vect{\hat{\rho'}}}{(\rho'^2+s'^2)^{3/2}} = 
				 \frac{q}{4\pi\epsilon_0}\frac{\gamma \rho\vect{\hat{\rho}}}{((\gamma s)^2+\rho^2)^{3/2}}=
                 \frac{q}{4\pi\epsilon_0} \frac{\rho\vect{\hat{\rho}}}{\gamma^2R^{*3}} \\
\vect{B}_\perp&= -\frac{\gamma   vE'_\perp}{c^2}\vect{\hat{\phi'}} =
		         -\frac{vE_\perp}{c^2}\vect{\hat{\phi}}
\end{align}
onde $R^* = \sqrt{s^2+\left(\rho/\gamma\right)^2}$

Agora podemos analisar a força exercida pela partícula fonte sobre a partícula teste usando a força de Lorentz
\begin{equation}
\vect{F} = \left(\vect{E} + \vect{v}\times\vect{B}\right)
\end{equation}
onde foi assumida carga unitária para a partícula teste, e os campos calculados anteriormente.

Assumindo que a velocidade da partícula teste é a mesma da partícula fonte (mesmo módulo e direção), as componentes longitudinal e transversal da força ficam:

\begin{align}
\vect{F}_{||} &= \vect{E}_{||} = -\frac{q}{4\pi\epsilon_0} \frac{s\vect{\hat{z}}}{\gamma^2R^{*3}}\\
\vect{F}_\perp&= \left(1+\frac{\vect{v}\times\vect{v}\times}{c^2}\right)\vect{E}_\perp = 
                 \left(1-\frac{v^2}{c^2}\right)\vect{E}_\perp = 
                 \frac{q}{4\pi\epsilon_0} \frac{\rho\vect{\hat{z}}}{\gamma^4R^{*3}}
\end{align}
onde vemos que a força longitudinal tende a zero proporcionalmente a $\gamma^{-2}$ quando $v \to c$ e que a força longitudinal tende a zero com $\gamma^{-4}$, se $s>\rho/\gamma$ e com $\gamma^{-1}$ se $s<\rho/\gamma$.

Agora, vamos assumir que a velocidade da partícula teste não é paralela à velocidade da partícula fonte
\begin{equation}
\vect{v_2} = v(\cos\delta \vect{\hat{z}} + \sin\delta\vect{\hat{\rho}})
\end{equation}

Assim, a força sofrida por essa partícula fica

\begin{equation}\begin{aligned}
\vect{F} &= \vect{E}_{||} + \vect{E}_\perp - v(\cos\delta \vect{\hat{z}} + \sin\delta\vect{\hat{\rho}}) \times \vect{\hat{\phi}}\frac{vE_\perp}{c^2} \\
&= \left(1-\frac{v^2}{c^2}\cos\delta\right)\vect{E}_\perp + \vect{E}_{||} + \frac{v^2}{c^2}E_\perp\sin\delta\vect{\hat{z}}
\end{aligned}\end{equation}

Olhando essa expressão, vemos que, conforme $v \to c$, a força pode ser expressa como:

\begin{equation}
\vect{F} = \left\{
\begin{aligned}
\left((1-\cos\delta)\vect{\hat{\rho}} + \sin\delta\vect{\hat{z}}\right) 
\frac{q}{4\pi\epsilon_0} \frac{\gamma\vect{\hat{\rho}}}{\rho^2} & &|s|<\rho/\gamma\\
0 & &|s|>\rho/\gamma
\end{aligned}\right.
\end{equation}





