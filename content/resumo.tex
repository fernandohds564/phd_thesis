Sirius é a nova fonte de luz síncrotron de quarta geração que está sendo construída em Campinas, Brasil, pelo Laboratório Nacional de Luz Síncrotron (LNLS). Com uma emitância natural de~\SI{250}{\pico\meter\radian}, radiação síncrotron de altíssimo brilho poderá ser gerada por até~\si{18}~dispositivos de inserção (DI) instalados nos trechos retos do anel de armazenamento e por~\si{20}~dipolos de alto campo (\SI{3.2}{\tesla}) presentes no centro de cada arco acromático da rede magnética. A câmara de vácuo padrão do anel será cilíndrica, feita de cobre e terá~\SI{12}{\milli\meter} de raio, que é um valor pequeno comparado com as câmaras de fontes de luz síncrotron de terceira geração, e os primeiros DIs previstos serão fora do vácuo e terão uma abertura bastante reduzida, com câmaras de apenas~\SI{2.4}{}~e~\SI{3.0}{\milli\meter}~de raio, na maioria dos casos. Adicionalmente, o sistema de vácuo do anel será distribuido, através da superposição de NEG na superfície interna das câmaras ao longo de todo o anel. Todos esses fatores intensificam os campos de arraste, ou impedâncias, da máquina, que podem gerar oscilações coerentes, deteriorando a qualidade da luz gerada, e causar perda total ou parcial do feixe, além de influenciar na dinâmica de equilíbrio do elétrons. Neste trabalho alguns dos principais componentes da câmara de vácuo foram modelados e seus campos de arraste calculados por meios semi-analíticos e numéricos e adicionados ao modelo total de impedância do anel. Com a aplicação de tal modelo para a primeira fase de operação, constatou-se que o feixe será instável nos planos transverais devido à oscilações causadas por acoplamento entre pacotes gerados pela impedância de parede resistiva, tornando necessária a instalação de sistemas de retroalimentação pacote por pacote para manter a estabilidade. Também foi previsto que o feixe ficará estável se o anel for operado com uma cromaticidade nominal de 2.8 em ambos os planos transversais. Constatou-se ainda que as mudanças de sintonia causadas por impedâncias quadrupolares dos DIs serão significativas, e o ponto de operação da máquina deverá ser mudado em função da corrente total. Os limiares das instabilidades relacionadas a oscilações intra-pacote estão muito acima da corrente nominal de operação e não serão um problema e não há previsão de instabilidades longitudinais de acoplamento entre pacotes, haja vista que a máquina operará com cavidades de RF supercondutoras. Na segunda fase de operação está prevista a instalação de uma cavidade Landau, que permitirá operação com corrente total mais alta, inclusive com pacotes bastante intensos no meio do trem. Apesar de não terem sido feitos cálculos para esse tipo de operação, os principais métodos e códigos desenvolvidos nesse trabalho podem ser diretamente usados para tal fim.
