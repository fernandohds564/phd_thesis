%%%%%%%%%%%%%%%%%%%%%%%%%%%%%%%%%%%%%%%%%%%%%%%%%%%%%%%%%%%%%%%%%%%%%%%%%%%%%%%%%%%%%%%%%%%%%%%%%%%%%%%%%
%%%%%%%%%%%%%%%%%%%%%%%%%%%%%%%%%%%%%%%%%%%%%%%%%%%%%%%%%%%%%%%%%%%%%%%%%%%%%%%%%%%%%%%%%%%%%%%%%%%%%%%%%
\chapter{Física de Aceleradores}

O propósito dessa seção é definir os aspectos fundamentais da dinâmica dos elétrons em um anel de armazenamento de elétrons em uma fonte de luz síncrotron dentro da aproximação de não interação entre partículas. Não há a intenção de ser completo ou rigoroso na definição das quantidades envolvidas, mas apenas familiarizar o leitor com a nomenclatura e principais propriedades do feixe. Referências serão indicadas para os leitores mais curiosos.

Um anel de armazenamento ideal possuiria as seguintes características:
\begin{itemize}
 \item Tempo de vida infinito: Os elétrons armazenados assim continuariam por um tempo infinito, sem colidirem com a câmara de vácuo.
 \item Tamanho transversal muito menor que o limite de difração: ou em outras palavras, emitância do feixe de elétrons muito menor que a emitância do feixe de fótons. Dessa forma, o tamanho do feixe de radiação na amostra seria definido apenas pelo princípio de incerteza dos fótons,
\end{itemize}

\todo{parte tirada do relatório de estágio}
Nesta parte do relatório, será estudada a dinâmina linear do movimento dos elétrons no anel de armazenamento. Em suma, essa abordagem considera o efeito dos campos produzidos por dipolos e quadrupolos, que são lineares com as coordenadas espaciais do movimento.

%%%%%%%%%%%%%%%%%%%%%%%%%%%%%%%%%%%%%%%%%%%%%%%%%%%%%%%%%%%%%%%%%%%%%%%%%%%%%%%%%%%%%%%%%%%%%%%%%%%%%%%%%
%%%%%%%%%%%%%%%%%%%%%%%%%%%%%%%%%%%%%%%%%%%%%%%%%%%%%%%%%%%%%%%%%%%%%%%%%%%%%%%%%%%%%%%%%%%%%%%%%%%%%%%%%
\section{Dinâmica Linear Transversal}

%%%%%%%%%%%%%%%%%%%%%%%%%%%%%%%%%%%%%%%%%%%%%%%%%%%%%%%%%%%%%%%%%%%%%%%%%%%%%%%%%%%%%%%%%%%%%%%%%%%%%%%%%
\subsection{Equações de Movimento}

A primeira etapa do estudo consiste em definir um sistema de coordenadas adequado, representado na \mbox{Figura~\ref{fig:coord}}. Trata-se de um sistema de Frenet-Serret com torção nula (órbita plana), acoplado a órbita ideal da partícula.
As equações que definem esse sistema são \cite{frenet}:
\begin{align} \label{eq:coord}
 \hat{\mathbf{s}}(s) &= \frac{d\mathbf{r}(s)}{d s} \nonumber\\
 \hat{\mathbf{x}}(s) &= -\rho(s)\,\hat{\mathbf{s}}'(s) \qquad
\hat{\mathbf{x}}'(s)=\frac{1}{\rho(s)}\hat{\mathbf{s}}(s) \\
 \hat{\mathbf{z}}(s) &= \hat{\mathbf{s}}(s)\times\hat{\mathbf{x}}(s)
\nonumber
\end{align}
onde $\rho(s)$ é o raio de curvatura da órbita, o apóstrofo indica derivação com relação a coordenada longitudinal, $s$, e os vetores unitários $\hat{\mathbf{s}}$, longitudinal, $\hat{\mathbf{x}}$, radial,
$\hat{\mathbf{z}}$, vertical, nessa ordem, formam uma base ortonormal dextrógira.

% \begin{figure}[!ht]
%  \center
%  %\includegraphics[scale=1.2]{Imagens/coord.png}
%  \caption{Sistema de coordenadas de Frenet-Serret usado no estudo de dinâmica de feixes em anéis de armazenamento.}
%  \label{fig:coord}
% \end{figure}

Os imãs de um anel de armazenamento são construídos e dispostos de forma que uma partícula com energia nominal, e tendo as condições iniciais corretas, se mova ao longo de uma órbita fechada ideal, enquanto as outras partículas apresentam
desvios em relação a essa órbita. Dessa forma, torna-se matematicamente mais fácil estudar o problema usando o sistema de coordenadas descrito acima, pois é possível truncar expansões dos campos magnéticos, além de que as expressões obtidas são de fácil interpretação.

Como já foi dito, os campos magnéticos tratados no estudo da dinâmica linear são aqueles gerados por dipolos e quadrupolos. Os dipolos são componentes magnéticos construídos para gerar um campo que não dependa das coordenadas transversais na proximidade da órbita ideal e perpendicular ao plano dessa
órbita. São eles que, juntamente com a energia nominal dos elétrons, definem a órbita fechada ideal e, consequentemente, seu raio de curvatura:
\begin{equation}
 \frac{1}{\rho(s)}=G(s)=\frac{ecB_0(s)}{E_0}
\end{equation}
em que $G(s)$ é a função curvatura, $e$ é a carga do elétron, $c$ a velocidade da luz, $B_0$ o campo gerado pelo dipolo e $E_0$ a energia nominal. Na equação acima fica clara a natureza puramente geométrica da função curvatura e sua periodicidade, além do fato de sua integral ao longo da órbita
ideal em todo o anel ser $2\pi$.

Já os quadrupolos, tem a função de focalizar ou defocalizar o feixe de elétrons, sendo projetados para gerar um campo nulo na órbita ideal, e um gradiente da seguinte forma:
\begin{equation}
 B_z(s,x,z)=\left(\frac{\partial B}{\partial x}\right)_{0,s}x \qquad \mathrm{e}
\qquad B_x(s,x,z)=\left(\frac{\partial B}{\partial x}\right)_{0,s}z
\end{equation}
onde a segunda equação é consequência de $\nabla \times B = 0$. A inexistência do termo $\partial B / \partial z$ é imposta na especificação do quadrupolo, devido ao fato dessa componente gerar um acoplamento entre os movimentos radial e vertical.
Alguns quadrupolos, chamados quadrupolos skew, são produzidos para terem apenas esse termo, e são inseridos no anel para reduzir o acoplamento gerado por campos espúrios, ou para aumentá-lo, dada a estreita relação desse fator com o tempo de vida do feixe de elétrons.
Também nesse caso pode-se definir uma função geométrica, a focalização:
\begin{equation}
  K_1(s)=\frac{ec}{E_0}\left(\frac{\partial B}{\partial x}\right)_{0,s}.
\end{equation}
que também possui periodicidade, como a função curvatura.

Os anéis de armazenamento são constituidos de vários superperíodos, ou seja, células idênticas repetidas para gerar uma órbita fechada. Nesses casos, a periodicidade das funções de curvatura e focalização são iguais ao comprimento de um superperíodo.

Com as definições acima, é possível obter a equação de movimento para um elétron. Há várias ferramentas matemáticas para isso. Dentre elas, destaca-se a formulação Hamiltoniana da mecânica clássica.

O procedimento a seguir nesse caso é, resumidamente, o seguinte: monta-se a Lagrangiana da partícula, que é o produto escalar do 4-vetor potencial eletromagnético com o 4-vetor velocidade e aplica-se a ela uma transformada de Legendre para substituir a variável velocidade pelo momento canônico. Após isso, faz-se uma transformação canônica na Hamiltoniana, para escrevê-la no sistema de coordenadas de Frenet-Serret, definido pela \mbox{equação~\eqref{eqcoord}}, e muda-se a variável independente, de $t$
(tempo)
para $s$.

Uma das vantagens deste modo de obtenção da equação de movimento é que até este ponto do desenvolvimento as equações são exatas, de modo que caso se queira estudar efeitos de ordem mais alta, tanto no desvio de energia da partícula, como de campo magnético, basta expandir a Hamiltoniana até tais ordens \cite{Wiedemann2 II}, além de também ser possível estudar o movimento longitudinal e seu acoplamento com a dinâmica transversal \cite{Lee}.

Contudo, como estamos interessados nas contribuições lineares, devemos considerar um desvio de energia dos elétrons até primeira ordem, $E=E_0+\boldsymbol{\epsilon}$, fazer a aproximação paraxial, em que
$\frac{\partial x}{\partial s}\ll 1$ e $\frac{\partial z}{\partial s} \ll 1$, ou seja, $p\approx p_s$ e considerar apenas os campos dipolares e quadrupolares discutidos acima. Com estas considerações, chega-se à seguinte equação \cite{Wiedemann3, Lee, Sands}:
\begin{align}
\label{eq:movhor}
 & x'' = K_x(s)x+G(s)\frac{\boldsymbol{\epsilon}}{E_0} \qquad
K_x(s)=-G^2(s)-K_1(s) &\\
\label{eq:movvert}
 & z'' = K_z(s)z \qquad \qquad \qquad \,\,\,\, K_z = K_1(s) &
\end{align}
onde o apóstrofo indica derivação com relação a coordenada longitudinal $s$. Nota-se na equação acima, pelas definições de $K_x(s)$ e $K_z(s)$, uma propriedade fundamental dos quadrupolos: quando um quadrupolo focaliza o feixe na direção radial (horizontal), ele defocaliza na direção vertical e vice-versa.

A derivação completa das equações de movimento expostas acima pela formulação Hamiltoniana pode ser encontrada nas referências \cite{Lee, Wiedemann3}, sendo que em \cite{Sands} e \cite{Wiedemann3} uma derivação diferente, baseada em argumentos geométricos sobre o movimento de uma partícula com energia
$E=E_0+\boldsymbol{\epsilon}$, é realizada.


%%%%%%%%%%%%%%%%%%%%%%%%%%%%%%%%%%%%%%%%%%%%%%%%%%%%%%%%%%%%%%%%%%%%%%%%%%%%%%%%%%%%%%%%%%%%%%%%%%%%%%%%%
\subsection{Oscilações Betatron}

Considerando apenas partículas com energia igual à nominal, ou seja, fazendo $\boldsymbol{\epsilon} = 0$, as \mbox{equações~\ref{eqmovhor} e \ref{eqmovvert}}, possuem a mesma forma, e podem ser estudadas simultaneamente.

Na maioria dos anéis, as funções de focalização e curvatura são constantes nos trechos com magnetos e nulas nos trechos sem magneto (livres). Dessa forma, a equação do movimento é a equação de Hill:

\begin{equation}\label{eqoscbet}
 y''+K(s)y=0
\end{equation}
onde $y$ representa as coordenadas transversais do movimento, $x$ e $z$. A solução dessa equação é dada por:
\begin{equation}
 y=\left\{\begin{array}{ll}
           a \cos\left(\sqrt{K}s+b\right) & 	K>0 \\
	   a s +b                      &	K=0 \\
	   a \cosh\left(\sqrt{-K}s+b\right) &	K<0
          \end{array}\right.
\end{equation}
onde as constantes $a$ e $b$ são determinadas pelas condições iniciais do problema. De forma que podemos escrever:
\begin{equation}
 \mathbf{y}(s)=M(s|s_0)\mathbf{y}(s_0)
\qquad \mathrm{com}\qquad
\mathbf{y}(s) = \left(\begin{array}{c}
			y(s) \\
			y'(s)
		       \end{array}\right)
\end{equation}
onde $M(s|s_0)$ é a matriz de transferência, que transporta a partícula do ponto inicial $s_0$ para o ponto final $s$.

Como a solução geral da equação de Hill é uma composição de duas soluções linearmente independentes, e as condições iniciais
\begin{equation} \label{eqcondinihill}
\mathbf{y_1}(s_0)=\left(\begin{array}{c}
			1 \\
			0
		       \end{array}\right)
\qquad \mathrm{e} \qquad
\mathbf{y_2}(s_0)=\left(\begin{array}{c}
			0 \\
			1
		       \end{array}\right)
\end{equation}
geram tais soluções, a forma geral da matriz de transferência é:
\begin{equation}\label{eqmatriztransf}
 M(s|s_0)=\left(\begin{array}{cc}
                  C(s,s_0)  &  S(s,s_0) \\
	          C'(s,s_0) &  S'(s,s_0)
		\end{array}\right).
\end{equation}
sendo que $C(s,s_0)$ corresponde à solução da primeira condição inicial de \mbox{\eqref{eqcondinihill}} e $S(s,s_0)$ é solução da segunda. Essas funções são análogas às funções seno e cosseno na resolução do oscilador harmônico.

% \begin{figure}[t]
%  \center
% % \includegraphics[scale=0.8]{Imagens/linha_de_transporte.png}
%  \caption{Linha de transporte genérica onde QF representa um quadrupolo focalizador, QD um defocalizador e B um dipolo \cite{Wiedemann3}.}
%  \label{fig:linhatransporte}
% \end{figure}

Uma propriedade da evolução de uma condição inicial para a final nesse tipo de problema é que a matriz de qualquer intervalo é o produto da matriz dos subintervalos. Assim, a matriz de transferência da linha de transporte representada na \mbox{Figura~\ref{fig:linhatransporte}} seria
\begin{equation}
M(s_8|s_0)=M_8M_7M_6M_5M_4M_3M_2M_1.
\end{equation}

De um modo geral, a matriz de uma volta no anel é dada por:
\begin{displaymath}
 M(s + P|s)=M^n(s)\qquad \mathrm{com} \qquad M(s)=M(s+L|s)
\end{displaymath}
onde $L$ é o comprimento de um superperíodo, $n$ é número de superperiodos e $P=n L$ é o perímetro (comprimento) do anel.

Devido ao grande número de revoluções que os elétrons fazem no anel, as matrizes devem ser limitadas, ou seja, todos os elementos de $M(s)$ devem ser finitos conforme o número de voltas cresce indefinidamente. Para definir a condição de estabilidade, analisaremos os autovalores da matriz de
transferência, que obedecem a seguinte equação:
\begin{equation}
 \lambda^2+\mathrm{Trace}\left(M(s)\right) \lambda +
\mathrm{det}\left(M(s)\right).
\end{equation}
Contudo, $\mathrm{det}\left(M(s)\right)=1$, por uma propriedade da equação de Hill \cite{Lee,Wiedemann3}, e $\mathrm{Trace}\left(M(s)\right)$ não depende de $s$, pois há uma relação de similaridade entre $M(s_1)$ e $M(s_2)$:
\begin{equation} \label{eqsimilaridade}
 M(s_2+L | s_1) = M(s_2) M(s_2 | s_1) = M(s_2 | s_1) M(s_1)
\end{equation}
Assim, fazendo $\mathrm{Trace}\left(M(s)\right)=2 \cos(\Phi)$, onde $\Phi$ é chamado de avanço de fase betatron do superperiodo, descobre-se que $\lambda_\pm = e^{\pm i \Phi}$, com $\Phi$ real se  $\mathrm{Trace}\left(M(s)\right)\leq 2$.

O critério de estabilidade descrito acima é fundamental para a escolha de novas redes magnéticas ou ópticas para um anel, sendo o primeiro vínculo a ser testado neste projeto.

Uma matriz $2\times2$ com as propriedades da matriz de transferência, $\mathrm{det}M=1$ e $\mathrm{Trace}M\leq 2$, pode ser parametrizada sem perda de generalidade da seguinte forma:

\begin{equation}
 M = \left(
\begin{array}{cc}
 \cos(\Phi) + \alpha \sin(\Phi)  & \beta \sin(\Phi) \\
 -\gamma \sin(\Phi) & \cos(\Phi) -\alpha \sin(\Phi)
\end{array} \right) = I \cos(\Phi) + J\sin(\Phi)
\end{equation}
onde $\alpha$, $\beta$ e $\gamma$ são os chamados parâmetros de Courant-Snyder, $\Phi$ é denominado avanço de fase betatron, $I$ é a matriz identidade e
\begin{equation}
 J= \left(\begin{array}{cc}
          \alpha  & \beta \\
          -\gamma & -\alpha
         \end{array} \right)
\quad \mathrm{com} \quad J^2=-I \, \Leftrightarrow
\, \beta \gamma = 1+ \alpha^2
\end{equation}

Usando as propriedades da matriz $J$ e a relação de similaridade  \eqref{eqsimilaridade}, podemos obter a evolução dos parâmetros de Courant-Snyder de um ponto a outro do anel:
\begin{equation}\label{eqevolparcouder}
\left(\begin{array}{c}
       \beta \\ \alpha \\ \gamma
      \end{array}\right) =
\left(\begin{array}{ccc}
       C^2    &   -2 C S      &  S^2   \\
       -C C'  &  C S' + C' S  &  -S S' \\
       C'^2   & -2 C' S'      &  S'^2
      \end{array}\right)
\left(\begin{array}{c}
       \beta_0 \\ \alpha_0 \\ \gamma_0
      \end{array}\right)
\end{equation}

Essa representação matricial das soluções da equação de Hill é muito importante para a simulação e construção de modelos matemáticos de anéis, devido à simplicidade dos cálculos e facilidade de implementação em uma linguagem computacional.
A maioria dos softwares de dinâmica de feixes existentes utiliza esse formalismo para calcular os parâmetros da óptica linear, como os parâmetros de Courant-Snyder e o avanço de fase.

Contudo, para compreendermos o significado físico dos parâmetros $\beta$, $\alpha$ e $\Phi$ e para obtermos outros resultados, é interessante analisarmos a natureza pseudo-harmônica do movimento dos elétrons. Para isso, utilizaremos o teorema de Floquet aplicado à equação de Hill.

Dada a equação \ref{eqoscbet}, o Teorema de Floquet afirma que existem duas soluções linearmente independentes da forma
\begin{equation}
 y_1(s)= w(s) e^{i \psi (s)}, \qquad y_2(s)= w(s) e^{-i \psi (s)}
\qquad \mathrm{com}\quad w(s+L)=w(s).
\end{equation}
Substituindo essas soluções na equação de Hill, obtemos:
\begin{align}
 & & w''= - K w + 1/w^3 \\
 & & \psi' = 1/w^2
\end{align}
onde, pela primeira equação, nota-se que $w$ não muda de sinal, pois uma vez próximo de $0$, o termo $1/w^3$ cresce indefinidamente, fazendo com que $w''$ cresça e $w$ aumente. Dessa forma, definimos $w$ positivo. Como as soluções $y_1$ e $y_2$ são linearmente independentes, podemos escrever a matriz
de um período em função delas:
\begin{equation}
 M(s)=\left(
\begin{array}{cc}
 \cos(\phi)-w w'\sin(\phi)    & w^2\sin(\phi) \\
 -\frac{1+w^2 w'^2}{w^2}\sin(\phi)  & \cos(\phi) + w w' \sin(\phi)
\end{array}\right)
\end{equation}
onde $\phi = \psi(s+L) - \psi(s)$. Comparando a equação acima, com a parametrização de Courant-Snyder, nota-se que:
\begin{equation}
 w^2 \equiv \beta(s) , \qquad \alpha \equiv -w w' = -\frac{\beta'(s)}{2},\qquad
\Phi \equiv \phi = \int_0^L \frac{d s}{\beta(s)}.
\end{equation}

Através das relações acima, da periodicidade de $\beta$, e da \mbox{equação \ref{eqevolparcouder}}, os parâmetros de Courant-Snyder e o avanço de fase ao longo do anel ficam univocamente definidos dada uma função $K(s)$, ou seja, dada uma rede magnética.

Agora, podemos escrever a solução geral da equação de Hill em sua forma pseudo-harmônica
\begin{equation}\label{eqpseudoharm}
 y(s)=\sqrt{\varepsilon} \sqrt{\beta_y(s)}\cos(\psi_y(s) + \phi_0),\quad
\psi_y(s)=\int_0^s \frac{d x}{\beta_y(x)}
\end{equation}
onde $\sqrt{\varepsilon}$ e $\phi_0$ são constantes determinadas pelas condições iniciais, sendo que definimos a dimensão de $\sqrt{\varepsilon}$ como m$^{1/2}$.

Analisando a equação acima, notamos que o movimento oscilatório dos elétrons em torno da órbita de referência é muito parecido com o movimento harmônico. As diferenças estão no fato de que amplitude do movimento é modulada pelo parâmetro $\beta$, chamado função betatron, e que o avanço de fase não é linear com a coordenada $s$ e também não é, em geral, multiplo de $2 \pi$ para uma volta completa, o que faz com que o movimento não seja periódico. A Figura \ref{fig:oscilacao_betatron} ilustra os argumentos acima.

% \begin{figure}[t]
%  \center
% % \includegraphics[scale=0.8]{Imagens/oscilacao_betatron.png}
%  \caption{Oscilações betatron de um elétron em várias revoluções. Linha tracejada representa a função betatron e linha sólida a trajetória. Adaptado de
%  \cite{Sands}.}
%  \label{fig:oscilacao_betatron}
% \end{figure}

A analogia com o movimento harmônico fica ainda mais clara quando determinamos o valor da constante $\sqrt{\varepsilon}$. Derivando a \mbox{equação \ref{eqpseudoharm}} e isolando a constante, obtemos:
\begin{equation}
 \varepsilon = \gamma x^2 + 2 \alpha x x' + \beta x'^2
\end{equation}
que é a equação de uma elipse no espaço de fase, \mbox{Figura \ref{fig:elipse_espacofase}}. Isso implica que, dado um ponto da coordenada longitudinal, o espaço de fase de todos os elétrons serão elipses de mesma orientação e excentricidade, determinadas pelos parâmetros de Courant-Snyder, mas com áreas $\pi \varepsilon$ diferentes, definidas pelas condições iniciais. Ainda, dado um elétron, a área de sua elipse do espaço de fase permanece constante ao longo de todo o anel.

% \begin{figure}[!b]
%  \center
% % \includegraphics[scale=0.5]{Imagens/elipse_espacofase.png}
%  \caption{Elipse do movimento de um elétron no espaço de fase para $s$ fixo~\cite{Wiedemann3}.}
%  \label{fig:elipse_espacofase}
% \end{figure}

Esta última propriedade é uma consequência direta do teorema de Liouville, que afirma que a densidade de pontos no espaço de fase não se altera para sistemas sob a ação de forças conservativas e em que a posição da partícula não dependa do seu momento linear \cite{Wiedemann3}. Também,ela é a principal motivação para se definir a emitância natural do feixe, visto que podemos escolher uma elipse que represente todo o feixe de elétrons e essa elipse apresentará a mesma área em todo o anel (a definição de emitância será dada na \mbox{seção \ref{efeitosemissao}}).

Um outro parâmetro fundamental das oscilações betatron é a normalização do avanço de fase de uma volta no anel de armazenamento. Define-se sintonia como:
\begin{equation}
 \nu_y = \frac{1}{2 \pi}\int^{s+P}_s \frac{d x}{\beta_y(x)}.
\end{equation}

O valor da sintonia influencia fortemente características como tempo de vida, abertura dinâmica e acoplamento, pois pequenos desvios nos valores de campo magnético dos dipolos e quadrupolos, a existência de elementos não lineares, como sextupolos e dispositivos de inserção, entre outras fontes, geram interferências que, quando entram em ressonância com as oscilações betatron, causam desvios arbitrariamente grandes na órbita, causando a a perda dos elétrons.

É possível mostrar \cite{Lee,Wiedemann3, Huth} que ressonâncias podem ser excitadas quando a seguinte relação é satisfeita
\begin{equation}
 m \nu_x + n \nu_z = l,\qquad m, n, l \in \mathbf{Z},
\end{equation}
onde $\nu_x$ e $\nu_z$ são as sintonias horizontal e vertical, respectivamente. Contudo, nem todas as ressonâncias são excitadas, de modo que as que devem ser evitadas são as de ordem, $r=|m|+|n|$, mais baixas.

Uma ressonância com fácil interpretação física é a de ordem 1, ou seja, quando pelo menos uma das duas sintonias é um número inteiro. Suponha que exista um erro dipolar em um ponto do anel. Se a sintonia é um número inteiro, os elétrons passarão por aquele ponto com a mesma fase e, portanto, as distorções causadas na órbita devido àquele erro se somarão, fazendo com que a amplitude da oscilação sempre aumente, até que o elétron colida com a câmara de vácuo do anel.

%%%%%%%%%%%%%%%%%%%%%%%%%%%%%%%%%%%%%%%%%%%%%%%%%%%%%%%%%%%%%%%%%%%%%%%%%%%%%%%%%%%%%%%%%%%%%%%%%%%%%%%%%
\subsection{Função Dispersão}

Os elétrons de um feixe em um anel de armazenamento possuem uma distribuição de energia em torno da nominal, resultado de um equilíbrio entre emissão de radiação e aceleração na cavidade de radio-frequência, além de outros fatores, como espalhamentos. Essa distribuição altera valores e gera novos parâmetros do movimento do feixe. Nessa parte do relatório serão definidas três consequências geradas por desvios pequenos (de primeira ordem), sendo a primeira a função dispersão.

Voltando à equação \ref{eqmovhor}, vemos que o desvio de energia $\boldsymbol{\epsilon}$ não influenciará o movimento vertical, mas dará origem a uma nova órbita fechada horizontal, $x_{\boldsymbol{\epsilon}}$, de modo que o deslocamento horizontal total será:
\begin{equation}
 x = x_\beta + x_{\boldsymbol{\epsilon}},\quad\mathrm{com}\quad
x_{\boldsymbol{\epsilon}}(s+P) = x_{\boldsymbol{\epsilon}}(s)
\end{equation}
onde $x_\beta$ denota as oscilações betatron estudadas no ítem anterior.

Devido à linearidade da equação de movimento, esta órbita será linear com o desvio de energia da partícula. Assim, definimos a função dispersão:

\begin{equation}
 x_{\boldsymbol{\epsilon}} (s) = \eta(s) \frac{\boldsymbol{\epsilon}}{E_0},\quad
\eta'' + K(s)\eta=G(s),
\quad \eta(s+P)=\eta(s).
\end{equation}

A solução completa da equação acima é a soma da solução da equação homogênea com uma solução particular. Como as funções $G(s)$ e $K(s)$ são constantes por trechos, podemos evoluir os valores da função dispersão de um ponto inicial para um ponto final usando a formulação matricial:
\begin{equation}
 \left(\begin{array}{c}
        \eta(s_2) \\
        \eta'(s_2)
       \end{array}\right)
 = M(s_2|s_1)
 \left(\begin{array}{c}
        \eta(s_1) \\
        \eta'(s_1)
       \end{array}\right)
 +
 \left(\begin{array}{c}
        d \\
        d'
       \end{array}\right)
\end{equation}
onde $M(s_2|s_1)$ é a matriz de transferência definida na \mbox{equação \ref{eqmatriztransf}} e $d$ é a solução particular, que pode ser obtida através da função de Green construída através das soluções principais da equação homogênea, $C(s_2,s_1)$ e $S(s_2,s_1)$, \cite{Wiedemann3}:
\begin{align}
 F(s_2,\tilde{s}) &= S(s_2,s_1) C(\tilde{s},s_1) - C(s_2,s_1) S(\tilde{s},s_1)
\\
d &= \int^{s_2}_{s_1} G(\tilde{s}) F(s_2,\tilde{s}) d \tilde{s}. &
\end{align}

Apesar deste formalismo matricial não gerar, inicialmente, funções periódicas, este é o método mais usado por softwares de dinâmica de feixe para a determinação de $\eta$ em anéis de armazenamento, devido à facilidade de programação e precisão dos cálculos.

Uma outra forma de obtenção da função dispersão, que satisfaz diretamente a condição de periodicidade, se dá pela utilização da Função de Green periódica da equação de Hill \cite{Lee}:
\begin{equation}
 \eta(s) =  \frac{\sqrt{\beta(s)}}{2\sin(\pi \nu)}\oint G(\tilde{s})
\sqrt{\beta(\tilde{s})} \cos \left(\pi \nu +|\psi(s) - \psi(\tilde{s})|\right)d
\tilde{s}.
\end{equation}
onde nota-se que $\eta$ diverge para sintonias inteiras.

Outro parâmetro fundamental para a óptica de um anel de armazenamento é o fator compactação de momento. De acordo com a métrica do sistema de coordenadas definido na \mbox{equação \ref{eqcoord}}, o comprimento de uma curva infinitesimal é dado por:
\begin{equation}
 (d l)^2 = (1+G x)^2(d s)^2+(d x)^2+(d z)^2 = \left((1+G x)^2+x'^2+
z'^2\right)\,(ds)^2.
\end{equation}
Fazendo $x=x_{\boldsymbol{\epsilon}}$, visto que em primeira ordem as oscilações betatron não alteram o comprimento da órbita, e considerando a aproximação paraxial, obtemos:
\begin{equation}
 P_{\boldsymbol{\epsilon}} = \oint (1+G(s) \eta(s) \delta)ds \quad \Rightarrow
\quad
\Delta P = \delta \oint G(s) \eta(s)\, d s
\end{equation}
onde $P_{\boldsymbol{\epsilon}}$ é o comprimento da nova órbita e $\delta = \boldsymbol{\epsilon} / E_0$ é o desvio relativo de energia. Define-se o fator compactação de momento como \cite{Lee}:
\begin{equation}
 \alpha_c \leq \frac{1}{P} \frac{d \Delta P}{d \delta} = \frac{1}{P}
\oint G(s) \eta(s) \, d s.
\end{equation}
onde observamos que $\alpha_c$ mede a magnitude da função dispersão nos trechos curvos do anel.

A importância do fator compactação de momento está no fato de que ele acopla os movimentos longitudinal e horizontal, sendo determinante para a estabilidade e tempo de amortecimento das oscilações de energia.

Por fim, vamos definir o conceito de cromaticidade. Quando um elétron passa por um quadrupolo com energia diferente da nominal, ele experimenta uma focalização diferente:
\begin{equation}
  K_\delta(s)=\frac{ec}{E_0(1+\delta)}\left(\frac{\partial B}{\partial
x}\right)_{0,s} \approx \frac{ec}{E_0}\left(\frac{\partial B}{\partial
x}\right)_{0,s}(1-\delta) = K_1(s)(1 - \delta)
\end{equation}
Inserindo esse termo na equação do movimento betatron \mbox{(equação \ref{eqoscbet})}, é possível demonstrar \cite{Lee, Wiedemann3} que haverá uma alteração na sintonia dada por
\begin{align}
 \Delta \nu_x \approx&-\left(\frac{1}{4\pi}\oint\beta_x K_1\,ds\right)\delta \\
 \Delta \nu_z \approx&\left(\frac{1}{4\pi}\oint\beta_z K_1\,ds\right)\delta
\end{align}

Define-se cromaticidade como:
\begin{equation}\label{eqcromaticidade}
 \xi_y \leq \frac{d (\Delta \nu_y)}{d \delta}
\end{equation}
sendo que a cromaticidade gerada apenas pelas focalizações da óptica linear é chamada cromaticidade natural, dada por:
\begin{align}
 \xi_x &\approx -\frac{1}{4 \pi} \oint \beta_x K_1\, d s  \\
 \xi_z &\approx \frac{1}{4 \pi} \oint \beta_z K_1\, d s.
\end{align}
A cromaticidade natural é sempre negativa em ambos os planos, pois a focalização é menos/mais efetiva para partículas com energia maior/menor, fazendo com que haja menos/mais oscilações bétatron, o que diminui/aumenta o valor das sintonias.

Valores negativos de cromaticidade pioram a performance de um anel de armazenamento devido a perda de partículas induzida pela mudança da sintonia e por excitarem uma instabilidade do feixe chamada \textit{head-tail} \cite{Wiedemann3}. Por isso, surge a necessidade de inserir sextupolos no anel para corrigir a cromaticidade, sendo que pode ser demonstrado que \cite{Wiedemann3}:
\begin{align}
 \xi_x &\approx\frac{1}{4 \pi} \oint(- \beta_x K_1+m\eta)\, d s \\
 \xi_z &\approx\frac{1}{4 \pi} \oint (\beta_z K_1-m\eta)\, d s.
\end{align}
onde $m=(ec/E_0)(\partial^2B/\partial x^2)$ é a ``força'' do sextupolo.

Pelo fato dos sextupolos serem elementos não lineares, sua inserção no anel altera toda a dinâmica do sistema, afetando a abertura dinâmica\footnote{Neste caso, entende-se por abertura dinâmica o conjunto de condições iniciais que uma partícula deve possuir para fazer um número elevado de revoluções de modo que sua trajetória permaneça limitada}, antes infinita, e excitando ressonâncias betatron de ordem mais alta. Por isso, ao construir uma óptica, deve-se levar em conta tais fatores para instalar os sextupolos em locais da rede que minimizem a força necessária para anular a cromaticidade e que tenham avanços de fase adequados \cite{Huth}.


%%%%%%%%%%%%%%%%%%%%%%%%%%%%%%%%%%%%%%%%%%%%%%%%%%%%%%%%%%%%%%%%%%%%%%%%%%%%%%%%%%%%%%%%%%%%%%%%%%%%%%%%%
%%%%%%%%%%%%%%%%%%%%%%%%%%%%%%%%%%%%%%%%%%%%%%%%%%%%%%%%%%%%%%%%%%%%%%%%%%%%%%%%%%%%%%%%%%%%%%%%%%%%%%%%%
\section{Dinâmica Longitudinal}

No desenvolvimento das equações de movimento da seção anterior não consideramos efeitos de perda ou ganho de energia pelos elétrons. Contudo, a emissão de radiação gera consequências no movimento em todas as direções.

Nessa seção estudaremos os principais efeitos que resultam na definição de parâmetros importantes para o desenvolvimento do projeto.

\subsection{Dinâmica longitudinal e oscilações de energia}

A quantidade de radiação emitida por um elétron depende do seu desvio de energia, de modo que para $\boldsymbol{\epsilon}$ pequeno podemos expandir:

\begin{equation}
 U_{\mathrm{rad}}(\boldsymbol{\epsilon}) = U_0 + Q \boldsymbol{\epsilon}
\end{equation}
onde $U_0$ é a energia perdida por partículas com energia igual a nominal e \cite{Sands}

\begin{equation}
 Q=\left(\frac{d U_\mathrm{rad}}{d \boldsymbol{\epsilon}}\right)_0 =
\frac{U_0}{E_0}\left[2+\mathcal{D}\right]\quad \text{com} \quad\mathcal{D} =
\frac{\oint \eta G(G^2+2 K_1) \, ds}{\oint G^2\,ds}.
\end{equation}

Toda energia perdida por radiação deve ser reposta para que os elétrons mantenham, em média, sua energia constante. Como campos magnéticos não fornecem energia para as partículas, essa reposição deve ser feita por campos elétricos.

O modo mais eficiente de fornecer energia para os elétrons e que caracteriza as fontes luz síncrotron é por meio de cavidades ressonantes que geram campo elétrico variável e tangencial à trajetória dos elétrons, as cavidades RF.

A frequência do modo ressonante dessas cavidades deve ser um múltiplo inteiro da frequência de revolução dos elétrons:

\begin{equation}
 f_\mathrm{rf}=kf_\mathrm{rev}
\end{equation}
onde $k$ é chamado de número harmônico, de modo que eles sempre passem pela cavidade quando o campo elétrico estiver em fase correta.

A primeira condição para se definir a fase do campo elétrico que gera um equilíbrio estável é que ela deve ser tal que a energia fornecida seja igual a perdida em uma volta por um elétron com energia nominal:

\begin{equation}
 e V_\mathrm{rf}(\phi)=U_0 ,
\end{equation}
onde $\phi$ é a fase do campo elétrico e $e$ a carga elétrica do elétron. Assim, se um elétron com energia $E_0$ passa pela cavidade quando ela se encontra nessa fase, ele repetirá esse processo indefinidamente. Esses elétrons são denominados elétrons síncronos.

A segunda condição exige que a derivada do potencial elétrico seja negativa, para que elétrons com energia um pouco diferente da nominal oscilem em torno da energia nominal com frequência $\Omega$, sofrendo um amortecimento em direção a região síncrona com constante $\alpha_{\boldsymbol{\epsilon}}$, sendo estes valores dados por \cite{Sands}:

\begin{equation}
 \alpha_{\boldsymbol{\epsilon}}=\frac{1}{2T_0}\frac{d U_\mathrm{rad}}{d E} =
 \frac{U_0}{2 T_0 E_0}J_{\boldsymbol{\epsilon}} \quad \text{com} \quad
 J_{\boldsymbol{\epsilon}} = 2+\mathcal{D}
\end{equation}

\begin{equation}
 \Omega^2 = \frac{\alpha_c e \dot{V}_0}{T_0 E_0}
\end{equation}
onde $T_0$ é o período de revolução dos elétrons, $J_{\boldsymbol{\epsilon}}$ é número de partição de amortecimento de energia e $\dot{V}$ é o módulo da derivada da voltagem de r.f. na fase síncrona. Geralmente, o amortecimento de energia é um processo lento comparado ao período de revolução, sendo que o tempo de amortecimento, inverso da constante $\alpha_{\boldsymbol{\epsilon}}$, é da ordem de milisegundos.

Juntamente com o amortecimento, há um termo de excitação que é gerado pelo processo de emissão de fótons. Para entender essa excitação, suponha que um elétron possua uma oscilação de energia da forma
\begin{equation}
 \boldsymbol{\epsilon}=A_0 e^{i \Omega(t-t_0)}
\end{equation}
onde desprezamos o amortecimento, devido à escala de tempo que queremos analisar ser muito menor.

% \begin{figure}[b]
%  \center
%  \includegraphics[scale=0.5]{Imagens/emissao_energia.png}
%  \caption{Variação da energia do elétron pela emissão de um fóton \cite{Sands}.}
%  \label{fig:emissaoenergia}
% \end{figure}

Agora suponha que este elétron emita um fóton no tempo $t_i$ \mbox{(Figura \ref{fig:emissaoenergia})}. Ele apresentará uma descontinuidade na energia, e passará a oscilar segundo a equação

\begin{equation}
 \boldsymbol{\epsilon}=A_0 e^{i \Omega(t-t_0)}-u e^{i \Omega(t-t_i)}
\end{equation}
de modo que sua nova amplitude de oscilação será dada por:

\begin{equation}
 A_1^2 = A_0^2 + u^2 - 2 A_0 u \cos(\Omega(t_i-t_0)).
\end{equation}

Fazendo a média temporal da equação acima, notamos que a amplitude aumenta devido ao processo de emissão, visto que devido à aleatoriedade dos tempos de emissão, a média da função cosseno é nula.

É possível demonstrar que sempre há um equilíbrio entre os efeitos de excitação por emissão e de amortecimento e que, nessa situação o desvio de energia dos elétrons têm uma Distribuição Normal com média nula e variância dada por \cite{Wiedemann3}:
\begin{equation}
 \sigma^2_{\boldsymbol{\epsilon}} = \frac{C_q \gamma^2_0
E_0^2}{J_{\boldsymbol{\epsilon}}} \frac{\oint G^3\, ds}{\oint G^2\, ds}
\end{equation}
onde $C_q = \mathrm{3,84\cdot 10^{-13}\, m}$ e $\gamma_0$ é a energia relativística.

O movimento longitudinal está acoplado com as oscilações de energia, pois, devido à função dispersão, partículas com energia diferente da nominal percorrerão o anel em diferentes órbitas fechadas. Isto fará com que o tempo de um ciclo e, consequentemente, sua posição longitudinal relativa à posição síncrona se alterem.

Esse acoplamento faz com que o movimento longitudinal também seja oscilatório e amortecido em torno da posição síncrona, com mesma frequência e amortecimento das oscilações de energia. Assim, os elétrons não se distribuem uniformemente ao longo do anel, mas se acumulam em pacotes, sendo que o número de pacotes é igual ao número harmônico.

O tamanho longitudinal dos pacotes é dado em função das características da cavidade r.f., do fator compactação de momento e da distribuição de energia \cite{Sands}:

\begin{equation}
 \sigma_s= \frac{\alpha_c}{\Omega E_0}\sigma_{\boldsymbol{\epsilon}}
\end{equation}

%%%%%%%%%%%%%%%%%%%%%%%%%%%%%%%%%%%%%%%%%%%%%%%%%%%%%%%%%%%%%%%%%%%%%%%%%%%%%%%%%%%%%%%%%%%%%%%%%%%%%%%%%
\subsection{Tamanho Transversal}

Na direção transversal também há efeitos de amortecimento e excitação do feixe relacionados à cavidade r.f.~e à emissão de radiação, efeitos estes que definem o tamanho do feixe nessas direções. Primeiramente, vamos analisar a direção horizontal.

% \begin{figure}[!b]
%  \center
%  \includegraphics[scale=0.4]{Imagens/emissao_horizontal.png}
%  \caption{Efeito do processo de emissão nas oscilações betatron \cite{Sands}.}
%  \label{fig:emissaohorizontal}
% \end{figure}

O cone de radiação emitida por um elétron tem um ângulo de divergência em relação à trajetória da partícula da ordem de $1/\gamma$ \cite{Lee}. Desse modo, pode-se considerar que o momento linear perdido por esse processo está na direção do movimento.

Se não fosse pela existência da função dispersão esse processo não alteraria a trajetória da partícula. Contudo, quando o elétron emite um fóton, ele perde energia e sua órbita fechada se altera (vide \mbox{(Figura \ref{fig:emissaohorizontal})}, fazendo com que as oscilações betatron também sofram um deslocamento e, como o movimento é contínuo:

\begin{equation}
 \delta x = \delta x_\beta + \eta \frac{d E}{E_0} = 0 \quad
\Rightarrow \quad \delta x_\beta =- \eta \frac{d E}{E_0}.
\end{equation}

É possível demonstrar que esta alteração no movimento gera um amortecimento na amplitude dado por:

\begin{equation}
 \frac{\delta a}{T_0a} = \frac{U_0}{2T_0 E}\mathcal{D}.
\end{equation}
Como o valor de $\mathcal{D}$ é em geral próximo da unidade, esse termo é positivo, ou seja, há um aumento na amplitude do movimento devido ao efeito da emissão de radiação.

% \begin{figure}[t]
%  \center
%  \includegraphics[scale=0.8]{Imagens/amortecimento_cavrf.png}
%  \caption{Amortecimento das oscilações betatron causado pela aceleração na cavidade r.f.~\cite{Sands}.}
%  \label{fig:amortecimentocavrf}
% \end{figure}

As cavidades r.f. não introduzem efeitos de amortecimento do tipo descrito para o processo de emissão porque elas são inseridas em trechos retos do anel, onde a perda ou ganho de energia não implicam em uma mudança imediata na órbita seguida pelo elétron.

A cavidade r.f.~também introduz um termo de amortecimento nas oscilações betatron: quando o elétron passa pela cavidade de r.f.~o momento linear que ele recebe, $\delta \textbf{P}$, está apenas na direção longitudinal, o que altera o ângulo do movimento betatron \mbox{(Figura \ref{fig:amortecimentocavrf})}, gerando um amortecimento das oscilações dado por \cite{Sands}:

\begin{equation}
  \frac{\delta a}{T_0a} = \frac{U_0}{2T_0 E}.
\end{equation}
Somando as duas contribuições, obtemos a constante de amortecimento das oscilações betatron horizontais:

\begin{equation}\label{constanteamorthor}
\alpha_x =  \frac{U_0}{2T_0 E}J_x , \quad \text{com} \quad J_x =(1-\mathcal{D}).
\end{equation}
onde $J_x$ é o número de partição de amortecimento da direção horizontal.

A constante de amortecimento descrita na \mbox{equação \ref{constanteamorthor}}, abrange apenas efeitos médios, relativos à perda de energia pelo processo emissão de radiação. Contudo, há também efeitos de excitação nas oscilações betatron, devido à efeitos quânticos (emissão quantizada e aleatória), que equilibram o efeito do amortecimento, gerando um valor de equilíbrio para a amplitude do movimento, $\sqrt{\varepsilon}$, dado por:

\begin{equation}
 \langle\varepsilon\rangle = C_q\frac{2 \gamma^2 \langle
\mathcal{H}/|\rho^3|\rangle}{J_x\langle 1/\rho^2 \rangle} ,\quad \text{com}
\quad \mathcal{H}=  \gamma \eta^2 + 2 \alpha \eta \eta' + \beta \eta'^2
\end{equation}
onde $C_q = \mathrm{3,83 \times 10^-13}$ m e $\langle \rangle$ representa a média ao longo da órbita de referência.

Também é possível mostrar que a distribuição de partículas na direção transversal segue uma distribuição normal \cite{Lee}, sendo que o valor $\langle\varepsilon\rangle$ corresponte a um sigma da distribuição de
amplitudes. Dessa forma, sabendo que o tamanho do feixe devido apenas às oscilações betatron é dado por

\begin{equation}
 \sigma_{x,\beta}^2(s) = \frac{1}{2}\beta_x(s)\langle\varepsilon\rangle,
\end{equation}
podemos definir a emitância natural do anel de armazenamento como:

\begin{equation}\label{eqemitancia}
 \varepsilon_0 \leq \frac{\sigma_{x,\beta}^2(s)}{\beta_x(s)} =
C_q\frac{\gamma^2 \langle \mathcal{H}/|\rho^3|\rangle}{J_x\langle 1/\rho^2
\rangle}
\end{equation}

O tamanho total do feixe horizontal é dado pela soma do tamanho devido às oscilações betatron com o introduzido pelas oscilações de energia, através da função dispersão. Matematicamente temos:
\begin{equation}
 x = x_\beta + \eta\frac{\boldsymbol{\epsilon}}{E_0} \quad \Rightarrow \quad
\sigma_x = \sqrt{\varepsilon_0 \beta(s)
+\eta^2 \left(\frac{\sigma_{\boldsymbol{\epsilon}}}{E_0}\right)^2}
\end{equation}

Na direção vertical a função dispersão é nula, fazendo com que tanto $\mathcal{D}$ como $\mathcal{H}$ sejam nulos. De acordo com os cálculos feitos acima, isso implicaria em uma emitância nula na direção vertical, pois haveria apenas um amortecimento causado pela cavidade r.f. Contudo, há um outro efeito de excitação quântica, que não foi considerado na derivação acima por ser $1/\gamma^2$ vezes menor que aquele analisado, gerado pela projeção da componente do momento da radiação emitida perpendicular à trajetória do elétron na direção vertical (matematicamente: $\mathbf{P_\perp} \cdot \mathbf{\hat{z}}$).

A constante de amortecimento para a direção vertical é dada por \cite{Sands}:
\begin{equation}
 \alpha_z = \frac{U_0}{2T_0 E_0}J_z , \quad \text{com} \quad J_z = 1
\end{equation}
de modo que os números de partição satisfazem o teorema de Robinson \cite{Robinson}:
\begin{equation}
 J_x + J_{\boldsymbol{\epsilon}} + J_z = 4 \quad \text{ou} \quad J_x +
J_{\boldsymbol{\epsilon}} = 3.
\end{equation}

Quando há campos magnéticos que acoplam os movimentos horizontal e vertical, a emitância natural é distribuida entre as direções $x$ e $z$ proporcionalmente ao campo, de modo que:
\begin{eqnarray}
& \varepsilon_x + \varepsilon_z = \varepsilon_0 & \\
& \varepsilon_z = \kappa \varepsilon_x
\end{eqnarray}
onde $\kappa$ é denominado coeficiente de acoplamento.

Quando há acoplamento, o tamanho vertical deixa de ser desprezível em relação ao horizontal, o que faz aumentar o volume dos pacotes, diminuindo a densidade de carga em cada pacote. Por sua vez, o tempo de vida do feixe é aumentado e o brilho da luz síncrotron diminuído.
