Sirius is the new fourth generation synchrotron light source that is being built in Campinas, Brazil, by the Brazilian Synchrotron Light Laboratory (LNLS). With a natural emittance of~\SI{250}{\pico\meter\radian}, extremely high brightness synchrotron light will be generated by, at most,~\SI{18}{}~insertion devices (IDs) installed in the straight sections of the storage ring and by~\SI{20}{}~superbends (\SI{3.2}{\tesla}) present in the center of each achromat of the magnetic lattice. The standard vacuum chamber will be round, made of copper, with a radius of~\SI{12}{\milli\meter}, which is small compared to third generation light sources chambers, and the first IDs planned will be out of vacuum and will have a very reduced gap, with chambers as small as~\SI{2.4}{}~and~\SI{3.0}{\milli\meter}, in most cases. Additionaly, the vacuum system will be distributed, through the use of NEG coating in the inner part of the chambers in the whole ring. All these factors intensify the impedance related effects of the machine, which can generate coherent oscillations, compromizing the quality of the light, cause total or partial beam loss and influence the equilibrium dynamics of the electrons. In this work some of the main components of the vacuum chamber were modelled and their wake fields were calculated with semi-analytical and numerical methods and added to the total impedance budget of the machine. With the application of this model to the first phase of operation, it was found that the beam will suffer from transverse coupled bunch resistive wall instability, making it necessary the installation of transverse bunch-by-bunch feedback systems in both planes. It was also predicted stability without feedback action if the ring operates with chromaticity larger than~\SI{2.8}{} in both planes. It was also found that the tune-shifts caused by quadrupolar impedances from the IDs are not negligible and the operation point of the machine will have to be changed as function of the total stored current. The thresholds for intra-bunch instabilities are much above the nominal operation current and will not be a problem in any of the three planes and there will be no longitudinal coupled-bunch motion as long as the ring operates with superconducting RF cavities. The installation of a Landau cavity is planned for the phase two of operation, which will allow higher total current in the machine and even high single-bunch current in the middle of the train. Even though it was not done any calculations for these conditions, the methods and codes developed in this work can be directly applied for those cases.
