\chapter{Impedance Calculations}

The Sirius impedance budget model is based on analytical calculations and numeric simulations with time and frequency solver codes. In the next sections we will describe how we have modeled some of the subsystems' impedance.

\section{Vacuum Chamber Impedance}

We can separate Sirius vacuum chambers into three groups:
\begin{description}
 \item[Straight Sections:] circular cross-section, $\Phi = \,$\SI{24}{\milli\meter};
 \item[Dipoles without radiation exit:] circular cross-section, $\Phi = \,$\SI{24}{\milli\meter};
 \item[Dipoles with radiation exit:] circular cross-section, $\Phi = \,$\SI{24}{\milli\meter} + a \SI{5}{\milli\meter} height ``nose'';
\end{description}

All chambers are made of cooper with thickness of \SI{1}{\milli\meter}. Thus, if we do not take into account curvature effects and the radiation exit, we can model the dipole chambers the same way as the straight section chambers.

Mounet and Métral \cite{mounet_metral2009} solved Maxwell equations analytically for the case of a macro-particle traveling longitudinally through an axisymmetric chamber composed by an arbitrary number of layers of different materials with generic electric permeability and magnetic permissivity. The authors presented general formulas for electromagnetic fields and impedances.

We have implemented on Matlab\textregistered Mounet-Métral formulas for the first order longitudinal, horizontal and vertical impedances. To validate the implementation, we compared with Chao's book calculation, as shown in figure \ref{fig:compare_mounet_chao}. In order for the results to be equivalent, we had to change the sign of the impedances calculated through Mounet-Métral formulas. This difference is due to the definition of the longitudinal coordinate: in Chao's book the distance $z$ between the source particle and the point where the impedance is being calculated is positive and in Mounet-Métral's report it is negative.

% \begin{figure}[!t]
%  \centering
%  \subfigure[\label{fig:compare_mounet_chao1}]{\includegraphics[width=0.45\textwidth]{figures/compare_mounet_chao1.png}}
%  \subfigure[\label{fig:compare_mounet_chao2}]{\includegraphics[width=0.45\textwidth]{figures/compare_mounet_chao2.png}}
%  \caption{Comparison between the implemented formulas from \cite{mounet_metral2009} and Chao's book \cite{Chao1993} calculation for an aluminum chamber, $\sigma =$ \SI{33.4}{\mega\siemens\per\meter}, with a radius of 5 cm.}
%  \label{fig:compare_mounet_chao}
% \end{figure}

The advantage of using Mounet-Métral formalism is that we can estimate the effect of NEG Coating on the impedance. Sirius will have about 90\% of the whole ring coated with NEG to improve the vacuum, i.e., all chambers described above will be coated. Figure \ref{fig:rw_with_neg} shows the comparison of the chamber wall impedance with and without the NEG coating and table \ref{tab:rw_with_neg} presents the main parameters values used for this calculation. 

\begin{table}[!b]
 \centering
 \caption{Main parameters used on the chamber wall impedance.}
 \label{tab:rw_with_neg}
 \begin{tabular}{lcl}\hline
  Chamber radius      & 12               & \si{\milli\meter} \\\hline
  Copper conductivity\cite{matwebsite} & 59 & \si{\mega\siemens\per\meter} \\\hline
  NEG thickness       & 1                & \si{\micro\meter}   \\\hline
  NEG conductivity    & 4.0              & \si{\mega\siemens\per\meter} \\\hline
 Length               & 480              & \si{\meter}   \\\hline
 \end{tabular}
\end{table}

We notice that the NEG coating has a similar effect on both planes, it begins to affect the imaginary impedance at lower frequencies than it does for the real one. These results are compatible with a previous study made by Nagaoka \cite{nagaoka2004}, and can be explained if we think that the wall impedance of a single thick metal layer is purely reactive in the limit of low frequencies. Also, we can infer that for very high frequencies the impedance with NEG coating tends to the impedance of a chamber with a infinite layer of NEG, which is very reasonable.

% \begin{figure}[!t]
%  \centering
%  \includegraphics[width=\textwidth]{figures/rw_with_neg.png}
%  \caption{NEG coating effect on transverse and longitudinal impedances. The ``Thick NEG'' curve is the unrealistic impedance of an infinitely thick layer of NEG, calculated to show how that the beam ``does not see'' the copper for high frequencies.}
%  \label{fig:rw_with_neg}
% \end{figure}

There is an uncertainty about NEG's conductivity which is presented in Table \ref{tab:neg_salad}. With this in mind, we calculated the impedance for each value in this table. The results are shown in figure \ref{fig:neg_salad}. We notice the value used to compose our model is lower than the others and gives a lower imaginary and higher real impedance.

\begin{table}[!t]
 \centering
 \caption{Several values for NEG conductivity}
 \label{tab:neg_salad}
\begin{tabular}{ccc}
Author                 & $\sigma$ [\si{\mega\siemens\per\meter}] & Explanation \\\hline
Nagaoka \cite{nagaoka2004}  & 4.0    & \begin{minipage}{0.6\textwidth}
                                        \vspace{1mm}
                                        Used this value to estimate the effect of NEG on Soleil wall impedance. Also, this is the conductivity of Vanadium, main component of the alloy.
                                        \vspace{1mm}
                                       \end{minipage} \\\hline
Nagaoka \cite{nagaoka2004}  & 0.0625 & \begin{minipage}{0.6\textwidth}
                                        \vspace{1mm}
                                        Said E. Plouviez measured this value, but I did not find the indicated reference.
                                        \vspace{1mm}
                                       \end{minipage} \\\hline
Métral \cite{metral_talk2011}& 0.04  & \begin{minipage}{0.6\textwidth}
                                        \vspace{1mm}
                                        Made the following reference: David Seebacher, F. Caspers, NEG properties in the microwave range, SPSU Meeting, 17th February, CERN. I couldn't find the file either.
                                        \vspace{1mm}
                                       \end{minipage} \\\hline  
Kersevan \cite{kersevan2002} & 0.2   & \begin{minipage}{0.6\textwidth}
                                        \vspace{1mm}
                                        Said Plouviez measured this value at \SI{14}{\giga\hertz}, but haven't given any reference.
                                        \vspace{1mm}
                                       \end{minipage} \\\hline                                                    
\end{tabular}
\end{table}


% \begin{figure}[!t]
%  \centering
%  \includegraphics[width=\textwidth]{figures/neg_salad.png}
%  \caption{Wall impedance for several values of NEG conductivity. The coating thickness is \SI{1}{\micro\meter} in all calculations.}
%  \label{fig:neg_salad}
% \end{figure}


\section{Small Gap Undulators}

By small gap undulators we mean the out of vacuum insertion devices that, on Sirius, will have a thick elliptical cooper vacuum chamber. We have modeled this impedance as a flat chamber, using the expression of the round chamber multiplied by Yokoya factors \cite{yokoya1993, gluckstern1993}. The relevant parameters of this modeling are presented in table \ref{tab:small_gap_undulators} and the impedance is plotted in figure \ref{fig:small_gap_undulators}.

\begin{table}[!t]
 \centering
 \caption{Main parameters used on the small gap undulators model.}
 \label{tab:small_gap_undulators}
 \begin{tabular}{lcl}\hline
  Chamber full gap      & 10               & \si{\milli\meter} \\\hline
  Copper conductivity\cite{matwebsite} & 59& \si{\mega\siemens\per\meter} \\\hline
  Length                & 3                & \si{\meter}   \\\hline
 \end{tabular}
\end{table}

% \begin{figure}[!t]
%  \centering
%  \includegraphics[width=\textwidth]{figures/small_gap_undulator.png}
%  \caption{Small gap undulator impedance.}
%  \label{fig:small_gap_undulators}
% \end{figure}


\section{In-vacuum Undulators}\label{sec:in-vacuum_undulators}

The in-vacuum undulators for Sirius will be made of NdFeB covered by a thin sheet of copper to diminish heating of the magnets and other wake fields issues, such as instabilities.

A good model for this element's impedance would be a flat multi-layer geometry. Even though Mounet and Métral \cite{mounet_metral2010} have developed general theory for this geometry we still have not successfully implemented their formulas.

The same authors, in another work \cite{mounet_metralhb2010} have demonstrated that application of Yokoya factors \cite{yokoya1993} are not valid for all frequencies but hold very well in their example for the range from \SI{1}{\mega\hertz} to \SI{1}{\tera\hertz}. 

Yokoya's theory was developed under the assumption of metallic chamber walls of constant electromagnetic properties, then, in our case, it is reasonable to expect that for the frequency range where the beam sees only the copper sheet we could use Mounet and Métral formulas for the round chamber multiplied by Yokoya's factors. Figure \ref{fig:compare_sheet_thick_flat} shows the impedance for several copper sheet thicknesses. We notice that down to \SI{100}{\mega\hertz} all curves agree with the flat chamber result and then begin to deviate in increasing order of thickness.

% \begin{figure}[!t]
%  \centering
%  \includegraphics[width=\textwidth]{figures/compare_sheet_thick_flat.png}
%  \caption{Model of in-vacuum undulator impedance using Mounet-Métral multi-layer round chamber formulas for several copper sheet thickness plus impedance calculated with thick copper layer flat chamber formula.}
%  \label{fig:compare_sheet_thick_flat}
% \end{figure}

The choice of the sheet thickness is mainly based on the heat deposition on the magnets. We want to minimize the total power loss by the beam and assure this power goes to the copper sheet and not to the magnets. In figure \ref{fig:heat_load_invac_magnets} we can notice that even for a \SI{50}{\micro\meter} thick sheet no heat goes to the magnets, at least not by radiation.

% \begin{figure}[!b]
%  \centering
%  \includegraphics[width=0.5\textwidth]{figures/heat_load_invac_magnets.png}
%  \caption{Beam power loss in function of copper sheet thickness for several bunch lengths.}
%  \label{fig:heat_load_invac_magnets}
% \end{figure}

Table \ref{tab:invac_undulators} shows the main material and geometry parameters used in the undulators modeling and figure \ref{fig:in_vacuum_undulators} shows the two types of in-vacuum undulators that will be present in the ring. We varied some orders of magnitude the values of the NdFeB parameters but it did not influenced the impedance significantly.

\begin{table}[!t]
 \centering
 \caption{Main parameters used on the in-vacuum undulators model.}
 \label{tab:invac_undulators}
 \begin{tabular}{lcl}\hline
  Chamber full gap      & 4 and 5.3            & \si{\milli\meter} \\\hline
  Copper conductivity\cite{matwebsite} & 59& \si{\mega\siemens\per\meter} \\\hline
  Copper sheet thickness &     50          & \si{\micro\meter} \\\hline
  NdFeB relative magnetic permeability & 10    & \\\hline
  NdFeB conductivity\cite{matwebsite} & 0.625& \si{\mega\siemens\per\meter}\\\hline
  Length                & 2                & \si{\meter}   \\\hline
 \end{tabular}
\end{table}

% \begin{figure}[!t]
%  \centering
%  \includegraphics[width=\textwidth]{figures/in_vacuum_undulators.png}
%  \caption{Two types of in-vacuum undulators that will be used on Sirius ring. The magnets installed at the high $\beta_x$ straight sections will have a gap of \SI{5.3}{\milli\meter}, while the ones placed at the low $\beta_x$ sections will have \SI{4}{\milli\meter}}
%  \label{fig:in_vacuum_undulators}
% \end{figure}


\section{Ferrite Kickers}

The impedance of this element is very tricky to model. We discuss the process adopted with more details in other report. Summarizing, it has two contributions: the coupled flux, which couples the beam, through the window frame, to the external circuit of the pulse generator, and the uncoupled flux, due to losses on the chamber walls. Figure \ref{fig:kicker_window_frame} shows the geometry of the kicker's window frame.

% \begin{figure}[!t]
%  \centering
%  \includegraphics[width=0.7\textwidth]{figures/kicker_window_frame.png}
%  \caption{Transverse cross section of the ferrite kickers that will be used for injection on Sirius.}
%  \label{fig:kicker_window_frame}
% \end{figure}

Regarding the coupled flux \cite{NassibianSacherer1979, VoelkerLambertson1989, DavinoHahn2003}, we ignored the effect of the Ti coating and the ceramics, because they're not important at low frequencies, below \SI{100}{\mega\hertz}, since the generator circuit does not have resonances at higher frequencies. The impedance we used for the generator is:
\begin{equation}
 Z_g = \frac{1}{1/R + i\omega C_p}
\end{equation}
which represents a RC circuit in parallel. Together with the inductance of the window frame, this impedance generates a peak centered at $\sim$\SI{50}{\mega\hertz}.


There are four possible models for the uncoupled flux:
\begin{description}
 \item[Tsutsui's Model:] Was created by Tsutsui \cite{tsutsui2000, tsutsui_vos2000} to model an in vacuum kicker, i.e. without the ceramic vacuum chamber and the Ti coating. For our case it is interesting to look at this model because it reproduces the fact that the beam sees both, the ferrite and the copper plates;
 \item[Worst Case:] Following the discussion carried out in section \ref{sec:in-vacuum_undulators}, we can use the Mounet-Métral formulas  multiplied by Yokoya factors with the following layers: Ti, ceramic, ferrite. This is considered worst case because the beam does not see the copper plates, that have high conductivity;
 \item[Best Case:] In this case we could use Mounet-Métral formulas with the layers: Ti, ceramic, copper. Now its the opposite, the beam does not see the low conductivity of the ferrite;
 \item[Average Case:] We can take the mean value of the last two models pondered by the ratio of the angles of direct exposure to the beam.
\end{description}

Figure \ref{fig:kicker_model} shows the results for the four cases with coupled flux already computed, plus an unrealistic case without Ti coating. We can perceive that the coating damp all the ceramic and ferrite resonances present in the ``Without Coating'' curve. This make us believe that it would also damp the uncoupled flux peak in the ``Tsutsui Model'', eliminating almost completely the longitudinal impedance and the above \SI{1}{\giga\hertz} transverse impedance even for the ``Worst Case'' scenario. However, the coating introduces an uncoupled flux peak in low frequencies that sum up with the originated by the coupled flux. This peak will introduce coupled bunch oscillations in the transverse plan, which are not very harmful since we will already need a transverse feedback system due to resistive wall instability.

% \begin{figure}[!t]
%  \centering
%  \includegraphics[width=\textwidth]{figures/kicker_model.png}
%  \caption{Impedances for four different models of the ferrite kicker magnets.}
%  \label{fig:kicker_model}
% \end{figure}

Below \SI{10}{\mega\hertz} we lose numeric precision on the calculation of the uncoupled flux and we decided to set this contribution to zero. This is not very important because the beam only have two or three lines below this frequency and the peak of the impedance is above this value.

Given the discussion above, we decided to use the ``Mean Case'' as the kicker impedance model in our budget. We understand it is a ``chimera'' that does not represent mathematically the problem, but if we think physically, it has all the important aspects of the geometry. This modeling is yet under study and improvements are being pursued, mainly by the implementation of the multi-layer flat chamber formulas.

Table \ref{tab:kicker_model_data} presents the values of the main parameters used in this model. Figure \ref{fig:permeability} shows the comparison between the magnetic permeability used in the model and the data taken from the CMD5005 data-sheet \cite{datasheetcmd5005}.

\begin{table}[!t]
 \centering
 \caption{Main parameters used on the kicker model.}
 \label{tab:kicker_model_data}
 \begin{tabular}{lcl}\hline
  Inner radius      &    10                & \si{\milli\meter} \\\hline
  Ti Conductivity\cite{matwebsite} & 1.8& \si{\mega\siemens\per\meter} \\\hline
  Ti thickness &     2             & \si{\micro\meter} \\\hline
  Ceramic electric permissivity & 9.3    & \\\hline
  Ceramic thickness & 7.5             & \si{\milli\meter}\\\hline
  Ferrite thickness & 20            & \si{\milli\meter} \\\hline
  Ferrite electric permissivity & 12 & \\\hline
  $R$               &  50           & \si{\ohm} \\\hline
  $C_p$             &  30          & \si{\pico\farad} \\\hline
  Length                & 0.6                & \si{\meter}   \\\hline
 \end{tabular}
\end{table}


% \begin{figure}[!t]
%  \centering
%  \subfigure[\label{fig:fitted_permeability}]{\includegraphics[width=0.45\textwidth]{figures/fitted_permeability.png}}
%  \subfigure[\label{fig:data_permeability}]{\includegraphics[width=0.45\textwidth]{figures/data_permeability.png}}
%  \caption{\subref{fig:fitted_permeability} Relative magnetic permeability used in the model. \subref{fig:data_permeability} Relative magnetic permeability taken from \cite{datasheetcmd5005}.}
%  \label{fig:permeability}
% \end{figure}


\section{Broad Band Impedance}

There are several components whose impedance were not calculated yet. For this reason, we have complemented our budget with a broad band resonator based on calculations and measurements of other synchrotron laboratories.

In the Table \ref{tab:broad_band_data} there are the parameters of the broad band resonators used in the impedance budget model and figure \ref{fig:broad_band} shows the impedance. For the longitudinal plane, we used the circumference scaled impedance from Soleil storage ring \cite{nagaoka2006} and for the transverse plane we used the circumference and beta scaled ESRF data, taken from \cite{nagaoka1999}, where the beta function scaling were made considering a medium value of \SI{20}{\meter} for ESRF.

\begin{table}[!hb]
 \centering
 \caption{Main parameters used on the broad-band model.}
 \label{tab:broad_band_data}
 \begin{tabular}{lccc}\hline
  Plane &  $R$ &  $f_R$  & $Q$\\\hline
  Longitudinal & \SI{5.3}{\kilo\ohm}& \SI{20}{\giga\hertz} &1\\\hline
  Horizontal & \SI{0.42}{\mega\ohm\per\meter} &\SI{22}{\giga\hertz} & 1 \\\hline
  Vertical   & \SI{0.42}{\mega\ohm\per\meter} &\SI{22}{\giga\hertz} & 1 \\\hline
 \end{tabular}
\end{table}


% \begin{figure}[!hb]
%  \centering
%  \includegraphics[width=\textwidth]{figures/broad_band.png}
%  \caption{Broad band impedance model used in our budget.}
%  \label{fig:broad_band}
% \end{figure}

\section{RF Cavity}

Even though Sirius RF cavity will be superconducting, we will commission the storage ring with a 5-Cells PETRA cavity and we must include its higher order modes in our budget.

to be done\dots

