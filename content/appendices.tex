\chapter{Lorentz Force Ultra-relativistic limit}\label{app:lorentz_cancel}
Intuitively, we tend to think the direct interaction between charged particles, such as electric repulsion, is the responsible for the collective effects observed in storage rings, however, as we will see in the subsequent analysis, this is not the main mechanism for ultra-relativistic particles.

\begin{figure}[hb!]
    \centering
    \label{fig:wake1}
    \begin{tikzpicture}[scale=1]
        \def\d{1cm}
        \draw[<->] (1,0) node[below]{$z$}
        -- ++(-\d,0) node[below left] {$S$}
        -- ++(0,\d) node[left] {$\rho$}; %coord sys S
        \draw[<->] (6*\d,0) node[below]{$z'$}
        -- ++(-\d,0) node[below left] {$S'$}
        -- ++(0,\d) node[left] {$\rho'$}; % coord sys S'
        \coordinate (V) at (0.5,0);
        \coordinate (Q1) at (4cm,1.5cm);
        \coordinate (Q2) at (0.5cm,2.5cm);
        \draw[->] (5*\d,0.5*\d) -- ++(V) node[above] {$\boldsymbol{v}$};
        \filldraw[fill=black] (Q1) circle[radius=0.05] node[above] {$q$}; % source particle
        \draw[->] (Q1) -- ++(V) node[above] {$\boldsymbol{v}$}; % velocity vector
        \filldraw[fill=black] (Q2)  circle[radius=0.05]; % test particle
        \draw[->] (Q2) -- ++(V) node[above] {$\boldsymbol{v}$}; % velocity vector
        \draw[dashed,|-|] ($(Q1)-(0,0.2)$)
        let \p1 = ($(Q2) - (Q1)$)
        in -- ++(\x1,0) node[midway,below] {$s$}; %horizontal distance
        \draw[dashed,|-|] ($(Q2)-(0.2,0)$)
        let \p1 = ($(Q1) - (Q2)$)
        in -- ++(0,\y1) node[midway,left] {$\rho$}; % vertical distance
        \draw[->] (Q1) -- ++($(Q2) - (Q1)$) node[midway,above] {$\boldsymbol{R}$}; %vector
    \end{tikzpicture}
    \caption{Duas partículas interagindo via campo direto.}
\end{figure}

To see this lets consider the interaction of a source particle $Q$ moving with velocity $\vect{v}=v{\vect{\hat{z}}}$ with a witness particle $q$ moving with the same velocity (parallel path) at a distance $s$ in the direction parallel to the movement and at a transverse distance $\rho$, as shown in Figure\,\ref{fig:wake1}. We want to determine the force that the source particle exerts on the witness particle. One way to do this is by calculating the electric field of the source particle in the co-moving frame of reference, $S'$, and Lorentz transforming it back to the laboratory's frame. After the math we obtain:

\begin{align}\label{eq:fields_free_particle}
    \vect{E} = \frac{q}{4\pi\epsilon_0}\frac{\vect{R}}{\gamma^2 R^{*3}}, & & \vect{B} = \frac{1}{c^2}\vect{v} \times \vect{E}
\end{align}
where $\vect{R}$ is the vector which connects both particles, going from the source to the witness and  $R^{*2} = s^2 + x^2/\gamma^2$, e $\gamma = 1/(1-v^2/c^2)$.

Combining equation\,\ref{eq:fields_free_particle} with the Lorentz force, we get the longitudinal and transverse force over the witness particle:
\begin{align}\label{eq:space_charge_force}
    F_l &= E_z = -\frac{q}{4\pi\epsilon_0}\frac{s}{\gamma^2\left(s^2+x^2/\gamma^2\right)^{3/2}}, \\
    F_t &= E_x - vB_y = -\frac{q}{4\pi\epsilon_0}\frac{x}{\gamma^4\left(s^2+x^2/\gamma^2\right)^{3/2}}
\end{align}

In accelerator physics the force $\vect{F}$ is known as space charge force. We can infer from equation\,\ref{eq:space_charge_force} that for any position $s$ and $x$, the longitudinal force is proportional to $\gamma^{-2}$ and $F_t \sim \gamma^{-4}$ if $s \gg x/\gamma$ and  $F_t \sim \gamma^{-1}$ if $s \approx 0$. This way, in the ultra-relativistic limit, $\gamma \to \infty$, the electromagnetic interaction between particles moving parallel to each other in free space is zero. It is easy to show that in this limit, if the movement of the particles is not parallel, there is an interaction force only for $s=0$, but, as their speed is the same, this situation can only happen for an infinitesimal time.

In this work we are interested in the the interaction between particles in the ultra-relativistic limit, $v \to c$. The space charge effects discussed above are despicable in this limit and the interaction between the particles is due to the presence of the walls of the vaccum chamber. Note that taking the limit $v \to c$ in the equation\,\ref{eq:fields_free_particle} and remembering that $s = vt - z$, we can write the electromagnetic field of a ultra-relativistic charge as
\begin{align}
    \vect{E} = \frac{q}{2\pi\epsilon_0}\frac{\vect{\hat{r}}}{r}\delta(z-ct), & & \vect{B} =\frac{1}{c}\vect{\hat{z}}\times\vect{E},
\end{align}
where $\vect{r} = \vect{\hat{x}}x + \vect{\hat{y}}y$ is a bidimensional vector in cylindrical coordinates ($\vect{\hat{x}}$ and $\vect{\hat{y}}$ are unit vectors in the $x$ e $y$ directions, respectively). The equations above show that the field is pancake-like and follow the beam as it travels througth the empty space. It is important to notice that this solution is steady-state, it was necessary an infinite amount of time before $t$ to build it and that's why there is no causal paradoxes in it.

\begin{figure}[hb!]
    \centering
    \label{fig:wake2}
    \begin{tikzpicture}
        \draw[very thick] (0,0) -- ++(10,0) (0,4) -- ++(10,0); %vacuum chamber
        \draw[dashed] (0,2) -- ++(10,0); %eixo de simetria
        \coordinate (V) at (0.5,0);
        \coordinate (Q1) at (4cm,2.2cm);
        \coordinate (Q2) at (0.5cm,2.5cm);
        \filldraw[fill=black] (Q1) circle[radius=0.05] node[left] {$q$}; % source particle
        \draw[->] (Q1) -- ++(V) node[above] {$\boldsymbol{v}$}; % velocity vector
        \draw[-{Stealth[length=10pt]}] (Q1) let \p1 = (Q1) in --(\x1,4);
        \draw[-{Stealth[length=10pt]}] (Q1) let \p1 = (Q1) in --(\x1,0) node[midway,right] {$\boldsymbol{E}$};
        \draw ($(Q1)+(0,1)$) circle[radius=0.2] node[right=0.2] {$\boldsymbol{B}$};
        \filldraw ($(Q1)+(0,1)$) circle[radius=0.07];
    \end{tikzpicture}
    \caption{Particles interacting in a perfectly conducting cylindrical tube.}
\end{figure}

Lets consider now a pipe with cylindrical symmetry\footnote{do not confuse cylindrical symmetry with cylinder. By cylindrical symmetry we mean a system with translational symmetry in one direction.}, hollow and with absolute vacuum in its interior, made of a perfect electric conductor material and with arbitrary cross section. If we put the particles of the previous example inside this pipe, moving parallel to symmetry axis, they will induce image charges in the surface of the wall which cancel the electromagnetic field inside the metal.

The image charges travel with the same velocity $\vect{v}$ of the particles (see Figure\,\ref{fig:wake2}). As they move in parallel paths with constant velocity, in the limit $v \to c$, according to the previous results, they do not interact, independently of how close they are from each other.

From this analysis we conclude that the interaction between particles in the ultra-relativistic limit can occur only for two reasons:
\begin{itemize}
    \item The wall is not perfectly conducting, or
    \item The pipe does not have cylindrical symmetry (which generally is due to the presence of RF cavities, flanges, bellows, beam position monitors, vacuum pumps, among other elements in the vacuum chamber of an accelerator).
\end{itemize}


\section{Duas partículas interagindo no vácuo}

\begin{figure}[hb!]
\centering
\begin{tikzpicture}[scale=1]
\def\d{1cm}
\draw[<->] (1,0) node[below]{$z$}
		-- ++(-\d,0) node[below left] {$S$}
        -- ++(0,\d) node[left] {$\rho$}; %coord sys S
\draw[<->] (6*\d,0) node[below]{$z'$}
		-- ++(-\d,0) node[below left] {$S'$}
        -- ++(0,\d) node[left] {$\rho'$}; % coord sys S'
\coordinate (V) at (0.5,0);
\coordinate (Q1) at (4cm,1.5cm);
\coordinate (Q2) at (0.5cm,2.5cm);
\draw[->] (5*\d,0.5*\d) -- ++(V) node[above] {$\boldsymbol{v}$};
\filldraw[fill=black] (Q1) circle[radius=0.05] node[above] {$q$}; % source particle
\draw[->] (Q1) -- ++(V) node[above] {$\boldsymbol{v}$}; % velocity vector
\filldraw[fill=black] (Q2)  circle[radius=0.05]; % test particle
\draw[->] (Q2) -- ++(V) node[above] {$\boldsymbol{v}$}; % velocity vector
\draw[dashed,|-|] ($(Q1)-(0,0.2)$)
				   let \p1 = ($(Q2) - (Q1)$)
                   in -- ++(\x1,0) node[midway,below] {$s$}; %horizontal distance
\draw[dashed,|-|] ($(Q2)-(0.2,0)$)
				   let \p1 = ($(Q1) - (Q2)$)
                   in -- ++(0,\y1) node[midway,left] {$\rho$}; % vertical distance
\draw[->] (Q1) -- ++($(Q2) - (Q1)$) node[midway,above] {$\boldsymbol{R}$}; %vector
\end{tikzpicture}
\caption{Duas partículas interagindo via campo direto.}
\end{figure}

Primeiro, campos gerados por $q_1$. No referencial $S'$:
\begin{align}
\vect{E'} &= \frac{q}{4\pi\epsilon_0} \frac{\vect{\hat{r'}}}{r'^2} \\
\vect{B'} &= \vect{0'}
\end{align}
assumindo que a partícula 1 está na origem do sistema de coordenadas $S'$.

Lembrando que a transformação entre coordenadas esféricas para cilíndricas são:
\begin{align}
r' &= \sqrt{s'^2 + \rho'^2} \\
\vect{\hat{r'}} &= \cos\theta'\vect{\hat{z'}} + \sin\theta'\vect{\hat{\rho'}} =
                  -\frac{s'}{r'}\vect{\hat{z'}} + \frac{\rho'}{r'}\vect{\hat{\rho'}}
\end{align}
onde a coordenada $\phi$ fica inalterada.

Assim podemos reescrever o campo elétrico em suas partes longitudinal e transversal:
\begin{align}
\vect{E'}_{||} &= \frac{q}{4\pi\epsilon_0} \frac{\cos\theta'\vect{\hat{z'}}}{r'^2} =
                 -\frac{q}{4\pi\epsilon_0} \frac{s'\vect{\hat{z'}}}{(\rho'^2+s'^2)^{3/2}} \\
\vect{E'}_{\perp} &= \frac{q}{4\pi\epsilon_0} \frac{\sin\theta'\vect{\hat{\rho'}}}{r'^2} =
                     \frac{q}{4\pi\epsilon_0} \frac{x'\vect{\hat{\rho'}}}{(\rho'^2+s'^2)^{3/2}}
\end{align}

Lembrando as equações de transformação de Lorentz para campos elétricos e magnéticos, para esse problema:

\begin{align}
 \vect{E}_{||} &= \vect{E'}_{||}\\
 \vect{B}_{||} &= \vect{B'}_{||}\\
 \vect{E}_\perp &= \gamma\left(\vect{E'}_\perp - \vect{v}\times\vect{B'}\right)\\
 \vect{B}_\perp &= \gamma\left(\vect{B'}_\perp + \frac{1}{c^2}\vect{v}\times\vect{E'}\right)
\end{align}

Ainda, as coordenadas espaciais são transformadas da seguinte maneira:

\begin{align}
\vect{\hat{\rho}} &=\vect{\hat{\rho'}}, \quad \vect{\hat{z}} = \vect{\hat{z'}}\\
\rho &= \rho', \quad z = \frac{z'}{\gamma}
\end{align}

Assim, podemos notar que:

\begin{align}
\vect{E}_{||} &= -\frac{q}{4\pi\epsilon_0} \frac{s'\vect{\hat{z'}}}{(\rho'^2+s'^2)^{3/2}} =
				 -\frac{q}{4\pi\epsilon_0} \frac{\gamma s\vect{\hat{z}}}{((\gamma s)^2+\rho^2)^{3/2}}=
                 -\frac{q}{4\pi\epsilon_0} \frac{s\vect{\hat{z}}}{\gamma^2R^{*3}} \\
\vect{E}_\perp&= \frac{q}{4\pi\epsilon_0}\frac{\gamma \rho'\vect{\hat{\rho'}}}{(\rho'^2+s'^2)^{3/2}} =
				 \frac{q}{4\pi\epsilon_0}\frac{\gamma \rho\vect{\hat{\rho}}}{((\gamma s)^2+\rho^2)^{3/2}}=
                 \frac{q}{4\pi\epsilon_0} \frac{\rho\vect{\hat{\rho}}}{\gamma^2R^{*3}} \\
\vect{B}_\perp&= -\frac{\gamma   vE'_\perp}{c^2}\vect{\hat{\phi'}} =
		         -\frac{vE_\perp}{c^2}\vect{\hat{\phi}}
\end{align}
onde $R^* = \sqrt{s^2+\left(\rho/\gamma\right)^2}$

Agora podemos analisar a força exercida pela partícula fonte sobre a partícula teste usando a força de Lorentz
\begin{equation}
\vect{F} = \left(\vect{E} + \vect{v}\times\vect{B}\right)
\end{equation}
onde foi assumida carga unitária para a partícula teste, e os campos calculados anteriormente.

Assumindo que a velocidade da partícula teste é a mesma da partícula fonte (mesmo módulo e direção), as componentes longitudinal e transversal da força ficam:

\begin{align}
\vect{F}_{||} &= \vect{E}_{||} = -\frac{q}{4\pi\epsilon_0} \frac{s\vect{\hat{z}}}{\gamma^2R^{*3}}\\
\vect{F}_\perp&= \left(1+\frac{\vect{v}\times\vect{v}\times}{c^2}\right)\vect{E}_\perp =
                 \left(1-\frac{v^2}{c^2}\right)\vect{E}_\perp =
                 \frac{q}{4\pi\epsilon_0} \frac{\rho\vect{\hat{z}}}{\gamma^4R^{*3}}
\end{align}
onde vemos que a força longitudinal tende a zero proporcionalmente a $\gamma^{-2}$ quando $v \to c$ e que a força longitudinal tende a zero com $\gamma^{-4}$, se $s>\rho/\gamma$ e com $\gamma^{-1}$ se $s<\rho/\gamma$.

Agora, vamos assumir que a velocidade da partícula teste não é paralela à velocidade da partícula fonte
\begin{equation}
\vect{v_2} = v(\cos\delta \vect{\hat{z}} + \sin\delta\vect{\hat{\rho}})
\end{equation}

Assim, a força sofrida por essa partícula fica

\begin{equation}\begin{aligned}
\vect{F} &= \vect{E}_{||} + \vect{E}_\perp - v(\cos\delta \vect{\hat{z}} + \sin\delta\vect{\hat{\rho}}) \times \vect{\hat{\phi}}\frac{vE_\perp}{c^2} \\
&= \left(1-\frac{v^2}{c^2}\cos\delta\right)\vect{E}_\perp + \vect{E}_{||} + \frac{v^2}{c^2}E_\perp\sin\delta\vect{\hat{z}}
\end{aligned}\end{equation}

Olhando essa expressão, vemos que, conforme $v \to c$, a força pode ser expressa como:

\begin{equation}
\vect{F} = \left\{
\begin{aligned}
\left((1-\cos\delta)\vect{\hat{\rho}} + \sin\delta\vect{\hat{z}}\right)
\frac{q}{4\pi\epsilon_0} \frac{\gamma\vect{\hat{\rho}}}{\rho^2} & &|s|<\rho/\gamma\\
0 & &|s|>\rho/\gamma
\end{aligned}\right.
\end{equation}
