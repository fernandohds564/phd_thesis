\chapter{Lorentz Force Ultra-relativistic limit}\label{app:lorentz_cancel}
Intuitively, we tend to think the direct interaction between charged particles, such as electric repulsion, is the responsible for the collective effects observed in storage rings, however, as we will see in the subsequent analysis, this is not the main mechanism for ultra-relativistic particles.

\begin{figure}[hb!]
    \centering
    \label{fig:wake1}
    \begin{tikzpicture}[scale=1]
        \def\d{1cm}
        \draw[<->] (1,0) node[below]{$z$}
        -- ++(-\d,0) node[below left] {$S$}
        -- ++(0,\d) node[left] {$\rho$}; %coord sys S
        \draw[<->] (6*\d,0) node[below]{$z'$}
        -- ++(-\d,0) node[below left] {$S'$}
        -- ++(0,\d) node[left] {$\rho'$}; % coord sys S'
        \coordinate (V) at (0.5,0);
        \coordinate (Q1) at (4cm,1.5cm);
        \coordinate (Q2) at (0.5cm,2.5cm);
        \draw[->] (5*\d,0.5*\d) -- ++(V) node[above] {$\boldsymbol{v}$};
        \filldraw[fill=black] (Q1) circle[radius=0.05] node[above] {$q$}; % source particle
        \draw[->] (Q1) -- ++(V) node[above] {$\boldsymbol{v}$}; % velocity vector
        \filldraw[fill=black] (Q2)  circle[radius=0.05]; % test particle
        \draw[->] (Q2) -- ++(V) node[above] {$\boldsymbol{v}$}; % velocity vector
        \draw[dashed,|-|] ($(Q1)-(0,0.2)$)
        let \p1 = ($(Q2) - (Q1)$)
        in -- ++(\x1,0) node[midway,below] {$s$}; %horizontal distance
        \draw[dashed,|-|] ($(Q2)-(0.2,0)$)
        let \p1 = ($(Q1) - (Q2)$)
        in -- ++(0,\y1) node[midway,left] {$\rho$}; % vertical distance
        \draw[->] (Q1) -- ++($(Q2) - (Q1)$) node[midway,above] {$\boldsymbol{R}$}; %vector
    \end{tikzpicture}
    \caption{Duas partículas interagindo via campo direto.}
\end{figure}

To see this lets consider the interaction of a source particle $Q$ moving with velocity $\vect{v}=v{\vect{\hat{z}}}$ with a witness particle $q$ moving with the same velocity (parallel path) at a distance $s$ in the direction parallel to the movement and at a transverse distance $\rho$, as shown in Figure\,\ref{fig:wake1}. We want to determine the force that the source particle exerts on the witness particle. One way to do this is by calculating the electric field of the source particle in the co-moving frame of reference, $S'$, and Lorentz transforming it back to the laboratory's frame. After the math we obtain:

\begin{align}\label{eq:fields_free_particle}
    \vect{E} = \frac{q}{4\pi\epsilon_0}\frac{\vect{R}}{\gamma^2 R^{*3}}, & & \vect{B} = \frac{1}{c^2}\vect{v} \times \vect{E}
\end{align}
where $\vect{R}$ is the vector which connects both particles, going from the source to the witness and  $R^{*2} = s^2 + x^2/\gamma^2$, e $\gamma = 1/(1-v^2/c^2)$.

Combining equation\,\ref{eq:fields_free_particle} with the Lorentz force, we get the longitudinal and transverse force over the witness particle:
\begin{align}\label{eq:space_charge_force}
    F_l &= E_z = -\frac{q}{4\pi\epsilon_0}\frac{s}{\gamma^2\left(s^2+x^2/\gamma^2\right)^{3/2}}, \\
    F_t &= E_x - vB_y = -\frac{q}{4\pi\epsilon_0}\frac{x}{\gamma^4\left(s^2+x^2/\gamma^2\right)^{3/2}}
\end{align}

In accelerator physics the force $\vect{F}$ is known as space charge force. We can infer from equation\,\ref{eq:space_charge_force} that for any position $s$ and $x$, the longitudinal force is proportional to $\gamma^{-2}$ and $F_t \sim \gamma^{-4}$ if $s \gg x/\gamma$ and  $F_t \sim \gamma^{-1}$ if $s \approx 0$. This way, in the ultra-relativistic limit, $\gamma \to \infty$, the electromagnetic interaction between particles moving parallel to each other in free space is zero. It is easy to show that in this limit, if the movement of the particles is not parallel, there is an interaction force only for $s=0$, but, as their speed is the same, this situation can only happen for an infinitesimal time.

In this work we are interested in the the interaction between particles in the ultra-relativistic limit, $v \to c$. The space charge effects discussed above are despicable in this limit and the interaction between the particles is due to the presence of the walls of the vaccum chamber. Note that taking the limit $v \to c$ in the equation\,\ref{eq:fields_free_particle} and remembering that $s = vt - z$, we can write the electromagnetic field of a ultra-relativistic charge as
\begin{align}
    \vect{E} = \frac{q}{2\pi\epsilon_0}\frac{\vect{\hat{r}}}{r}\delta(z-ct), & & \vect{B} =\frac{1}{c}\vect{\hat{z}}\times\vect{E},
\end{align}
where $\vect{r} = \vect{\hat{x}}x + \vect{\hat{y}}y$ is a bidimensional vector in cylindrical coordinates ($\vect{\hat{x}}$ and $\vect{\hat{y}}$ are unit vectors in the $x$ e $y$ directions, respectively). The equations above show that the field is pancake-like and follow the beam as it travels througth the empty space. It is important to notice that this solution is steady-state, it was necessary an infinite amount of time before $t$ to build it and that's why there is no causal paradoxes in it.

\begin{figure}[hb!]
    \centering
    \label{fig:wake2}
    \begin{tikzpicture}
        \draw[very thick] (0,0) -- ++(10,0) (0,4) -- ++(10,0); %vacuum chamber
        \draw[dashed] (0,2) -- ++(10,0); %eixo de simetria
        \coordinate (V) at (0.5,0);
        \coordinate (Q1) at (4cm,2.2cm);
        \coordinate (Q2) at (0.5cm,2.5cm);
        \filldraw[fill=black] (Q1) circle[radius=0.05] node[left] {$q$}; % source particle
        \draw[->] (Q1) -- ++(V) node[above] {$\boldsymbol{v}$}; % velocity vector
        \draw[-{Stealth[length=10pt]}] (Q1) let \p1 = (Q1) in --(\x1,4);
        \draw[-{Stealth[length=10pt]}] (Q1) let \p1 = (Q1) in --(\x1,0) node[midway,right] {$\boldsymbol{E}$};
        \draw ($(Q1)+(0,1)$) circle[radius=0.2] node[right=0.2] {$\boldsymbol{B}$};
        \filldraw ($(Q1)+(0,1)$) circle[radius=0.07];
    \end{tikzpicture}
    \caption{Particles interacting in a perfectly conducting cylindrical tube.}
\end{figure}

Lets consider now a pipe with cylindrical symmetry\footnote{do not confuse cylindrical symmetry with cylinder. By cylindrical symmetry we mean a system with translational symmetry in one direction.}, hollow and with absolute vacuum in its interior, made of a perfect electric conductor material and with arbitrary cross section. If we put the particles of the previous example inside this pipe, moving parallel to symmetry axis, they will induce image charges in the surface of the wall which cancel the electromagnetic field inside the metal.

The image charges travel with the same velocity $\vect{v}$ of the particles (see Figure\,\ref{fig:wake2}). As they move in parallel paths with constant velocity, in the limit $v \to c$, according to the previous results, they do not interact, independently of how close they are from each other.

From this analysis we conclude that the interaction between particles in the ultra-relativistic limit can occur only for two reasons:
\begin{itemize}
    \item The wall is not perfectly conducting, or
    \item The pipe does not have cylindrical symmetry (which generally is due to the presence of RF cavities, flanges, bellows, beam position monitors, vacuum pumps, among other elements in the vacuum chamber of an accelerator).
\end{itemize}


\chapter{Duas partículas interagindo no vácuo}

\begin{figure}[hb!]
\centering
\begin{tikzpicture}[scale=1]
\def\d{1cm}
\draw[<->] (1,0) node[below]{$z$}
		-- ++(-\d,0) node[below left] {$S$}
        -- ++(0,\d) node[left] {$\rho$}; %coord sys S
\draw[<->] (6*\d,0) node[below]{$z'$}
		-- ++(-\d,0) node[below left] {$S'$}
        -- ++(0,\d) node[left] {$\rho'$}; % coord sys S'
\coordinate (V) at (0.5,0);
\coordinate (Q1) at (4cm,1.5cm);
\coordinate (Q2) at (0.5cm,2.5cm);
\draw[->] (5*\d,0.5*\d) -- ++(V) node[above] {$\boldsymbol{v}$};
\filldraw[fill=black] (Q1) circle[radius=0.05] node[above] {$q$}; % source particle
\draw[->] (Q1) -- ++(V) node[above] {$\boldsymbol{v}$}; % velocity vector
\filldraw[fill=black] (Q2)  circle[radius=0.05]; % test particle
\draw[->] (Q2) -- ++(V) node[above] {$\boldsymbol{v}$}; % velocity vector
\draw[dashed,|-|] ($(Q1)-(0,0.2)$)
				   let \p1 = ($(Q2) - (Q1)$)
                   in -- ++(\x1,0) node[midway,below] {$s$}; %horizontal distance
\draw[dashed,|-|] ($(Q2)-(0.2,0)$)
				   let \p1 = ($(Q1) - (Q2)$)
                   in -- ++(0,\y1) node[midway,left] {$\rho$}; % vertical distance
\draw[->] (Q1) -- ++($(Q2) - (Q1)$) node[midway,above] {$\boldsymbol{R}$}; %vector
\end{tikzpicture}
\caption{Duas partículas interagindo via campo direto.}
\end{figure}

Primeiro, campos gerados por $q_1$. No referencial $S'$:
\begin{align}
\vect{E'} &= \frac{q}{4\pi\epsilon_0} \frac{\vect{\hat{r'}}}{r'^2} \\
\vect{B'} &= \vect{0'}
\end{align}
assumindo que a partícula 1 está na origem do sistema de coordenadas $S'$.

Lembrando que a transformação entre coordenadas esféricas para cilíndricas são:
\begin{align}
r' &= \sqrt{s'^2 + \rho'^2} \\
\vect{\hat{r'}} &= \cos\theta'\vect{\hat{z'}} + \sin\theta'\vect{\hat{\rho'}} =
                  -\frac{s'}{r'}\vect{\hat{z'}} + \frac{\rho'}{r'}\vect{\hat{\rho'}}
\end{align}
onde a coordenada $\phi$ fica inalterada.

Assim podemos reescrever o campo elétrico em suas partes longitudinal e transversal:
\begin{align}
\vect{E'}_{||} &= \frac{q}{4\pi\epsilon_0} \frac{\cos\theta'\vect{\hat{z'}}}{r'^2} =
                 -\frac{q}{4\pi\epsilon_0} \frac{s'\vect{\hat{z'}}}{(\rho'^2+s'^2)^{3/2}} \\
\vect{E'}_{\perp} &= \frac{q}{4\pi\epsilon_0} \frac{\sin\theta'\vect{\hat{\rho'}}}{r'^2} =
                     \frac{q}{4\pi\epsilon_0} \frac{x'\vect{\hat{\rho'}}}{(\rho'^2+s'^2)^{3/2}}
\end{align}

Lembrando as equações de transformação de Lorentz para campos elétricos e magnéticos, para esse problema:

\begin{align}
 \vect{E}_{||} &= \vect{E'}_{||}\\
 \vect{B}_{||} &= \vect{B'}_{||}\\
 \vect{E}_\perp &= \gamma\left(\vect{E'}_\perp - \vect{v}\times\vect{B'}\right)\\
 \vect{B}_\perp &= \gamma\left(\vect{B'}_\perp + \frac{1}{c^2}\vect{v}\times\vect{E'}\right)
\end{align}

Ainda, as coordenadas espaciais são transformadas da seguinte maneira:

\begin{align}
\vect{\hat{\rho}} &=\vect{\hat{\rho'}}, \quad \vect{\hat{z}} = \vect{\hat{z'}}\\
\rho &= \rho', \quad z = \frac{z'}{\gamma}
\end{align}

Assim, podemos notar que:

\begin{align}
\vect{E}_{||} &= -\frac{q}{4\pi\epsilon_0} \frac{s'\vect{\hat{z'}}}{(\rho'^2+s'^2)^{3/2}} =
				 -\frac{q}{4\pi\epsilon_0} \frac{\gamma s\vect{\hat{z}}}{((\gamma s)^2+\rho^2)^{3/2}}=
                 -\frac{q}{4\pi\epsilon_0} \frac{s\vect{\hat{z}}}{\gamma^2R^{*3}} \\
\vect{E}_\perp&= \frac{q}{4\pi\epsilon_0}\frac{\gamma \rho'\vect{\hat{\rho'}}}{(\rho'^2+s'^2)^{3/2}} =
				 \frac{q}{4\pi\epsilon_0}\frac{\gamma \rho\vect{\hat{\rho}}}{((\gamma s)^2+\rho^2)^{3/2}}=
                 \frac{q}{4\pi\epsilon_0} \frac{\rho\vect{\hat{\rho}}}{\gamma^2R^{*3}} \\
\vect{B}_\perp&= -\frac{\gamma   vE'_\perp}{c^2}\vect{\hat{\phi'}} =
		         -\frac{vE_\perp}{c^2}\vect{\hat{\phi}}
\end{align}
onde $R^* = \sqrt{s^2+\left(\rho/\gamma\right)^2}$

Agora podemos analisar a força exercida pela partícula fonte sobre a partícula teste usando a força de Lorentz
\begin{equation}
\vect{F} = \left(\vect{E} + \vect{v}\times\vect{B}\right)
\end{equation}
onde foi assumida carga unitária para a partícula teste, e os campos calculados anteriormente.

Assumindo que a velocidade da partícula teste é a mesma da partícula fonte (mesmo módulo e direção), as componentes longitudinal e transversal da força ficam:

\begin{align}
\vect{F}_{||} &= \vect{E}_{||} = -\frac{q}{4\pi\epsilon_0} \frac{s\vect{\hat{z}}}{\gamma^2R^{*3}}\\
\vect{F}_\perp&= \left(1+\frac{\vect{v}\times\vect{v}\times}{c^2}\right)\vect{E}_\perp =
                 \left(1-\frac{v^2}{c^2}\right)\vect{E}_\perp =
                 \frac{q}{4\pi\epsilon_0} \frac{\rho\vect{\hat{z}}}{\gamma^4R^{*3}}
\end{align}
onde vemos que a força longitudinal tende a zero proporcionalmente a $\gamma^{-2}$ quando $v \to c$ e que a força longitudinal tende a zero com $\gamma^{-4}$, se $s>\rho/\gamma$ e com $\gamma^{-1}$ se $s<\rho/\gamma$.

Agora, vamos assumir que a velocidade da partícula teste não é paralela à velocidade da partícula fonte
\begin{equation}
\vect{v_2} = v(\cos\delta \vect{\hat{z}} + \sin\delta\vect{\hat{\rho}})
\end{equation}

Assim, a força sofrida por essa partícula fica

\begin{equation}\begin{aligned}
\vect{F} &= \vect{E}_{||} + \vect{E}_\perp - v(\cos\delta \vect{\hat{z}} + \sin\delta\vect{\hat{\rho}}) \times \vect{\hat{\phi}}\frac{vE_\perp}{c^2} \\
&= \left(1-\frac{v^2}{c^2}\cos\delta\right)\vect{E}_\perp + \vect{E}_{||} + \frac{v^2}{c^2}E_\perp\sin\delta\vect{\hat{z}}
\end{aligned}\end{equation}

Olhando essa expressão, vemos que, conforme $v \to c$, a força pode ser expressa como:

\begin{equation}
\vect{F} = \left\{
\begin{aligned}
\left((1-\cos\delta)\vect{\hat{\rho}} + \sin\delta\vect{\hat{z}}\right)
\frac{q}{4\pi\epsilon_0} \frac{\gamma\vect{\hat{\rho}}}{\rho^2} & &|s|<\rho/\gamma\\
0 & &|s|>\rho/\gamma
\end{aligned}\right.
\end{equation}


\chapter{Causality and catch up distance}

If one particle moves in a straight line at light speed, the electromagnetic field scattered by the discontinuities of the chamber will not catch up with it and will not affect the charges travelling ahead of it. The field will only interact with the charges moving behind of the source particle. Such property is known as causality.

Even though this property is not strictly true in the real world, because particles always travel at speeds lower than the light's, it is true in most practical cases, as we will see bellow.

Lets try to calculate the distance $z$ where the field generated by some discontinuity in the vacuum chamber will catch up with a witness particle at a distance $s$ behind the source particle. At the time $t=0$ the source particle passes through the discontinuity and an electromagnetic wave is generated with its wave front travelling at the speed of light in all directions, forming a sphere of radius $R$, see FigureXX. At any given time after this, the following relation holds:
\begin{align}
ct = R \quad vt = z && \Rightarrow && R = \frac{z}{\beta} \quad \text{where} \,\, \beta = \frac{v}{c}
\end{align}
where $z$ is the distance travelled by the source particle. Besides that, at the specific time when the wake catchs up with the witness particle, the following relation is valid:
\begin{align}
R^2 = b^2 + (z-s)^2  \Rightarrow z^2(\frac{1}{\beta^2}-1) + 2sz - (b^2 + s^2) = 0  \Rightarrow \\
z = -\gamma^2 \beta^2 s + \sqrt{s^2\gamma^4\beta^4 + \gamma^2\beta^2\left(b^2 + s^2\right)}  = \gamma^2 \beta^2 s\left(-1 + \sqrt{1 + \frac{1}{\gamma^2\beta^2}\left(1 + \frac{b^2}{s^2}\right)}\right)
\end{align}
where $b$ is the distance from the discontinuity to the trajectory of the particles and $\gamma = 1/\sqrt{1-\beta^2}$ is the relativistic energy. FigureXX shows a graphic of this function, normalized by the distance $b$. We notice that, for $s=0$, which means the field catching up with the source particle, $z = \gamma\beta b$. For the case of Sirius, $\gamma \approx 5870$ and $\beta \approx 1$, if $ b = 2$mm, $z \approx 12$m.

\begin{figure}[hb!]
\centering
\label{fig:catch_up}
\begin{tikzpicture}
\draw[very thick] (0,0) -- ++(10,0) (0,4) -- ++(10,0); %vacuum chamber
\draw[very thick] (4.8,4) to [out=-90,in=0] (5,3.8) to [out=180,in=-90] (5.2,4);
\draw[dashed] (0,2) -- ++(10,0); %eixo de simetria
\coordinate (V) at (0.5,0);
\coordinate (Q1) at (4cm,2.2cm);
\coordinate (Q2) at (0.5cm,2.5cm);
\filldraw[fill=black] (Q1) circle[radius=0.05] node[left] {$q$}; % source particle
\draw[->] (Q1) -- ++(V) node[above] {$\boldsymbol{v}$}; % velocity vector
\draw[-{Stealth[length=10pt]}] (Q1) let \p1 = (Q1) in --(\x1,4);
\draw[-{Stealth[length=10pt]}] (Q1) let \p1 = (Q1) in --(\x1,0) node[midway,right] {$\boldsymbol{E}$};
\draw ($(Q1)+(0,1)$) circle[radius=0.2] node[right=0.2] {$\boldsymbol{B}$};
\filldraw ($(Q1)+(0,1)$) circle[radius=0.07];
\end{tikzpicture}
\caption{The catch up distance.}
\end{figure}


\chapter{Check Panofsky-Wenzel}\label{app:check_panofky}

As the velocity of the particle is in the longitudinal direction we have
\begin{align}
    \dertot{a}{t} = \derpar{a}{t} + v\derpar{a}{s}
\end{align}
where $a$ can be any of the components of vector potential or the scalar potential. This way, since all fields and potentials go to zero in infinity, we always have this equality for the integral of these quantities:
\begin{align}
    \infint{t}{\left(\derpar{a}{t} + v\derpar{a}{s}\right)|_{s=vt-z}} = 0.
\end{align}
Besides, since the integration in time happens with $s=vt-z$, we have
\begin{align}
    \derpar{}{z}\left(\infint{t}{a}\right) =
    \infint{t}{\derpar{a}{s}\derpar{s}{z}} =
    -\infint{t}{\derpar{a}{s}}
\end{align}
Also, lets remember the relations between the electromagnetic fields and the potentials:
\begin{align}
    E_x &= -\derpar{A_x}{t} - \derpar{\phi}{x} &
        & B_x = \derpar{A_y}{s} - \derpar{A_s}{y}\\
    E_y &= -\derpar{A_y}{t} - \derpar{\phi}{y} &
        & B_y = \derpar{A_s}{x} - \derpar{A_x}{s}\\
    E_s &= -\derpar{A_s}{t} - \derpar{\phi}{s} &
        & B_s = \derpar{A_x}{y} - \derpar{A_y}{x}\\
\end{align}
\begin{align}
    w_s = \derpar{W}{z}\nonumber
    &= \frac{c}{qQ} \derpar{}{z}\left(\infint{t}{\left(vA_s-\phi\right)}\right)
    = \frac{c}{qQ} \infint{t}{\left(-v\derpar{A_s}{s}+\derpar{\phi}{s}\right)}\\
    &= \frac{c}{qQ} \infint{t}{\left(\derpar{A_s}{t}+\derpar{\phi}{s}\right)}
    =-\frac{c}{qQ} \infint{t}{E_s}\\[1cm]
    w_x = \derpar{W}{x}\nonumber
    &=\frac{c}{qQ} \infint{t}{\left(v\derpar{A_s}{x}-\derpar{\phi}{x}\right)}\\
    &= \frac{c}{qQ} \infint{t}{\left(vB_y+v\derpar{A_x}{s} + E_x + \derpar{A_x}{t}\right)}
    = \frac{c}{qQ} \infint{t}{\left(E_x + vB_y\right)}\\[1cm]
    w_y = \derpar{W}{y}\nonumber
    &=\frac{c}{qQ} \infint{t}{\left(v\derpar{A_s}{y}-\derpar{\phi}{y}\right)}\\
    &=\frac{c}{qQ} \infint{t}{\left(-vB_x+v\derpar{A_y}{s} + E_y + \derpar{A_y}{t}\right)}
    = \frac{c}{qQ} \infint{t}{\left(E_y + vB_x\right)}
\end{align}

%%%%%
%%%%%
%%%%
%%%%%
%%%%%

\chapter{Symmetry analysis}\label{app:symmetry_analysis}

First lets consider the general expansion of the Wake potential of a bunch in the transverse coordinates of the centroid of the source $(x_s,y_s)$ and the integration path $(x,y)$ up to third order:

\begin{align}
W(\vect{x},s) &= W_0(s) + \sum_{i=1}^4 M_i(s) x_i + \frac12\sum_{i,j=1}^4 D_{ij}(s) x_i x_j + \frac13\sum_{i,j,k=1}^4 Q_{ijk}(s) x_i x_j x_k
\end{align}
where we considered $\vect{x} = (x_1,x_2,x_3,x_4) = (x_s,y_s,x,y)$ and because of the commutative property of multiplication, we can set $D_{ij} = D_{ji}$ and $Q_{ijk}=Q_{kij}=Q_{jki}=Q_{ikj}=Q_{jik}=Q_{kji}$ without loss of generality. With these considerations, number of independent components of $D$ is 10 and $Q$ is 20. Besides that, the fact that $W$ is an harmonic function of the transverse coordinates of the integration path, imposes that
\begin{align}
D_{33} &= - D_{44} \\
Q_{33i} &= - Q_{44i}, \quad \text{with} \quad i=1,2,3,4
\end{align}
which leaves only nine independent components of $D$ and sixteen of $Q$.

Thus, the wake forces become:
\begin{align}
F_L(\vec{x},s) = W'(\vec{x},s) &= W_0' + \sum_{i=1}^4 M_i' x_i + \frac12\sum_{i,j=1}^4 D_{ij}' x_i x_j + \frac13\sum_{i,j,k=1}^4 Q_{ijk}' x_i x_j x_k \\
F_x(\vec{x},s) = \derpar{W(\vect{x},s)}{x} &= M_3 + \sum_{i=1}^4 D_{3i} x_i + \sum_{i,j=1}^4 Q_{3ij} x_i x_j \\
F_y(\vec{x},s) = \derpar{W(\vect{x},s)}{y} &= M_4 + \sum_{i=1}^4 D_{4i} x_i + \sum_{i,j=1}^4 Q_{4ij} x_i x_j
\end{align}

Generally we are interested in obtaining the linear terms as function of the transverse coordinates correct up to second order. It means we want to isolate the linear from the quadract terms in the simulations. Thus, lets keep only the second order terms in the wake forces above.
\begin{align}
F_L(\vect{x},s) &= W_0' + \vect{M'}^T \cdot \vect{x} + \frac12\vect{x}^T \cdot \tensor{D'}  \cdot \vect{x} \\
F_x(\vect{x},s) &= M_x + \vect{D_x}^T \cdot \vect{x} +        \vect{x}^T \cdot \tensor{Q_x} \cdot \vect{x} \\
F_y(\vect{x},s) &= M_y + \vect{D_y}^T \cdot \vect{x} +        \vect{x}^T \cdot \tensor{Q_y} \cdot \vect{x}
\end{align}
where
\begin{align}
\vect{M} &= \begin{pmatrix*}[r] M_{1}\\ M_{2}\\ M_{3}\\ M_{4}\end{pmatrix*} \\
\tensor{D}  &= \begin{pmatrix*}[r] D_{11} & D_{12} & D_{13} & D_{14} \\
                                   D_{21} & D_{22} & D_{23} & D_{24} \\
                                   D_{31} & D_{32} & D_{33} & D_{34} \\
                                   D_{41} & D_{42} & D_{43} & D_{44}
               \end{pmatrix*} =
               \begin{pmatrix*}[r] D_{11} & D_{12} & D_{13} & D_{14} \\
                                   D_{12} & D_{22} & D_{23} & D_{24} \\
                                   D_{13} & D_{23} & D_{33} & D_{34} \\
                                   D_{14} & D_{24} & D_{34} & -D_{33}
               \end{pmatrix*}\\
\vect{D_x} &= \tensor{D}\cdot \vect{\hat{x}} \\
\vect{D_y} &= \tensor{D}\cdot \vect{\hat{y}} \\
\tensor{Q_x}&= \begin{pmatrix*}[r] Q_{311} & Q_{312} & Q_{313} & Q_{314} \\
                                   Q_{321} & Q_{322} & Q_{323} & Q_{324} \\
                                   Q_{331} & Q_{332} & Q_{333} & Q_{334} \\
                                   Q_{341} & Q_{342} & Q_{343} & Q_{344}
               \end{pmatrix*} =
               \begin{pmatrix*}[r] Q_{113} & Q_{123} & Q_{133} & Q_{134} \\
                                   Q_{123} & Q_{223} & Q_{233} & Q_{234} \\
                                   Q_{133} & Q_{233} & Q_{333} & Q_{334} \\
                                   Q_{134} & Q_{234} & Q_{334} & -Q_{333}
               \end{pmatrix*}\\
\tensor{Q_y}&= \begin{pmatrix*}[r] Q_{411} & Q_{412} & Q_{413} & Q_{414} \\
                                   Q_{421} & Q_{422} & Q_{423} & Q_{424} \\
                                   Q_{431} & Q_{432} & Q_{433} & Q_{434} \\
                                   Q_{441} & Q_{442} & Q_{443} & Q_{444}
               \end{pmatrix*} =
               \begin{pmatrix*}[r] Q_{114} & Q_{124} & Q_{134} & Q_{133} \\
                                   Q_{124} & Q_{224} & Q_{234} & Q_{233} \\
                                   Q_{134} & Q_{234} & Q_{334} & -Q_{333} \\
                                   Q_{133} & Q_{233} &-Q_{333} & -Q_{334}
               \end{pmatrix*}\\
\end{align}



Due to their importance, some components of the $D$ tensor have a name: the $D_{31}$ and $D_{42}$ are called dipolar wakes; and $D_{33}$ and $D_{44}$ are the quadrupolar wakes. While the first generates coherent tune-shifts and instabilities, the later is the responsible for incoherent tune-shifts of the beam. The other terms are the skew components, which are zero for most practical cases, as we will see below.

When the geometry has symmetry the number of independent compononts in $M$, $D$ and $Q$ are reduced even more.
When the symmetry occurs in one plane, the number of independent components is two, five and eight, respectivelly.Below we list examples for the most practical cases:

\begin{itemize}
\item symmetry in the $yz$ plane, or $x=0$.
\begin{align}
W(x_s,y_s,x,y,s) = & W(-x_s,y_s,-x,y,s) \Rightarrow \\
M_2= & M_4=0\\
D_{12}=D_{14}= & D_{23}=D_{34}=0\\
Q_{111}=Q_{122}= & Q_{113}=Q_{223}=0\\
Q_{133}=Q_{333}= & Q_{124}=Q_{234}=0
\end{align}
\begin{align}
\vect{M}&= \begin{pmatrix} M_{1}\\ 0\\ M_{3}\\ 0\end{pmatrix} &
\tensor{D} &= \begin{pmatrix} D_{11} &    0   & D_{13} &    0    \\
                                  0   & D_{22} &    0   &  D_{24} \\
                               D_{13} &    0   & D_{33} &    0    \\
                                  0   & D_{24} &    0   & -D_{33}
               \end{pmatrix}\\
\tensor{Q_x}&=\begin{pmatrix}    0    & Q_{123} &    0    & Q_{134} \\
                               Q_{123} &    0    & Q_{233} &    0    \\
                                  0    & Q_{233} &    0    & Q_{334} \\
                               Q_{134} &    0    & Q_{334} &    0
               \end{pmatrix} &
\tensor{Q_y}&=\begin{pmatrix} Q_{114} &    0    & Q_{134} &    0    \\
                                  0    & Q_{224} &    0    & Q_{233} \\
                               Q_{134} &    0    & Q_{334} &    0    \\
                                  0    & Q_{233} &    0    & -Q_{334}
               \end{pmatrix}
\end{align}

\item symmetry in the $xz$ plane, or $y=0$.
\begin{align}
W(x_s,y_s,x,y,s) = & W(x_s,-y_s,x,-y,s) \Rightarrow \\
M_1= & M_3=0\\
D_{12}=D_{14}= & D_{23}=D_{34}=0\\
Q_{112}=Q_{222}= & Q_{123}=Q_{233}=0\\
Q_{114}=Q_{224}= & Q_{134}=Q_{334}=0
\end{align}
\begin{align}
\vect{M}    &= \begin{pmatrix}   0  \\ M_{2}\\  0  \\ M_{4}\end{pmatrix} &
\tensor{D}  &= \begin{pmatrix} D_{11} &    0   & D_{13} &    0   \\
                                  0   & D_{22} &    0   & D_{24} \\
                               D_{13} &    0   & D_{33} &    0   \\
                                  0   & D_{24} &    0   & -D_{33}
               \end{pmatrix}\\
\tensor{Q_x}&= \begin{pmatrix} Q_{113} &    0    & Q_{133} &    0    \\
                                  0    & Q_{223} &    0    & Q_{234} \\
                               Q_{133} &    0    & Q_{333} &    0    \\
                                  0    & Q_{234} &    0    & -Q_{333}
               \end{pmatrix} &
\tensor{Q_y}&= \begin{pmatrix}    0    & Q_{124} &    0    & Q_{133} \\
                               Q_{124} &    0    & Q_{234} &    0    \\
                                  0    & Q_{234} &    0    & -Q_{333} \\
                               Q_{133} &    0    &-Q_{333} &    0
               \end{pmatrix}
\end{align}

\item symmetry in the plane $y=x$.
\begin{align}
W(x_s,y_s,x,y,s) =& W(y_s,x_s,y,x,s) \Rightarrow \\
M_1=M_2, \quad & M_3=M_4\\
D_{11}=D_{22}, \quad D_{13}=D_{24}, \quad & D_{14}=D_{23}, \quad D_{33}=0 \\
Q_{111}=Q_{222}, \quad Q_{112}= Q_{122}, \quad & Q_{113}=Q_{224}, \quad Q_{114}= Q_{223} \\
Q_{123}=Q_{124}, \quad Q_{133}=-Q_{233}, \quad & Q_{134}=Q_{234}, \quad Q_{333}=-Q_{334}
\end{align}
\begin{align}
\vect{M}  &= \begin{pmatrix} M_{1}\\ M_{1}\\ M_{3}\\ M_{3}\end{pmatrix} &
\tensor{D}   &=\begin{pmatrix} D_{11} & D_{12} & D_{13} & D_{14} \\
                               D_{12} & D_{11} & D_{14} & D_{13} \\
                               D_{13} & D_{14} &    0   & D_{34} \\
                               D_{14} & D_{13} & D_{34} &    0
               \end{pmatrix}\\
\tensor{Q_x}&= \begin{pmatrix} Q_{113} & Q_{123} & Q_{133} & Q_{134} \\
                               Q_{123} & Q_{114} &-Q_{133} & Q_{134} \\
                               Q_{133} &-Q_{133} & Q_{333} &-Q_{333} \\
                               Q_{134} & Q_{134} &-Q_{333} &-Q_{333}
               \end{pmatrix} &
\tensor{Q_y} &=\begin{pmatrix} Q_{114} & Q_{123} & Q_{134} & Q_{133} \\
                               Q_{123} & Q_{113} & Q_{134} &-Q_{133} \\
                               Q_{134} & Q_{134} &-Q_{333} &-Q_{333} \\
                               Q_{133} &-Q_{133} &-Q_{333} & Q_{333}
               \end{pmatrix}
\end{align}

\item symmetry in the plane $y=-x$.
\begin{align}
W(x_s,y_s,x,y,s) =& W(-y_s,-x_s,-y,-x,s) \Rightarrow \\
M_1=-M_2, \quad & M_3=-M_4\\
D_{11}=D_{22}, \quad D_{13}=D_{24}, \quad & D_{14}=D_{23}, \quad D_{33}=0 \\
Q_{111}=-Q_{222}, \quad Q_{112}=-Q_{122}, \quad & Q_{113}=-Q_{224}, \quad Q_{114}=-Q_{223} \\
Q_{123}=-Q_{124}, \quad Q_{133}= Q_{233}, \quad & Q_{134}=-Q_{234}, \quad Q_{333}= Q_{334}
\end{align}
\begin{align}
\vect{M}  &= \begin{pmatrix} M_{1}\\ -M_{1}\\ M_{3}\\ -M_{3}\end{pmatrix} &
\tensor{D}   &=\begin{pmatrix} D_{11} & D_{12} & D_{13} & D_{14} \\
                               D_{12} & D_{11} & D_{14} & D_{13} \\
                               D_{13} & D_{14} &    0   & D_{34} \\
                               D_{14} & D_{13} & D_{34} &    0
               \end{pmatrix}\\
\tensor{Q_x}&= \begin{pmatrix} Q_{113} & Q_{123} & Q_{133} & Q_{134} \\
                               Q_{123} &-Q_{114} & Q_{133} &-Q_{134} \\
                               Q_{133} & Q_{133} & Q_{333} & Q_{333} \\
                               Q_{134} &-Q_{134} & Q_{333} &-Q_{333}
               \end{pmatrix} &
\tensor{Q_y} &=\begin{pmatrix} Q_{114} &-Q_{123} & Q_{134} & Q_{133} \\
                              -Q_{123} &-Q_{113} &-Q_{134} & Q_{133} \\
                               Q_{134} &-Q_{134} & Q_{333} &-Q_{333} \\
                               Q_{133} & Q_{133} &-Q_{333} &-Q_{333}
               \end{pmatrix}
\end{align}
\end{itemize}

Below we combine some of the above mentioned symmetries which are very commom in the simulations performed for accelerators elements:

\begin{itemize}
\item x=0 and y=0.
\begin{align}
\vect{M}= \vect{0} & &
\tensor{D} = \begin{pmatrix} D_{11} &    0   & D_{13} &    0    \\
                                  0   & D_{22} &    0   &  D_{24} \\
                               D_{13} &    0   & D_{33} &    0    \\
                                  0   & D_{24} &    0   & -D_{33}
               \end{pmatrix} & &
\tensor{Q_x}=\tensor{0} & &
\tensor{Q_y}=\tensor{0}
\end{align}

\item $y=-x$ and $y=x$.
\begin{align}
\vect{M}= \vect{0} & &
\tensor{D}   &=\begin{pmatrix} D_{11} & D_{12} & D_{13} & D_{14} \\
                               D_{12} & D_{11} & D_{14} & D_{13} \\
                               D_{13} & D_{14} &    0   & D_{34} \\
                               D_{14} & D_{13} & D_{34} &    0
               \end{pmatrix}& &
\tensor{Q_x}=\tensor{0} & &
\tensor{Q_y}=\tensor{0}
\end{align}

\item $x=0$, $y=0$, $y=-x$ and $y=x$.
\begin{align}
\vect{M}= \vect{0} & &
\tensor{D}   &=\begin{pmatrix} D_{11} &   0    & D_{13} &   0    \\
                                 0    & D_{11} &    0   & D_{13} \\
                               D_{13} &   0    &    0   &   0    \\
                                 0    & D_{13} &    0   &   0
               \end{pmatrix} & &
\tensor{Q_x}=\tensor{0} & &
\tensor{Q_y}=\tensor{0}
\end{align}
\end{itemize}
where we notice that when there is at least 2 planes of symmetry, all the odd order terms are zero.

\chapter{Wake Calculation From GdfiDL Simulations}

One simulation in GDFIDL consists of passing a linear gaussian bunch with the velocity of light in the longitudinal direction and with a specific transverse position, say $(x_s,y_s)=(d_x,d_y)$ through the simulated structure solving Maxwell equations in time. While doing this, the code saves in memory the wake potential $W(d_x,d_y,x,y,s)$ for all transverse positions $(x,y)$ of integration and all $s$. This procedure is very time consuming and we gerally try to perform the mininum amount of simulations possible to get the results we need.

Lets suppose a simulation was performed with the position of the source in $(x_s,y_s)=(d,0)$. If we calculate the Horizontal Wake potential in a path where $y=0$ then, up to second order in the transverse coordinates we can write:

\begin{align}
F_L(d,0,x,0) &= W_0  +  M_1d  +  M_3x + D_{11}d^2 + D_{13}dx + D_{33}x^2 \\
F_x(d,0,x,0) &= M_x + D_{x3} x + D_{x1} d + Q_{x33} x^2 + Q_{x11} d^2 + Q_{x13} x d\\
F_y(d,0,x,0) &= M_y + D_{y3} x + D_{y1} d + Q_{y33} x^2 + Q_{y11} d^2 + Q_{y13} x d
\end{align}

Now, integrating the total wake in 4 different $x$ points we get:
\begin{align}
	F_1 = F_x(d,x_1) &= M_x  +  D_{x3} x_1  +  D_{x1} d  +  Q_{x33} x_1^2  +  Q_{x11} d^2  +  Q_{x13} x_1 d \\
    F_2 = F_x(d,x_2) &= M_x  +  D_{x3} x_2  +  D_{x1} d  +  Q_{x33} x_2^2  +  Q_{x11} d^2  +  Q_{x13} x_2 d \\
    F_3 = F_x(d,x_3) &= M_x  +  D_{x3} x_3  +  D_{x1} d  +  Q_{x33} x_3^2  +  Q_{x11} d^2  +  Q_{x13} x_3 d \\
    F_4 = F_x(d,x_4) &= M_x  +  D_{x3} x_4  +  D_{x1} d  +  Q_{x33} x_4^2  +  Q_{x11} d^2  +  Q_{x13} x_4 d
\end{align}

To extract the component $D_x$ from the total wake we do:
\begin{align}
\Delta_1 = \frac{F_1}{x_1} - \frac{F_2}{x_2} &= \left(M_x  +  D_{x1}d  +  Q_{x11}d^2\right)\left(\frac{1}{x_1}-\frac{1}{x_2}\right) + Q_{x33}(x_1-x_2) \\
\Delta_2 = \frac{F_3}{x_3} - \frac{F_4}{x_4} &= \left(M_x  +  D_{x1}d  +  Q_{x11}d^2\right)\left(\frac{1}{x_3}-\frac{1}{x_4}\right) + Q_{x33}(x_3-x_4)
\end{align}
then, remembering $(1/a-1/b)/(a-b) = -1/ab$ we get:
\begin{align}
	\frac{\Delta_1}{x_1-x_2} - \frac{\Delta_2}{x_3-x_4} &= \left(M_x  +  D_{x1} d  +  Q_{x11} d^2\right)\left(\frac{1}{x_3x_4} - \frac{1}{x_1x_2}\right)
\end{align}
\begin{align}
	M_x  +  D_{x1} d +  Q_{x11} d^2 &= \left(\frac{x_1x_2x_3x_4}{x_3x_4 - x_1x_2}\right)\left(\frac{\Delta_1}{x_1-x_2} - \frac{\Delta_2}{x_3-x_4}\right)
\end{align}

Note that in general to isolate the dipolar component $D_x$, it is necessary to perform another two simulations. If another simulation is performed with the source at $(x_s,y_s) = (-d,0)$ it is also possible to isolate the dipolar component.

Now, lets try to extract the quadrupolar component
\begin{align}
\Delta_1 = \frac{F_1 - F_2}{x_1-x_2} &= D_{x3} + Q_{x13}d + Q_{x33}(x_1 + x_2)
\end{align}
Notice that, if we choose $x_2=-x_1$ the contribution from $Q_{x33}$ is canceled and
\begin{align}
	D_{x3} + Q_{x13}d = \Delta_1
\end{align}
where it is clear that is necessary other simulation to extract the quadrupolar wake.

\chapter{Cálculo dos wake-potential a partir do ECHOzR}

De acordo com a referência, temos que:

\begin{align}
W_{||}(x_0,y_0,x,y,s) &= \frac1w\sum_{m=1}^\infty W_m(y_0,y,s)\sin(k_{x,m}x_0)\sin(k_{x,m}x), \\
W_y(x_0,y_0,x,y,s) &= \frac1w\sum_{m=1}^\infty k_{x,m}W_{y,m}(y_0,y,s)\sin(k_{x,m}x_0)\sin(k_{x,m}x), \\
W_x(x_0,y_0,x,y,s) &= \frac1w\sum_{m=1}^\infty k_{x,m}W_{x,m}(y_0,y,s)\sin(k_{x,m}x_0)\cos(k_{x,m}x),
\end{align}
where $w$ is the half-width of the structure, $0<x<2w$ is the horizontal position of the trailing particle, $0<x_0<2w$ is the horizontal position of the source particle, $y$ is vertical position of the trailing particle, $y_0$ is vertical position of the source particle, $s=z-ct$ is the position of the trailing particle relative to the source particle and
\begin{equation}
k_{x,m} = \frac{\pi}{2w}m
\end{equation}
The other terms are given by
\begin{align}
W_m(y_0,y,s)     &= W^{cc}_m(s)\cosh(k_{x,m}y_0)\cosh(k_{x,m}y) + W^{ss}_m(s)\sinh(k_{x,m}y_0)\sinh(k_{x,m}y), \\
W_{y,m}(y_0,y,s) &= S^{cc}_m(s)\cosh(k_{x,m}y_0)\sinh(k_{x,m}y) + S^{ss}_m(s)\sinh(k_{x,m}y_0)\cosh(k_{x,m}y), \\
W_{x,m}(y_0,y,s) &= S^{cc}_m(s)\cosh(k_{x,m}y_0)\cosh(k_{x,m}y) + S^{ss}_m(s)\sinh(k_{x,m}y_0)\sinh(k_{x,m}y),
\end{align}
where
\begin{equation}
S^{cc}_m = \int_{-\infty}^s W^{cc}_m(s')\mathrm{d}s', \qquad S^{ss}_m = \int_{-\infty}^s W^{ss}_m(s')\mathrm{d}s'
\end{equation}
and the quantities $W^{cc}_m(s')$ and $W^{ss}_m(s')$ are related to the output of the softwares ECHOzR and ECHO2D by:
\begin{align}
W^{cc}_m(s') = \frac{W_m^M(y_0,y,s)}{\cosh(k_{x,m}y_0)\cosh(k_{x,m}y)} \\
W^{ss}_m(s') = \frac{W_m^E(y_0,y,s)}{\sinh(k_{x,m}y_0)\sinh(k_{x,m}y)}
\end{align}
where $W_m^M(y_0,y,s)$ and $W_m^M(y_0,y,s)$ are the results of the simulation with magnetic and electric boundary conditions, respectively.


Our objective is to obtain the formulas for the monopole longitudinal, dipole and quarupolar transverse wake functions at the point $\vec{v} = (x_0=w,y_0=0,x=w,y=0)$ from these results. The monopolar wakes are readly obtained:
\begin{align}
W_m(0,0,s) = W^{cc}_m(s) &\Rightarrow W_{||}(\vec{v}) = \frac1w\sum^\infty_{m=1,\mathrm{odd}} W^{cc}_m(s), \\
W_{y,m}(0,0,s) = 0 &\Rightarrow W_y(\vec{v}) = 0, \\
W_{x,m}(0,0,s) = S^{cc}_m(s) &\Rightarrow W_x(\vec{v}) = \frac1w\sum^\infty_{m=1} S^{cc}_m(s)\frac{\sin(m\pi)}{2} = 0
\end{align}
To calculate the dipolar and quadrupolar wakes, we need to take the first derivative of the transverse wakes at the point of interest, $\vec{v}$. It is easy to see that the first derivatives of the longitudinal wake are zero. For the transverse:

\begin{align}
W_{y,d}(s) &= \left.\frac{\mathrm{d}}{\mathrm{d}y_0}W_y\right|_{\vec{v}} = \frac1w\sum^\infty_{m=1,\mathrm{odd}} k_{x,m}^2 S^{ss}_m(s) = \frac1w\int_{-\infty}^s\sum^\infty_{m=1,\mathrm{odd}} k_{x,m}^2 W^{ss}_m(s') \mathrm{d}s',\\
W_{y,q}(s) &= \left.\frac{\mathrm{d}}{\mathrm{d}y}W_y\right|_{\vec{v}} = \frac1w\sum^\infty_{m=1,\mathrm{odd}} k_{x,m}^2 S^{cc}_m(s) = \frac1w\int_{-\infty}^s\sum^\infty_{m=1,\mathrm{odd}} k_{x,m}^2 W^{cc}_m(s') \mathrm{d}s', \\
W_{x,d}(s) &= \left.\frac{\mathrm{d}}{\mathrm{d}x_0}W_x\right|_{\vec{v}} = \frac1w\sum^\infty_{m=1,\mathrm{even}} k_{x,m}^2 S^{cc}_m(s) = \frac1w\int_{-\infty}^s\sum^\infty_{m=1,\mathrm{even}}k_{x,m}^2 W^{cc}_m(s') \mathrm{d}s', \\
W_{x,q}(s) &= \left.\frac{\mathrm{d}}{\mathrm{d}x}W_x\right|_{\vec{v}} = -\frac1w\sum^\infty_{m=1,\mathrm{odd}} k_{x,m}^2 S^{cc}_m(s) =-\frac1w\int_{-\infty}^s\sum^\infty_{m=1,\mathrm{odd}} k_{x,m}^2 W^{cc}_m(s') \mathrm{d}s'
\end{align}
and all the skew terms are zero.
