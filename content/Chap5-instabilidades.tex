\chapter{Collective Effects Related to Wake Fields}

The storage ring's global properties relevant to the calculation of collective instabilities are presented in table \ref{tab:sr_main_properties}

\begin{table}[!ht]
 \centering
 \caption{Storage ring's main properties for instabilities estimations.}
 \label{tab:sr_main_properties}
 \begin{tabular}{lcl}
  Circumference & 518.25 & \si{\meter}\\\hline
  Revolution Period & 1.73 & \si{\micro\second}\\\hline
  Revolution Frequency & 0.578 & \si{\mega\hertz}\\\hline
  Harmonic Number      & 864   & \\\hline
  Momentum compaction  & \SI{1.7e-4}{}\\\hline
  Energy               & 3      & \si{\giga\electronvolt}\\\hline
  Horizontal Tune      & 46.171 & \\\hline
  Vertical Tune        & 14.147 & \\\hline
 \end{tabular}
\end{table}

In our studies of collective effects on Sirius, we identified five stages in the operation of the ring since the commissioning until the full-current mode. The main characteristics of these stages are presented in table \ref{tab:sirius_stages}.

\begin{table}[!hb]
 \centering
 \caption{Sirius stages of operation}
 \label{tab:sirius_stages}
 \begin{tabular}{lccccc}
 Stage  & Comm.  & IDs  & IDS + Landau & Full-Current & SB \\\hline
 Current [\si{\milli\ampere}] & 10     & 100  &  100        & 300  & 2    \\\hline
 Bunch Length [\si{\milli\meter}] & 3.3 & 3.4 &  13.4       & 13.4 & 3.8    \\\hline
 Hor. damp. time [\si{\milli\second}]  & 16  & 9.4 &  9.4   & 9.4  &9.4 \\\hline
 Ver. damp. time [\si{\milli\second}]  & 21  & 12.5&  12.5  & 12.5 &12.5   \\\hline
 Long. damp. time [\si{\milli\second}] & 12.5& 7.5 &  7.5   & 7.5  &7.5  \\\hline
 \end{tabular}
\end{table}

The impedance budget model we are using is presented in table \ref{tab:impedance_budget} and the total impedances are plotted in figure \ref{fig:impedance_budget}.

\begin{table}[!ht]
 \centering
 \caption{Impedance Budget Model}
 \label{tab:impedance_budget}
 \begin{tabular}{lccc}
 Element                  &  Number  & $<\beta_x>$ & $<\beta_y>$ \\\hline
Wall With Coating         & 1        &   7         &   11        \\\hline  
In-vac. Und. @ low Betax  & 4        &       4     &       1.5   \\\hline      
In-vac. Und. @ high betax & 1        &         16  &         4   \\\hline        
Small Gap Undulators      & 2        &        16   &        4.5 \\\hline       
Ferrite Kickers           &   4      &       17    &       4   \\\hline        
Broad Band                &    1     &      6.8    &      11   \\\hline       
\end{tabular}
\end{table}

% \begin{figure}[!ht]
%  \centering
%  \includegraphics[width=0.95\textwidth]{figures/impedance_budget.png}
%  \caption{Impedance budget model for Sirius, with transverse impedance multiplied by the average beta functions in the location of each element. The kicker impedance seems discontinuous because it is negative at some frequencies.}
%  \label{fig:impedance_budget}
% \end{figure}

In the subsequent subsections we will estimate the main instabilities thresholds and strengths for each operation stage. The calculations are based on the solution of the linearized Vlasov equation for equally spaced and populated Gaussian bunches \cite{Chao1993, chin1985, cai2011}.

So far we have implemented four codes on Matlab\textregistered to calculate instabilities:
 \paragraph{Transverse Mode Coupling:} It is based on the development made by Chin \cite{chin1985} and it does basically what MOSES \cite{moses1986} does, except that:
 \begin{description}
  \item It handles arbitrary impedances;
  \item It handles the multi-bunch case;
  \item It cannot handle tune spread.
 \end{description}
To check the reliability of the code we compared it with MOSES for a broad band impedance for different chromaticities. Figure \ref{fig:comp_moses_matlab} shows the results.

% \begin{figure}[!ht]
%  \centering
%  \subfigure[\label{fig:tmci_chrom0}]{
%  \begin{minipage}{0.49\textwidth}
%   \includegraphics[width=\textwidth]{figures/tmci_matlab_chrom0.png}
%  \end{minipage}
%  \begin{minipage}{0.415\textwidth}
%   \includegraphics[width=\textwidth]{figures/tmci_moses_imag_chrom0.png}
% \includegraphics[width=\textwidth]{figures/tmci_moses_real_chrom0.png}
% \end{minipage}}
 
%  \subfigure[\label{fig:tmci_chrom01}]{
%  \begin{minipage}{0.49\textwidth}
%   \includegraphics[width=\textwidth]{figures/tmci_matlab_chrom01.png}
%  \end{minipage}
%  \begin{minipage}{0.415\textwidth}
%   \includegraphics[width=\textwidth]{figures/tmci_moses_imag_chrom01.png}  \includegraphics[width=\textwidth]{figures/tmci_moses_real_chrom01.png}
% \end{minipage}}
%  \caption{Comparison between MOSES (right) and our code (left) for:  \subref{fig:tmci_chrom0} zero chromaticity and \subref{fig:tmci_chrom01} 0.1 normalized chromaticity.}
%  \label{fig:comp_moses_matlab}
% \end{figure}

 \paragraph{Longitudinal Mode Coupling:} It does the same as the previous code, but for the longitudinal plane. We still have to systematically check this implementation with tracking codes, but preliminary results are positive.
 
 \paragraph{Transverse Coupled Bunch Instabilities:} Based on the effective impedance approximation \cite{Chao1993}. It were checked by comparisons with the transverse mode coupling code, since the results for coupled bunch instabilities are contained on it.
 
 \paragraph{Longitudinal Coupled Bunch Instabilities:} Idem to the previous one.
 
 
\section{Coupled Bunch Instabilities}\label{sec:coupled_bunch_instabilities}

It is important to know the coupled bunch instability threshold at zero chromaticity during the commissioning. In this phase the in-vacuum undulators' gap will be opened and we must calculate the instability without these elements. Figure \ref{fig:cbi_thres} shows the results. We notice that we can run the machine in the nominal current of the commissioning stage without the feedback system.

% \begin{figure}[!ht]
%  \centering
%  \subfigure[\label{fig:cbi_h_thres}]{\includegraphics[width=\textwidth]{figures/cbi_h_thres.png}}
%  \subfigure[\label{fig:cbi_v_thres}]{\includegraphics[width=\textwidth]{figures/cbi_v_thres.png}}
%  \caption{Coupled Bunch instability thresholds at zero chromaticity during commissioning for the:  \subref{fig:cbi_h_thres} Horizontal and \subref{fig:cbi_v_thres} Vertical Planes.}
%  \label{fig:cbi_thres}
% \end{figure}

Figure \ref{fig:cbi_h}/\ref{fig:cbi_v} shows the characterization of the coupled bunch instabilities for the horizontal/vertical plane for the operational modes of interest. We notice that both tunes being above integer contribute to diminish the estimated growth rates. This way, operating the ring at a normalized chromaticity of 0.05/0.1 in the horizontal/vertical plane, which corresponds to an unnormalized chromaticity of 2.3/1.4, we can avoid this kind of instabilities.

% \begin{figure}[H]
%  \centering
%  \subfigure[\label{fig:cbi_h_id}]{\includegraphics[width=\textwidth]{figures/cbi_h_id.png}}
%  \subfigure[\label{fig:cbi_h_idlan}]{\includegraphics[width=\textwidth]{figures/cbi_h_idlan.png}}
%  \subfigure[\label{fig:cbi_h_fc}]{\includegraphics[width=\textwidth]{figures/cbi_h_fc.png}}
%  \caption{Coupled Bunch instabilities for the horizontal plane at stages:  \subref{fig:cbi_h_id} IDs. \subref{fig:cbi_h_idlan} \hbox{IDs + Landau cavity}. \subref{fig:cbi_h_fc} Full current.}
%  \label{fig:cbi_h}
% \end{figure}

% \begin{figure}[H]
%  \centering
%  \subfigure[\label{fig:cbi_v_id}]{\includegraphics[width=\textwidth]{figures/cbi_v_id.png}}
%  \subfigure[\label{fig:cbi_v_idlan}]{\includegraphics[width=\textwidth]{figures/cbi_v_idlan.png}}
%  \subfigure[\label{fig:cbi_v_fc}]{\includegraphics[width=\textwidth]{figures/cbi_v_fc.png}}
%  \caption{Coupled Bunch instabilities for the vertical plane at:  \subref{fig:cbi_v_id} IDs stage. \subref{fig:cbi_v_idlan} IDs + Landau cavity stage. \subref{fig:cbi_v_fc} Full current stage.}
%  \label{fig:cbi_v}
% \end{figure}



\section{Single Bunch Instabilities}


In this section we will analyze the single bunch instabilities, they will be important not only for the characterization of the single bunch operational mode, but also to have a worst case scenario for the hybrid-filling pattern. Table \ref{tab:impedance_effect_sb} and Figures \ref{fig:lmci} and \ref{fig:tmci} show some results.

We notice that the broad band impedance overcomes all others, including the resistive wall for the three planes, being the transverse ones the most prejudiced. From this observation we can infer two things:
\begin{enumerate}
 \item The broad band used is over estimated and we need to vary some of the parameters, mainly the shunt resistance to obtain some specification for the ring impedance;
 \item The stability of the ring relies on the yet unknown, and we need to evaluate other element's impedances in order to have a more realistic view.
\end{enumerate}


\begin{table}[!ht]
 \centering
 \caption{Impedance Budget effects on the SB operational mode}
 \label{tab:impedance_effect_sb}
 \begin{tabular}{lccc}
\multirow{2}{*}{Element}&$\kappa_L$&$\beta_x \kappa_x$& $\beta_y \kappa_y$\\
 &[\si{\volt\per\pico\coulomb}] &[\si{\kilo\volt\per\pico\coulomb}] &     [\si{\kilo\volt\per\pico\coulomb}]\\\hline 
Wall With Coating           & 4          &-7.4    &   -12    \\\hline     
In-vac. Und. @ low Betax    & 0.28       &-2.7    &   -2     \\\hline
In-vac. Und. @ high Betax   & 0.053      &-1.2    &   -0.58  \\\hline     
Small Gap Undulators        & 0.084      &-0.5    &   -0.28  \\\hline     
Ferrite Kickers             & 0.24       &-0.29   &   -0.1   \\\hline     
Broad Band                  & 26         &-58     &   -94    \\\hline
Total                       & 30.4       &-69.9   &   -108   \\\hline
\end{tabular}
\end{table}


% \begin{figure}[H]
%  \centering
%  \subfigure[\label{fig:lmci_all-imped}]{\includegraphics[width=0.49\textwidth]{figures/lmci_all-imped.png}}
%  \subfigure[\label{fig:lmci_sem-bbr}]{\includegraphics[width=0.49\textwidth]{figures/lmci_sem-bbr.png}}
%  \caption{\subref{fig:lmci_all-imped} Longitudinal mode coupling instability with the whole impedance budget taken into account. \subref{fig:lmci_sem-bbr} Without broad band resonator.}
%  \label{fig:lmci}
% \end{figure}

% \begin{figure}[!t]
%  \centering
%  \subfigure[\label{fig:tmci_all-imped}]{\includegraphics[width=0.49\textwidth]{figures/tmci_h_all-imped.png} 
%  \includegraphics[width=0.49\textwidth]{figures/tmci_v_all-imped.png}}
%  \subfigure[\label{fig:tmci_sem-bbr}]{\includegraphics[width=0.49\textwidth]{figures/tmci_h_sem-bbr.png}
%  \includegraphics[width=0.49\textwidth]{figures/tmci_v_sem-bbr.png}}
%  \caption{\subref{fig:tmci_all-imped} Transverse mode coupling instability with the whole impedance budget taken into account. \subref{fig:tmci_sem-bbr} Without broad band resonator.}
%  \label{fig:tmci}
% \end{figure}


Based on the results of section \ref{sec:coupled_bunch_instabilities} we investigated the effect of positive chromaticity on single bunch instabilities. As can be noticed in figure \ref{fig:tmci_chrom} the instability is not completely damped, but its growth rates do not increase as fast as they would with zero chromaticity. 

It seems that with positive chromaticity it is easier for a conventional feedback system to control the beam, however, in order to understand better the behavior of this instability we need to look at the evolution of the distribution function calculated through the eigenvector decomposition.

% \begin{figure}
%  \centering
%  \subfigure[\label{fig:tmci_h_chrom}]{\includegraphics[width=0.49\textwidth]{figures/tmci_h_chrom.png}}
%  \subfigure[\label{fig:tmci_v_chrom}]{\includegraphics[width=0.49\textwidth]{figures/tmci_v_chrom.png}}
%  \caption{\subref{fig:tmci_h_chrom} Transverse mode coupling instability with the whole impedance budget taken into account. \subref{fig:tmci_v_chrom} Without broad band resonator.}
%  \label{fig:tmci_chrom}
% \end{figure}
