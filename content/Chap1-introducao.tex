\chapter{Introdução} 	\label{cap:intro}

Ao projetar um acelerador de partículas, primeiramente é feito um estudo da interação entre uma única partícula e os campos externos, gerados pelos elementos magnéticos e elétricos do acelerador. Nesse estudo, aspectos complexos como a dinâmica não linear, erros de campo, multipolos e erros de alinhamento são considerados.

Nos aceleradores atuais, como por exemplo os anéis de armazenamento de fontes de Luz Síncrotron, há uma necessidade crescente de se obter feixes cada vez mais intensos e colimados e menores nas direções transversais . Essas características fazem com que o estudo de outro tipo de interação seja crucial para o
\textit{design} de um acelerador. Essa interação é de origem coletiva, ou seja, está relacionada a efeitos que campos gerados pelas próprias partículas do feixe causam nas outras.

Efeitos coletivos geralmente são pequenos quando comparados com os dos campos externos e são tratados como uma perturbação. Contudo, conforme a intensidade do feixe aumenta, esses campos auto-induzidos se tornam cada vez mais intensos e podem gerar instabilidades coletivas, que fazem com que o feixe seja perdido
ou então sofra oscilações, coerentes ou incoerentes, que deturpam as características desejáveis para a radiação síncrotron gerada.

Desde Janeiro de 2011 está sendo desenvolvido um novo modo de operação para o anel de armazenamento de elétrons do LNLS, o UVX. O objetivo desse projeto é diminuir a emitância da máquina de 100~nm.rad para valores menores que 50~nm.rad.

Um modo de operação já foi desenvolvido teoricamente, simulado, e implementado no anel, como descrito em \cite{Fernando}. Testes com baixa corrente evidenciaram uma diminuição no tamanho horizontal do feixe compatível com o esperado teoricamente. Contudo, quando testado com alta corrente no anel (250~mA), o feixe apresentou uma redução de tamanho horizontal muito menor, além de demonstrar instabilidade e alto acoplamento transversal.

Tais indícios sugeriram que uma instabilidade coletiva tinha sido ativada devido ao aumento da densidade de elétrons no feixe. Tendo isso em mente, começamos a desenvolver um estudo de efeitos coletivos para determinar a instabilidade e tentar curá-la.

Além desse objetivo mais imediato, o estudo de efeitos coletivos também visa adquirir um conhecimento que será muito importante para o design da nova fonte de luz síncrotron do LNLS, o Sirius \cite{Sirius}, haja vista que esses efeitos são os principais limitantes de desempenho das fontes de terceira geração.


%%%%%%%%%%%%%%%%%%%%%%%%%%%%%%%%%%%%%%%%%%%%%%%%%%%%%%%%%%%%%%%%%%%%%%%%%%%%%%%%%%%%%%%%%%%%%%%%%%%%%%%%%
%%%%%%%%%%%%%%%%%%%%%%%%%%%%%%%%%%%%%%%%%%%%%%%%%%%%%%%%%%%%%%%%%%%%%%%%%%%%%%%%%%%%%%%%%%%%%%%%%%%%%%%%%
\section{Modelo do Acelerador}
Um anel de armazenamento de elétrons consiste em uma rede de elementos magnéticos e elétricos que confinam o movimento das partículas em uma órbita fechada para que elas realizem um número elevado de revoluções.

O estudo da dinâmica de interação entre as partículas armazenadas e os campos externos que condicionam seu movimento geralmente é feito em um sistema de coordenadas adequado que mede os desvios das trajetórias em relação à de uma partícula ideal, com energia nominal e posições transversais e longitudinal adequadas.

Esse estudo pode se tornar bastante complicado e por isso um modelo simplificado é usado quando abordamos o tema de instabilidades coletivas. Nesse modelo, apenas as características fundamentais da dinâmica são mantidas, para facilitar a interpretação física e a resolução matemática das equações.

Considera-se que o anel de armazenamento é circular, com raio $R = L/2\pi$, e que os elétrons sofrem oscilações harmônicas nas três direções: longitudinal, $s$, radial, $x$ e vertical $z$, de modo que a equação de movimento é dada por:
\begin{equation}\label{modelo}
 u'' + \omega_u^2 u = 0,
\end{equation}
onde $u = x, z, s$ e $\omega_u$ é a frequência fundamental de oscilação naquela direção, dada por:
\begin{equation}\nonumber
  \omega_u = \frac{\nu_u}{R},
\end{equation}
com $\nu_u$ sendo a sintonia bétatron para as oscilações transversais e síncrotron para as longitudinais.

Quando uma força, além daquelas provenientes dos elementos da rede, atua sobre as partículas, o movimento passa a ser forçado e \eqref{modelo} fica:
\begin{equation}\label{modeloforcado}
 u'' + \omega_u^2 u = \frac{F(t)}{m\gamma},
\end{equation}
supondo que a solução para a equação acima é oscilatória com frequência $\Omega \approx \omega_u$, veremos, por substituição na equação de movimento, que a variação na frequência de oscilação será dada por:
\begin{equation}
\Delta \omega_u \equiv \Omega - \omega_u \approx - \frac{F(t)}{2\omega_u \gamma m u_0}e^{i\Omega t}.
\end{equation}
onde a parte imaginária de $\Delta \omega$ representa um termo exponencial de amortecimento ou excitação no movimento oscilatório. Assim, podemos definir a taxa de crescimento da instabilidade como:
\begin{equation}
 \tau^{-1} \equiv \Im\{\Delta \omega_u\}
\end{equation}
onde taxas de crescimento positivas indicam a existência de instabilidade e taxas negativas implicam em maior amortecimento do feixe.

Os modelos de efeitos coletivos que serão discutidos terão a finalidade de descrever força $F(t)$, de forma que possamos analisar o desvio de frequência e determinar a ocorrência ou não da instabilidade.

Um modelo um pouco mais elaborado que o descrito aqui consideraria os mecanismos de amortecimento natural do feixe, de forma que a existência da instabilidade dependeria de sua taxa de crescimento ser maior ou menor que a taxa de amortecimento natural.

%%%%%%%%%%%%%%%%%%%%%%%%%%%%%%%%%%%%%%%%%%%%%%%%%%%%%%%%%%%%%%%%%%%%%%%%%%%%%%%%%%%%%%%%%%%%%%%%%%%%%%%%%
%%%%%%%%%%%%%%%%%%%%%%%%%%%%%%%%%%%%%%%%%%%%%%%%%%%%%%%%%%%%%%%%%%%%%%%%%%%%%%%%%%%%%%%%%%%%%%%%%%%%%%%%%
\section{Efeitos Coletivos}
A figura de mérito de uma Fonte de Luz Síncrotron é o brilho da luz gerada, que pode ser definido como:
\begin{equation}\label{eq:defbrilho}
 B=\frac{\dot{N}_\gamma}{4\pi^2\sigma_x\sigma_y\sigma_{x'}\sigma_{y'}\mathrm{d}E
/E_\gamma}\quad \left(\mathrm{\frac{\text{fótons} \cdot s^{-1}}{mm^2 \cdot
mrad^2 \cdot 0,1\% bandwidth}}\right)
\end{equation}
onde $\sigma_u$ e $\sigma_{u'}$, com $u= x, y$, são os desvios padrão das distribuições espaciais e angulares do feixe de elétrons, $\dot{N}_\gamma$ é o fluxo de elétrons integrado em um intervalo de 0.1\% de banda de energia, $\mathrm{d}E/E_\gamma$.

Assim, estamos interessados em determinar e evitar todos os efeitos coletivos que degradem o brilho da máquina, que podem ser
\begin{itemize}
 \item Oscilações transversais: que aumentam o tamanho e a divergência do feixe de elétrons.
 \item Oscilações longitudinais: que implicam em um aumento do espalhamento de energia, $\sigma_E$, dos elétrons e um consequente aumento do tamanho horizontal, além de alargamento das linhas de energia dos onduladores.
 \item Instabilidades: limitam a corrente máxima do anel, $I$, e consequentemente o fluxo de fótons, $\dot{N}_\gamma \sim I$.
\end{itemize}

Os efeitos coletivos se manifestam de duas maneiras: a primeira é por meio da interação eletromagnética direta entre as partículas, denominada efeito das cargas espaciais (\textit{space charge} em Inglês) e a outra é através de \textit{Wake Fields}, que são o resultado da interação dos campos eletromagnéticos gerados pelos elétrons com as paredes mais próximas e atuam em partículas subsequentes \cite{Khan}.

O efeito das cargas espaciais é o mais conhecido dentre os efeitos coletivos, mas é pequeno para anéis de armazenamento de elétrons, porque ele decresce com o aumento da energia relativística $\gamma$. Tal propriedade pode ser percebida calculando a força de atração que uma linha de corrente com densidade de carga $\lambda$ uniforme, se movendo com velocidade $\beta c$ exerce sobre uma carga $q$  a uma distância $r$ viajando paralelamente com mesma velocidade:
\begin{equation}
 F= q(E - \beta B)=\frac{q\lambda}{2\pi \epsilon_0 \gamma^2 r},
\end{equation}
onde fica evidente o comportamento decrescente da força de interação com o aumento da energia.

Por outro lado, os \textit{Wake Fields} não dependem da energia do feixe, mas sim da geometria da câmara de vácuo e das características dos materiais que a constituem.

Um dos tipos de \textit{Wake Fields} mais conhecidos, que possui uma solução analítica simples de ser obtida é o da parede resistiva. Este problema consiste em uma câmara de vácuo cilíndrica de raio $b$, infinitamente espessa e longa, formada por um material de condutividade $\sigma$. A uma distância $a<b$ do seu centro há uma carga $q$ viajando longitudinalmente com velocidade $c$ e a uma distância $z$ atrás dela há outra carga $q$, também viajando com velocidade $c$.

A resolução desse problema envolve expandir as densidades de carga e de corrente que geram os \textit{Wake Fields} em termos de anéis de carga  concêntricos ao eixo de simetria da câmara de vácuo:
\begin{align}
\rho(r,\theta,s,t)  = \sum_{n=0}^\infty \rho_m &=  \sum_{m=0}^\infty
\frac{I_m}{(1+\delta_{0m})\pi a^{m+1}} \delta(r-a) \delta(s-ct) \cos(m\theta)\\
\vec{j}(r,\theta,s,t) = \sum_{m=0}^\infty\vec{j}_m &=\sum_{m=0}^\infty c \rho_m
\hat{s}
\end{align}
onde $I_m = q a^m$ é o momento de multipolo de ordem $m$ da partícula. Usar essa base para expansão da carga facilita bastante os cálculos porque o problema se resume a resolver as Equações de Maxwell para a m-ésima componente dos campos.

Como a câmara de vácuo não é perfeitamente condutora, há campos não nulos nessa região também. Nesse caso, as fontes são dadas por:
\begin{equation}
 \rho = 0, \qquad \vec{j} = \sigma \vec{E}.
\end{equation}

Após escrever as Equações de Maxwell em coordenadas cilíndricas e substituir as densidades de carga e corrente nas expressões, é possível determinar a dependência em $\theta$ dos campos. Também, como há simetria longitudinal, os campos não dependem da posição relativa ao anel $s$, mas apenas da distância da partícula fonte, $z=s-ct$, de modo que $z>0$ corresponde a posições a frente da fonte.

Devido a essa propriedade, que acopla o tempo e a coordanada longitudinal, é interessante fazer uma Transformada de Fourier em $z$ para transformar a equação diferencial parcial em $r,t$ e $s$ em uma ordinária apenas em $r$. Assim, escrevemos:
\begin{align}
 (E_s,E_r,B_\theta)(r,\theta,z) &= \cos(m\theta)
\int_{-\infty}^\infty\!\!\frac{dk}{2\pi}(\tilde{E}_s,\tilde{E}_r,\tilde{B} _\theta)(r,k),\\
 (B_s,B_r,E_\theta)(r,\theta,z) &= \sin(m\theta)
\int_{-\infty}^\infty\!\!\frac{dk}{2\pi}(\tilde{B}_s,\tilde{B}_r,\tilde{E} _\theta)(r,k).
\end{align}

Aplicando as condições de contorno de que o campo não pode divergir em $r=0$ e de continuidade dos campos tangenciais na parede da câmara, obtém-se as expressões para todas as regiões do espaço. Dessa forma, é possível determinar a força que atua em uma partícula teste a uma distância $z$ atrás da fonte:
\begin{align}\nonumber \label{eq:res.wall}
(F_\lVert)_m = \frac{eI_m}{\pi b^{2m+1}(1+\delta_{0m})}
\sqrt{\frac{c}{\sigma}}r^m \cos(m\theta) \frac{1}{|z|^{3/2}}\\
(\vec{F}_\perp)_m = \frac{2eI_m}{\pi b^{2m+1}} \sqrt{\frac{c}{\sigma}}
m r^{m-1} \frac{1}{|z|^{1/2}} \left(\hat{r}\cos(m\theta) -\hat{\theta}
\sin(m\theta)\right).
\end{align}
A solução acima é uma aproximação válida para uma região
\begin{displaymath}
\eta^{1/3}b\ll|z|\ll \frac{b}{\eta}, \qquad \eta \equiv \frac{c}{4\pi \sigma b},
\end{displaymath}
sendo que para uma câmara de vácuo de alumínio com $5 cm$ de raio, temos que $6\times10^{-6} m\ll |z| \ll 3\times10^7 m$. A solução exata, assim como a resolução detalhada desse problema pode ser encontrada em \cite{Chao}.

A simplicidade de \eqref{eq:res.wall} e facilidade na resolução são consequências das simetrias longitudinal e axial do problema original, que permitiram o desacoplamento dos multipolos e a dependência apenas em $|z|$. Quando uma dessas simetrias é quebrada, a determinação dos campos e das forças que atuam sobre o feixe fica muito complicada, de modo que aproximações devem ser feitas.

Uma dessas aproximações é a de feixe rígido, que já foi usada implicitamente na solução da parede resistiva. Ela consiste em não considerar o efeito causado pela ação do \textit{Wake Field} no feixe, ou seja, tanto as posições transversais como a longitudinal das partículas do feixe são fixas. Apesar dessa aproximação não ser auto-consistente e não prever o surgimento de instabilidades coletivas, ela é valida para anéis de armazenamento de elétrons, em que $\beta \approx 1$, e permite obter expressões  que são usadas posteriormente em modelos de instabilidades.

%%%%%%%%%%%%%%%%%%%%%%%%%%%%%%%%%%%%%%%%%%%%%%%%%%%%%%%%%%%%%%%%%%%%%%%%%%%%%%%%%%%%%%%%%%%%%%%%%%%%%%%%%
\section{\textit{Wake Functions}}

Em geral, para seções da câmara de vácuo que não possuem simetria translacional, o \textit{Wake Field} e a \textit{Wake Force} que atua sobre uma partícula prova passam a depender das variáveis $s$ e $t$ separadamente e as equações se tornam muito complicadas para serem resolvidas.

Para tentar recuperar a dependência apenas de $z = s - \beta c t$, ao invés de tentarmos determinar a força instantânea que partícula sente, vamos olhar para o impulso recebido quando ela passa por toda a seção geradora do \textit{wake field}. Matematicamente, usando a aproximação de feixe rígido, temos:

\begin{equation}
\beta c \Delta\vec{ p}(x,y,z) = \int^{L/2}_{-L/2}\!\! ds
\vec{F}(x,y,s,(s-z)/\beta c)
\end{equation}
onde o $L/2$ nos limites de integração não representa o comprimento da seção que gera o \textit{Wake Field} mas sim um comprimento adequado para calcular a integral de modo que os campos elétrico e magnético nesse ponto sejam iguais, ou seja, não sofram mais o efeito da assimetria gerada pela estrutura. Por exemplo, se a estrutura em questão é periódica, o limite de integração pode ser o período da estrutura, se é do tipo cavidade, o limite de integração deve ser muito maior que o tamanho da cavidade.

As duas aproximações feitas acima (feixe rígido e limites de integração) impõem restrições sobre o momento que a partícula recebe dos \textit{Wake Field}. Tais restrições são evidenciadas pelo Teorema de \textit{Panofsky-Wenzel} \cite{Bio}:
\begin{equation}
 \vec{\nabla} \times \Delta\vec{ p} = \vec{0}
\quad \Rightarrow \left\lbrace 
\begin{array}{l}
(\vec{\nabla} \times \Delta \vec{p})_s = 0
\xrightarrow{coord.~cartesianas} \frac{\partial \Delta p_x}{\partial y}
= \frac{\partial \Delta p_y}{\partial x} \\
\frac{\partial \Delta \vec{ p}_\bot}{\partial z} =
\vec{\nabla}_\bot \Delta p_s
\end{array} \right.
\end{equation}
onde $\Delta\vec{ p}$ é o momento linear ganho por uma partícula de prova a uma distância $z$ atrás da fonte.

Na demonstração do teorema acima são usadas apenas as equações de Maxwell, a força de Lorentz e as duas aproximações. Portanto, ele não depende das condições de contorno e nem da fonte, além de não exigir que $\beta = 1$ (apenas que ele seja alto o suficiente para a aproximação de feixe rígido ser válida).

Há um corolário do Teorema de \textit{Panofsky-Wenzel} que afirma:
\begin{equation}
 \beta = 1\quad \Rightarrow \quad \vec{\nabla}_\bot \cdot
\Delta\vec{ p}_\bot = 0.
\end{equation}

Devido a sua generalidade e poder de simplificação, esse teorema é fundamental para o estudo de efeitos coletivos. Uma aplicação direta pode ser obtida o escrevendo em coordenadas cilíndricas:
\begin{align}
 \frac{\partial }{\partial r} (r\Delta p_\theta)& = \frac{\partial}{\partial
\theta}(\Delta p_r) \\
 \frac{\partial}{\partial z}(\Delta p_\theta) &= \frac{1}{r}
\frac{\partial}{\partial \theta}(\Delta p_s) \\
 \frac{\partial}{\partial z}(\Delta p_r) &= \frac{\partial}{\partial r}(\Delta
p_s) \\
(\beta = 1) \quad  \frac{\partial}{\partial r}(r\Delta p_r) &=
\frac{\partial}{\partial \theta}(\Delta p_\theta).
\end{align}

As equações acima podem ser facilmente resolvidas de modo que a solução pode ser convenientemente escrita da seguinte forma:
\begin{align}
\label{eq:W}
 \beta c \Delta \vec{p}_\bot &= -q \sum_{m=0}^\infty I_m W_m(z) m
r^{m-1}\left(\hat{r}\cos(m\theta) - \hat{\theta}\sin(m\theta)\right)\\
\label{eq:W'}
\beta c \Delta p_s &= -q \sum_{m=0}^\infty I_m W'_m(z) r^{m}\cos(m\theta)
\end{align}
onde $I_m$ é o m-ésimo multipolo da distribuição de carga\footnote{para uma carga pontual a uma distância $a$ do eixo de simetria $I_m=qa^m$.} que gera o \textit{Wake Field} e $q$ é a carga da partícula de prova. As funções $W_m$ e $W'_m$, que estão relacionadas entre si por
\begin{equation}
 W'_m(z)=\frac{d}{dz}W_m(z),
\end{equation}
são chamadas \textit{Wake Functions} longitudinal e transversal, respectivamente, e são determinadas pelas condições de contorno do problema, devendo ser resolvidas independentemente.

Apesar de a forma exata das \textit{Wake Functions} depender do problema específico a ser tratado, é possível inferir algumas de suas propriedades gerais, assim como algumas propriedades do impulso.

A primeira propriedade é devida à causalidade: $W_m(z) = 0$ e $W'_m(z)=0$ para $z>0$, ou seja, uma partícula que está a frente do feixe não sofre ação do \textit{Wake Field} gerado por ele. Apesar dessa característica ser restrita para $\beta = 1$ ela é sempre aplicada, por ser uma aproximação válida em anéis
reais.

Outras propriedades envolvem o fato delas serem reais e possuírem dimensão $[W'_k] = V/C/m^{2k}$ e $[W_k] = V/C/m^{2k-1}$, sendo que essa última propriedade é importante porque, para problemas com simetria rotacional, a m-ésima componente do impulso transversal é escalada por $(a/b)^{2m-1}$ e do longitudinal por $(a/b)^{2m}$, onde $a$ é o tamanho transversal do feixe e $b$ o raio da câmara de vácuo. Como $a\ll b$, os modos mais baixos dominam, de forma que a maioria das instabilidades longitudinais são geradas por $W'_0$ e a maioria das instabilidades transversais por $W_1$\footnote{Visto que $W_0$ não tem sentido físico por $(\Delta \vec{p}_\bot)_0 = \vec{0}$ (ver \eqref{eq:W}).} \cite{Chao, Bio}.

Ainda, por argumentos de conservação de energia \cite{Chao, Bio} é possível demonstrar que $ W'_m(0^-)>0$ e decresce conforme $|z|$ aumenta, pois uma partícula imediatamente atrás de uma outra deve sofrer uma força retardadora, caso contrário o sistema ganharia energia indefinidamente. Isso implica, pelo Teorema de \textit{Panofsky-Wenzel}, que $W_m(z)$ começa em $0$ (não demonstrável, mas a maioria das \textit{Wake Functions} seguem essa regra) e cresce negativamente e monotonicamente com $|z|$, para $|z|$ pequeno. Essa análise permite concluir que em anéis onde predominam \textit{Wake Fields} transversais, feixes curtos são preferíveis e em máquinas onde os longitudinais dominam feixes longos são melhores.

Uma característica importante da transferência de momento linear que podemos tirar de \eqref{eq:W'} ocorre quando $m=0$, ou seja, no caso da força monopolar. Nesse caso, $\Delta p_s$ é independente das coordenadas transversais da partícula prova, de modo que todas as partículas em um determinado plano sentem a mesma força, proporcional à $W'_0$. Esse tipo de impulso pode levar a um auto-empacotamento do feixe ou então à \textit{microwave instability}.

Ainda, para $m=1$ (dipolo), o impulso transversal é constante com as coordenadas transversais, enquanto o impulso longitudinal depende linearmente da distância $x$ (considerando $\hat{x}$ a direção em que a carga geradora foi posta), de modo que partículas em lados opostos do eixo sofrem impulsos opostos. Esse tipo
de característica distorce o feixe em um formato de banana e pode levar à instabilidade chamada \textit{Beam Breakup} em que o feixe é distorcido até atingir a câmara de vácuo.

\subsection{Impedâncias}

Assim como foi feito na resolução da \textit{resistive wall} é interessante descrever os \textit{wake fields} no domínio da frequência, pois o tratamento analítico das equações geralmente é simplificado quando estamos nesse espaço, além do que a interpretação das equações também se torna mais simples nesse caso, como é o caso das estruturas ressonantes, como a cavidade de RF, que permitem que apenas campos com frequências específicas ressoem por um longo tempo, e possam agir sobre o feixe, gerando instabilidades.

Desse modo, podemos definir impedância como a Transformada de Fourier das
\textit{Wake Function}:
\begin{align}
Z^\lVert_m (\omega)&=\int^\infty_\infty\!\!\frac{dz}{c}e^{-i\omega z/c}W'_m(z)\\
Z^\perp_m (\omega)&= i\int^\infty_\infty \!\! \frac{dz}{c}e^{-i\omega z/c}W_m(z)
\end{align}
onde $Z^\lVert_m$ é a impedância longitudinal e $Z^\perp_m$ a impedância transversal.

Algumas propriedades gerais da impedância são dadas abaixo:
\begin{itemize}
 \item $Z_m^\lVert(-\omega) = [ Z_m^\lVert(-\omega)]^*$ e $Z_m^\perp(\omega) = - [Z_m^\perp(\omega)]^*$ : devido ao fato de $W_m(z)$ ser real;
\item $Z_m^\Vert(\omega) = \frac{\omega}{c}Z_m^\perp(\omega)$, para coordenadas cilíndricas: devido ao Teorema de \textit{Panofsky-Wenzel};
\item $\Re\{Z_m^\Vert(\omega)\}\geq 0$ e $\Re\{Z_m^\perp(\omega)\}\geq 0$, se $\omega\geq 0$: porque a energia do feixe não pode aumentar sem a existência de forças externas.
\end{itemize}

A interpretação física das impedâncias é análoga à da impedância de um circuito elétrico. Para ilustrar a validade dessa analogia, vejamos o seguinte exemplo: considere um feixe com momento m-polar que gera um \textit{Wake Field} dado por
\begin{equation}
 P_m = \hat{P}_m e^{-i\omega(t-s/v)} \nonumber
\end{equation}
e uma partícula com carga $q$ a uma distância $z$ da fonte. O potencial elétrico longitudinal sentido por essa partícula será (ver\eqref{eq:W'}) \cite{Bio}:
\begin{equation}\label{eq:pot.imp}
\hat{V} =-\frac{\hat{P}_m I_m}{q} Z_m^\lVert
\end{equation}
onde $I_m$ é o m-ésimo multipolo da partícula teste. Para a componente monopolar \eqref{eq:pot.imp} se reduz à $V = Z^\lVert_0 P_0$, que é idêntica à expressão para um circuito elétrico.

Em referência a essa analogia, impedâncias de cavidades, que possuem picos de ressonância para determinadas frequências são modeladas como um circuito RLC em paralelo:
\begin{equation}
 Z_m^\lVert(\omega)= \frac{R_s}{1+iQ
\left(\frac{\omega_R}{\omega}-\frac{\omega}{\omega_R}\right)}
\end{equation}
onde $\omega_R = 1/\sqrt{LC}$ é a frequência de ressonância e $Q=R_s \sqrt{C/L}$ é o fator de qualidade da cavidade. Mesmo em casos em que a geometria não é do tipo estrutura ressonante, é comum classificar as impedâncias em capacitiva ($\Im\{Z_m^\lVert\}>0$) ou indutiva ($\Im\{Z_m^\lVert\}<0$), dependo do sinal de sua parte imaginária \cite{Chao}.

Outra propriedade da impedância está ligada a perda de energia. Existe três meios pelos quais um feixe pode perder energia em um anel de armazenamento: por emissão de radiação, colisão com gases residuais e por \textit{Wake Fields}; sendo que essa última é chamada de perda parasita.

Para mostrar a relação da perda parasita com a impedância, consideremos que o feixe possui uma distribuição de carga 
\begin{equation}\nonumber
\rho(\tau)= e N\lambda(\tau),\qquad \text{com} \quad\int_{-T_0/2}^{T_0/2} \!\! d \tau \lambda(\tau) = 1,
\end{equation}
onde $\tau$ é o avanço das partículas em relação à síncrona, $N$ é o número de partículas no feixe, $e$ é a carga elétrica de cada partícula e $T_0$ é o período de revolução no anel. Segundo \eqref{eq:W'}, a energia ganha por uma partícula do feixe devido a interação com os \textit{Wake Field} gerados por toda a distribuição em uma volta é
\begin{equation}\nonumber
\Delta \epsilon(\tau) = -e^2 N \int_{-T_0/2}^{\tau} \!\! d\tau'
W'_0(\tau-\tau')\lambda(\tau') = -e^2 N \omega_0\sum_{n=-\infty}^{\infty}
Z_0^\lVert(n\omega_0) \tilde{\lambda}_n e^{in\omega_0\tau}
\end{equation}
onde na última igualdade foi usado o Teorema da Convolução, e
\begin{displaymath}
 \tilde{\lambda}_n =
\frac{1}{2\pi}\int_{-T_0/2}^{T_0/2}\!\!d\tau\lambda(\tau)e^{-in\omega_0\tau}
\end{displaymath}
com $\omega_0 = 2\pi/T_0$. Assim, a energia média ganha por partícula por volta é
\begin{equation}\nonumber
\overline{\Delta \epsilon} = \int_{-T_0/2}^{T_0/2}\!\! d\tau \lambda(\tau)
\Delta \epsilon(\tau) = -e^2 N\omega_0 \sum_{n=-\infty}^{\infty}
Z_0^\lVert(n\omega_0) |\tilde{\lambda}_n|^2.
\end{equation}
Contudo, como $\lambda(\tau)$ é real, $|\tilde{\lambda}_n|^2$ é par; e como $ Z_0^\lVert(-\omega) = [Z_0^\lVert(\omega)]^*$, apenas a parte real da impedância contribui para a perda de energia:
\begin{equation}\label{eq:energiaperdida}
\overline{\Delta \epsilon} = -e^2 N\omega_0 \sum_{n=-\infty}^{\infty}
\Re\{Z_0^\lVert(n\omega_0)\} |\tilde{\lambda}_n|^2.
\end{equation}

Analisando \eqref{eq:energiaperdida} fica mais evidente a analogia da impedância com àquela de circuitos elétricos, pois apenas a parte real contribui com a dissipação de energia pelo feixe, enquanto a parte imaginária vai gerar alterações na fase deste. Também, nota-se que a energia perdida por partícula cresce linearmente com o número de elétrons no feixe.

%%%%%%%%%%%%%%%%%%%%%%%%%%%%%%%%%%%%%%%%%%%%%%%%%%%%%%%%%%%%%%%%%%%%%%%%%%%%%%%%%%%%%%%%%%%%%%%%%%%%%%%%%
%%%%%%%%%%%%%%%%%%%%%%%%%%%%%%%%%%%%%%%%%%%%%%%%%%%%%%%%%%%%%%%%%%%%%%%%%%%%%%%%%%%%%%%%%%%%%%%%%%%%%%%%%
\section{Modelos de Instabilidades Coletivas}

Podemos classificar os \textit{Wake Fields} de acordo com seu alcance. Campos que ressoam e decaem lentamente, correspondentes a bandas finas de impedância, são de longo alcance, enquanto aqueles que decaem rápido, correspondentes a bandas largas de impedância, são chamados de campos de curto alcance.

Considerando essa classificação,as instabilidades coletivas também podem ser dividas em duas classes:
\begin{itemize}
 \item Instabilidades tipo Robinson ou inter-pacotes: resultantes da ação de \textit{Wake Fields} de longo alcance, geram oscilações dos pacotes, com amplitude proporcional à corrente do feixe. Apesar de existirem para todos os valores de corrente, na prática elas são visíveis para correntes nas quais a taxa de crescimento das amplitudes de oscilação é maior que a dos mecanismos de amortecimento natural do feixe. 
 \item Instabilidades de modo acoplado ou intra-pacotes: são fenômenos que relacionam partículas de um mesmo pacote. Neste caso, a instabilidade ocorre acima de um determinado valor de corrente do feixe.
\end{itemize}

A solução exata do comportamento dos elétrons estocados, sujeitos a forças de interação entre si, só é possível através da resolução da equação de Vlasov \cite{Chao, Bio}, que considera uma distribuição contínua de partículas no espaço de fase. Todavia, modelos de macro-partículas, que consideram  que os
pacotes do feixe consistem em poucas partículas interagindo, geralmente uma ou duas, são fáceis de resolver, fornecem resultados satisfatórios para a maioria dos casos e possibilitam uma interpretação física dos fenômenos envolvidos. 

%%%%%%%%%%%%%%%%%%%%%%%%%%%%%%%%%%%%%%%%%%%%%%%%%%%%%%%%%%%%%%%%%%%%%%%%%%%%%%%%%%%%%%%%%%%%%%%%%%%%%%%%%
\subsection{Modelos de uma partícula}

Considera-se que cada pacote do feixe é formado por apenas uma macro-partícula, não apresentando estrutura interna. Esses modelos são bons para descrever as instabilidades tipo Robinson, pois a consideração feita é válida quando as partículas de um mesmo pacote sofrem oscilações coerentes. Tendo isso em mente, vamos analisar as direções longitudinal e transversal separadamente.

%%%%%%%%%%%%%%%%%%%%%%%%%%%%%%%%%%%%%%%%%%%%%%%%%%%%%%%%%%%%%%%%%%%%%%%%%%%%%%%%%%%%%%%%%%%%%%%%%%%%%%%%%
\subsubsection{Longitudinal}

Primeiramente, consideremos que apenas um pacote esteja circulando no anel. Consideremos também que apenas a \textit{Wake Function} monopolar ($m=0$) contribui. Para esse caso, a força que atua sobre o pacote é dada por \cite{Chao}:
\begin{equation}\label{eq:1partforcalongit}
 \frac{F(t)}{z_0} \approx \frac{N^2 e^2 \eta}{z_0cT_0}\sum^\infty_{k=-\infty}
 W'_0 (kT_0) + (z(t-kT_0)-z(t))(W'_0)'(kT_0)
\end{equation}
onde $\eta$ é aproximadamente o fator compactação de momento da rede magnética, $N$ é o número de partículas no pacote e a soma foi estendida até infinito pela propriedade de causalidade da \textit{Wake Function}. Desse modo, a variação da frequência é dada por:
\begin{equation}
 \Delta \omega = -\frac{Ne^2 \eta}{2\omega_s\gamma m_e c T_0}
\sum^\infty_{k=-\infty} \left(e^{i\Omega k T_0}-1\right)(W'_0)'(kT_0).
\end{equation}
onde nota-se que o primeiro termo de \eqref{eq:1partforcalongit} não foi considerado, por se tratar de um efeito estático, que apenas desloca a posição de equilíbrio do feixe.

Em termos da impedância, a taxa de crescimento da instabilidade pode ser escrita da seguinte maneira:
\begin{equation}\label{long1b}
 \tau^{-1}=\frac{Ne^2 \eta}{2\omega_s E T_0^2} \sum^\infty_{p=-\infty}
(p\omega_0+\omega_s)\Re \{Z^\lVert_0(p\omega_0+\omega_s)\}
\end{equation}

Em um anel de armazenamento, grande parte da impedância longitudinal é oriunda da cavidade de radio-frequência, ou seja, é do tipo banda fina. Para esse tipo de impedância, nota-se que apenas dois termos da somatória, os valores positivo e negativo de $p$ que mais se aproximam da ressonância, contribuem significativamente para a taxa de crescimento, ver Figura \ref{fig:narrow}, de modo que \eqref{long1b} pode ser aproximada para: 
\begin{equation} \label{narrowband}
 \tau^{-1}=\frac{Ne^2 \eta}{2\omega_s E T_0^2}|p|\omega_0
\Re\{-Z^\lVert_0(-|p|\omega_0 +\omega_s) + Z^\lVert_0(|p|\omega_0 +\omega_s)\}.
\end{equation}
Assim, se o pico da impedância for maior que $p\omega$ a soma será positiva e haverá instabilidade, caso contrário haverá amortecimento.

% \begin{figure}[!t]
% \center
%  \includegraphics[scale=0.5]{Imagens/narrowband.png}
%  \caption{Ilustração de \eqref{long1b} aplicada a uma impedância de banda fina, em cinza. Cada ponto corresponde a um termo da somatória. Nesse exemplo, a soma dos termos é positiva e, consequentemente a taxa de crescimento. (retirado de \cite{Khan}).}
%  \label{fig:narrow}
% \end{figure}

Especificamente para o caso da cavidade RF, sabemos que seu modo fundamental está próximo de $h\omega$. Assim, se o harmônico fundamental da cavidade for um pouco maior que o produto acima, haverá uma instabilidade longitudinal. Essa instabilidade foi a primeira a ser estudada e é chamada de Instabilidade de Robinson.

Considerando que $h$ pacotes compõem o feixe, eles oscilarão em um de seus auto-modos, em que a diferença de fase entre sucessivos pacotes é dada por $2\pi\mu/h$, com $\mu=0,...,h-1$. assim, as taxas de crescimento devem ser calculadas para cada modo:
\begin{equation}
\tau^{-1}_\mu=\frac{h N e^2 \eta}{2\omega_s E T_0^2}
\sum^\infty_{p=-\infty} (ph\omega_0+\mu\omega_0+\omega_s)
\Re\{Z^\lVert_0(ph\omega_0+\mu\omega_0+\omega_s)\}
\end{equation}

Esses modos podem interagir, por exemplo, com os HOMs das cavidades de radio-frequência e produzir instabilidades que vão ou não surgir, dependendo do auto-modo de ressonância do feixe.

%%%%%%%%%%%%%%%%%%%%%%%%%%%%%%%%%%%%%%%%%%%%%%%%%%%%%%%%%%%%%%%%%%%%%%%%%%%%%%%%%%%%%%%%%%%%%%%%%%%%%%%%%
\subsection{Transversal}

Considerando novamente a situação \textit{single-bunch} e que apenas a \textit{wake function} transversal dipolar ($m=1$) contribui, a força que atua sobre o pacote será:
\begin{equation}\label{eq:trans1b}
 \frac{F(t)}{y_0}=\frac{N^2 e^2}{y_0 c T_0} \sum^\infty_{k=-\infty}
y(t-kT_0)W_1(kT_0)
\end{equation}
o que implica em uma variação de frequência dada por:
\begin{equation}
 \Delta \omega = - i\frac{Ne^2}{2\omega_\beta \gamma m_e c T^2_0}
\sum^\infty_{p=-\infty} Z^\perp_1 (p \omega_0 + \Omega)
\end{equation}
de modo que a taxa de crescimento é:
\begin{equation}\label{resistive}
\tau^{-1} = -\frac{Ne^2 c}{2\omega_\beta E T^2_0} \sum^\infty_{p=-\infty}
\Re\{Z^\perp_ 1 (p \omega_0 + \omega_\beta)\}
\end{equation}

Para a expressão acima, é importante analisar dois tipos de impedâncias. As de banda fina, que podem gerar uma versão transversal da instabilidade de Robinson, e as de parede resistiva, que além de causarem instabilidades entre-pacotes, cobrem um longo campo de frequências, caindo com $1/\sqrt{|\omega|}$, o que exige a consideração de vários termos na soma para obter a taxa de crescimento com precisão (ver Figura \ref{fig:resistive}).

% \begin{figure}[!h]
% \center
%  \includegraphics[scale=0.5]{Imagens/resistive.png}
%  \caption{Ilustração de \eqref{resistive} assumindo uma impedância de parede resistiva, em cinza. Cada ponto corresponde a um termo da somatória. Nesse exemplo, a soma dos termos é negativa e a taxa de crescimento positiva.  (retirado de \cite{Khan}).}
%  \label{fig:resistive}
% \end{figure}

Para o caso da impedância de parede resistiva, a soma de termos positivos e negativos correspondentes, leva a conclusão de que para valores de sintonia cuja parte fracionária é menor que $0.5$ o movimento é estável, enquanto valores maiores são instáveis \cite{Khan}.

De modo similar à direção longitudinal, $h$ pacotes equidistantes oscilarão em seus auto-modos. Também, de acordo com \eqref{eq:trans1b}, a taxa de crescimento transversal é dada por:
\begin{equation}
\tau^{-1}_\mu = \frac{h Ne^2 c}{2\omega_\beta E T^2_0} \sum^\infty_{p=-\infty}
\Re\{Z^\perp_ 1 (p h\omega_0 +\mu\omega_0 + \omega_\beta)\}.
\end{equation}

Neste caso, a taxa de crescimento é sempre positiva para impedâncias de parede resistiva, ou seja, sempre haverá instabilidade de parede resistiva para modos de operação \textit{multi-bunch}. Ainda, simulações mostram que o sistema tente a evoluir para o modo de oscilação que maximiza a taxa de crescimento da instabilidade \cite{Khan}.

%%%%%%%%%%%%%%%%%%%%%%%%%%%%%%%%%%%%%%%%%%%%%%%%%%%%%%%%%%%%%%%%%%%%%%%%%%%%%%%%%%%%%%%%%%%%%%%%%%%%%%%%%
\subsection{Modelos de poucas partículas}

Neste tipo de modelos, considera-se que os pacotes são formados por duas ou mais partículas distribuídas longitudinalmente que interagem entre si. São aplicados, principalmente, no estudo de instabilidades do tipo intra-pacotes, geradas por \textit{wake fields} de curto alcance. Aqui serão considerados apenas modelos de duas macro-partículas, uma na parte da frente do pacote, chamada de cabeça , e outra na parte de trás, chamada cauda.

O movimento transversal acoplado dessas duas macro-partículas é modelado da seguinte forma:

\begin{align}
 \ddot{y}_1(t)+\omega^2_\beta(\delta_E)y_1(t)&=0& \\
 \ddot{y}_2(t)+\omega^2_\beta(\delta_E)y_2(t)&=\frac{Ne^2 W_1}{2\gamma c
T_0m_e}y_1.&
\end{align}
onde $y_1$ se refere à cabeça e $W_1$ é assumido constante no comprimento do pacote. As equações acima são válidas para a primeira metade do movimento síncrotron, sendo que na segunda metade ($t>T_s/2$) as partículas trocam de posição e as equações também devem ter seus índices trocados.

A variação da sintonia bétatron com a energia pode ser expressa por:
\begin{equation}
 \omega_\beta(\delta_E)=\omega_\beta(0)+\omega_0 \xi \delta_E =
\omega_\beta(0)+\omega_0 \xi \frac{\omega_s\hat{z}}{c \eta} \cos(\omega_st)
\end{equation}
onde $\omega_s$ é a frequência síncrotron e $\xi$ é a cromaticidade.

Resolvendo o sistema de equações, chega-se a:
\begin{equation}
\left(\begin{array}{c}\tilde{y}_1 \\ \tilde{y}_2 \end{array} \right)_{t=T_s/2}
=  e^{-i\omega_\beta T_s/2} 
\left(\begin{array}{cc} 1 & 0 \\ ia & 1 \end{array}\right)
\left(\begin{array}{c}\tilde{y}_1 \\ \tilde{y}_2 \end{array} \right)_{t=0}
\end{equation}
onde
\begin{equation}
 a = \frac{\pi N e~2 W_1}{4\gamma c T_0 m_e \omega_\beta \omega_s}\left(
1+i\frac{4 \xi \omega_0 \hat{z}}{\pi c \eta}\right).
\end{equation}

Após uma revolução síncrotron completa teremos:
\begin{equation}
\left(\begin{array}{c}\tilde{y}_1 \\ \tilde{y}_2 \end{array} \right)_{t=T_s}
=  e^{-i\omega_\beta T_s} 
\left(\begin{array}{cc} 1-a^2 & ia \\ ia & 1 \end{array}\right)
\left(\begin{array}{c}\tilde{y}_1 \\ \tilde{y}_2 \end{array} \right)_{t=0}.
\end{equation}

Os autovalores da matriz de transferência acima são dados por:
\begin{align}
 \lambda &= e^{\pm i\phi}\quad \text{com} \quad \cos(\phi) \equiv
1-\frac{a^2}{2}, \quad \text{se} \quad a\leq 2 \\
 \lambda &= e^{\pm i\mu}\quad \text{com} \quad \cosh(\mu) \equiv \frac{a^2}{2}
- 1, \quad \text{se} \quad a > 2.
\end{align}

Com esses resultados é possível analisar qualitativamente dois tipos de instabilidades. A primeira é chamada de efeito \textit{head-tail} (cabeça-cauda em português). No caso em que $|a| << 1$, que é válido para fontes de radiação síncrotron operando no modo \textit{multi-bunch}, a aproximação $\phi \approx a$ é válida. Os auto-vetores correspondentes aos autovalores determinados acima descrevem dois modos de oscilação das macro-partículas com diferenças de fases relativas de 0 e $\pi$.

Considerando $\eta$ positivo, vemos que se $\xi<0$ o modo 0 cresce com uma taxa $\Im\{a\}$ enquanto o modo $\pi$ é amortecido com a mesma taxa. Daí vem a necessidade de introduzir sextupolos nos anéis de armazenamento. O objetivo é manter a cromaticidade nula ou com um valor positivo pequeno, para amortecer o modo 0. Nesse caso, o modo $\pi$ seria excitado, contudo o modelo de duas partículas superestima sua taxa de crescimento, sendo que na prática ela é pequena em comparação com a do modo 0.

Nota-se que a instabilidade \textit{head-tail} ocorre para todos os valores de corrente. Ainda, aplicando-se um alto valor positivo de cromaticidade é possível amortecer oscilações de multi-pacotes, a custa de uma abertura dinâmica e tempo de vida reduzidos.

A outra instabilidade é conhecida por vários nomes na literatura:  \textit{head-tail turbulence, strong head-tail instability, transverse microwave instability, transverse mode coupling}. Quando a condição $|a| << 1$ não é mais válida, o movimento não pode ser mais visto como dois modos separados, de modo que para $\xi = 0$ a amplitude se mantém confinada, se $a<2$. Contudo, para valores maiores que esse, há instabilidade até mesmo com a cromaticidade nula e as taxas de crescimento são da ordem do amortecimento natural por emissão de radiação síncrotron \cite{Khan}.

%%%%%%%%%%%%%%%%%%%%%%%%%%%%%%%%%%%%%%%%%%%%%%%%%%%%%%%%%%%%%%%%%%%%%%%%%%%%%%%%%%%%%%%%%%%%%%%%%%%%%%%%%
%%%%%%%%%%%%%%%%%%%%%%%%%%%%%%%%%%%%%%%%%%%%%%%%%%%%%%%%%%%%%%%%%%%%%%%%%%%%%%%%%%%%%%%%%%%%%%%%%%%%%%%%%
\section{Medidas contra Efeitos Coletivos}

Após ter feito uma análise dos mecanismos que geram instabilidades coleticas em feixes não-contínuos, é possível identificar algumas medidas úteis para evitar esse tipo de problema em anéis de armazenamento.

Uma delas é a minimização da impedância da câmara de vácuo. Como foi estudado, a impedância é determinada pela geometria e condutividade da parede da câmara de vácuo. 

Em anéis de armazenamento há muitos locais em que a câmara é interrompida ou então sofre alterações em sua seção transversal. Geralmente a impedância introduzida por essas descontinuidades pode ser consideravelmente minimizadas se os seguintes critérios forem obedecidos \cite{Khan}:
\begin{itemize}
 \item Interrupções da câmara, como em flanges ou \textit{bellows} devem possuir pontes metálicas para que a corrente imagem flua;
 \item Mudanças inevitáveis no formato da câmara, como em \textit{tapers} nos finais de onduladores, devem ser projetados com ângulos pequenos (tipicamente da ordem de \SI{10}{\degree} ou menos);
 \item Buracos ou fendas na câmara, por exemplo portas de bombas de vácuo, portas de radiação síncrotron, ou o canal de injeção do feixe, devem ser projetados de modo que uma parede lisa fique mais perto do feixe que a interrupção.
\end{itemize}

No caso de cavidades de radio-frequência a maior contribuição para a impedância vem dos modos harmônicos de ordem mais altas, chamados HOMs (\textit{higher-order modes}). Para combater esses efeitos, geralmente emprega-se absorvedores de frequência de bandas largas ou então antenas para extrair modos particulares \cite{YellowCERN95}. Além desses métodos, muito esforço é investido no desenvolvimento de geometrias que possuam HOMs menos intensos.

Para as impedâncias de parede resistiva, geralmente as medidas mais comuns a se tomar são fazer câmaras largas e com material de boa condutividade elétrica. Se a seção transversal tiver que ser pequena, como no caso dos onduladores, a escolha do material da câmara é de importância primordial \cite{Khan}.

As medidas discutidas até agora não se aplicam para os casos em que tem-se uma dada câmara de vácuo, ou seja, uma máquina que era estável, mas devido a alguma mudança no modo operação ou em algum elemento do anel, assim como na corrente estocada, passou a demonstrar instabilidades. Para esses casos, as variáveis
que podem ser alteradas são reduzidas, pois parâmetros como circunferência, energia do feixe, perdas por radiação e frequências síncrotron e bétatron geralmente são dedicadas a outras necessidades.

Um modo eficiente de suprimir instabilidades é por meio do aumento da cromaticidade da máquina, que desloca o espectro do feixe. Contudo, esse método envolve aumentar a força dos sextupolos em regiões dispersivas e, consequentemente, diminuir a abertura dinâmica e o tempo de vida \cite{Khan}.

O efeito de impedâncias de banda fina podem ser minimizados alterando sua frequência, para diminuir a interação com o espectro do feixe. Para cavidades de RF, isso pode ser feito deformando-as mecanicamente, por meio do aumento de sua temperatura, por exemplo \cite{112Khan}. Todavia, deve-se fazer essas alterações tendo sempre mais de um grau de liberdade, para que apenas os HOMs sejam alterados, mantendo o modo fundamental constante.

\textit{Landau damping} é um tipo de amortecimento gerado pelos componentes não lineares dos potenciais definidos pelos campos magnéticos guia e pela voltagem RF. Nas direções transversais ele é introduzido por sextupolos, e na longitudinal pela forma senoidal do potencial. Um amortecimento adicional pode ser introduzido na direção longitudinal adicionando uma voltagem de RF com um múltiplo do modo fundamental, por meio das cavidades Landau \cite{114Khan}.

Além desses modos passivos de amortecimento das instabilidades, há os ativos. Entre eles, estão inclusos a modulação de fase da cavidade RF e o uso de  sistemas de \textit{Feedback}. O primeiro método já é empregado no UVX desde 2003, quando foi instalada uma nova cavidade de RF no anel, para a instalação de dispositivos de inserção, que possuia um HOM muito elevado (ver \cite{rfUVX}).

Os Sistemas de \textit{Feedback}, cujo princípio de funcionamento será discutido na seção \ref{feedback}, são muito empregados nas máquinas atuais para amortecimento de instabilidades, estando presentes na maioria das máquinas de terceira geração. Contudo, eles são úteis em apenas em uma banda estreita de taxas de crescimento, podendo não ser capazes de controlar taxas duas ordens de grandeza maiores que as de amortecimento do feixe \cite{Khan}, além de combaterem apenas instabilidades do tipo Robison. Por isso, os métodos passivos continuam sendo muito importantes para a estabilidade do feixe.