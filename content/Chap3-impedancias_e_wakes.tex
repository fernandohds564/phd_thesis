\chapter{Wakes and Impedances}\label{cap:wake_impedances}

%%%%%%%%%%%%%%%%%%%%%%%%%%%%%%%%%%%%%%%%%%%%%%%%%%%%%%%%%%%%%%%%%%%%%%%%%%%%%%%%%%%%%%%%%%%%%%%%%%%%%%%%%
%%%%%%%%%%%%%%%%%%%%%%%%%%%%%%%%%%%%%%%%%%%%%%%%%%%%%%%%%%%%%%%%%%%%%%%%%%%%%%%%%%%%%%%%%%%%%%%%%%%%%%%%%
\section{\engw{Wake Fields}}

In this section we are going to introduce the concept of wake field in particle accelerators and try to understand its foundations and main properties for the subsequent analysis of its influence over the movement of the charged particles in the accelerator. There are several approaches in the literature to explain this subject, the most appreciated by the author is the one presented in reference \cite{Stupakov2000a}, which will be reproduced in parts here.

%%%%%%%%%%%%%%%%%%%%%%%%%%%%%%%%%%%%%%%%%%%%%%%%%%%%%%%%%%%%%%%%%%%%%%%%%%%%%%%%%%%%%%%%%%%%%%%%%%%%%%%%%
\subsection{Interaction Mechanisms}\label{ssec:interation_mechanisms}

Intuitively, we tend to think the direct interaction between charged particles, such as electric repulsion, is the responsible for the collective effects observed in storage rings, however, as we will see in the subsequent analysis, this is not the main mechanism for ultra-relativistic particles.

\begin{figure}[hb!]
\centering
\label{fig:wake1}
\begin{tikzpicture}[scale=1]
\def\d{1cm}
\draw[<->] (1,0) node[below]{$z$} 
		-- ++(-\d,0) node[below left] {$S$} 
        -- ++(0,\d) node[left] {$\rho$}; %coord sys S
\draw[<->] (6*\d,0) node[below]{$z'$}
		-- ++(-\d,0) node[below left] {$S'$} 
        -- ++(0,\d) node[left] {$\rho'$}; % coord sys S'
\coordinate (V) at (0.5,0);
\coordinate (Q1) at (4cm,1.5cm);
\coordinate (Q2) at (0.5cm,2.5cm);
\draw[->] (5*\d,0.5*\d) -- ++(V) node[above] {$\boldsymbol{v}$};
\filldraw[fill=black] (Q1) circle[radius=0.05] node[above] {$q$}; % source particle
\draw[->] (Q1) -- ++(V) node[above] {$\boldsymbol{v}$}; % velocity vector
\filldraw[fill=black] (Q2)  circle[radius=0.05]; % test particle
\draw[->] (Q2) -- ++(V) node[above] {$\boldsymbol{v}$}; % velocity vector
\draw[dashed,|-|] ($(Q1)-(0,0.2)$)
				   let \p1 = ($(Q2) - (Q1)$)
                   in -- ++(\x1,0) node[midway,below] {$s$}; %horizontal distance
\draw[dashed,|-|] ($(Q2)-(0.2,0)$)
				   let \p1 = ($(Q1) - (Q2)$)
                   in -- ++(0,\y1) node[midway,left] {$\rho$}; % vertical distance
\draw[->] (Q1) -- ++($(Q2) - (Q1)$) node[midway,above] {$\boldsymbol{R}$}; %vector
\end{tikzpicture}
\caption{Duas partículas interagindo via campo direto.}
\end{figure}

To see this lets consider the interaction of a source particle $Q$ moving with velocity $\vect{v}=v{\vect{\hat{z}}}$ with a witness particle $q$ moving with the same velocity (parallel path) at a distance $s$ in the direction parallel to the movement and at a transverse distance $\rho$, as shown in Figure\,\ref{fig:wake1}. We want to determine the force that the source particle exerts on the witness particle. One way to do this is by calculating the electric field of the source particle in the co-moving frame of reference, $S'$, and Lorentz transforming it back to the laboratory's frame. After the math we obtain:

\begin{align}
 \label{eq:fields_free_particle}
 \vect{E} = \frac{q}{4\pi\epsilon_0}\frac{\vect{R}}{\gamma^2 R^{*3}}, & & \vect{B} = \frac{1}{c^2}\vect{v} \times \vect{E} 
\end{align}
where $\vect{R}$ is the vector which connects both particles, going from the source to the witness and  $R^{*2} = s^2 + x^2/\gamma^2$, e $\gamma = 1/(1-v^2/c^2)$.

Combining equation\,\ref{eq:fields_free_particle} with the Lorentz force, we get the longitudinal and transverse force over the witness particle:
\begin{align}\label{eq:space_charge_force}
 F_l &= E_z = -\frac{q}{4\pi\epsilon_0}\frac{s}{\gamma^2\left(s^2+x^2/\gamma^2\right)^{3/2}}, \\
 F_t &= E_x - vB_y = -\frac{q}{4\pi\epsilon_0}\frac{x}{\gamma^4\left(s^2+x^2/\gamma^2\right)^{3/2}}
\end{align}

In accelerator physics the force $\vect{F}$ is known as space charge force. We can infer from equation\,\ref{eq:space_charge_force} that for any position $s$ and $x$, the longitudinal force is proportional to $\gamma^{-2}$ and $F_t \sim \gamma^{-4}$ if $s \gg x/\gamma$ and  $F_t \sim \gamma^{-1}$ if $s \approx 0$. This way, in the ultra-relativistic limit, $\gamma \to \infty$, the electromagnetic interaction between particles moving parallel to each other in free space is zero. It is easy to show that in this limit, if the movement of the particles is not parallel, there is an interaction force only for $s=0$, but, as their speed is the same, this situation can only happen for an infinitesimal time.


In this work we are interested in the the interaction between particles in the ultra-relativistic limit, $v \to c$. The space charge effects discussed above are despicable in this limit and the interaction between the particles is due to the presence of the walls of the vaccum chamber. Note that taking the limit $v \to c$ in the equation\,\ref{eq:fields_free_particle} and remembering that $s = vt - z$, we can write the electromagnetic field of a ultra-relativistic charge as
\begin{align}
\vect{E} = \frac{q}{2\pi\epsilon_0}\frac{\vect{\hat{r}}}{r}\delta(z-ct), & & \vect{B} =\frac{1}{c}\vect{\hat{z}}\times\vect{E},
\end{align}
where $\vect{r} = \vect{\hat{x}}x + \vect{\hat{y}}y$ is a bidimensional vector in cylindrical coordinates ($\vect{\hat{x}}$ and $\vect{\hat{y}}$ are unit vectors in the $x$ e $y$ directions, respectively). The equations above show that the field is pancake-like and follow the beam as it travels througth the empty space. It is important to notice that this solution is steady-state, it was necessary an infinite amount of time before $t$ to build it and that's why there is no causal paradoxes in it.

\begin{figure}[hb!]
\centering
\label{fig:wake2}
\begin{tikzpicture}
\draw[very thick] (0,0) -- ++(10,0) (0,4) -- ++(10,0); %vacuum chamber
\draw[dashed] (0,2) -- ++(10,0); %eixo de simetria
\coordinate (V) at (0.5,0);
\coordinate (Q1) at (4cm,2.2cm);
\coordinate (Q2) at (0.5cm,2.5cm);
\filldraw[fill=black] (Q1) circle[radius=0.05] node[left] {$q$}; % source particle
\draw[->] (Q1) -- ++(V) node[above] {$\boldsymbol{v}$}; % velocity vector
\draw[-{Stealth[length=10pt]}] (Q1) let \p1 = (Q1) in --(\x1,4);
\draw[-{Stealth[length=10pt]}] (Q1) let \p1 = (Q1) in --(\x1,0) node[midway,right] {$\boldsymbol{E}$};
\draw ($(Q1)+(0,1)$) circle[radius=0.2] node[right=0.2] {$\boldsymbol{B}$};
\filldraw ($(Q1)+(0,1)$) circle[radius=0.07];
\end{tikzpicture}
\caption{Particles interacting in a perfectly conducting cylindrical tube.}
\end{figure}

Lets consider now a pipe with cylindrical symmetry\footnote{do not confuse cylindrical symmetry with cylinder. By cylindrical symmetry we mean a system with translational symmetry in one direction.}, hollow and with absolute vacuum in its interior, made of a perfect electric conductor material and with arbitrary cross section. If we put the particles of the previous example inside this pipe, moving parallel to symmetry axis, they will induce image charges in the surface of the wall which cancel the electromagnetic field inside the metal. 

The image charges travel with the same velocity $\vect{v}$ of the particles (see Figure\,\ref{fig:wake2}). As they move in parallel paths with constant velocity, in the limit $v \to c$, according to the previous results, they do not interact, independently of how close they are from each other.

From this analysis we conclude that the interaction between particles in the ultra-relativistic limit can occur only for two reasons:
\begin{itemize}
    \item The wall is not perfectly conducting, or
    \item The pipe does not have cylindrical symmetry (which generally is due to the presence of RF cavities, flanges, bellows, beam position monitors, vacuum pumps, among other elements in the vacuum chamber of an accelerator).
\end{itemize}

%%%%%%%%%%%%%%%%%%%%%%%%%%%%%%%%%%%%%%%%%%%%%%%%%%%%%%%%%%%%%%%%%%%%%%%%%%%%%%%%%%%%%%%%%%%%%%%%%%%%%%%%%
\subsection{Causality and catch up distance}

If one particle moves in a straight line at light speed, the electromagnetic field scattered by the discontinuities of the chamber will not catch up with it and will not affect the charges travelling ahead of it. The field will only interact with the charges moving behind of the source particle. Such property is known as causality.

Even though this property is not strictly true in the real world, because particles always travel at speeds lower than the light's, it is true in most practical cases, as we will se bellow.

Lets try to calculate the distance $z$ where the field generated by some discontinuity in the vacuum chamber will catch up with a witness particle at a distance $s$ behind the source particle. At the time $t=0$ the source particle passes through the discontinuity and an electromagnetic wave is generated with its wave front travelling at the speed of light in all directions, forming a sphere of radius $R$, see FigureXX. At any given time after this, the following relation holds:
\begin{align}
ct = R \quad vt = z && \Rightarrow && R = \frac{z}{\beta} \quad \text{where} \,\, \beta = \frac{v}{c}
\end{align}
where $z$ is the distance travelled by the source particle. Besides that, at the specific time when the wake catchs up with the witness particle, the following relation is valid:
\begin{align}
R^2 = b^2 + (z-s)^2 && \Rightarrow &&
z^2(\frac{1}{\beta^2}-1) + 2sz - (b^2 + s^2) = 0 && \Rightarrow \\
z = -\gamma^2 \beta^2 s + \sqrt{s^2\gamma^4\beta^4 + \gamma^2\beta^2\left(b^2 + s^2\right)} && = \gamma^2 \beta^2 s\left(-1 + \sqrt{1 + \frac{1}{\gamma^2\beta^2}\left(1 + \frac{b^2}{s^2}\right)}\right) &&
\end{align}
where $b$ is the distance from the discontinuity to the trajectory of the particles and $\gamma = 1/\sqrt{1-\beta^2}$ is the relativistic energy. FigureXX shows a graphic of this function, normalized by the distance $b$. We notice that, for $s=0$, which means the field catching up with the source particle, $z = \gamma\beta b$. For the case of Sirius, $\gamma \approx 5870$ and $\beta \approx 1$, if $ b = 2$mm, $z \approx 12$m.

\begin{figure}[hb!]
\centering
\label{fig:catch_up}
\begin{tikzpicture}
\draw[very thick] (0,0) -- ++(10,0) (0,4) -- ++(10,0); %vacuum chamber
\draw[very thick] (4.8,4) to [out=-90,in=0] (5,3.8) to [out=180,in=-90] (5.2,4);
\draw[dashed] (0,2) -- ++(10,0); %eixo de simetria
\coordinate (V) at (0.5,0);
\coordinate (Q1) at (4cm,2.2cm);
\coordinate (Q2) at (0.5cm,2.5cm);
\filldraw[fill=black] (Q1) circle[radius=0.05] node[left] {$q$}; % source particle
\draw[->] (Q1) -- ++(V) node[above] {$\boldsymbol{v}$}; % velocity vector
\draw[-{Stealth[length=10pt]}] (Q1) let \p1 = (Q1) in --(\x1,4);
\draw[-{Stealth[length=10pt]}] (Q1) let \p1 = (Q1) in --(\x1,0) node[midway,right] {$\boldsymbol{E}$};
\draw ($(Q1)+(0,1)$) circle[radius=0.2] node[right=0.2] {$\boldsymbol{B}$};
\filldraw ($(Q1)+(0,1)$) circle[radius=0.07];
\end{tikzpicture}
\caption{The catch up distance.}
\end{figure}


%%%%%%%%%%%%%%%%%%%%%%%%%%%%%%%%%%%%%%%%%%%%%%%%%%%%%%%%%%%%%%%%%%%%%%%%%%%%%%%%%%%%%%%%%%%%%%%%%%%%%%%%%
\subsection{Definição de Wake}\label{ssec:wake_definition}

The Electromagnetic interaction of charged particles with the environment generally has a small effect when compared with the effect of the guiding electric and magnetic fields of the accelerators components and can be treated as a perturbation. In an zeroth order approximation we can assume the beam moves with constant speed in a straight line and solve Maxwell equations 
A interação eletromagnética de partículas carregadas com o ambiente normalmente tem um efeito pequeno quando comparado com o efeito de campos elétricos e magnéticos externos dos aceleradores e pode ser considerada com uma perturbação. Em uma aproximação de ordem zero, podemos assumir que o feixe se move com velocidade constante em uma linha reta, e então resolvemos as equações de Maxwell, encontramos os campos e computamos o efeitos desses campos no movimento das partículas. Com essa abordagem negligenciamos efeitos de segunda ordem porque o movimento em uma órbita perturbada podem gerar apenas uma pequena mudança nos campos computados pela aproximação de ordem zero. Essas correções são geralmente pequenas, especialmente para partículas ultra-relativísticas.

Outra importante característica da interação entre o campo eletromagnético gerado e as partículas é que em muitos casos de importância prática eles estão localizados em uma região pequena comparada com o tamanho da órbita do feixe. Ela também ocorre uma escala de tempo muito menor que a de oscilação do feixe no acelerador (como os períodos bétatron e síncrotron). Isso nos permite considerar essa interação dentro da aproximação de impulso e caracterizá-la pelo momento transferido para a partícula.

Dessa forma, podemos introduzir a noção de \engw{wake} da seguinte maneira. Considere a partícula 1, com carga $q$ se movendo ao longo do eixo $z$ com uma velocidade próxima à da luz, $v\approx c$, de modo que $z=ct$ (veja a figura \ref{fig:4}). Uma partícula 2 com carga unitária se move paralelamente à partícula 1, com a mesma velocidade, a uma distância $s$ com deslocamento transversal $\vect{\rho}$ relativo ao eixo $z$. O vetor $\vect{\rho}$ é um vetor bi-dimensional perpendicular ao eixo $z$, $\vect{\rho} = (x,y)$. Apesar de as partículas viajarem no vácuo, há contornos materiais no problema que espalham o campo eletromagnético que gera uma interação entre as partículas.

Assumindo que as equações de Maxwell foram resolvidas e que os campos gerados pela partícula 1 foram encontrados, podemos calcular a mudança no momento $\Delta \vect{p}$ da segunda partícula causada por esse campo como uma função do deslocamento $\vect{\rho}$ e da distância $s$,

\begin{equation}
 \Delta \vect{p}(\vect{\rho},s) = \defint{t}{\left[\vect{E}(\vect{\rho},z,t) + \vect{\hat{z}}\times \vect{B}(\vect{\rho},z,t)\right]_{z=ct-s}}{-\infty}{\infty}.
\end{equation}

Note que a integral é feita sobre uma linha reta --- a órbita não perturbada da segunda partícula. Os limites de integração são estendidos de menos para mais infinito assumindo que a integral converge.

Como a dinâmica do feixe é diferente nas direções longitudinal e transversal, é útil separar o momento longitudinal $\Delta p_z$ da componente transversal $\vect{\Delta p}_\perp$. Dessa forma, com uma convenção de sinal e um fator de normalização $c/q$, podemos definir as chamadas \engw{wake functions}, ou simplesmente \engw{wakes} longitudinal e transversal,

\begin{equation}\label{eq:wake_definition}\begin{aligned}
    w_l(\vect{\rho},s) &= -\frac{c}{q} \Delta p_z = -\frac{c}{q} \udefint{t}{E_z|_{z=ct-s}}, \\
    \vect{w}_t(\vect{\rho},s) &= \frac{c}{q} \vect{\Delta p}_\perp = \frac{c}{q} \udefint{t}{\left[\vect{E}_\perp + \vect{\hat{z}}\times \vect{B}\right]_{z=ct-s}}
\end{aligned}\end{equation}

Note o sinal de menos na definição de $w_l$ --- ele é introduzido para que um \engw{wake} longitudinal positivo corresponda a uma perda de energia da partícula teste (caso ambas a partícula fonte e teste tenham o mesmo sinal de carga). Os \engw{wakes} definidos tem dimensão de \si{\volt\per\coulomb} no Sistema Internacional de Unidades.

Por causa do princípio de causalidade o \engw{wakefield} não se propaga a frente da partícula fonte, então
\begin{equation}
    w_l(\vect{\rho},s) = 0, \qquad \vect{w}_t(\vect{\rho},s) = \vect{0}, \qquad \mathrm{para } \quad s < 0.
\end{equation}

Na definição acima foi assumido que o campo eletromagnético estava localizado no espaço e no tempo e que a integral na equação \ref{eq:wake_definition} converge. Contudo, há casos em que isso não é verdade e a fonte do \engw{wake} é distribuida uniformemente em um longo caminho, como é o caso do \engw{wake} de parede resistiva de uma câmara de vácuo. Nesse caso é mais conveniente introduzir o \engw{wake} por unidade de comprimento, descartando a integração na equação \ref{eq:wake_definition}:

\begin{equation}\begin{aligned}
    w_l(\vect{\rho},s) &= -\frac1q E_z|_{z=ct-s}, \\
    \vect{w}_t(\vect{\rho},s) &= \frac1q\left[\vect{E}_\perp + \vect{\hat{z}}\times \vect{B}\right]_{z=ct-s}.
\end{aligned}\end{equation}

Nessa definição os \engw{wakes} adquirem uma dimensão adicional de inverso de comprimento e tem dimensão \si{\volt\per\coulomb\per\meter} no Sistema Internacional de Unidades.

%%%%%%%%%%%%%%%%%%%%%%%%%%%%%%%%%%%%%%%%%%%%%%%%%%%%%%%%%%%%%%%%%%%%%%%%%%%%%%%%%%%%%%%%%%%%%%%%%%%%%%%%%
\subsection{Teorema de Panofsky-Wenzel}

Várias relações gerais entre os \engw{wakes} longitudinal e transversal podem ser obtidas das equações de Maxwell sem que seja necessário especificar as condições de contorno para os campos.

Vamos introduzir o vetor $\vect{R} =(\vect{\rho},-s)$ (o sinal de menos na frente do $s$ é devido ao fato de $s$ ser positivo para posições atrás da partícula fonte) e considerar o momento $\vect{\Delta p}$ na equação \ref{eq:wake_definition} como uma função de $\vect{R}$. Vamos assumir que o campo eletromagnético é especificado através do potencial vetor $\vect{A}(\vect{r},t)$ e o potencial escalar $\phi(\vect{r},t)$, e computar
$\vect{\Delta p}$ para os dados campos. É conveniente usar a formulação Lagrangiana para as equações de movimento,

\begin{equation}\label{eq:euler_lagrange}
    \dertot{}{t}\derpar{L}{\vect{v}} = \derpar{L}{\vect{r}} = \vect{\nabla}L,
\end{equation}
com a Lagrangiana para a partícula teste com carga unitária é
\begin{equation}\label{eq:lagrangiana_charged_part}
   L = -mc^2 \sqrt{1 - \frac{v^2}{c^2}} + \frac1c \vect{Av} - \phi
\end{equation}
substituindo a equação \ref{eq:lagrangiana_charged_part} na equação \ref{eq:euler_lagrange} obtemos:

\begin{equation}
 \dertot{}{t} \left(\vect{p} + \frac1c \vect{A}\right) = \vect{\nabla}\left(\frac1c \vect{Av} - \phi\right).
\end{equation}
onde $\vect{p} = m\gamma\vect{v}$.

Agora, integrando esta equação ao longo da órbita da partícula teste, $x=\mathrm{const}$, $y=\mathrm{const}$ e $z = ct-s$, e assumindo que os campos $\vect{A}$ e $\phi$ vão a zero no infinito, encontramos

\begin{equation}
 \vect{\Delta p}(\vect{R}) = \udefint{t}{\vect{\nabla}\left(\frac1c\vect{Av} - \phi \right) = \frac{q}{c} \vect{\nabla_R}W(\mathrm{R})}.
\end{equation}
onde introduzimos o \engw{wake potential} $W$,
\begin{equation}
 W(\vect{R}) = \frac{c}{q}\udefint{t}{\left(\frac1c \vect{Av} -\phi\right)}
      \overset{\vect{v} \approx c\vect{\hat{z}}}{=}
                 \frac{c}{q}\udefint{t}{\left(A_z -\phi\right)}.
\end{equation}

Assim, provamos uma relação que estabelece que todas as três componentes do vetor $\vect{\Delta p}$ podem ser obtidas derivando uma única função escalar $W$. Relembrando a relação entre os componentes de $\vect{\Delta p}$ e os \engw{wakes}, \ref{eq:wake_definition}, descobrimos que

\begin{equation}\label{eq:wake_function_definition}
 w_l = - \derpar{W}{(-s)} = \derpar{W}{s},\qquad \vect{w}_l = \vect{\nabla_\rho}W,
\end{equation}
e, consequentemente
\begin{equation}\label{eq:panofsky_wenzel_theorem}
 \derpar{\vect{w}_t}{s} = \vect{\nabla_\rho}w_l.
\end{equation}

Esta relação é comumente chamada de teorema de Panofsky-Wenzel. Note que $\nabla_\rho$ é um gradiente bidimensional com respeito às coordenadas $x$ e $y$.

Uma das aplicações computacionais mais importantes do teorema de Panofsky-Wenzel é que o conhecimento do \engw{wake function}, $w_l$, longitudinal nos permite encontrar o \engw{wake} transversal,$\vect{w}_t$ por meio de uma integração simples da equação \ref{eq:panofsky_wenzel_theorem}.

Outra propriedade importante de $W$ é que ele é uma função harmônica das variáveis $x$ e $y$,
\begin{equation}\label{eq:wake_potential_harmonic}
 \Delta_\perp W \equiv \derpar[2]{W}{x} + \derpar[2]{W}{y} = 0.
\end{equation}
Para provar isso vamos usar o fato que ambos $\vect{A}$ e $\phi$ satisfazem a equação de onda no espaço livre, $(\partial^2/\partial t^2 - c^2 \Delta)\vect{A} = \vect{0}$ e $(\partial^2/\partial t^2 - c^2 \Delta)\phi = 0$. Dessa forma,

\begin{equation}\begin{aligned}
0
&=\frac{c}{q}\udefint{t}{\left(\derpar[2]{}{t}-c^2\Delta\right)\!\!(A_z-\phi)}\\
&=\frac{c}{q}\left[\udefint{t}{\left(\derpar[2]{}{t} - c^2\derpar[2]{}{z}\right)} -
	            c^2\udefint{t}{\left(\derpar[2]{}{x} + \derpar[2]{}{y}\right)}\right]\!\!(A_z-\phi)\\
&=\frac{c}{q}\udefint{t}{\left(\derpar{}{t}+c\derpar{}{z}\right)
				     \!\!\left(\derpar{}{t}-c\derpar{}{z}\right)\!\!(A_z -\phi)}
   -c^2\!\left(\derpar[2]{W}{x} + \derpar[2]{W}{y}\right)
\end{aligned}\end{equation}

A última integral nessa equação é nula porque
\begin{equation}
    \derpar{}{t} + c\derpar{}{z} \approx \derpar{}{t} + \vect{v\nabla} = \dertot{}{t}
\end{equation}
e
\begin{equation}
  \udefint{t}{\left(\derpar{}{t} + c\derpar{}{z}\right)\!\!
             \left(\derpar{}{t} - c\derpar{}{z}\right)\!\!(A_z - \phi)}
  = \udefint{t}{\dertot{}{t}\!\!\left(\derpar{}{t}-c\derpar{}{z}\right)\!\!(A_z - \phi)}
  =  0.
\end{equation}

%%%%%%%%%%%%%%%%%%%%%%%%%%%%%%%%%%%%%%%%%%%%%%%%%%%%%%%%%%%%%%%%%%%%%%%%%%%%%%%%%%%%%%%%%%%%%%%%%%%%%%%%%
\subsection{Sistemas Com Um Eixo de Simetria}
Na seção \ref{ssec:wake_definition} nós definimos o \engw{wake} como uma função do deslocamento da partícula teste relativo ao caminho da partícula fonte. Em aplicações práticas nós também estamos interessados em saber como o \engw{wake} depende da trajetória da partícula fonte. Assumiremos que o sistema em consideração tem um eixo de simetria e vamos escolhe-lo como o eixo $z$ do sistema de coordenadas, veja \ref{fig:5}. Agora a partícula fonte, $1$, se move na direção $z$ com um deslocamento dado pelo vetor $\vect{\rho'}$, e a partícula teste viaja paralelamente à partícula fonte, com a mesma velocidade, a uma distância $s$ atrás da fonte com um deslocamento $\vect{\rho}$ relativo ao eixo. Os vetores $\vect{\rho'}$ e $\vect{\rho}$ são os vetores bidimensionais perpendiculares ao eixo $z$. O \engw{wake} ainda é definido pela equação \eqref{eq:wake_definition} mas agora ele será considerado como uma função de $\vect{\rho'}$, $\vect{\rho}$ e $s$
\begin{equation}\begin{aligned}
w_l &= w_l(\vect{\rho},\vect{\rho'},s), \\
\vect{w}_t &= \vect{w}_t(\vect{\rho},\vect{\rho'},s).
\end{aligned}\end{equation}

Normalmente a câmara de vácuo é projetada de forma que o eixo do sistema serve como uma órbita ideal para o feixe. Assim, desvios desse eixo são relativamente pequenos e ambos os vetores $\vect{\rho'}$ e $\vect{\rho}$ são tipicamente muito menores que a câmara de vácuo, de forma que em $w_l$ podemos negligenciá-los e introduzir um \engw{wake} longitudinal que só depende de $s$,

\begin{equation}
	w_l(s) = w_l(\vect{0},\vect{0},s).
\end{equation}

Se os elementos que formam a câmara de vácuo também tem alguma simetria (por exemplo, se eles tem seção transversal circular, elíptica ou retangular), o \engw{wake} transverso no eixo, onde $(\vect{\rho},\vect{\rho'}) = (\vect{0},\vect{0})$, é nulo, $\vect{w}_t(\vect{0},\vect{0},s)=\vect{0}$. Para pequenos valores de $(\vect{\rho},\vect{\rho'})$ nós podemos expandir $\vect{w}_t(\vect{\rho},\vect{\rho'},s)$ mantendo apenas os termos lineares. Dessa forma, obtemos uma relação tensorial entre os \engw{wakes} transversos e os deslocamentos,

\begin{equation}
	\vect{w}_t(\vect{\rho},\vect{\rho'},s) = \overleftrightarrow{\vect{W}_1}(s)\vect{\rho} +
    										 \overleftrightarrow{\vect{W}_2}(s)\vect{\rho'},
\end{equation}
onde $\overleftrightarrow{\vect{W}_1}$ e $\overleftrightarrow{\vect{W}_2}$ são tensores bidimensionais de ordem 2. 

\subsection{Sistemas com Simetria axial}

Em um sistema com simetria axial, o \engw{wake potential}, $W$, depende apenas dos módulos de $\vect{\rho}$ e $\vect{\rho'}$ e do ângulo, $\theta$ entre eles. Sempre podemos escolher um sistema de coordenadas tal que o vetor $\vect{\rho'}$ fique no plano $xz$, veja a figura \ref{fig:6}, de forma que $W$ será uma função periódica e par \todo{entender porque é par} do ângulo $\theta$ em um sistema de coordenadas cilíndrico. Decompondo $W$ em séries de Fourier em $\theta$ temos:

\begin{equation}
	W(\rho,\rho',\theta,s) = \sum_{m=0}^\infty W_m (\rho,\rho',s) \cos(m\theta).
\end{equation}

Inserindo essa equação na equação \eqref{eq:wake_potential_harmonic}, temos

\begin{equation}
	\sum_{m=0}^\infty \left(\frac1\rho\derpar{}{\rho}\rho\derpar{W_m}{\rho} - 
    					    \frac{m^2}{\rho^2}W_m\right)\cos(m\theta) = 0
\end{equation}
de onde podemos encontrar a dependência explícita em $\rho$ de $W$,
\begin{equation}\label{eq:wake_potential_of_rho}
	W_m(\rho,\rho',s) = A_m(\rho',s)\rho^m.
\end{equation}
Na equação \eqref{eq:wake_potential_of_rho} a solução singular na origem, $W_m \propto \rho^{-m}$ foi descartada.

Também é possível encontrar a dependência de $W_m$ em função de $\rho'$, \todo{Achar essa derivação e incluir aqui} ver \cite{Bane_PAC1983}, que é
\begin{equation}
	A_m(\rho',s) = F_m(s)\rho'^m.
\end{equation}

Usando a equação \eqref{eq:wake_function_definition} podemos calcular os \engw{wake functions}
\begin{equation}
	w_l = \sum w_l^{(m)}, \qquad \vect{w}_t = \sum \vect{w}_t^{(m)}
\end{equation}
onde
\begin{equation}\label{eq:wake_function_cylindrical}\begin{aligned}
w_l^{(m)} &= \rho'^m\rho^mF'_m(s)\cos(m\theta),\\
\vect{w}_t^{(m)} &= m\rho'^m\rho^{m-1} F_m(s)\left[\vect{\hat{r}}\cos(m\theta) - 
												  \vect{\hat{\theta}}\sin(m\theta)\right]
\end{aligned}\end{equation}
onde $\vect{\hat{r}}$ e $\vect{\hat{\theta}}$ são vetores unitários nas direções radial e azimutal no sistema cilíndrico de coordenadas e $F'_m$ é a derivada de $F_m$ em relação à $s$. Lembre que nessas equações nós assumimos que a partícula fonte está em $\theta = 0$.

As equações \eqref{eq:wake_function_cylindrical} são válidas para valores arbitrários de $\rho$ e $\rho'$. Próximo ao eixo, onde os deslocamentos são pequenos, os termos de ordem mais alta, com valores grandes de $m$, também ficam pequenos. Nesse caso podemos manter apenas os termos não nulos de mais baixa ordem,
\begin{equation}\label{eq:wake_function_expanded}\begin{aligned}
	w_l & \equiv w_l^{(0)} = F'_0(s), \\
    \vect{w}_t & \equiv \vect{w}_t^{(1)} = F_1(s)\rho'\left(\vect{\hat{r}}\cos(\theta) - \vect{\hat{\theta}}\sin(\theta)\right) = \vect{\rho'} F_1(s)
\end{aligned}\end{equation}
onde a última igualdade da última equação é justificada porque $\vect{\hat{r}}\cos(\theta) - \vect{\hat{\theta}}\sin(\theta) = \vect{\hat{x}}$, que é a direção em que está a partícula fonte, de acordo com nossa definição inicial. 

Geralmente o \engw{wake} transversal definido na equação \eqref{eq:wake_function_expanded} é redefinido como
\begin{equation}
	w_t(s) \equiv \frac{|\vect{w}_t|}{|\vect{\rho'}|} = F_1(s)
\end{equation}
e adquire a unidade \si{\volt\per\coulomb\per\meter}. De acordo com essa definição, um \engw{wake} transversal positivo significa um impulso na direção do deslocamento da partícula fonte (caso ambas cargas tenham o mesmo sinal).


%%%%%%%%%%%%%%%%%%%%%%%%%%%%%%%%%%%%%%%%%%%%%%%%%%%%%%%%%%%%%%%%%%%%%%%%%%%%%%%%%%%%%%%%%%%%%%%%%%%%%%%%%
%%%%%%%%%%%%%%%%%%%%%%%%%%%%%%%%%%%%%%%%%%%%%%%%%%%%%%%%%%%%%%%%%%%%%%%%%%%%%%%%%%%%%%%%%%%%%%%%%%%%%%%%%
\section{Impedâncias}

Nesta seção vamos definir o conceito de impedância e derivar algumas de suas propriedades.

%%%%%%%%%%%%%%%%%%%%%%%%%%%%%%%%%%%%%%%%%%%%%%%%%%%%%%%%%%%%%%%%%%%%%%%%%%%%%%%%%%%%%%%%%%%%%%%%%%%%%%%%%
\subsection{Definição de Impedância}
O conhecimento dos \engw{wake functions} longitudinal e transversal nos dá uma informação completa, dentro da aproximação de feixe rígido, sobre a interação eletromagnética do feixe com o seu ambiente. Contudo, em muitos casos, especialmente no estudo de instabilidades do feixe, é mais conveniente usar a Transformada de Fourier dos \engw{wake functions} ou as impedâncias. Também, geralmente é mais fácil calcula a impedância para uma dada geometria da câmara de vácuo ao invés da \engw{wake function}.

Por razões históricas a impedâncias longitudinal, $Z_l$, e transversal, $Z_t$, são definidas como a Transformada de Fourier dos \engw{wakes} com fatores multiplicativos diferentes,
\begin{equation}\label{eq:impedances_definition}\begin{aligned}
Z_l(\omega) &= \frac1c \defint{s}{w_l(s)e^{i\omega s/c}}{0}{\infty},\\
Z_t(\omega) &= -\frac{i}{c} \defint{s}{w_t(s)e^{i\omega s/c}}{0}{\infty}.
\end{aligned}\end{equation}
Note que a integração na equação \eqref{eq:impedances_definition} pode ser estendida para a região de valores negativos de $s$, porque $w_l$ e $w_t$ são nulos naquela região. Além disso, a impedância pode ser definida para valores complexos de $\omega$, desde que $\Im(\omega) > 0 $ para que a integral convirja. Dessa forma, a impedância é uma função analítica no plano superior da variável complexa $\omega$.

Um aspecto importante a respeito da definição de impedância é que há divergências em sua definição na literatura. Por exemplo, as referências \cite{Zotter1993} e \cite{Wilson1987} definem a impedância longitudinal como o complexo conjugado da equação \eqref{eq:impedances_definition}. Nesse trabalho estamos seguindo a definição das referências \cite{CHao1993,Stupakov2000a,Heifets1991}.

{\huge Até aqui}

%%%%%%%%%%%%%%%%%%%%%%%%%%%%%%%%%%%%%%%%%%%%%%%%%%%%%%%%%%%%%%%%%%%%%%%%%%%%%%%%%%%%%%%%%%%%%%%%%%%%%%%%%
%%%%%%%%%%%%%%%%%%%%%%%%%%%%%%%%%%%%%%%%%%%%%%%%%%%%%%%%%%%%%%%%%%%%%%%%%%%%%%%%%%%%%%%%%%%%%%%%%%%%%%%%%
\section{Cálculo dos wake-potential a partir do ECHOzR}

De acordo com a referência, temos que:

\begin{align}
W_{||}(x_0,y_0,x,y,s) &= \frac1w\sum_{m=1}^\infty W_m(y_0,y,s)\sin(k_{x,m}x_0)\sin(k_{x,m}x), \\
W_y(x_0,y_0,x,y,s) &= \frac1w\sum_{m=1}^\infty k_{x,m}W_{y,m}(y_0,y,s)\sin(k_{x,m}x_0)\sin(k_{x,m}x), \\
W_x(x_0,y_0,x,y,s) &= \frac1w\sum_{m=1}^\infty k_{x,m}W_{x,m}(y_0,y,s)\sin(k_{x,m}x_0)\cos(k_{x,m}x),
\end{align}
where $w$ is the half-width of the structure, $0<x<2w$ is the horizontal position of the trailing particle, $0<x_0<2w$ is the horizontal position of the source particle, $y$ is vertical position of the trailing particle, $y_0$ is vertical position of the source particle, $s=z-ct$ is the position of the trailing particle relative to the source particle and 
\begin{equation}
k_{x,m} = \frac{\pi}{2w}m
\end{equation}
The other terms are given by
\begin{align}
W_m(y_0,y,s)     &= W^{cc}_m(s)\cosh(k_{x,m}y_0)\cosh(k_{x,m}y) + W^{ss}_m(s)\sinh(k_{x,m}y_0)\sinh(k_{x,m}y), \\
W_{y,m}(y_0,y,s) &= S^{cc}_m(s)\cosh(k_{x,m}y_0)\sinh(k_{x,m}y) + S^{ss}_m(s)\sinh(k_{x,m}y_0)\cosh(k_{x,m}y), \\
W_{x,m}(y_0,y,s) &= S^{cc}_m(s)\cosh(k_{x,m}y_0)\cosh(k_{x,m}y) + S^{ss}_m(s)\sinh(k_{x,m}y_0)\sinh(k_{x,m}y), 
\end{align}
where
\begin{equation}
S^{cc}_m = \int_{-\infty}^s W^{cc}_m(s')\mathrm{d}s', \qquad S^{ss}_m = \int_{-\infty}^s W^{ss}_m(s')\mathrm{d}s'
\end{equation}
and the quantities $W^{cc}_m(s')$ and $W^{ss}_m(s')$ are related to the output of the softwares ECHOzR and ECHO2D by:
\begin{align}
W^{cc}_m(s') = \frac{W_m^M(y_0,y,s)}{\cosh(k_{x,m}y_0)\cosh(k_{x,m}y)} \\
W^{ss}_m(s') = \frac{W_m^E(y_0,y,s)}{\sinh(k_{x,m}y_0)\sinh(k_{x,m}y)}
\end{align}
where $W_m^M(y_0,y,s)$ and $W_m^M(y_0,y,s)$ are the results of the simulation with magnetic and electric boundary conditions, respectively.


Our objective is to obtain the formulas for the monopole longitudinal, dipole and quarupolar transverse wake functions at the point $\vec{v} = (x_0=w,y_0=0,x=w,y=0)$ from these results. The monopolar wakes are readly obtained:
\begin{align}
W_m(0,0,s) = W^{cc}_m(s) &\Rightarrow W_{||}(\vec{v}) = \frac1w\sum^\infty_{m=1,\mathrm{odd}} W^{cc}_m(s), \\
W_{y,m}(0,0,s) = 0 &\Rightarrow W_y(\vec{v}) = 0, \\
W_{x,m}(0,0,s) = S^{cc}_m(s) &\Rightarrow W_x(\vec{v}) = \frac1w\sum^\infty_{m=1} S^{cc}_m(s)\frac{\sin(m\pi)}{2} = 0
\end{align}
To calculate the dipolar and quadrupolar wakes, we need to take the first derivative of the transverse wakes at the point of interest, $\vec{v}$. It is easy to see that the first derivatives of the longitudinal wake are zero. For the transverse:

\begin{align}
W_{y,d}(s) &= \left.\frac{\mathrm{d}}{\mathrm{d}y_0}W_y\right|_{\vec{v}} = \frac1w\sum^\infty_{m=1,\mathrm{odd}} k_{x,m}^2 S^{ss}_m(s) = \frac1w\int_{-\infty}^s\sum^\infty_{m=1,\mathrm{odd}} k_{x,m}^2 W^{ss}_m(s') \mathrm{d}s',\\
W_{y,q}(s) &= \left.\frac{\mathrm{d}}{\mathrm{d}y}W_y\right|_{\vec{v}} = \frac1w\sum^\infty_{m=1,\mathrm{odd}} k_{x,m}^2 S^{cc}_m(s) = \frac1w\int_{-\infty}^s\sum^\infty_{m=1,\mathrm{odd}} k_{x,m}^2 W^{cc}_m(s') \mathrm{d}s', \\
W_{x,d}(s) &= \left.\frac{\mathrm{d}}{\mathrm{d}x_0}W_x\right|_{\vec{v}} = \frac1w\sum^\infty_{m=1,\mathrm{even}} k_{x,m}^2 S^{cc}_m(s) = \frac1w\int_{-\infty}^s\sum^\infty_{m=1,\mathrm{even}}k_{x,m}^2 W^{cc}_m(s') \mathrm{d}s', \\
W_{x,q}(s) &= \left.\frac{\mathrm{d}}{\mathrm{d}x}W_x\right|_{\vec{v}} = -\frac1w\sum^\infty_{m=1,\mathrm{odd}} k_{x,m}^2 S^{cc}_m(s) =-\frac1w\int_{-\infty}^s\sum^\infty_{m=1,\mathrm{odd}} k_{x,m}^2 W^{cc}_m(s') \mathrm{d}s'
\end{align}
and all the skew terms are zero.

%%%%%%%%%%%%%%%%%%%%%%%%%%%%%%%%%%%%%%%%%%%%%%%%%%%%%%%%%%%%%%%%%%%%%%%%%%%%%%%%%%%%%%%%%%%%%%%%%%%%%%%%%
%%%%%%%%%%%%%%%%%%%%%%%%%%%%%%%%%%%%%%%%%%%%%%%%%%%%%%%%%%%%%%%%%%%%%%%%%%%%%%%%%%%%%%%%%%%%%%%%%%%%%%%%%
\section{Wake Calculation From GdfiDL Simulations}

First lets consider the general expansion of the Wake potential of a bunch in the transverse coordinates of the centroid of the source $(x_s,y_s)$ and the integration path $(x,y)$ up to third order:

\begin{align}
W(\vect{x},s) &= W_0(s) + \sum_{i=1}^4 M_i(s) x_i + \frac12\sum_{i,j=1}^4 D_{ij}(s) x_i x_j + \frac13\sum_{i,j,k=1}^4 Q_{ijk}(s) x_i x_j x_k
\end{align}
where we considered $\vect{x} = (x_1,x_2,x_3,x_4) = (x_s,y_s,x,y)$ and because of the commutative property of multiplication, we can set $D_{ij} = D_{ji}$ and $Q_{ijk}=Q_{kij}=Q_{jki}=Q_{ikj}=Q_{jik}=Q_{kji}$ without loss of generality. With these considerations, number of independent components of $D$ is 10 and $Q$ is 20. Besides that, the fact that $W$ is an harmonic function of the transverse coordinates of the integration path, imposes that
\begin{align}
D_{33} &= - D_{44} \\
Q_{33i} &= - Q_{44i}, \quad \text{with} \quad i=1,2,3,4
\end{align}
which leaves only nine independent components of $D$ and sixteen of $Q$.

Thus, the wake forces become:
\begin{align}
F_L(\vec{x},s) = W'(\vec{x},s) &= W_0' + \sum_{i=1}^4 M_i' x_i + \frac12\sum_{i,j=1}^4 D_{ij}' x_i x_j + \frac13\sum_{i,j,k=1}^4 Q_{ijk}' x_i x_j x_k \\
F_x(\vec{x},s) = \derpar{W(\vect{x},s)}{x} &= M_3 + \sum_{i=1}^4 D_{3i} x_i + \sum_{i,j=1}^4 Q_{3ij} x_i x_j \\
F_y(\vec{x},s) = \derpar{W(\vect{x},s)}{y} &= M_4 + \sum_{i=1}^4 D_{4i} x_i + \sum_{i,j=1}^4 Q_{4ij} x_i x_j
\end{align}

Generally we are interested in obtaining the linear terms as function of the transverse coordinates correct up to second order. It means we want to isolate the linear from the quadract terms in the simulations. Thus, lets keep only the second order terms in the wake forces above.
\begin{align}
F_L(\vect{x},s) &= W_0' + \vect{M'}^T \cdot \vect{x} + \frac12\vect{x}^T \cdot \tensor{D'}  \cdot \vect{x} \\
F_x(\vect{x},s) &= M_x + \vect{D_x}^T \cdot \vect{x} +        \vect{x}^T \cdot \tensor{Q_x} \cdot \vect{x} \\
F_y(\vect{x},s) &= M_y + \vect{D_y}^T \cdot \vect{x} +        \vect{x}^T \cdot \tensor{Q_y} \cdot \vect{x}
\end{align}
where
\begin{align}
\vect{M} &= \begin{pmatrix*}[r] M_{1}\\ M_{2}\\ M_{3}\\ M_{4}\end{pmatrix*} \\
\tensor{D}  &= \begin{pmatrix*}[r] D_{11} & D_{12} & D_{13} & D_{14} \\
                                   D_{21} & D_{22} & D_{23} & D_{24} \\
                                   D_{31} & D_{32} & D_{33} & D_{34} \\
                                   D_{41} & D_{42} & D_{43} & D_{44} 
               \end{pmatrix*} = 
               \begin{pmatrix*}[r] D_{11} & D_{12} & D_{13} & D_{14} \\
                                   D_{12} & D_{22} & D_{23} & D_{24} \\
                                   D_{13} & D_{23} & D_{33} & D_{34} \\
                                   D_{14} & D_{24} & D_{34} & -D_{33}  
               \end{pmatrix*}\\
\vect{D_x} &= \tensor{D}\cdot \vect{\hat{x}} \\
\vect{D_y} &= \tensor{D}\cdot \vect{\hat{y}} \\
\tensor{Q_x}&= \begin{pmatrix*}[r] Q_{311} & Q_{312} & Q_{313} & Q_{314} \\
                                   Q_{321} & Q_{322} & Q_{323} & Q_{324} \\
                                   Q_{331} & Q_{332} & Q_{333} & Q_{334} \\
                                   Q_{341} & Q_{342} & Q_{343} & Q_{344} 
               \end{pmatrix*} = 
               \begin{pmatrix*}[r] Q_{113} & Q_{123} & Q_{133} & Q_{134} \\
                                   Q_{123} & Q_{223} & Q_{233} & Q_{234} \\
                                   Q_{133} & Q_{233} & Q_{333} & Q_{334} \\
                                   Q_{134} & Q_{234} & Q_{334} & -Q_{333} 
               \end{pmatrix*}\\
\tensor{Q_y}&= \begin{pmatrix*}[r] Q_{411} & Q_{412} & Q_{413} & Q_{414} \\
                                   Q_{421} & Q_{422} & Q_{423} & Q_{424} \\
                                   Q_{431} & Q_{432} & Q_{433} & Q_{434} \\
                                   Q_{441} & Q_{442} & Q_{443} & Q_{444} 
               \end{pmatrix*} = 
               \begin{pmatrix*}[r] Q_{114} & Q_{124} & Q_{134} & Q_{133} \\
                                   Q_{124} & Q_{224} & Q_{234} & Q_{233} \\
                                   Q_{134} & Q_{234} & Q_{334} & -Q_{333} \\
                                   Q_{133} & Q_{233} &-Q_{333} & -Q_{334} 
               \end{pmatrix*}\\
\end{align}



Due to their importance, some components of the $D$ tensor have a name: the $D_{31}$ and $D_{42}$ are called dipolar wakes; and $D_{33}$ and $D_{44}$ are the quadrupolar wakes. While the first generates coherent tune-shifts and instabilities, the later is the responsible for incoherent tune-shifts of the beam. The other terms are the skew components, which are zero for most practical cases, as we will see below.

\subsection{Symmetry analysis}

When the geometry has symmetry the number of independent compononts in $M$, $D$ and $Q$ are reduced even more.
When the symmetry occurs in one plane, the number of independent components is two, five and eight, respectivelly.Below we list examples for the most practical cases:

\begin{itemize}
\item symmetry in the $yz$ plane, or $x=0$.
\begin{align}
W(x_s,y_s,x,y,s) = & W(-x_s,y_s,-x,y,s) \Rightarrow \\
M_2= & M_4=0\\
D_{12}=D_{14}= & D_{23}=D_{34}=0\\
Q_{111}=Q_{122}= & Q_{113}=Q_{223}=0\\
Q_{133}=Q_{333}= & Q_{124}=Q_{234}=0
\end{align}
\begin{align}
\vect{M}&= \begin{pmatrix} M_{1}\\ 0\\ M_{3}\\ 0\end{pmatrix} &
\tensor{D} &= \begin{pmatrix} D_{11} &    0   & D_{13} &    0    \\
                                  0   & D_{22} &    0   &  D_{24} \\
                               D_{13} &    0   & D_{33} &    0    \\
                                  0   & D_{24} &    0   & -D_{33}  
               \end{pmatrix}\\
\tensor{Q_x}&=\begin{pmatrix}    0    & Q_{123} &    0    & Q_{134} \\
                               Q_{123} &    0    & Q_{233} &    0    \\
                                  0    & Q_{233} &    0    & Q_{334} \\
                               Q_{134} &    0    & Q_{334} &    0    
               \end{pmatrix} &
\tensor{Q_y}&=\begin{pmatrix} Q_{114} &    0    & Q_{134} &    0    \\
                                  0    & Q_{224} &    0    & Q_{233} \\
                               Q_{134} &    0    & Q_{334} &    0    \\
                                  0    & Q_{233} &    0    & -Q_{334} 
               \end{pmatrix}
\end{align}

\item symmetry in the $xz$ plane, or $y=0$.
\begin{align}
W(x_s,y_s,x,y,s) = & W(x_s,-y_s,x,-y,s) \Rightarrow \\
M_1= & M_3=0\\
D_{12}=D_{14}= & D_{23}=D_{34}=0\\
Q_{112}=Q_{222}= & Q_{123}=Q_{233}=0\\
Q_{114}=Q_{224}= & Q_{134}=Q_{334}=0
\end{align}
\begin{align}
\vect{M}    &= \begin{pmatrix}   0  \\ M_{2}\\  0  \\ M_{4}\end{pmatrix} &
\tensor{D}  &= \begin{pmatrix} D_{11} &    0   & D_{13} &    0   \\
                                  0   & D_{22} &    0   & D_{24} \\
                               D_{13} &    0   & D_{33} &    0   \\
                                  0   & D_{24} &    0   & -D_{33}  
               \end{pmatrix}\\
\tensor{Q_x}&= \begin{pmatrix} Q_{113} &    0    & Q_{133} &    0    \\
                                  0    & Q_{223} &    0    & Q_{234} \\
                               Q_{133} &    0    & Q_{333} &    0    \\
                                  0    & Q_{234} &    0    & -Q_{333} 
               \end{pmatrix} &
\tensor{Q_y}&= \begin{pmatrix}    0    & Q_{124} &    0    & Q_{133} \\
                               Q_{124} &    0    & Q_{234} &    0    \\
                                  0    & Q_{234} &    0    & -Q_{333} \\
                               Q_{133} &    0    &-Q_{333} &    0    
               \end{pmatrix}
\end{align}

\item symmetry in the plane $y=x$.
\begin{align}
W(x_s,y_s,x,y,s) =& W(y_s,x_s,y,x,s) \Rightarrow \\
M_1=M_2, \quad & M_3=M_4\\
D_{11}=D_{22}, \quad D_{13}=D_{24}, \quad & D_{14}=D_{23}, \quad D_{33}=0 \\
Q_{111}=Q_{222}, \quad Q_{112}= Q_{122}, \quad & Q_{113}=Q_{224}, \quad Q_{114}= Q_{223} \\
Q_{123}=Q_{124}, \quad Q_{133}=-Q_{233}, \quad & Q_{134}=Q_{234}, \quad Q_{333}=-Q_{334}
\end{align}
\begin{align}
\vect{M}  &= \begin{pmatrix} M_{1}\\ M_{1}\\ M_{3}\\ M_{3}\end{pmatrix} &
\tensor{D}   &=\begin{pmatrix} D_{11} & D_{12} & D_{13} & D_{14} \\
                               D_{12} & D_{11} & D_{14} & D_{13} \\
                               D_{13} & D_{14} &    0   & D_{34} \\
                               D_{14} & D_{13} & D_{34} &    0    
               \end{pmatrix}\\
\tensor{Q_x}&= \begin{pmatrix} Q_{113} & Q_{123} & Q_{133} & Q_{134} \\
                               Q_{123} & Q_{114} &-Q_{133} & Q_{134} \\
                               Q_{133} &-Q_{133} & Q_{333} &-Q_{333} \\
                               Q_{134} & Q_{134} &-Q_{333} &-Q_{333} 
               \end{pmatrix} &
\tensor{Q_y} &=\begin{pmatrix} Q_{114} & Q_{123} & Q_{134} & Q_{133} \\
                               Q_{123} & Q_{113} & Q_{134} &-Q_{133} \\
                               Q_{134} & Q_{134} &-Q_{333} &-Q_{333} \\
                               Q_{133} &-Q_{133} &-Q_{333} & Q_{333} 
               \end{pmatrix}
\end{align}

\item symmetry in the plane $y=-x$.
\begin{align}
W(x_s,y_s,x,y,s) =& W(-y_s,-x_s,-y,-x,s) \Rightarrow \\
M_1=-M_2, \quad & M_3=-M_4\\
D_{11}=D_{22}, \quad D_{13}=D_{24}, \quad & D_{14}=D_{23}, \quad D_{33}=0 \\
Q_{111}=-Q_{222}, \quad Q_{112}=-Q_{122}, \quad & Q_{113}=-Q_{224}, \quad Q_{114}=-Q_{223} \\
Q_{123}=-Q_{124}, \quad Q_{133}= Q_{233}, \quad & Q_{134}=-Q_{234}, \quad Q_{333}= Q_{334}
\end{align}
\begin{align}
\vect{M}  &= \begin{pmatrix} M_{1}\\ -M_{1}\\ M_{3}\\ -M_{3}\end{pmatrix} &
\tensor{D}   &=\begin{pmatrix} D_{11} & D_{12} & D_{13} & D_{14} \\
                               D_{12} & D_{11} & D_{14} & D_{13} \\
                               D_{13} & D_{14} &    0   & D_{34} \\
                               D_{14} & D_{13} & D_{34} &    0    
               \end{pmatrix}\\
\tensor{Q_x}&= \begin{pmatrix} Q_{113} & Q_{123} & Q_{133} & Q_{134} \\
                               Q_{123} &-Q_{114} & Q_{133} &-Q_{134} \\
                               Q_{133} & Q_{133} & Q_{333} & Q_{333} \\
                               Q_{134} &-Q_{134} & Q_{333} &-Q_{333} 
               \end{pmatrix} &
\tensor{Q_y} &=\begin{pmatrix} Q_{114} &-Q_{123} & Q_{134} & Q_{133} \\
                              -Q_{123} &-Q_{113} &-Q_{134} & Q_{133} \\
                               Q_{134} &-Q_{134} & Q_{333} &-Q_{333} \\
                               Q_{133} & Q_{133} &-Q_{333} &-Q_{333} 
               \end{pmatrix}
\end{align}
\end{itemize}

Below we combine some of the above mentioned symmetries which are very commom in the simulations performed for accelerators elements:

\begin{itemize}
\item x=0 and y=0.
\begin{align}
\vect{M}= \vect{0} & &
\tensor{D} = \begin{pmatrix} D_{11} &    0   & D_{13} &    0    \\
                                  0   & D_{22} &    0   &  D_{24} \\
                               D_{13} &    0   & D_{33} &    0    \\
                                  0   & D_{24} &    0   & -D_{33}  
               \end{pmatrix} & &
\tensor{Q_x}=\tensor{0} & &
\tensor{Q_y}=\tensor{0}
\end{align}

\item $y=-x$ and $y=x$.
\begin{align}
\vect{M}= \vect{0} & &
\tensor{D}   &=\begin{pmatrix} D_{11} & D_{12} & D_{13} & D_{14} \\
                               D_{12} & D_{11} & D_{14} & D_{13} \\
                               D_{13} & D_{14} &    0   & D_{34} \\
                               D_{14} & D_{13} & D_{34} &    0    
               \end{pmatrix}& &
\tensor{Q_x}=\tensor{0} & &
\tensor{Q_y}=\tensor{0}
\end{align}

\item $x=0$, $y=0$, $y=-x$ and $y=x$.
\begin{align}
\vect{M}= \vect{0} & &
\tensor{D}   &=\begin{pmatrix} D_{11} &   0    & D_{13} &   0    \\
                                 0    & D_{11} &    0   & D_{13} \\
                               D_{13} &   0    &    0   &   0    \\
                                 0    & D_{13} &    0   &   0    
               \end{pmatrix} & &
\tensor{Q_x}=\tensor{0} & &
\tensor{Q_y}=\tensor{0}
\end{align}
\end{itemize}
where we notice that when there is at least 2 planes of symmetry, all the odd order terms are zero.

\subsection{GDFIDL Simulations}

One simulation in GDFIDL consists of passing a linear gaussian bunch with the velocity of light in the longitudinal direction and with a specific transverse position, say $(x_s,y_s)=(d_x,d_y)$ through the simulated structure solving Maxwell equations in time. While doing this, the code saves in memory the wake potential $W(d_x,d_y,x,y,s)$ for all transverse positions $(x,y)$ of integration and all $s$. This procedure is very time consuming and we gerally try to perform the mininum amount of simulations possible to get the results we need.

Lets suppose a simulation was performed with the position of the source in $(x_s,y_s)=(d,0)$. If we calculate the Horizontal Wake potential in a path where $y=0$ then, up to second order in the transverse coordinates we can write:

\begin{align}
F_L(d,0,x,0) &= W_0  +  M_1d  +  M_3x + D_{11}d^2 + D_{13}dx + D_{33}x^2 \\
F_x(d,0,x,0) &= M_x + D_{x3} x + D_{x1} d + Q_{x33} x^2 + Q_{x11} d^2 + Q_{x13} x d\\
F_y(d,0,x,0) &= M_y + D_{y3} x + D_{y1} d + Q_{y33} x^2 + Q_{y11} d^2 + Q_{y13} x d
\end{align}

Now, integrating the total wake in 4 different $x$ points we get:
\begin{align}
	F_1 = F_x(d,x_1) &= M_x  +  D_{x3} x_1  +  D_{x1} d  +  Q_{x33} x_1^2  +  Q_{x11} d^2  +  Q_{x13} x_1 d \\
    F_2 = F_x(d,x_2) &= M_x  +  D_{x3} x_2  +  D_{x1} d  +  Q_{x33} x_2^2  +  Q_{x11} d^2  +  Q_{x13} x_2 d \\
    F_3 = F_x(d,x_3) &= M_x  +  D_{x3} x_3  +  D_{x1} d  +  Q_{x33} x_3^2  +  Q_{x11} d^2  +  Q_{x13} x_3 d \\
    F_4 = F_x(d,x_4) &= M_x  +  D_{x3} x_4  +  D_{x1} d  +  Q_{x33} x_4^2  +  Q_{x11} d^2  +  Q_{x13} x_4 d
\end{align}

To extract the component $D_x$ from the total wake we do:
\begin{align}
\Delta_1 = \frac{F_1}{x_1} - \frac{F_2}{x_2} &= \left(M_x  +  D_{x1}d  +  Q_{x11}d^2\right)\left(\frac{1}{x_1}-\frac{1}{x_2}\right) + Q_{x33}(x_1-x_2) \\
\Delta_2 = \frac{F_3}{x_3} - \frac{F_4}{x_4} &= \left(M_x  +  D_{x1}d  +  Q_{x11}d^2\right)\left(\frac{1}{x_3}-\frac{1}{x_4}\right) + Q_{x33}(x_3-x_4)
\end{align}
then, remembering $(1/a-1/b)/(a-b) = -1/ab$ we get: 
\begin{align}
	\frac{\Delta_1}{x_1-x_2} - \frac{\Delta_2}{x_3-x_4} &= \left(M_x  +  D_{x1} d  +  Q_{x11} d^2\right)\left(\frac{1}{x_3x_4} - \frac{1}{x_1x_2}\right)
\end{align}
\begin{align}
	M_x  +  D_{x1} d +  Q_{x11} d^2 &= \left(\frac{x_1x_2x_3x_4}{x_3x_4 - x_1x_2}\right)\left(\frac{\Delta_1}{x_1-x_2} - \frac{\Delta_2}{x_3-x_4}\right)
\end{align}

Note that in general to isolate the dipolar component $D_x$, it is necessary to perform another two simulations. If another simulation is performed with the source at $(x_s,y_s) = (-d,0)$ it is also possible to isolate the dipolar component.

Now, lets try to extract the quadrupolar component
\begin{align}
\Delta_1 = \frac{F_1 - F_2}{x_1-x_2} &= D_{x3} + Q_{x13}d + Q_{x33}(x_1 + x_2)
\end{align}
Notice that, if we choose $x_2=-x_1$ the contribution from $Q_{x33}$ is canceled and
\begin{align}
	D_{x3} + Q_{x13}d = \Delta_1
\end{align}
where it is clear that is necessary other simulation to extract the quadrupolar wake.

%%%%%%%%%%%%%%%%%%%%%%%%%%%%%%%%%%%%%%%%%%%%%%%%%%%%%%%%%%%%%%%%%%%%%%%%%%%%%%%%%%%%%%%%%%%%%%%%%%%%%%%%%
%%%%%%%%%%%%%%%%%%%%%%%%%%%%%%%%%%%%%%%%%%%%%%%%%%%%%%%%%%%%%%%%%%%%%%%%%%%%%%%%%%%%%%%%%%%%%%%%%%%%%%%%%
\section{Panofski-Wenzel}

O conceito de wake-function e impedância tem seu surgimento motivado pelo teorema de Panofski-Wenzel. Esse teorema é resultado de duas aproximações feitas na análise do problema de duas partículas relativísticas interagindo entre si por meio de campos eletromagnéticos espalhados pela câmara de vácuo do anel de armazenamento. Por sua central importância para o subsequente desenvolvimento desse trabalho, vamos olhar esse teorema com um pouco mais de detalhe.

Primeiro, vamos considerar a imagem apresentada na figura X, ou seja, consideremos uma situação em que temos uma partícula de carga $Q$ atravessando a estrutura ilustrada com velocidade $\vec{v}$ ao longo da câmara de vácuo de um anel de armazenamento. A uma distância longitudinal $z$ dela, uma partícula de prova com carga $q$ também atravessa a estrutura na mesma direção. Estamos interessados em saber qual é a força que essa carga de prova sentirá ao longo de sua passagem por toda a estrutura. Pela equação de Lorentz, temos a todo instante:
\begin{equation}
 \vec{F} = q\left(\vec{E} + \vec{v} \times \vec{B}\right)
\end{equation}
onde $\vec{E}$ e $\vec{B}$ é o campo eletromagnético na posição da partícula que está sentindo a força. Esse campo pode ser decomposto como a soma de vários campos com origem e interpretação física diferentes: O campo externo, gerado pelos ímãs e cavidades de RF que guiam o feixe e o campo de interação, que foi gerado pela partícula fonte, de carga $Q$. O campo de interação ainda pode ser decomposto em duas contribuições distintas: o campo direto, que existiria mesmo sem a presença da câmara de vácuo e o campo indireto, resultado da interação dessa carga com a câmara de vácuo. Para obter a transferência de momento total sentida pela carga de prova ao longo de toda estrutura, temos que integrar a equação acima na trajetória da partícula. Ao realizarmos esse procedimento, logo vemos que a solução auto-consistente desse problema é muito complexa, pois a trajetória da partícula prova depende da força que ela sentiu nos tempos anteriores, assim como da trajetória da carga fonte, devido à dependência dos campos eletromagnéticos que são gerados por ela.

Assim, para conseguir seguir mais adiante na análise desse problema devemos fazer aproximações. A primeira delas é considerar que ambas as partículas são rígidas e possuem velocidades paralelas uma com a outra e paralelas ao eixo de simetria da câmara de vácuo (caso a câmara de vácuo não tenha simetria, considera-se a velocidade das partículas paralelas à direção em que o feixe de elétrons deverá passar). Essa consideração simplifica drasticamente a análise do problema, como veremos logo a seguir.

Antes de seguir com a demonstração do teorema, vamos justificar essa aproximação. Na maioria dos aceleradores de partículas as partículas estão no limite ultra-relativístico, em que sua velocidade é aproximadamente a da luz e varia muito pouco com variação de momento. Isso implica que nesse limite as partículas são bastante rígidas

